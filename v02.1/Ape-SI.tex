\chapter{Sistema Internacional d'Unitats (SI)} \index{sistema internacional d'unitats}

S'expliquen a continuaci\'{o} q\"{u}estions relacionades amb el sistema
internacional d'unitats (SI), el qual est\`{a} definit pel {"<}Bureau
International des Poids et Mesures{">} (\textsf{BIPM}). Podeu trobar
m\'{e}s informaci\'{o} a l'adre\c{c}a: \href{http://www.bipm.fr/}{www.bipm.fr}.\index{BIPM@\textsf{BIPM}}

\section{Unitats fonamentals}
\index{sistema internacional d'unitats!unitats fonamentals}

En la Taula \vref{taula:SI-fonamentals} es poden veure les unitats
fonamentals del sistema internacional d'unitats.

\begin{table}[h]
   \caption{\label{taula:SI-fonamentals} Unitats fonamentals de l'SI}
   \begin{center}\begin{tabular}{llc}
   \toprule[1pt]
   Magnitud & Unitat & S\'{\i}mbol \\
   \midrule
   longitud & metre & m \\
   massa & quilogram & kg \\
   temps & segon & s\\
   intensitat de corrent el\`{e}ctric & ampere & A \\
   temperatura termodin\`{a}mica & kelvin & K\\
   quantitat de mat\`{e}ria & mol & mol \\
   intensitat lluminosa & candela &  cd \\
   \bottomrule[1pt]
   \end{tabular} \end{center}
\end{table}
\index{metre} \index{quilogram} \index{segon} \index{amper}
\index{kelvin} \index{mol} \index{candela} \index{longitud}
\index{massa} \index{temps} \index{intensitat de corrent el\`{e}ctric}
\index{temperatura!termodin\`{a}mica} \index{quantitat de mat\`{e}ria}
\index{intensitat lluminosa} \index{m} \index{kg} \index{s}
\index{A} \index{K} \index{cd}

Es presenten a continuaci\'{o}, de forma breu, les definicions
d'aquestes unitats fonamentals; entre par\`{e}ntesis, s'indica l'any en
qu\`{e} la {"<}Conf\'{e}rence G\'{e}n\'{e}rale des Poids et Mesures{">} les va posar en
vigor.

\begin{list}{}
   {\setlength{\labelwidth}{22mm} \setlength{\leftmargin}{22mm} \setlength{\labelsep}{2mm}}
   \item[\textbf{metre}:] Longitud de la traject\`{o}ria recorreguda per la llum
   en el buit, durant un interval de temps igual a 1/299792458\unit{s} (1983).
   \item[\textbf{quilogram}:] Massa del prototip internacional del quilogram, fet de plat\'{\i}-iridi i
    conservat al {"<}Bureau International des Poids et Mesures{">} (1901).
   \item[\textbf{segon}:] Durada de 9192631770 per\'{\i}odes de la
   radiaci\'{o} corresponent a la transici\'{o} entre els dos nivells
  hiperfins, de l'estat fonamental de l'\`{a}tom de cesi-133 (1967).
   \item[\textbf{ampere}:] Intensitat d'un corrent constant,
   que mantinguda en dos conductors para{\l.l}els rectilinis de longitud
   infinita, de secci\'{o} transversal negligible, i situats a una
   dist\`{a}ncia l'un de l'altre d'un metre, en el buit, produeix una for\c{c}a entre
   aquests dos conductors de $2\cdot10^{-7}\unit{N/m}$ (1948).
   \item[\textbf{kelvin}:] Fracci\'{o} 1/273,16 de la temperatura
   termodin\`{a}mica corresponent al punt triple de l'aigua (1967).
   \item[\textbf{mol}:] Quantitat de mat\`{e}ria que cont\'{e} tantes
   entitats elementals, com \`{a}toms hi ha en 0,012\unit{kg} de carboni-12 (1971).
   \item[\textbf{candela}:] Intensitat lluminosa, en una direcci\'{o} determinada,
   d'una font que emet radiaci\'{o} monocrom\`{a}tica de freq\"{u}\`{e}ncia $540\cdot10^{12}\unit{Hz}$, i
   que t\'{e} una intensitat radiant en aquesta direcci\'{o} de $\frac{1}{683}\unit{W/sr}$ (1979).
\end{list}

\section{Unitats derivades}
\index{sistema internacional d'unitats!unitats derivades}

En la Taula \vref{taula:SI-derivades} es presenta una llista no
exhaustiva, d'unitats derivades del sistema internacional d'unitats.

\begin{longtable}[h]{llclc}
   \caption{\label{taula:SI-derivades} Algunes unitats derivades de
   l'SI}\\
   \toprule[1pt]
    \multirow{2}{15mm}{\rule{0mm}{6mm}Magnitud} & \multirow{2}{15mm}{\rule{0mm}{6mm}Unitat}  &
    \multirow{2}{15mm}{\rule{0mm}{6mm}S\'{\i}mbol}  & \multicolumn{2}{c}{Equival\`{e}ncia en unitats SI}\\
    \cmidrule(rl){4-5}
    &  &   & fonamentals & altres\\
   \midrule
   \endfirsthead
   \caption[]{Algunes unitats derivades de l'SI (\emph{ve de la p\`{a}gina
   anterior})}\\
   \toprule[1pt]
    \multirow{2}{15mm}{\rule{0mm}{6mm}Magnitud} & \multirow{2}{15mm}{\rule{0mm}{6mm}Unitat}  &
    \multirow{2}{15mm}{\rule{0mm}{6mm}S\'{\i}mbol}  & \multicolumn{2}{c}{Equival\`{e}ncia en unitats SI}\\
    \cmidrule(rl){4-5}
    &  &  & fonamentals & altres\\
   \midrule
   \endhead
   \midrule
   \multicolumn{5}{r}{(\emph{continua a la p\`{a}gina seg\"{u}ent})}
   \endfoot
   \endlastfoot
   angle pla & radiant & rad  & \unit{m\cdot m^{-1} = 1} & --- \\
   angle s\`{o}lid & estereoradiant & sr &\unit{m^2\cdot m^{-2} = 1} & ---  \\
   freq\"{u}\`{e}ncia & hertz & Hz & \unit{s^{-1}} & --- \\
   for\c{c}a & newton & N & \unit{m\cdot kg\cdot s^{-2}} & --- \\
   pressi\'{o} & pascal & Pa  & \unit{m^{-1}\cdot kg\cdot s^{-2}} & \unit{N/m^2} \\
   energia, treball & joule & J & \unit{m^2\cdot kg\cdot s^{-2}} & \unit{N\cdot m}\\
   pot\`{e}ncia & watt & W & \unit{m^2\cdot kg\cdot s^{-3}}  & \unit{J/s}\\
   c\`{a}rrega el\`{e}ctrica & coulomb & C  & \unit{s\cdot A} &  ---\\
   potencial el\`{e}ctric & volt & V & \unit{m^2\cdot kg\cdot s^{-3}\cdot A^{-1}}  & \unit{W/A}\\
   capacitat el\`{e}ctrica & farad & F   & \unit{m^{-2}\cdot kg^{-1}\cdot s^4\cdot A^2}& \unit{C/V}\\
   resist\`{e}ncia el\`{e}ctrica & ohm &  \unit{\ohm}  & \unit{m^2\cdot kg\cdot s^{-3}\cdot A^{-2}} & \unit{V/A}\\
   conduct\`{a}ncia el\`{e}ctrica & siemens &  S  & \unit{m^{-2}\cdot kg^{-1}\cdot s^3\cdot A^2} & \unit{A/V}\\
   flux magn\`{e}tic & weber &  Wb  & \unit{m^2\cdot kg\cdot s^{-2}\cdot A^{-1}} & \unit{V\cdot s}\\
   densitat de flux magn\`{e}tic & tesla &  T  & \unit{kg\cdot s^{-2}\cdot A^{-1}} & \unit{Wb/m^2}\\
   induct\`{a}ncia & henry &  H  & \unit{m^2\cdot kg\cdot s^{-2}\cdot A^{-2}} & \unit{Wb/A}\\
   temperatura Celsius & grau Celsius &  \celsius & \unit{K} & --- \\
   flux llumin\'{o}s & lumen & lm  & \unit{cd}& \unit{cd\cdot sr}\\
   i{\l.l}uminaci\'{o} & lux & lx & \unit{m^{-2}\cdot cd} & \unit{lm/m^2} \\
   activitat  d'un radion\'{u}clid & becquerel & Bq& \unit{s^{-1}} & --- \\
   dosi absorbida & gray & Gy  & \unit{m^2\cdot s^{-2}}& \unit{J/kg}\\
   dosi equivalent & sievert & Sv  & \unit{m^2\cdot s^{-2}}& \unit{J/kg}\\
   viscositat din\`{a}mica & pascal segon & \unit{Pa\cdot s}&
   \unit{m^{-1}\cdot kg\cdot s^{-1}} & --- \\
   tensi\'{o} superficial & newton per metre & \unit{N/m} &
   \unit{kg\cdot s^{-2}} & ---\\
    intensitat de camp el\`{e}ctric & volt per metre & \unit{V/m}& \unit{m\cdot kg\cdot s^{-3}\cdot A^{-1}} & --- \\
    permitivitat & farad per metre & \unit{F/m}& \unit{m^{-3}\cdot kg^{-1}\cdot s^4\cdot A^2} & --- \\
   permeabilitat & henry per metre & \unit{H/m} & \unit{m\cdot kg\cdot s^{-2}\cdot A^{-2}} & ---\\
    \bottomrule[1pt]
\end{longtable}
\index{radiant} \index{estereoradiant} \index{hertz} \index{newton}
\index{pascal} \index{joule} \index{watt} \index{coulomb}
\index{volt} \index{farad} \index{ohm} \index{siemens} \index{weber}
\index{tesla} \index{henry} \index{lumen} \index{lux}
\index{becquerel} \index{gray} \index{sievert} \index{grau Celsius}
\index{angle pla}  \index{angle s\`{o}lid} \index{freq\"{u}\`{e}ncia}
\index{for\c{c}a} \index{pressi\'{o}} \index{energia} \index{pot\`{e}ncia}
\index{carrega electrica@c\`{a}rrega el\`{e}ctrica} \index{potencial
el\`{e}ctric} \index{capacitat} \index{resist\`{e}ncia} \index{conduct\`{a}ncia}
\index{flux magn\`{e}tic} \index{densitat de flux magn\`{e}tic}
\index{induct\`{a}ncia} \index{temperatura!Celsius} \index{flux
llumin\'{o}s} \index{iluminacio@i{\l.l}uminaci\'{o}} \index{activitat  d'un
radion\'{u}clid} \index{dosi absorbida}  \index{dosi equivalent}
\index{intensitat de camp el\`{e}ctric} \index{viscositat din\`{a}mica}
\index{tensi\'{o} superficial} \index{permitivitat}
\index{permeabilitat} \index{rad} \index{sr} \index{Hz} \index{N}
\index{Pa} \index{J} \index{W} \index{C} \index{V} \index{F}
\index{$\Omega$} \index{S} \index{Wb} \index{T} \index{H}
\index{\celsius} \index{lm} \index{lx} \index{Bq} \index{Gy}
\index{Sv}

\section{Prefixes}
\index{sistema internacional d'unitats!prefixes}

En la Taula \vref{taula:SI-prefixes} es presenta una llista, amb els
prefixes que es poden anteposar a les unitats del sistema
internacional d'unitats, per tal de formar els seus m\'{u}ltiples i
subm\'{u}ltiples.

\begin{table}[h]
   \caption{\label{taula:SI-prefixes} Prefixes de  l'SI}
   \begin{center}\begin{tabular}{llccllc}
   \toprule[1pt]
   \multicolumn{3}{c}{M\'{u}ltiples} & & \multicolumn{3}{c}{Subm\'{u}ltiples}\\
   \cmidrule(rl){1-3} \cmidrule(rl){5-7}
   factor & nom & s\'{\i}mbol & & factor & nom & s\'{\i}mbol\\
   \midrule
    $10^{24}$ &  yotta & Y & & $10^{-24}$ & yocto & y \\
    $10^{21}$ &  zetta & Z & & $10^{-21}$ & zepto & z \\
    $10^{18}$ &  exa & E & & $10^{-18}$ & atto & a \\
    $10^{15}$ &  peta & P & & $10^{-15}$ & femto & f \\
    $10^{12}$ &  tera & T & & $10^{-12}$ & pico & p \\
    $10^{9}$ &  giga & G & & $10^{-9}$ & nano & n \\
    $10^{6}$ &  mega & M & & $10^{-6}$ & micro & $\mu$ \\
    $10^{3}$ &  kilo & k & & $10^{-3}$ & mi{\l.l}i & m \\
    $10^{2}$ &  hecto & h & & $10^{-2}$ & centi & c \\
    $10^{1}$ &  deca & da & & $10^{-1}$ & deci & d \\
    \bottomrule[1pt]
   \end{tabular} \end{center}
\end{table}
\index{yotta} \index{zetta} \index{exa} \index{peta} \index{tera} \index{giga} \index{mega}
\index{kilo} \index{hecto} \index{deca} \index{deci} \index{centi} \index{mili} \index{micro}
\index{nano} \index{pico} \index{femto} \index{atto} \index{zepto} \index{yocto}

\section{Normes d'escriptura}
\index{sistema internacional d'unitats!normes d'escriptura}

Es presenten a continuaci\'{o}, algunes normes aplicables a l'escriptura
de les unitats del sistema internacional d'unitats.

Els s\'{\i}mbols de les unitats no canvien de forma en el plural, no han
d'utilitzar-se abreviatures, ni han d'afegir-se o suprimir-se
lletres. Exemple: 150\unit{kg}, 25\unit{m},  33\unit{cm^3}
(incorrecte: 150 Kgs, 25 mts, 33 cc).

Els s\'{\i}mbols de les unitats no han d'anar seguits d'un punt (no s\'{o}n
abreviatures), llevat que es trobin al final d'una oraci\'{o}.

Els s\'{\i}mbols de les unitats s'escriuen a la dreta dels valors
num\`{e}rics, separats per un espai en blanc. Exemple: 25\unit{V},
40\unit{\celsius} (incorrecte: 25V, 40\celsius).

En el cas de s\'{\i}mbols d'unitats derivades, formats pel producte
d'altres unitats, el producte s'indicar\`{a} mitjan\c{c}ant un punt volat o
un espai en blanc. Exemple: 24\unit{N\cdot m}, 24\unit{N\,m}. En
aquest darrer cas, cal tenir en compte  l'ordre en qu\`{e} s'escriuen
les unitats, ja que algunes combinacions poden produir confusi\'{o}, i
\'{e}s millor evitar-les. Exemple: 24\unit{N\,m} (24 newton metre) i
24\unit{m\,N} (24 metre newton) s\'{o}n expressions equivalents, per\`{o}
aquesta darrera pot ser confosa amb 24\unit{mN} (24 mi{\l.l}inewton).

En el cas de s\'{\i}mbols d'unitats derivades, formats per la divisi\'{o}
d'altres unitats, la divisi\'{o} s'indicar\`{a} mitjan\c{c}ant una l\'{\i}nia
inclinada o horitzontal, o mitjan\c{c}ant pot\`{e}ncies negatives. Exemple:
100\unit{m/s}, 100\unit{\frac{m}{s}}, 100\unit{m\cdot s^{-1}}.

En el cas anterior, quan s'utilitza la l\'{\i}nia inclinada i hi ha m\'{e}s
d'una unitat en el denominador, aquestes unitats s'han d'escriure
entre par\`{e}ntesis. Exemple: 5\unit{m\cdot kg/(s^3\cdot A)}
(incorrecte: 5\unit{m\cdot kg/s^3\cdot A}, 5\unit{m\cdot kg/s^3/
A}).

No ha de deixar-se cap espai en blanc entre el s\'{\i}mbol d'un prefixe i
el s\'{\i}mbol d'una unitat. Exemple: 12\unit{pF}, 3\unit{GHz}
(incorrecte: 12\unit{p\,F}, 3\unit{G\,Hz}).

El grup format pel s\'{\i}mbol d'un prefixe i el s\'{\i}mbol d'una unitat
esdev\'{e} un nou s\'{\i}mbol inseparable (formant un m\'{u}ltiple o subm\'{u}ltiple
de la unitat), i pot ser pujat a una pot\`{e}ncia positiva o negativa, i
combinat amb altres s\'{\i}mbols. Exemple: 20\unit{km^2}, 1\unit{V/cm}.

Tan sols es permet un prefixe davant d'una unitat. Exemple:
8\unit{nm} (incorrecte: 8 m\micro m).

En el cas dels s\'{\i}mbols d'unitats derivades, formades per la divisi\'{o}
d'altres unitats, l'\'{u}s de prefixes en el numerador i denominador de
forma simult\`{a}nia pot causar confusi\'{o}, i \'{e}s preferible, per tant,
utilitzar una alta combinaci\'{o} d'unitats on nom\'{e}s el numerador o el
denominador tinguin prefix. Exemple: 10\unit{kV/mm} \'{e}s acceptable,
per\`{o} \'{e}s preferible utilitzar 10\unit{MV/m}.

De forma an\`{a}loga, el mateix \'{e}s aplicable als s\'{\i}mbols d'unitats
derivades, formades pel producte d'altres unitats.  Exemple:
10\unit{MV\cdot ms} \'{e}s acceptable, per\`{o} \'{e}s preferible utilitzar
10\unit{kV\cdot s}.

Els noms de les unitats de l'SI s'escriuen en min\'{u}scula, excepte en
el cas de {"<}grau Celsius{">}, i a l'inici d'una oraci\'{o}.

Les unitats que tenen noms provinents de noms propis, s'han
d'escriure tal com apareixen en les taules
\vref{taula:SI-fonamentals} i \vref{taula:SI-derivades}, i no s'han
de traduir. Exemple: 50 newton, 300 joule (incorrecte: 50 neuton,
300 joul).


 Quan el nom d'una unitat
cont\'{e} un prefixe, ambdues parts s'han d'escriure juntes. Exemple: 1
mi{\l.l}igram (incorrecte: 1 mi{\l.l}i gram, 1 mi{\l.l}i-gram).

En el cas  d'unitats derivades que s'expressen amb divisions o
productes, s'utilitza la preposici\'{o} {"<}per{">} entre dos noms d'unitats
per indicar la divisi\'{o}, i no s'utilitza cap paraula per indicar el
producte. Exemple: 100 \unit{V/m} \'{e}s equivalent a 100 volt per metre
(incorrecte:
 100 volt/metre), 20 \unit{\ohm\cdot m} \'{e}s equivalent a 20 ohm metre
(incorrecte: 20 ohm per metre).

El valor d'una quantitat ha d'expressar-se  utilitzant \'{u}nicament una
unitat. Exemple: 10,234\unit{m} (incorrecte: 10 m 23 cm 4 mm).

Quan s'expressa el valor d'una quantitat, \'{e}s incorrecte afegir
lletres o altres s\'{\i}mbols a la unitat; qualsevol informaci\'{o}
addicional necess\`{a}ria, ha d'afegir-se a la quantitat; Exemple:
U\ped{pp} = 1000\unit{V} (incorrecte: U = 1000\unit{V\ped{pp}}).

Quan s'indiquen valors de magnituds amb les seves desviacions,
s'indiquen intervals, o s'expressen diversos valors num\`{e}rics, les
unitats han de ser presents en cadascun dels valors, o s'han d'usar
par\`{e}ntesis si es vol posar les unitats nom\'{e}s al final. Exemple:
$63{,}2\unit{m} \pm 0{,}1\unit{m}$, $(63{,}2 \pm 0{,}1)\unit{m}$
(incorrecte: $63{,}2 \pm 0{,}1\unit{m}$, $63{,}2\unit{m} \pm
0{,}1$), 0\unit{V} a 5\unit{V}, (0 a 5)\unit{V} (incorrecte: 0 a
5\unit{V}), $50\unit{m}\times 50\unit{m}$ (incorrecte: $50\times
50\unit{m}$), 127\unit{s} + 3\unit{s} = 130\unit{s}, (127 +
3)\unit{s} = 130\unit{s} (incorrecte: 127 + 3\unit{s} =
130\unit{s}).
