\begin{thebibliography}{99}

    \addcontentsline{toc}{chapter}{Bibliografia}

    \bibitem{KOP} \textsc{Helmut Kopka, Patrick W. Daly}. \textsl{A Guide To \LaTeX}.  Addison-Wesley, 3rd edition, 1999.
    \bibitem{GRZ} \textsc{George Gr\"{a}tzer}. \textsl{More Math Into \LaTeX}.  Springer, 4th edition, 2007.
    \bibitem{GOO} \textsc{Michel Goossens, Frank Mittelbach, Sebstian Rahtz, Denis Roegel, Herbert Vo{\ss}}. \textsl{The \LaTeX{} Graphics Companion}.  Addison-Wesley, 2nd edition, 2008.
    \bibitem{VALa} \textsc{Gabriel Valiente Feruglio}. \textsl{Composici\'{o} de textos cient\'{\i}fics amb \LaTeX}.  Edicions UPC, 1998.
    \bibitem{VALb} \textsc{Gabriel Valiente Feruglio}. \textsl{Modern Catalan Typographical Conventions}.  TUGBoat, 16(3), 329-338, 1995.
    \bibitem{BEC} \textsc{Claudio Beccari}. \textsl{Typesetting mathematics for science and technology according to ISO 31/XL}.  TUGBoat, 18(1), 39-48, 1997.
    \bibitem{WIL} \textsc{J. William Howard, Jr}. \textsl{Graecum est: el uso del griego en textos electr\'{o}nicos de car\'{a}cter cient\'{\i}fico-t\'{e}cnico}.  Panace@, VI(19), 45-54, 2005.


    \bibitem{SCH} \textsc{Joel L. Schiff}. \textsl{The Laplace Transform: Theory and Applications}.  Springer, 1999.
    \bibitem{RJB} \textsc{R. J. Beerends, H. G. ter Morsche, J. C. van den Berg, E. M. van de Vrie}. \textsl{Fourier and Laplace Transforms}.  Cambridge University Press, 2003.


    \bibitem{RASa} \textsc{Enrique Ras}. \textsl{Teor\'{\i}a de circuitos. Fundamentos}.  Marcombo Boixareu Editores, 3\textordfeminine\ edici\'{o}n, 1977.
    \bibitem{RASb} \textsc{Enrique Ras}. \textsl{Transformadores. De potencia, medida y protecci\'{o}n}.  Marcombo Boixareu Editores, 6\textordfeminine\ edici\'{o}n, 1985.
    \bibitem{RASc} \textsc{Enrique Ras}. \textsl{Teor\'{\i}a de l\'{\i}neas el\'{e}ctricas (Volumen I)}.  Marcombo Boixareu Editores, 2\textordfeminine\ edici\'{o}n, 1986.
    \bibitem{RASd} \textsc{Enrique Ras}. \textsl{Redes el\'{e}ctricas y multipolos}.  Marcombo Boixareu Editores, 1980.
    \bibitem{RASe} \textsc{Enrique Ras}. \textsl{An\'{a}lisis de Fourier y c\'{a}lculo operacional aplicados a la electrotecnia}.  Marcombo Boixareu Editores, 1979.

    \bibitem{COR} \textsc{Felipe C\'{o}rcoles L\'{o}pez, Joaquim Pedra Dur\'{a}n, Miquel Salichs Vivancos}. \textsl{Transformadores}.  Edicions UPC, 2004.


    \bibitem{CHA} \textsc{Stephen J. Chapman}. \textsl{M\'{a}quinas El\'{e}ctricas}.  McGraw-Hill, 4\textordfeminine\ edici\'{o}n, 2005.
    \bibitem{FIT} \textsc{A. E. Fitzgerald, Charles Kingsley Jr., Stephen D. Umans}. \textsl{Electric Machinery}.  McGraw-Hill, 6th edition, 2003.


    \bibitem{GRA} \textsc{John J. Grainger, William D. Stevenson Jr}. \textsl{An\'{a}lisis de Sistemas de Potencia}.  McGraw-Hill, 1996.
    \bibitem{HAD} \textsc{Hadi Saadat}. \textsl{Power System Analysis}.  McGraw-Hill, 2nd edition, 2004.
    \bibitem{DUN} \textsc{J. Duncan Glover, Mulukutla S. Sarma, Thomas J. Overbye}. \textsl{Power System Analysis \& Design}.  CENGAGE Learning, 5th edition (SI), 2011.

    \bibitem{BLAa} \textsc{J. Lewis Blackburn}. \textsl{Simmetrical Components for Power Systems Engineering}.  Marcel Dekker, Inc, 1993.
    \bibitem{BLAb} \textsc{J. Lewis Blackburn, Thomas J. Domin}. \textsl{Protective Relaying. Principles and Applications}.  CRC Press, 3rd edition, 2007.
    \bibitem{REI} \textsc{Donald Reimert}. \textsl{Protective Relaying for Power Generation Systems}.  CRC Press, 2006.


    \bibitem{TLE} \textsc{Nasser D. Tleis}. \textsl{Power Systems Modelling and Fault Analysis -- Theory and Practise}.  ELSEVIER, 2008.


    \bibitem{CAP} \textsc{Robert Capella}. \textsl{Protecciones el\'{e}ctricas en MT}.  Publicaci\'{o}n T\'{e}cnica de Schneider 071, mayo 2003.
    \bibitem{LLO} \textsc{Manuel Llorente Ant\'{o}n}. \textsl{L\'{\i}neas y cables}.  Publicaci\'{o}n T\'{e}cnica de Schneider 073, enero 2003.
    \bibitem{PAS} \textsc{Jean Pasteau}. \textsl{Envolventes y grados de protecci\'{o}n}.  Cuaderno T\'{e}cnico de Schneider 166, febrero 2001.
    \bibitem{FONa} \textsc{Paola Fonti}. \textsl{Transformadores de intensidad: c\'{o}mo determinar sus especificaciones}.  Cuaderno T\'{e}cnico de Schneider 194, agosto 2000.
    \bibitem{FONb} \textsc{Paola Fonti}. \textsl{Transformadores de intensidad: errores de especificaci\'{o}n y soluciones}.  Cuaderno T\'{e}cnico de Schneider 195, diciembre 2001.

\end{thebibliography}
