\chapter{Normatives Diverses}

\section{Numeraci\'{o} ANSI de dispositius el\`{e}ctrics} \index{numeraci\'{o} ANSI de dispositius el\`{e}ctrics}
\index{ANSI!C37.2}

Es d\'{o}na a continuaci\'{o} una llista de la numeraci\'{o} de dispositius
el\`{e}ctrics segons la norma \textsf{ANSI C37.2}, amb una breu
explicaci\'{o} de la seva funci\'{o}.

\begin{multicols}{2}
\begin{list}{}
{\setlength{\labelwidth}{6mm} \setlength{\leftmargin}{6mm}
\setlength{\labelsep}{2mm}}

\item[\textbf{1}] \index{element principal} \textbf{Element principal}. \'{E}s un dispositiu
iniciador, com ara un commutador de control, un rel\`{e} de tensi\'{o}, un
interruptor de nivell, etc., que serveix per posar en marxa o fora
de servei un aparell, ja sigui directament, o b\'{e}  mitjan\c{c}ant altres
dispositius, com ara rel\`{e}s de protecci\'{o}, o rel\`{e}s temporitzats.

\item[\textbf{2}] \index{rel\`{e}!de tancament o arrencada, amb retard} \textbf{Rel\`{e}
de tancament o arrencada, amb retard de temps}. \'{E}s el que
proporciona un retard de temps entre les operacions d'una seq\"{u}\`{e}ncia
autom\`{a}tica, o d'un sistema de protecci\'{o}, excepte quan aquest retard
\'{e}s proporcionat espec\'{\i}ficament pels dispositius 48, 62 o 79,
descrits m\'{e}s endavant. S'utilitza principalment com a protecci\'{o} de
la discrep\`{a}ncia de pols d'un interruptor.

\item[\textbf{3}] \index{rel\`{e}!de comprovaci\'{o} o de bloqueig} \textbf{Rel\`{e} de comprovaci\'{o} o
de bloqueig}. \'{E}s el que actua en resposta a la posici\'{o} d'una s\`{e}rie
d'altres dispositius (o d'una s\`{e}rie de condicions predeterminades)
en un equip, per tal de permetre que una seq\"{u}\`{e}ncia d'operaci\'{o}
continu\"{\i}, o per tal de parar-la, o per proporcionar una prova de la
posici\'{o} d'aquests dispositius o d'aquestes condicions.

\item[\textbf{4}] \index{contactor!principal} \textbf{Contactor principal}. \'{E}s un dispositiu,
generalment governat pel dispositiu n\'{u}mero 1 i pels dispositius de perm\'{\i}s i protecci\'{o}
que calgui, que serveix per obrir i tancar els circuits de control necessaris per tal de
posar un equip en marxa, o per parar-lo.

\item[\textbf{5}] \index{dispositiu!de parada}  \textbf{Dispositiu de parada}. \'{E}s el que
t\'{e} com a funci\'{o} principal, deixar fora de servei un equip i
mantenir-lo en aquest estat; la seva actuaci\'{o} pot ser manual o
el\`{e}ctrica. Queda exclosa la funci\'{o} de bloqueig el\`{e}ctric en
situacions anormals (vegeu la funci\'{o} 86).

\item[\textbf{6}] \index{interruptor!d'arrencada} \textbf{Interruptor d'arrencada}. \'{E}s
el que t\'{e} com a funci\'{o} principal connectar una m\`{a}quina a la seva font de tensi\'{o} d'arrencada.

\item[\textbf{7}] \index{interruptor!d'\`{a}node} \textbf{Interruptor d'\`{a}node}. \'{E}s el que
s'utilitza en els circuits dels \`{a}nodes d'un rectificador de
pot\`{e}ncia, principalment per interrompre el circuit del rectificador
en  cas de produir-s'hi un arc el\`{e}ctric.

\item[\textbf{8}] \index{dispositiu!de desconnexi\'{o} de
l'energia de control} \textbf{Dispositiu de desconnexi\'{o} de l'energia
de control}. \'{E}s un element de desconnexi\'{o} (commutador de ganiveta,
interruptor de bloc o fusibles connectables) que s'utilitza per
connectar i desconnectar la font d'energia de control,  a la barra
de tensi\'{o} de control o a l'equip al qual doni servei. Es considera
que l'energia de control inclou a l'energia auxiliar que alimenta a
aparells, com ara motors petits o calefactors.

\item[\textbf{9}] \index{dispositiu!d'inversi\'{o}} \textbf{Dispositiu d'inversi\'{o}}. \'{E}s el
que s'utilitza per invertir les connexions del camp d'una m\`{a}quina, o
per realitzar qualsevol altra funci\'{o}  d'inversi\'{o}.

\item[\textbf{10}] \index{commutador!de seq\"{u}\`{e}ncia} \textbf{Commutador de seq\"{u}\`{e}ncia}. \'{E}s un
dispositiu que s'utilitza pera canviar la seq\"{u}\`{e}ncia de connexi\'{o} o
desconnexi\'{o} d'unitats, en un equip de m\'{u}ltiples unitats.

\item[\textbf{11}] \textbf{Reservat per a  futures aplicacions}.
L'\textsf{USBR}\footnote{{"<}United States Bureau of Reclamation{">}}
 li assigna la funci\'{o}: transformador de pot\`{e}ncia de control.

\item[\textbf{12}] \index{dispositiu!d'exc\'{e}s de velocitat} \textbf{Dispositiu d'exc\'{e}s de velocitat}. \'{E}s normalment un
interruptor de velocitat, de connexi\'{o} directa, que
actua quan la m\`{a}quina  s'embala.

\item[\textbf{13}] \index{dispositiu!de velocitat sincr\`{o}nica} \textbf{Dispositiu de
velocitat sincr\`{o}nica}. \'{E}s un element, com ara un interruptor de
velocitat centr\'{\i}fuga, un rel\`{e} de freq\"{u}\`{e}ncia de lliscament, un rel\`{e}
de tensi\'{o}, o qualsevol altre aparell que actua a, aproximadament, la
velocitat sincr\`{o}nica d'una m\`{a}quina.

\item[\textbf{14}] \index{dispositiu!de manca de velocitat} \textbf{Dispositiu de manca de velocitat}. \'{E}s el que actua quan la velocitat d'una m\`{a}quina baixa per sota d'un valor determinat.

\item[\textbf{15}] \index{dispositiu!igualador de velocitat o freq\"{u}\`{e}ncia}
\textbf{Dispositiu igualador de velocitat o freq\"{u}\`{e}ncia}. \'{E}s el que
actua per tal d'igualar i mantenir la velocitat o la  freq\"{u}\`{e}ncia
d'una m\`{a}quina o d'un sistema a un cert valor, aproximadament igual
al  d'una altra m\`{a}quina o sistema.

\item[\textbf{16}] \textbf{Reservat per a  futures aplicacions}.
L'\textsf{USBR}\footnotemark[1] li assigna la funci\'{o}: carregador de
bateries.

\item[\textbf{17}] \index{commutador!de \guillemotleft{}shunt\guillemotright{} o de desc\`{a}rrega} \textbf{Commutador
de  {"<}shunt{">} o de desc\`{a}rrega}. \'{E}s el que serveix per obrir i tancar
un circuit {"<}shunt{">} entre els extrems de qualsevol aparell (excepte
una resist\`{e}ncia), com ara el camp d'una m\`{a}quina, un condensador o
una react\`{a}ncia. Queden exclosos els elements que realitzen les
funcions de {"<}shunt{">} necess\`{a}ries per arrancar una m\`{a}quina, mitjan\c{c}ant
els dispositius 6, 42, o equivalents; tamb\'{e} queda exclosa la funci\'{o}
del dispositiu 73, el qual serveix per a l'operaci\'{o} de resist\`{e}ncies.

\item[\textbf{18}] \index{dispositiu!d'acceleraci\'{o} o desacceleraci\'{o}} \textbf{Dispositiu d'acceleraci\'{o} o desacceleraci\'{o}}. \'{E}s
el que s'utilitza per tancar o per causar el tancament dels circuits
que serveixen per augmentar o disminuir la velocitat d'una m\`{a}quina.

\item[\textbf{19}] \index{contactor!de transici\'{o} d'arrencada a marxa normal}
\textbf{Contactor de transici\'{o} d'arrencada a marxa normal}. La seva
funci\'{o} \'{e}s fer la transfer\`{e}ncia de les connexions de l'alimentaci\'{o}
d'arrencada, a la de marxa normal d'una m\`{a}quina.

\item[\textbf{20}] \index{valvula@v\`{a}lvula} \textbf{V\`{a}lvula}. S'assigna aquest n\'{u}mero a una v\`{a}lvula
utilitzada en un circuit de buit, d'aire, de gas, d'oli, d'aigua,
etc., quan s'acciona el\`{e}ctricament o quan t\'{e} accessoris el\`{e}ctrics,
com ara commutadors auxiliars.

\item[\textbf{21}] \index{rel\`{e}!de dist\`{a}ncia} \textbf{Rel\`{e} de dist\`{a}ncia}. \'{E}s el que actua
quan l'admit\`{a}ncia, la imped\`{a}ncia o la react\`{a}ncia d'un circuit surt fora d'un cert l\'{\i}mit.

\item[\textbf{22}] \index{interruptor!igualador} \textbf{Interruptor igualador}.  \'{E}s el
que serveix per connectar i desconnectar les connexions igualadores
o d'equilibri d'intensitat del camp d'una m\`{a}quina, o per regular
equips en una  insta{\l.l}aci\'{o} de  m\'{u}ltiples unitats.

\item[\textbf{23}] \index{dispositiu!controlador de temperatura} \textbf{Dispositiu
controlador de temperatura}. \'{E}s el que actua per tal de fer pujar la
temperatura d'un lloc o d'un aparell, quan aquesta temperatura baixa
per sota d'un cert l\'{\i}mit, o a l'inrev\'{e}s, el que actua per tal de fer
 baixar la temperatura d'un lloc o d'un aparell, quan aquesta
temperatura  puja per sobre d'un cert l\'{\i}mit. Un exemple seria un
term\`{o}stat; en canvi, un dispositiu per regular la temperatura dins
d'un marge estret, es designaria amb el n\'{u}mero 90.

\item[\textbf{24}] \textbf{Reservat per a  futures aplicacions}.
L'\textsf{USBR}\footnotemark[1] li assigna la funci\'{o}: interruptor o
contactor d'uni\'{o} de barres.

\item[\textbf{25}] \index{dispositiu!de sincronitzaci\'{o}}\index{dispositiu!de comprovaci\'{o} de
sincronisme} \textbf{Dispositiu de sincronitzaci\'{o} o de comprovaci\'{o}
de sincronisme}. \'{E}s el que actua quan dos circuits de corrent altern
s\'{o}n dins dels l\'{\i}mits desitjats de tensi\'{o}, freq\"{u}\`{e}ncia i angle de
fase, per permetre la connexi\'{o} en para{\l.l}el d'aquests dos circuits.


\item[\textbf{26}] \index{dispositiu!t\`{e}rmic} \textbf{Dispositiu t\`{e}rmic}. \'{E}s el que
actua quan la temperatura del camp {"<}shunt{">} o del bobinat esmorte\"{\i}dor
d'una m\`{a}quina, la temperatura d'una resist\`{e}ncia de limitaci\'{o} de
c\`{a}rrega, o la temperatura d'un l\'{\i}quid, etc., supera un valor
determinat. Tamb\'{e} actua si la temperatura de l'aparell protegit cau
per sota d'un valor determinat.

\item[\textbf{27}] \index{rel\`{e}!de m\'{\i}nima tensi\'{o}} \textbf{Rel\`{e} de m\'{\i}nima tensi\'{o}}. \'{E}s el que
actua quan la tensi\'{o} baixa per sota d'un l\'{\i}mit determinat.

\item[\textbf{28}] \index{detector!de flama} \textbf{Detector de flama}. La seva funci\'{o} \'{e}s
detectar l'exist\`{e}ncia de flama en el pilot o cremador principal de, per exemple, una
caldera o una turbina de gas.

\item[\textbf{29}] \index{contactor!d'a\"{\i}llament} \textbf{Contactor d'a\"{\i}llament}. \'{E}s el que
s'utilitza amb l'\'{u}nic prop\`{o}sit de desconnectar un circuit d'un
altre,  a  causa de maniobres    d'emerg\`{e}ncia,  de manteniment o de
prova.

\item[\textbf{30}] \index{rel\`{e}!anunciador} \textbf{Rel\`{e} anunciador}. \'{E}s un dispositiu de
reposici\'{o} no autom\`{a}tica, que d\'{o}na una s\`{e}rie d'indicacions visuals
individuals, de les funcions d'aparells de protecci\'{o}, i que es pot
disposar tamb\'{e} per efectuar una funci\'{o} de bloqueig.

\item[\textbf{31}] \index{dispositiu!d'excitaci\'{o} separada} \textbf{Dispositiu d'excitaci\'{o}
separada}. \'{E}s el que connecta un circuit, com ara el camp {"<}shunt{">}
d'una commutatriu, a una font d'excitaci\'{o} separada, durant el proc\'{e}s
d'arrencada. Tamb\'{e} s'utilitza per energitzar el circuit d'encesa
d'un rectificador de pot\`{e}ncia.


\item[\textbf{32}] \index{rel\`{e}!direccional de pot\`{e}ncia} \textbf{Rel\`{e} direccional de
pot\`{e}ncia}. \'{E}s el que actua quan se supera un valor determinat del
flux de pot\`{e}ncia en un sentit donat. Tamb\'{e} actua per causa d'una
inversi\'{o} de pot\`{e}ncia, originada per un arc el\`{e}ctric en el circuit
an\`{o}dic o cat\`{o}dic d'un rectificador de pot\`{e}ncia.

\item[\textbf{33}] \index{commutador!de posici\'{o}} \textbf{Commutador de posici\'{o}}. \'{E}s el que
obre o tanca un contacte, quan un dispositiu principal o una part d'un aparell que no tingui un n\'{u}mero funcional de dispositiu, arriba a una posici\'{o} determinada.

\item[\textbf{34}] \index{dispositiu!principal de seq\"{u}\`{e}ncia} \textbf{Dispositiu principal de
 seq\"{u}\`{e}ncia}. \'{E}s un element, com ara un selector de contactes m\'{u}ltiples, o com ara un
 dispositiu programable,
 que fixa la seq\"{u}\`{e}ncia d'operaci\'{o}
de dispositius principals, durant l'arrencada i la parada, o durant altres operacions
que requereixin una seq\"{u}\`{e}ncia.

\item[\textbf{35}] \index{dispositiu!per operar
escombretes} \index{dispositiu!per posar en curt circuit anells de
frec} \textbf{Dispositiu per operar escombretes o per posar en curt
circuit anells de frec}. \'{E}s el que serveix per elevar, baixar o
desviar les escombretes d'una m\`{a}quina, o per posar en curt circuit
els seus anells de frec. Tamb\'{e} serveix per fer o desfer els
contactes d'un rectificador mec\`{a}nic.

\item[\textbf{36}] \index{dispositiu!de polaritat}
\index{dispositiu!de tensi\'{o} de polaritzaci\'{o}} \textbf{Dispositiu de
polaritat o de tensi\'{o} de polaritzaci\'{o}}. \'{E}s el que acciona o permet
l'accionament d'altres dispositius, tan sols amb una polaritat
donada, o el que verifica la pres\`{e}ncia d'una tensi\'{o} de polaritzaci\'{o}
en un equip.

\item[\textbf{37}] \index{rel\`{e}!de baixa intensitat o baixa pot\`{e}ncia} \textbf{Rel\`{e} de baixa
intensitat o baixa pot\`{e}ncia}. \'{E}s el que actua quan la intensitat o la pot\`{e}ncia cauen per
sota d'un valor determinat.

\item[\textbf{38}] \index{dispositiu!protector de coixinets}
\textbf{Dispositiu protector de coixinets}. \'{E}s el que actua amb una
temperatura excessiva dels coixinets, o amb condicions mec\`{a}niques
an\`{o}males que poden derivar en una temperatura excessiva dels
coixinets.

\item[\textbf{39}] \index{detector!de condicions mec\`{a}niques}
\textbf{Detector de condicions mec\`{a}niques}. \'{E}s el que actua davant
de situacions mec\`{a}niques anormals (excepte les que tenen lloc en els
coixinets d'una m\`{a}quina, funci\'{o} 38), com ara vibraci\'{o} excessiva,
excentricitat, etc.

\item[\textbf{40}] \index{rel\`{e}!de camp} \textbf{Rel\`{e} de camp}. \'{E}s el que actua quan es
d\'{o}na un valor massa baix de la intensitat de camp d'una m\`{a}quina, o quan es d\'{o}na un valor
massa gran de la component reactiva del corrent d'armadura en una m\`{a}quina de corrent
altern, la qual cosa indica una excitaci\'{o} de camp massa baixa.

\item[\textbf{41}] \index{interruptor!de camp} \textbf{Interruptor de camp}. \'{E}s un dispositiu
que actua per tal de connectar o desconnectar l'excitaci\'{o} del camp
d'una m\`{a}quina.

\item[\textbf{42}] \index{interruptor!de marxa} \textbf{Interruptor de marxa}. \'{E}s un
dispositiu que t\'{e} per funci\'{o} principal connectar una m\`{a}quina a la
seva font de tensi\'{o} de funcionament.

\item[\textbf{43}] \index{dispositiu!de transfer\`{e}ncia} \textbf{Dispositiu de transfer\`{e}ncia}. \'{E}s
un element, accionat manualment, que efectua la transfer\`{e}ncia dels circuits de control, per tal
 de modificar el proc\'{e}s d'operaci\'{o} d'equips de connexi\'{o} o d'altres dispositius.

\item[\textbf{44}] \index{rel\`{e}!de seq\"{u}\`{e}ncia d'arrencada de grup} \textbf{Rel\`{e} de seq\"{u}\`{e}ncia
d'arrencada de grup}. \'{E}s el que actua per arrancar la seg\"{u}ent unitat
disponible, en un equip de m\'{u}ltiples unitats, quan falla o quan no
est\`{a} disponible la unitat que normalment hauria d'arrencar.

\item[\textbf{45}] \index{detector!de condiciones atmosf\`{e}riques} \textbf{Detector de condiciones
atmosf\`{e}riques}. \'{E}s el que actua davant de condicions atmosf\`{e}riques anormals, com ara fums
perillosos, gasos explosius, foc, etc.

\item[\textbf{46}] \index{rel\`{e}!de seq\"{u}\`{e}ncia negativa d'intensitat} \textbf{Rel\`{e} de
seq\"{u}\`{e}ncia negativa d'intensitat}. \'{E}s un rel\`{e} que actua quan les
intensitats polif\`{a}siques estan en seq\"{u}\`{e}ncia inversa o
desequilibrades, o quan contenen una component de seq\"{u}\`{e}ncia negativa
superior a un cert l\'{\i}mit.

\item[\textbf{47}] \index{rel\`{e}!de seq\"{u}\`{e}ncia de fase de tensi\'{o} } \textbf{Rel\`{e}
de seq\"{u}\`{e}ncia de fase de tensi\'{o}}. \'{E}s el que actua amb un valor donat
de tensi\'{o}, quan es d\'{o}na la seq\"{u}\`{e}ncia de fases desitjada.

\item[\textbf{48}] \index{rel\`{e}!de seq\"{u}\`{e}ncia incompleta} \textbf{Rel\`{e} de seq\"{u}\`{e}ncia
incompleta}. \'{E}s el que torna un equip a la seva posici\'{o} normal  i
l'enclava, si la seq\"{u}\`{e}ncia normal d'arrencada, de funcionament o de
parada no s'ha completat degudament en un interval de temps
determinat.

\item[\textbf{49}] \index{rel\`{e}!t\`{e}rmic d'una m\`{a}quina o d'un transformador}
\textbf{Rel\`{e} t\`{e}rmic d'una m\`{a}quina o d'un transformador}. \'{E}s el que
actua quan la temperatura d'un element d'una m\`{a}quina o d'un
transformador (normalment un debanat), per on circula el corrent,
supera un valor determinat.

\item[\textbf{50}] \index{rel\`{e}!instantani de sobreintensitat o de velocitat d'augment
d'intensitat} \textbf{Rel\`{e} instantani de sobreintensitat o de velocitat d'augment
d'intensitat}. \'{E}s el que actua instant\`{a}niament quan es d\'{o}na un valor excessiu de la
intensitat o de la  velocitat d'augment de la intensitat.

\item[\textbf{51}] \index{rel\`{e}!temporitzat de sobreintensitat de corrent altern}
\textbf{Rel\`{e} temporitzat de sobreintensitat de corrent altern}. \'{E}s
un rel\`{e} amb una caracter\'{\i}stica de temps inversa o definida, que
actua amb una certa temporitzaci\'{o}, quan es d\'{o}na un valor excessiu de
la intensitat.

\item[\textbf{52}] \index{interruptor!de corrent altern} \textbf{Interruptor de corrent altern}. \'{E}s
 el que s'utilitza per tancar i obrir un circuit de pot\`{e}ncia de corrent altern sota condicions
normals, de falta o d'emerg\`{e}ncia.

\item[\textbf{53}] \index{rel\`{e}!d'excitatriu o de generador de corrent continu}
\textbf{Rel\`{e} d'excitatriu o de generador de corrent continu}. \'{E}s el
que for\c{c}a la creaci\'{o} del camp d'una m\`{a}quina de corrent continu
durant l'arrencada, o el que actua quan la tensi\'{o} d'una m\`{a}quina ha
arribat a un valor determinat.

\item[\textbf{54}] \index{interruptor!d'alta velocitat, de corrent continu}
\textbf{Interruptor d'alta velocitat, de corrent continu}. \'{E}s el que
actua per tal de reduir el corrent d'un circuit principal, en un
temps inferior a 0,01\unit{s} despr\'{e}s d'haver-se produ\"{\i}t un corrent
massa elevat, o una velocitat de creixement d'aquest corrent massa
elevada.

\item[\textbf{55}] \index{rel\`{e}!de factor de pot\`{e}ncia} \textbf{Rel\`{e} de factor de pot\`{e}ncia}.
\'{E}s el que actua quan el factor de potencia en un circuit de corrent altern no arriba o
sobrepassa un valor determinat.

\item[\textbf{56}] \index{rel\`{e}!d'aplicaci\'{o} del camp} \textbf{Rel\`{e} d'aplicaci\'{o} del camp}.
\'{E}s el que s'utilitza per controlar autom\`{a}ticament l'aplicaci\'{o} de l'excitaci\'{o} de camp d'un
motor de corrent altern, en un punt predeterminat en el cicle de lliscament.

\item[\textbf{57}] \index{dispositiu!de curt circuit o de posada a terra}
\textbf{Dispositiu de curt circuit o de posada a terra}. \'{E}s el que
opera en un circuit principal per tal de curtcircuitar-lo  o
posar-lo a terra, en resposta a ordres autom\`{a}tiques o manuals.

\item[\textbf{58}] \index{rel\`{e}!de fallada de rectificador de pot\`{e}ncia} \textbf{Rel\`{e} de
fallada de rectificador de pot\`{e}ncia}. \'{E}s el que actua a causa de la
fallada d'un o m\'{e}s \`{a}nodes d'un rectificador de pot\`{e}ncia, o a causa
de la fallada d'un d\'{\i}ode a conduir o bloquejar pr\`{o}piament.

\item[\textbf{59}] \index{rel\`{e}!de sobretensi\'{o}} \textbf{Rel\`{e} de sobretensi\'{o}}. \'{E}s el que
actua quan la tensi\'{o} supera un valor determinat.

\item[\textbf{60}] \index{rel\`{e}!d'equilibri de tensi\'{o} o corrent} \textbf{Rel\`{e} d'equilibri de
tensi\'{o} o corrent}. \'{E}s el que actua amb una difer\`{e}ncia de tensi\'{o} o
corrent entre dos circuits.

\item[\textbf{61}] \textbf{Reservat per a  futures aplicacions}.

\item[\textbf{62}] \index{rel\`{e}!de parada o obertura, amb retard} \textbf{Rel\`{e} de
parada o obertura, amb retard de temps}. \'{E}s el que s'utilitza
conjuntament amb el dispositiu que inicia la parada total o la
indicaci\'{o} de parada o obertura, en una seq\"{u}\`{e}ncia autom\`{a}tica.

\item[\textbf{63}] \index{rel\`{e}!de pressi\'{o} de gas, l\'{\i}quid o buit} \textbf{Rel\`{e} de pressi\'{o}
de gas, l\'{\i}quid o buit}. \'{E}s el que actua a un valor determinat de
pressi\'{o} de l\'{\i}quid o gas, o per a una determinada velocitat de
variaci\'{o} d'aquesta pressi\'{o}.

\item[\textbf{64}] \index{rel\`{e}!de protecci\'{o} de terra} \textbf{Rel\`{e} de protecci\'{o} de terra}.
\'{E}s el que actua davant d'un defecte a terra de l'a\"{\i}llament d'una
m\`{a}quina. Aquesta funci\'{o} s'aplica nom\'{e}s a un rel\`{e} que detecti el pas
del corrent des de la carcassa  d'una m\`{a}quina a terra, o a un rel\`{e}
que detecti un terra en un circuit normalment no connectat a terra;
no s'aplica a un dispositiu connectat en el circuit secundari d'un
transformador d'intensitat, que estigui connectat en el circuit de
pot\`{e}ncia d'un sistema posat normalment a terra.

\item[\textbf{65}] \index{regulador} \textbf{Regulador}. \'{E}s un equip format per elements
el\`{e}ctrics, mec\`{a}nics o flu\'{\i}dics,  que controla el flux d'aigua,
vapor, etc.,  a una m\`{a}quina motriu, per tal d'arrancar-la, mantenir
la seva velocitat o parar-la.

\item[\textbf{66}] \index{rel\`{e}!de passos} \textbf{Rel\`{e} de passos}. \'{E}s el que actua per tal
de permetre un nombre especificat d'operacions d'un dispositiu
donat, o b\'{e}, un nombre especificat d'operacions successives amb un
interval donat de temps entre cadascuna. Tamb\'{e} pot actuar per
permetre l'energitzaci\'{o} peri\`{o}dica d'un circuit, o per accelerar una
m\`{a}quina a baixa velocitat.

\item[\textbf{67}] \index{rel\`{e}!direccional de sobreintensitat de corrent altern}
\textbf{Rel\`{e} direccional de sobreintensitat de corrent altern}. \'{E}s
el que actua a partir d'un valor determinat de circulaci\'{o}
d'intensitat  de corrent altern, en un sentit donat.

\item[\textbf{68}] \index{rel\`{e}!de bloqueig} \textbf{Rel\`{e} de bloqueig}. \'{E}s el que inicia un
senyal pilot per bloquejar o disparar, quan hi ha faltes externes en
una l\'{\i}nia de transmissi\'{o}, o en altres aparells, sota certes
condicions; pot cooperar tamb\'{e} amb altres dispositius, per tal de
bloquejar el dispar o per bloquejar el reenganxament en una condici\'{o}
de p\`{e}rdua de sincronisme.

\item[\textbf{69}] \index{dispositiu!controlador
de permissiu} \textbf{Dispositiu controlador de permissiu}. \'{E}s
generalment, un interruptor auxiliar de dues posicions, accionat
manualment, el qual permet en una posici\'{o}, el tancament d'un
interruptor o la posada en servei d'un equip, i en l'altra posici\'{o},
impedeix l'accionament de l'interruptor o de l'equip.

\item[\textbf{70}] \index{reostat@re\`{o}stat} \textbf{Re\`{o}stat}. \'{E}s un dispositiu utilitzat per
variar la resist\`{e}ncia d'un circuit, en resposta a algun m\`{e}tode de control.

\item[\textbf{71}] \index{rel\`{e}!de nivell de l\'{\i}quid o gas} \textbf{Rel\`{e} de nivell de l\'{\i}quid
o gas}. \'{E}s el que actua a partir d'un valor determinat del nivell
d'un l\'{\i}quid o d'un gas, o a partir de determinades velocitats de
variaci\'{o} d'aquests nivells.

\item[\textbf{72}] \index{interruptor!de corrent continu} \textbf{Interruptor de corrent
continu}. \'{E}s el que s'utilitza per tancar i obrir un circuit de pot\`{e}ncia de corrent continu
 sota condicions normals, de falta o d'emerg\`{e}ncia.

\item[\textbf{73}] \index{contactor!de resist\`{e}ncia de c\`{a}rrega} \textbf{Contactor de resist\`{e}ncia
 de c\`{a}rrega}. \'{E}s el que s'utilitza per posar en curt circuit o per commutar un gra\'{o} de c\`{a}rrega,
 destinat a limitar o a desviar la c\`{a}rrega, en un circuit de pot\`{e}ncia.

\item[\textbf{74}] \index{rel\`{e}!d'alarma} \textbf{Rel\`{e} d'alarma}. \'{E}s qualsevol altre rel\`{e},
diferent al dispositiu 30, que s'utilitza per actuar una alarma
visible o audible.

\item[\textbf{75}] \index{mecanisme!de canvi de posici\'{o}} \textbf{Mecanisme de canvi
de posici\'{o}}. \'{E}s el que s'utilitza per moure un dispositiu d'un equip
des d'una posici\'{o} a una altra; un  exemple, seria el mecanisme
utilitzat per canviar un interruptor entre les posicions de
connectat, desconnectat i prova.

\item[\textbf{76}] \index{rel\`{e}!de sobreintensitat de corrent continu} \textbf{Rel\`{e} de
sobreintensitat de corrent continu}. \'{E}s el que actua quan la intensitat en un circuit de
corrent continu, sobrepassa un valor determinat.

\item[\textbf{77}] \index{transmissor d'impulsos} \textbf{Transmissor d'impulsos}. \'{E}s un
 dispositiu que s'utilitza per generar o transmetre  impulsos, a trav\'{e}s d'un circuit de
telemetria o fil pilot, a un dispositiu d'indicaci\'{o} o recepci\'{o}
remot.

\item[\textbf{78}] \index{rel\`{e}!de  mesura de l'angle de fase o de protecci\'{o} de
desfase} \textbf{Rel\`{e} de  mesura de l'angle de fase o de protecci\'{o}
de desfase}. \'{E}s el que actua a partir d'un valor determinat de
l'angle de fase entre dues tensions o dues intensitats, o entre una
tensi\'{o} i una intensitat.

\item[\textbf{79}] \index{rel\`{e}!de reenganxament de corrent altern} \textbf{Rel\`{e} de
reenganxament de corrent altern}. \'{E}s el que controla el reenganxament i enclavament d'un
interruptor de corrent altern.

\item[\textbf{80}] \index{rel\`{e}!de flux de l\'{\i}quids o gasos} \textbf{Rel\`{e} de flux de l\'{\i}quids
o gasos}.
 \'{E}s el que actua a partir d'un valor determinat del flux d'un l\'{\i}quid o d'un gas, o de
 la  velocitat de variaci\'{o} d'aquest flux.

\item[\textbf{81}] \index{rel\`{e}!de freq\"{u}\`{e}ncia} \textbf{Rel\`{e} de freq\"{u}\`{e}ncia}. \'{E}s el que actua
davant una variaci\'{o} de la freq\"{u}\`{e}ncia o de la seva velocitat de variaci\'{o}.

\item[\textbf{82}] \index{rel\`{e}!de reenganxament de corrent continu} \textbf{Rel\`{e} de
reenganxament de corrent continu}. \'{E}s el que controla el tancament i el reenganxament d'un
interruptor de corrent continu, generalment responent a les condicions de c\`{a}rrega del
circuit.

\item[\textbf{83}] \index{rel\`{e}!de selecci\'{o} o transfer\`{e}ncia  del control autom\`{a}tic}
\textbf{Rel\`{e} de selecci\'{o} o transfer\`{e}ncia del control autom\`{a}tic}. \'{E}s
el que actua per tal d'escollir autom\`{a}ticament entre certes fonts
d'alimentaci\'{o} o entre certes condicions d'un equip; tamb\'{e} \'{e}s el que
efectua autom\`{a}ticament una operaci\'{o} de transfer\`{e}ncia.

\item[\textbf{84}] \index{mecanisme!d'accionament} \textbf{Mecanisme d'accionament}. \'{E}s un
mecanisme o un servo-mecanisme el\`{e}ctric complet,  d'un canviador de
preses, d'un regulador d'inducci\'{o} o de qualsevol altre aparell
similar, que no tingui n\'{u}mero de funci\'{o} propi assignat.

\item[\textbf{85}] \index{rel\`{e}!receptor d'ones portadores o fil pilot} \textbf{Rel\`{e}
receptor d'ones portadores o fil pilot}. \'{E}s un rel\`{e} actuat per un
senyal d'una ona portadora o per un fil pilot de corrent continu, provocat
per l'actuaci\'{o} d'una protecci\'{o} direccional.

\item[\textbf{86}] \index{rel\`{e}!d'enclavament} \textbf{Rel\`{e} d'enclavament}. \'{E}s un rel\`{e}
accionat el\`{e}ctricament, amb reposici\'{o} manual o el\`{e}ctrica, que actua
per parar i mantenir un equip fora de servei, quan hi ha condiciones
anormals.

\item[\textbf{87}] \index{rel\`{e}!de protecci\'{o} diferencial} \textbf{Rel\`{e} de protecci\'{o}
diferencial}. \'{E}s el que actua a partir d'una  difer\`{e}ncia quantitativa de dues intensitats
o d'algunes altres magnituds el\`{e}ctriques.

\item[\textbf{88}] \index{motor o grup moto-generador auxiliar} \textbf{Motor o grup
moto-generador auxiliar}. \'{E}s un dispositiu que s'utilitza per
accionar equips auxiliares.

\item[\textbf{89}] \index{desconnectador de l\'{\i}nia} \textbf{Desconnectador de l\'{\i}nia}. \'{E}s
un dispositiu que s'utilitza com a desconnectador o a\"{\i}llador en un
circuit de pot\`{e}ncia de corrent continu o altern, sempre que aquest
dispositiu sigui operat el\`{e}ctricament o tingui accessoris el\`{e}ctrics.

\item[\textbf{90}] \index{dispositiu!de regulaci\'{o}} \textbf{Dispositiu de regulaci\'{o}}. \'{E}s el que
actua per tal de regular una magnitud, com ara la tensi\'{o}, la intensitat, la potencia,
la velocitat, la freq\"{u}\`{e}ncia, etc., a un valor determinat.

\item[\textbf{91}] \index{rel\`{e}!direccional de tensi\'{o}} \textbf{Rel\`{e} direccional de tensi\'{o}}.
\'{E}s el que actua quan la tensi\'{o} entre els extrems oberts d'un
interruptor o contactor, sobrepassa un valor determinat, en un
sentit donat.

\item[\textbf{92}] \index{rel\`{e}!direccional de tensi\'{o} i pot\`{e}ncia} \textbf{Rel\`{e} direccional
de tensi\'{o} i pot\`{e}ncia}. \'{E}s el que permet o ocasiona la connexi\'{o} de
dos circuits, quan la difer\`{e}ncia de tensi\'{o} entre ambd\'{o}s supera un
valor determinat, en un cert sentit, i ocasiona la desconnexi\'{o} dels
dos circuits, quan la pot\`{e}ncia circulant supera un valor determinat,
en el sentit contrari.

\item[\textbf{93}] \index{contactor!de canvi del camp} \textbf{Contactor de canvi del camp}. \'{E}s el
que actua per tal de augmentar o disminuir el valor de l'excitaci\'{o}
d'una m\`{a}quina.

\item[\textbf{94}] \index{rel\`{e}!de dispar o dispar lliure} \textbf{Rel\`{e} de dispar o dispar
lliure}. \'{E}s el que actua per tal de disparar o permetre disparar un
interruptor, un contactor, etc., o per evitar un reenganxament
immediat d'un interruptor, en el cas que hagi d'obrir i l'ordre de
tancament sigui mantinguda.

\item[\textbf{95}] \textbf{Espec\'{\i}fic}.\footnote{Utilitzat en insta{\l.l}acions
individuals per a aplicacions concretes, quan cap de les funcions 1
a 94 no \'{e}s apropiada.} L'\textsf{USBR}\footnotemark[1] li assigna la
funci\'{o}: rel\`{e} o contactor de tancament.

\item[\textbf{96}] \textbf{Espec\'{\i}fic}.\footnotemark[2]

\item[\textbf{97}] \textbf{Espec\'{\i}fic}.\footnotemark[2]


\item[\textbf{98}] \textbf{Espec\'{\i}fic}.\footnotemark[2] L'\textsf{USBR}\footnotemark[1] li assigna la
funci\'{o}: rel\`{e} de p\`{e}rdua d'excitaci\'{o}.

\item[\textbf{99}] \textbf{Espec\'{\i}fic}.\footnotemark[2] L'\textsf{USBR}\footnotemark[1] li assigna la
funci\'{o}: detector d'arc el\`{e}ctric.

\end{list}
\end{multicols}


\section{Grau de protecci\'{o} IP} \index{IP} \index{grau de protecci\'{o}} \index{CEI!60529}

La codificaci\'{o} {"<}International Protection{">} (\textsf{IP}), segons la
norma \textsf{CEI 60529}, s'utilitza per descriure el grau de
protecci\'{o}  proporcionat pels elements envoltants d'equips el\`{e}ctrics, contra
la penetraci\'{o} de cossos s\`{o}lids estranys i contra els efectes nocius
de l'aigua.

La codificaci\'{o} consisteix en les sigles \textsf{\textbf{IP}}
seguides per dues xifres, m\'{e}s una lletra addicional (opcional) i una
lletra suplement\`{a}ria (opcional); quan el grau de protecci\'{o}
corresponent a una de les dues xifres no s'utilitzi, perqu\`{e} no sigui
necessari o perqu\`{e} no sigui conegut, es reempla\c{c}ar\`{a} la xifra en
q\"{u}esti\'{o} per una \textsf{\textbf{X}}. Es defineix a continuaci\'{o} el
significat de les xifres i lletres que formen el codi \textsf{IP}:

\textbf{1a xifra}. Indica el grau de protecci\'{o} de les persones contra els contactes amb
parts en tensi\'{o} o amb peces en moviment, i el grau de protecci\'{o} dels equips contra la
penetraci\'{o} de cossos s\`{o}lids i pols. Els valors possibles s\'{o}n els seg\"{u}ents:
\begin{list}{}
   {\setlength{\labelwidth}{10mm} \setlength{\leftmargin}{10mm} \setlength{\labelsep}{2mm}}
   \item[\textbf{0}] Sense cap protecci\'{o} en particular.
   \item[\textbf{1}] Protecci\'{o} contra l'entrada de cossos s\`{o}lids superiors a 50\unit{mm},
   com per exemple,   contactes involuntaris de la m\`{a}.
   \item[\textbf{2}] Protecci\'{o} contra l'entrada de cossos superiors a 12\unit{mm}, com per exemple,
   contactes involuntaris dels dits de la m\`{a}.
   \item[\textbf{3}] Protecci\'{o} contra l'entrada de cossos superiors a 2,5\unit{mm},
   com per exemple, eines o cables.
   \item[\textbf{4}] Protecci\'{o} contra l'entrada de cossos superiors a 1\unit{mm}.
   \item[\textbf{5}] Protecci\'{o} contra la pols. Es permet la seva entrada all\`{a} on no sigui perjudicial.
   \item[\textbf{6}] Protecci\'{o} total contra la pols.
\end{list}

\textbf{2a xifra}. Indica el grau de protecci\'{o} dels equips contra
l'entrada d'aigua. Els valors possibles s\'{o}n els seg\"{u}ents:
\begin{list}{}
   {\setlength{\labelwidth}{10mm} \setlength{\leftmargin}{10mm} \setlength{\labelsep}{2mm}}
   \item[\textbf{0}] Sense cap protecci\'{o} en particular.
   \item[\textbf{1}] Protecci\'{o} contra la caiguda vertical de gotes d'aigua.
   \item[\textbf{2}] Protecci\'{o} contra la caiguda de gotes d'aigua fins a 15\unit{\degree} de la  vertical.
   \item[\textbf{3}] Protecci\'{o} contra la caiguda de pluja fina (pulveritzada) fins a 60\unit{\degree} de la  vertical.
   \item[\textbf{4}] Protecci\'{o} contra la caiguda d'aigua en totes les direccions.
   \item[\textbf{5}] Protecci\'{o} contra aigua llan\c{c}ada a raig amb m\`{a}negues.
   \item[\textbf{6}] Protecci\'{o} contra aigua llan\c{c}ada a raigs forts o per cops de mar.
   \item[\textbf{7}] Protecci\'{o} contra la immersi\'{o} temporal.
   \item[\textbf{8}] Protecci\'{o} contra la immersi\'{o} prolongada o a gran pressi\'{o}.
\end{list}


\textbf{Lletra addicional (opcional)}. En alguns casos la protecci\'{o}
proporcionada pels elements envoltants contra l'acc\'{e}s a les parts
perilloses \'{e}s millor que la indicada per la primera xifra del codi;
en aquests casos es pot caracteritzar aquesta protecci\'{o} amb una
lletra addicional, afegida despr\'{e}s de les dues xifres; aix\`{o} permet
tenir obertures adequades per a la ventilaci\'{o},  guardant a l'hora el
grau requerit de protecci\'{o} de les persones. Els valors possibles s\'{o}n
els seg\"{u}ents:
\begin{list}{}
   {\setlength{\labelwidth}{10mm} \setlength{\leftmargin}{10mm} \setlength{\labelsep}{2mm}}
   \item[\textbf{A}] Els  cossos estranys de di\`{a}metre superior a
   50\unit{mm}    poden penetrar en l'element envoltant, per\`{o} tan sols d'una forma volunt\`{a}ria i deliberada.
   \item[\textbf{B}] Els  cossos estranys de di\`{a}metre superior a 12\unit{mm}
    poden penetrar en l'element envoltant, per\`{o} un dit de la m\`{a} no ha de poder entrar m\'{e}s de 80\unit{mm}, i
    ha de quedar per tant, a una dist\`{a}ncia    suficient de les parts perilloses.
   \item[\textbf{C}] Els  cossos estranys de di\`{a}metre superior a 2,5\unit{mm}
   poden penetrar en l'element envoltant, per\`{o} un filferro d'acer d'aquest di\`{a}metre i 100\unit{mm}
   de longitud ha de quedar a una dist\`{a}ncia suficient de les parts perilloses.
   \item[\textbf{D}] Els  cossos estranys de di\`{a}metre superior a 1\unit{mm}
   poden penetrar en l'element envoltant, per\`{o} un filferro d'acer d'aquest di\`{a}metre i 100\unit{mm}
   de longitud, ha de quedar a una dist\`{a}ncia suficient de les parts perilloses.
\end{list}

\textbf{Lletra suplement\`{a}ria (opcional)}. El codi \textsf{IP} accepta tamb\'{e} algunes
lletres suplement\`{a}ries al final, per tal d'afegir una informaci\'{o} concreta. Els valors
possibles s\'{o}n els seg\"{u}ents:
\begin{list}{}
   {\setlength{\labelwidth}{10mm} \setlength{\leftmargin}{10mm} \setlength{\labelsep}{2mm}}
   \item[\textbf{H}] Material d'alta tensi\'{o}.
   \item[\textbf{M}] En m\`{a}quines rotatives indica que els assajos s'han realitzat amb el
    rotor girant.
   \item[\textbf{S}] En m\`{a}quines rotatives indica que els assajos s'han realitzat amb el
    rotor parat.
   \item[\textbf{W}] Protecci\'{o} contra la intemp\`{e}rie.
\end{list}


Algunes versions antigues del codi \textsf{IP} poden tenir una tercera xifra que indica la resist\`{e}ncia a impactes mec\`{a}nics, no obstant, avui en dia la norma \textsf{CEI 60529} ja no recull aquesta tercera xifra, i la resist\`{e}ncia a impactes mec\`{a}nics ve indicada pel codi \textsf{IK} (vegeu l'apartat seg\"{u}ent). Aquesta tercera xifra donava el grau de resist\`{e}ncia de l'element envoltant a una certa energia d'impacte:
\begin{list}{}
   {\setlength{\labelwidth}{10mm} \setlength{\leftmargin}{10mm} \setlength{\labelsep}{2mm}}
   \item[\textbf{0}] Cap resist\`{e}ncia en particular a l'impacte.
   \item[\textbf{1}] Resisteix una energia d'impacte de 0{,}225\unit{J}, equivalent a l'impacte d'una massa de 150\unit{g} deixada anar des d'una altura de 15\unit{cm}.
   \item[\textbf{2}] Resisteix una energia d'impacte de 0{,}375\unit{J}, equivalent a l'impacte d'una massa de 250\unit{g} deixada anar des d'una altura de 15\unit{cm}.
   \item[\textbf{3}] Resisteix una energia d'impacte de 0{,}5\unit{J}, equivalent a l'impacte d'una massa de 250\unit{g} deixada anar des d'una altura de 20\unit{cm}.
   \item[\textbf{5}] Resisteix una energia d'impacte de 2\unit{J}, equivalent a l'impacte d'una massa de 500\unit{g} deixada anar des d'una altura de 40\unit{cm}.
   \item[\textbf{7}] Resisteix una energia d'impacte de 6\unit{J}, equivalent a l'impacte d'una massa de 1{,}5\unit{kg} deixada anar des d'una altura de 40\unit{cm}.
   \item[\textbf{9}]Resisteix una energia d'impacte de 20\unit{J}, equivalent a l'impacte d'una massa de 5\unit{kg} deixada anar des d'una altura de 40\unit{cm}.
\end{list}

\section{Codi IK de resist\`{e}ncia a impactes} \index{IK} \index{resist\`{e}ncia a impactes} \index{EN!50102}

Actualment, el codi \textsf{IK} definit en la norma \textsf{EN 50102} \'{e}s el que defineix la resist\`{e}ncia  als impactes mec\`{a}nics d'un element  envoltant. Aquest codi est\`{a} format per les lletres \textsf{IK} seguides d'un n\'{u}mero de dues xifres, que d\'{o}na el grau de resist\`{e}ncia de l'element envoltant a una certa energia d'impacte:

\begin{list}{}
   {\setlength{\labelwidth}{10mm} \setlength{\leftmargin}{10mm} \setlength{\labelsep}{2mm}}
   \item[\textbf{00}] Cap resist\`{e}ncia en particular a l'impacte.
   \item[\textbf{01}] Resisteix una energia d'impacte de 0{,}15\unit{J}, equivalent a l'impacte d'una massa de 200\unit{g} deixada anar des d'una altura de 7{,}5\unit{cm}.
   \item[\textbf{02}] Resisteix una energia d'impacte de 0{,}2\unit{J}, equivalent a l'impacte d'una massa de 200\unit{g} deixada anar des d'una altura de 10\unit{cm}.
   \item[\textbf{03}] Resisteix una energia d'impacte de 0{,}35\unit{J}, equivalent a l'impacte d'una massa de 200\unit{g} deixada anar des d'una altura de 17{,}5\unit{cm}.
   \item[\textbf{04}] Resisteix una energia d'impacte de 0{,}5\unit{J}, equivalent a l'impacte d'una massa de 200\unit{g} deixada anar des d'una altura de 25\unit{cm}.
   \item[\textbf{05}] Resisteix una energia d'impacte de 0{,}7\unit{J}, equivalent a l'impacte d'una massa de 200\unit{g} deixada anar des d'una altura de 35\unit{cm}.
   \item[\textbf{06}]Resisteix una energia d'impacte de 1\unit{J}, equivalent a l'impacte d'una massa de 500\unit{g} deixada anar des d'una altura de 20\unit{cm}.
   \item[\textbf{07}]Resisteix una energia d'impacte de 2\unit{J}, equivalent a l'impacte d'una massa de 500\unit{g} deixada anar des d'una altura de 40\unit{cm}.
   \item[\textbf{08}]Resisteix una energia d'impacte de 5\unit{J}, equivalent a l'impacte d'una massa de 1{,}7\unit{kg} deixada anar des d'una altura de 29{,}5\unit{cm}.
   \item[\textbf{09}]Resisteix una energia d'impacte de 10\unit{J}, equivalent a l'impacte d'una massa de 5\unit{kg} deixada anar des d'una altura de 20\unit{cm}.
   \item[\textbf{10}]Resisteix una energia d'impacte de 20\unit{J}, equivalent a l'impacte d'una massa de 5\unit{kg} deixada anar des d'una altura de 40\unit{cm}.
\end{list}



\section{Codi NEMA d'elements envoltants}
\index{codis NEMA d'elements envoltants} \index{NEMA!250}

La {"<}National Electrical Manufacturers Association{">}
codifica els elements envoltants en la norma \textsf{NEMA 250}, de manera similar al codi \textsf{IP}, segons el seu grau de protecci\'{o} contra elements externs nocius. Podeu trobar m\'{e}s informaci\'{o} a l'adre\c{c}a: \href{http://www.nema.org/prod/be/enclosures/}{www.nema.org/prod/be/enclosures/}.

Aquesta norma defineix els seg\"{u}ents valors:

\begin{list}{}
   {\setlength{\labelwidth}{10mm} \setlength{\leftmargin}{10mm} \setlength{\labelsep}{2mm}}
   \item[\textbf{1}] Protecci\'{o} contra la pols, per\`{o} no de forma total, i contra les esquitxades suaus o indirectes; impedeix el contacte accidental amb els equips interns. S'utilitza en interiors en condicions atmosf\`{e}riques normals.
   \item[\textbf{2}] Com el tipus 1, i a m\'{e}s ofereix protecci\'{o} total contra el goteig.
   \item[\textbf{3}] Protecci\'{o} contra la pols, la pluja, l'aiguaneu i la neu; impedeix el contacte accidental amb els equips interns. S'utilitza en interiors o exteriors i pot suportar la formaci\'{o} de gel al seu damunt.
   \item[\textbf{3R}] Com el tipus 3, per\`{o} sense protecci\'{o} contra la pols.
   \item[\textbf{3S}] Com el tipus 3, i a m\'{e}s els mecanismes externs han de ser operables quan s'hi dipositi gel.
    \item[\textbf{4}] Protecci\'{o} contra la pols, la pluja, l'aiguaneu, la neu, les esquitxades i els raigs d'aigua directes; impedeix el contacte accidental amb els equips interns. S'utilitza en interiors o exteriors i pot suportar la formaci\'{o} de gel al seu damunt.
   \item[\textbf{4X}] Com el tipus 4, i a m\'{e}s ofereix protecci\'{o} contra la corrosi\'{o}.
   \item[\textbf{5}] Protecci\'{o} contra la pols i contra les esquitxades suaus o indirectes; impedeix el contacte accidental amb els equips interns. S'utilitza en interiors.
   \item[\textbf{6}] Protecci\'{o} contra els raigs d'aigua directes i contra l'entrada d'aigua en ser submergit un temps curt; impedeix el contacte accidental amb els equips interns. S'utilitza en interiors o exteriors i pot suportar la formaci\'{o} de gel al seu damunt.
   \item[\textbf{6P}] Com el tipus 6, per\`{o} protegit contra l'entrada d'aigua en ser submergit durant un temps m\'{e}s llarg.
   \item[\textbf{7}] S'utilitza en interiors, en ambients perillosos definits per la norma \textsf{NEC} com a classe I, grups A, B, C o D.
   \item[\textbf{8}] Com el tipus 7, per\`{o} d'\'{u}s interior i exterior.
   \item[\textbf{9}] S'utilitza en interiors i exteriors, en ambients perillosos definits per la norma \textsf{NEC} com a classe II, grups E, F o G.
   \item[\textbf{10}] Compleix el requisits del {"<}Mine Safety and Health Administration{">} 30 CFR part 18.
   \item[\textbf{11}] Protecci\'{o} contra l'efecte corrosiu de l\'{\i}quids i gasos.
   \item[\textbf{12}] Protecci\'{o} contra la pols en suspensi\'{o} i contra les esquitxades suaus o indirectes; impedeix el contacte accidental amb els equips interns. S'utilitza en interiors.
   \item[\textbf{12K}] Com el tipus 12, per\`{o} l'element envoltant pot tenir obertures.
   \item[\textbf{13}] Protecci\'{o} contra la pols en suspensi\'{o} i contra les esquitxades o ruixats d'aigua, oli o l\'{\i}quids no corrosius; impedeix el contacte accidental amb els equips interns. S'utilitza en interiors.
\end{list}

\pagebreak La Taula \vref{taula:nema-IP} es pot utilitzar per trobar el codi \textsf{IP} equivalent a un codi \textsf{NEMA} donat; no ha d'utilitzar-se per a la conversi\'{o} contr\`{a}ria.
\begin{table}[htb]
   \caption{\label{taula:nema-IP} Conversi\'{o} de codis \textsf{NEMA} a codis \textsf{IP}}
   \begin{center}\begin{tabular}{cc}
   \toprule[1pt]
   Codi \textsf{NEMA} & Codi \textsf{IP} equivalent \\
   \midrule
   1 & \textsf{IP}10 \\
   2 & \textsf{IP}11 \\
   3 & \textsf{IP}54 \\
   3R & \textsf{IP}14 \\
   3S & \textsf{IP}54 \\
   4 i 4X & \textsf{IP}56 \\
   5 & \textsf{IP}52 \\
   6 i 6P & \textsf{IP}67\\
   12 i 12K & \textsf{IP}52 \\
   13 & \textsf{IP}54 \\
   \bottomrule[1pt]
   \end{tabular} \end{center}
\end{table}



\section{Classes NEMA d'a\"{\i}llaments t\`{e}rmics en motors}
\index{classes NEMA d'a\"{\i}llaments t\`{e}rmics} \index{NEMA}

Quan es posa en marxa un motor, la seva temperatura comen\c{c}a a pujar
per sobre de la temperatura ambient a causa del corrent que circula
pels seus debanats.

La {"<}National Electrical Manufacturers Association{">} (\textsf{NEMA}),
defineix diverses classes d'a\"{\i}llament t\`{e}rmic, depenent de
l'increment global de temperatura perm\`{e}s respecte de la temperatura
ambient, que fixa en 40\unit{\celsius};\index{temperatura!ambient}
per a cada classe es permet un increment addicional de temperatura
en el punt m\'{e}s calent, situat en el centre dels debanats del
motor.\index{temperatura!en el punt m\'{e}s calent}

En la Taula \vref{taula:classes-nema} es donen els valors dels increments de temperatura permesos per a cadascuna de les diferents classes d'a\"{\i}llament, partint d'una temperatura ambient de 40\unit{\celsius}.
\begin{table}[htb]
   \caption{\label{taula:classes-nema} Classes \textsf{NEMA} d'a\"{\i}llaments t\`{e}rmics en motors}
   \begin{center}\begin{tabular}{cr<{\hspace{6em}}r<{\hspace{8em}}}
   \toprule[1pt]
   Classe & \multicolumn{1}{c}{Increment global de temperatura} & \multicolumn{1}{c}{Increment addicional de temperatura} \\
   NEMA &   \multicolumn{1}{c}{sobre la temperatura ambient}  & \multicolumn{1}{c}{en el punt m\'{e}s calent} \\
   \midrule
   A & 60\unit{\celsius} & 5\unit{\celsius}   \\
   B & 80\unit{\celsius} & 10\unit{\celsius}   \\
   F & 105\unit{\celsius} & 10\unit{\celsius}   \\
   H & 125\unit{\celsius} & 15\unit{\celsius}   \\
   \bottomrule[1pt]
   \end{tabular} \end{center}
\end{table}
\index{A} \index{B} \index{F} \index{H}


\section{Norma CEI  d'interruptors autom\`{a}tics  de baixa tensi\'{o}}


Es donen a continuaci\'{o} algunes definicions incloses en la norma \textsf{CEI 60947-2}, referents als interruptors autom\`{a}tics de baixa tensi\'{o}, \'{e}s a dir, tensions que no passin dels 1000\unit{V} en corrent altern, o dels 1500\unit{V} en corrent continu.\index{CEI!60947-2}

Aquesta norma defineix un interruptor autom\`{a}tic de la manera seg\"{u}ent: Aparell mec\`{a}nic de connexi\'{o} capa\c{c} d'establir, de suportar i d'interrompre les intensitats en les condicions normals d'un circuit, aix\'{\i} com d'establir, de suportar durant un temps especificat i  d'interrompre les intensitats en les condicions anormals especificades d'un circuit, com per exemple les que apareixen durant un curt circuit.

Un interruptor autom\`{a}tic es diu que \'{e}s limitador de corrent, quan el seu temps d'obertura \'{e}s particularment breu, per tal d'evitar que el corrent que s'origina en un curt circuit arribi al seu valor m\`{a}xim.

Quan tenim dos  dispositius de protecci\'{o} (interruptors autom\`{a}tics, fusibles, etc.) en s\`{e}rie, la selectivitat es diu que \'{e}s total si per a qualsevol nivell d'intensitat, el dispositiu situat aig\"{u}es avall obre sempre abans que ho faci el dispositiu situat aig\"{u}es amunt. La selectivitat es diu que \'{e}s parcial, si l'obertura del dispositiu situat aig\"{u}es avall, abans que ho faci el dispositiu situat aig\"{u}es amunt, nom\'{e}s est\`{a} garantida fins a una cert nivell d'intensitat $I\ped{s}$, anomenada intensitat l\'{\i}mit de selectivitat; $I\ped{s}$ correspon al punt d'intersecci\'{o} de les caracter\'{\i}stiques intensitat--temps dels dos dispositius de protecci\'{o}. En el cas d'interruptors autom\`{a}tics es defineixen dues categories d'\'{u}s:\index{Is@$I\ped{s}$}
 \begin{list}{}
   {\setlength{\labelwidth}{10mm} \setlength{\leftmargin}{10mm} \setlength{\labelsep}{2mm}}
   \item[\textbf{A}] Interruptors autom\`{a}tics que no estan espec\'{\i}ficament preparats per ser selectius en condicions de curt circuit, amb altres dispositius de protecci\'{o} situats en s\`{e}rie  aig\"{u}es avall.
   \item[\textbf{B}] Interruptors autom\`{a}tics espec\'{\i}ficament concebuts per ser selectius en condicions de curt circuit, amb altres dispositius de protecci\'{o} situats en s\`{e}rie  aig\"{u}es avall. Aquest interruptors han d'especificar el seu corrent admissible de curta durada $I\ped{cw}$.
\end{list}

Es relacionen a continuaci\'{o} les definicions de diversos par\`{a}metres del interruptors autom\`{a}tics; es d\'{o}na entre par\`{e}ntesis el nom equivalent en angl\`{e}s:
\begin{list}{}
   {\setlength{\labelwidth}{10mm} \setlength{\leftmargin}{10mm} \setlength{\labelsep}{2mm}}
   \item[$\boldsymbol{U\ped{e}}$] Tensi\'{o} nominal d'operaci\'{o} ({"<}rated operational voltage{">}).\index{Ue@$U\ped{e}$}
   \item[$\boldsymbol{U\ped{i}}$] Tensi\'{o} nominal d'a\"{\i}llament ({"<}rated insulation voltage{">}). \'{E}s el valor de tensi\'{o} utilitzat en els assajos diel\`{e}ctrics de l'interruptor;  el valor m\'{e}s elevat d'$U\ped{e}$ no pot ser mai superior a $U\ped{i}$.\index{Ui@$U\ped{i}$}
    \item[$\boldsymbol{U\ped{imp}}$] Tensi\'{o} nominal d'impuls suportada ({"<}rated impulse withstand voltage{">}). \'{E}s el valor de pic d'una tensi\'{o} d'impuls, de forma i polaritat predeterminades, que l'interruptor pot suportar.\index{Uimp@$U\ped{imp}$}
   \item[$\boldsymbol{I\ped{th}}$] Corrent t\`{e}rmic convencional a l'aire lliure ({"<}conventional free-air thermal current{">}).  \'{E}s el valor  m\`{a}xim del corrent a utilitzar en els assajos d'escalfament de l'interruptor sense cap element envoltant, a l'aire lliure.\index{Ith@$I\ped{th}$}
   \item[$\boldsymbol{I\ped{the}}$] Corrent t\`{e}rmic convencional dins d'un element envoltant ({"<}conventional enclosed thermal current{">}).  \'{E}s el valor  m\`{a}xim del corrent a utilitzar en els assajos d'escalfament de l'interruptor, quan est\`{a} situat dins d'un element envoltant especificat.\index{Ithe@$I\ped{the}$}
    \item[$\boldsymbol{I\ped{u}}$] Corrent nominal ininterromput ({"<}rated uninterrupted current{">}).  \'{E}s el valor d'intensitat, fixat pel fabricant, que l'interruptor pot suportar de manera ininterrompuda.\index{Iu@$I\ped{u}$}
    \item[$\boldsymbol{I\ped{n}}$] Corrent nominal ({"<}rated current{">}).  En els interruptors autom\`{a}tics \'{e}s equivalent a $I\ped{u}$ i t\'{e} el mateix valor que $I\ped{th}$.\index{In@$I\ped{n}$}
    \item[$\boldsymbol{I\ped{cu}}$] Poder nominal de tall \'{u}ltim en curt circuit ({"<}rated ultimate  short-circuit beaking capacity{">}). \'{E}s la capacitat que t\'{e} l'interruptor d'obrir en curt circuit a la tensi\'{o} nominal d'operaci\'{o}, en un cilce d'assaig del tipus O--t--CO (obrir, tancar i obrir); s'expressa pel valor efica\c{c} sim\`{e}tric del corrent esperat, en kA. Despr\'{e}s de l'assaig, no es requereix que l'interruptor pugui suportar en r\`{e}gim continu la seva intensitat nominal.\index{Icu@$I\ped{cu}$}
    \item[$\boldsymbol{I\ped{cs}}$] Poder nominal de tall de servei en curt circuit ({"<}rated service  short-circuit beaking capacity{">}). \'{E}s la capacitat que t\'{e} l'interruptor d'obrir en curt circuit a la tensi\'{o} nominal d'operaci\'{o}, en un cilce d'assaig del tipus O--t--CO--t--CO (obrir, tancar i obrir, tancar i obrir); s'expressa pel valor  del corrent esperat, en kA, corresponent a un dels percentatges d'$I\ped{cu}$ especificats en la Taula \vref{taula:IcsIcu}, arrodonit al valor enter m\'{e}s pr\`{o}xim. Despr\'{e}s de l'assaig, l'interruptor cal que pugui suportar en r\`{e}gim continu la seva intensitat nominal.\index{Ics@$I\ped{cs}$}
         \begin{table}[h]
           \caption{\label{taula:IcsIcu} Valors d'$I\ped{cs}$ respecte d'$I\ped{cu}$}
           \begin{center}\begin{tabular}{cc}
           \toprule[1pt]
           Categoria d'\'{u}s &  Valors possibles d'$I\ped{cs}\, [\% I\ped{cu}]$ \\
           \midrule
           A &  25, 50, 75, 100   \\
           B & 50, 75, 100   \\
           \bottomrule[1pt]
           \end{tabular} \end{center}
         \end{table}
    \item[$\boldsymbol{I\ped{cm}}$] Poder nominal de tancament en curt circuit ({"<}rated short-circuit making capacity{">}). \'{E}s la capacitat que t\'{e} l'interruptor de tancar en curt circuit a la tensi\'{o} nominal d'operaci\'{o}, per a un factor de pot\`{e}ncia especificat en corrent altern, o per a una constant de temps especificada en corrent continu; s'expressa pel valor m\`{a}xim de pic del corrent esperat. En el cas de corrent altern ha de complir-se: $I\ped{cm} \geq n I\ped{cu}$; els valor possibles del par\`{a}metre $n$ poden veure's en la Taula \vref{taula:IcmIcu}.\index{Icm@$I\ped{cm}$}
        \begin{table}[h]
           \caption{\label{taula:IcmIcu} Valors d'$n$ respecte d'$I\ped{cu}$}
           \begin{center}\begin{tabular}{ccc}
           \toprule[1pt]
           $I\ped{cu}$  [kA] &  Factor de pot\`{e}ncia & $n$ \\
           \midrule
           $4{,}5 \leq I\ped{cu} \leq 6$\phantom{0} & 0{,}7\phantom{0}  & 1{,}5   \\
           \phantom{0,}$6 < I\ped{cu} \leq 10$         & 0{,}5\phantom{0}  & 1{,}7   \\
           \phantom{,}$10 < I\ped{cu} \leq 20$       & 0{,}3\phantom{0}  & 2{,}0   \\
           \phantom{,}$20 < I\ped{cu} \leq 50$       & 0{,}25 &  2{,}1   \\
           \phantom{$0{,}0<{}$}$I\ped{cu} > 50$     & 0{,}2\phantom{0}  & 2{,}2   \\
           \bottomrule[1pt]
           \end{tabular} \end{center}
         \end{table}
    \item[$\boldsymbol{I\ped{cw}}$] Corrent nominal de curta durada admissible ({"<}rated short-time withstand current{">}). \'{E}s el corrent que pot suportar l'interruptor durant un temps convencional, sense danyar-se i sense alterar les seves caracter\'{\i}stiques, obtenint-se aix\'{\i} la possibilitat de ser selectiu amb altres dispositius de protecci\'{o} situats en s\`{e}rie  aig\"{u}es avall; s'expressa pel valor efica\c{c} sim\`{e}tric del corrent esperat, en kA. El temps m\'{\i}nim que ha de suportar la intensitat \'{e}s 0,05\unit{s}, i els valors preferits s\'{o}n: 0,05\unit{s}, 0,1\unit{s}, 0,25\unit{s}, 0,5\unit{s} i 1\unit{s}. El valor m\'{\i}nim que ha de tenir $I\ped{cw}$ pot veure's en la Taula\vref{taula:InIcw}. \index{Icw@$I\ped{cw}$}
        \begin{table}[h]
           \caption{\label{taula:InIcw} Valors d'$I\ped{cw}$ respecte d'$I\ped{n}$}
           \begin{center}\begin{tabular}{cc}
           \toprule[1pt]
           $I\ped{n}$  [A] &  Valor m\'{\i}nim d'$I\ped{cw}$ \\
           \midrule
           $I\ped{n} \leq 2500$  & m\`{a}xim entre $12  I\ped{n}$ i 15\unit{kA}  \\
           $I\ped{n} > 2500$  & 30\unit{kA}   \\
           \bottomrule[1pt]
           \end{tabular} \end{center}
         \end{table}
\end{list}





\section{\`{A}mbit d'aplicaci\'{o} de diverses Normes CEI}\index{CEI}\label{sec:normes_IEC}

Es relacionen a continuaci\'{o} diverses normes \textsf{CEI} agrupades pel seu \`{a}mbit d'aplicaci\'{o}. Podeu trobar el llistat complet de les normes \textsf{CEI} a l'adre\c{c}a: \href{http://www.iec.ch/standardsdev/publications/}{www.iec.ch/standardsdev/publications/}.

\subsection*{Aparellatge de baixa tensi\'{o}}
\begin{dinglist}{'167}
    \item \textbf{CEI 60947-1}. Low-voltage switchgear and controlgear -- General rules.\index{CEI!60947-1}
    \item \textbf{CEI 60947-2}. Low-voltage switchgear and controlgear -- Circuit-breakers.\index{CEI!60947-2}
    \item \textbf{CEI 60947-3}. Low-voltage switchgear and controlgear -- Switches, disconnectors, switch-dis\-con\-nec\-tors and fuse-combination units.\index{CEI!60947-3}
    \item \textbf{CEI 60947-4-1}. Low-voltage switchgear and controlgear -- Contactors and Motor Starters -- Electromechanical Contactors and Motor Starters.\index{CEI!60947-4-1}
    \item \textbf{CEI 60947-4-2}. Low-voltage switchgear and controlgear -- Contactors and Motor Starters -- A.C. Semiconductor Motor Controllers and Starters.\index{CEI!60947-4-2}
    \item \textbf{CEI 60947-4-3}. Low-voltage switchgear and controlgear -- Contactors and Motor Starters -- A.C. Semiconductor Controllers and Contactors for non-motor Loads.\index{CEI!60947-4-3}
\end{dinglist}

\subsection*{Coordinaci\'{o} d'a\"{\i}llaments}
\begin{dinglist}{'167}
    \item \textbf{CEI 60071-1}. Insulation co-ordination -- Definitions, principles and rules.\index{CEI!60071-1}
    \item \textbf{CEI 60071-2}. Insulation co-ordination -- Application guide.\index{CEI!60071-2}
    \item \textbf{CEI 60071-3}. Insulation co-ordination -- Phase to phase insulation coordination. Principles, rules and application guide.\index{CEI!60071-3}
    \item \textbf{CEI 60071-4}. Insulation co-ordination -- Computational guide to insulation co-ordination and modelling of electrical networks.\index{CEI!60071-4}
    \item \textbf{CEI 60071-5}. Insulation co-ordination -- Procedures for high-voltage direct current (HVDC) converter stations.\index{CEI!60071-5}
\end{dinglist}

\subsection*{Fusibles de baixa tensi\'{o}}
\begin{dinglist}{'167}
    \item \textbf{CEI 60269-1}. Low-voltage fuses -- General requirements.\index{CEI!60269-1}
    \item \textbf{CEI 60269-2}. Low-voltage fuses -- Supplementary requirements for fuses for use by authorized persons
          (fuses mainly for industrial application) -- Examples of standardized systems of fuses A to J.\index{CEI!60269-2}
\end{dinglist}

\subsection*{Proteccions el\`{e}ctriques}
\begin{dinglist}{'167}
    \item \textbf{CEI 60255-3}. Electrical relays -- Single input energizing quantity measuring relays with dependent Low-voltage fuses - General requirements.\index{CEI!60255-03@60255-3}
    \item \textbf{CEI 60255-6}. Electrical relays -- Measuring relays with more than one input energizing quantity.\index{CEI!60255-06@60255-6}
    \item \textbf{CEI 60255-8}. Electrical relays -- Thermal electrical relays.\index{CEI!60255-08@60255-8}
    \item \textbf{CEI 60255-12}. Electrical relays -- Directional relays and power relays with two input energizing quantities.\index{CEI!60255-12}
    \item \textbf{CEI 60255-13}. Electrical relays -- Biased (percentage) differential relays.\index{CEI!60255-13}
    \item \textbf{CEI 60255-16}. Electrical relays -- Impedance measuring relays.\index{CEI!60255-16}
\end{dinglist}

\subsection*{Representaci\'{o} i simbologia}
\begin{dinglist}{'167}
    \item \textbf{CEI 60027-1}. Letter symbols to be used in electrical technology -- General.\index{CEI!60027-1}
    \item \textbf{CEI 60027-2}. Letter symbols to be used in electrical technology -- Telecommunications and electronics.\index{CEI!60027-2}
    \item \textbf{CEI 60027-3}. Letter symbols to be used in electrical technology -- Logarithmic and related quantities, and their units.\index{CEI!60027-3}
    \item \textbf{CEI 60027-4}. Letter symbols to be used in electrical technology -- Symbols for quantities to be used for rotating electrical machines.\index{CEI!60027-4}
    \item \textbf{CEI 60027-6}. Letter symbols to be used in electrical technology -- Control technology.\index{CEI!60027-6}
\end{dinglist}


\subsection*{Termoparells} \index{termoparells}
\begin{dinglist}{'167}
    \item \textbf{CEI 60584-1}. Thermocouples -- Reference tables.\index{CEI!60584-1}
    \item \textbf{CEI 60584-2}. Thermocouples -- Tolerances.\index{CEI!60584-2}
    \item \textbf{CEI 60584-3}. Thermocouples -- Extension and compensating cables - Tolerances and identification system.\index{CEI!60584-3}
\end{dinglist}


\subsection*{Transformadors de mesura i protecci\'{o}}
\begin{dinglist}{'167}
    \item \textbf{CEI 60185}. Current transformers.\index{CEI!60185}
    \item \textbf{CEI 60186}. Voltage transformers.\index{CEI!60186}
    \item \textbf{CEI 60044-1}. Instrument Transformers -- Current transformers.\index{CEI!60044-1}
    \item \textbf{CEI 60044-2}. Instrument Transformers -- Inductive voltage transformers.\index{CEI!60044-2}
    \item \textbf{CEI 60044-3}. Instrument Transformers -- Combined transformers.\index{CEI!60044-3}
\end{dinglist}


\subsection*{Transformadors de pot\`{e}ncia}
\begin{dinglist}{'167}
    \item \textbf{CEI 60076-1}. Power transformers -- General.\index{CEI!60076-01@60076-1}
    \item \textbf{CEI 60076-2}. Power transformers -- Temperature rise.\index{CEI!60076-02@60076-2}
    \item \textbf{CEI 60076-3}. Power transformers -- Insulation levels, dielectric tests and external clearances in air.\index{CEI!60076-03@60076-3}
    \item \textbf{CEI 60076-4}. Power transformers -- Guide to the lightning impulse and switching impulse testing - Power transformers and reactors.\index{CEI!60076-04@60076-4}
    \item \textbf{CEI 60076-5}. Power transformers -- Ability to withstand short circuit.\index{CEI!60076-05@60076-5}
    \item \textbf{CEI 60076-6}. Power transformers -- Reactors.\index{CEI!60076-06@60076-6}
    \item \textbf{CEI 60076-7}. Power transformers -- Loading guide for oil-immersed power transformers.\index{CEI!60076-07@60076-7}
    \item \textbf{CEI 60076-8}. Power transformers -- Application guide.\index{CEI!60076-08@60076-8}
    \item \textbf{CEI 60076-10}. Power transformers -- Determination of sound levels.\index{CEI!60076-10}
    \item \textbf{CEI 60076-11}. Power transformers -- Dry-type transformers.\index{CEI!60076-11}
\end{dinglist}


\section{\`{A}mbit d'aplicaci\'{o} de diverses Normes IEEE}\index{IEEE}\label{sec:normes_IEEE}

Es relacionen a continuaci\'{o} diverses normes \textsf{IEEE} agrupades pel seu \`{a}mbit d'aplicaci\'{o}. Podeu trobar el llistat complet de les normes \textsf{IEEE} a l'adre\c{c}a: \href{http://ieeexplore.ieee.org/xpl/standards.jsp}{ieeexplore.ieee.org/xpl/standards.jsp}.

\subsection*{Bateries i altres equips de corrent continu}
\begin{dinglist}{'167}
    \item \textbf{IEEE 450}. Recommended Practice for  Maintenance, Testing and Replacement of Vented Lead-Acid Batteries for Stationary Applications.\index{IEEE!450}
    \item \textbf{IEEE 484}. Recommended Practice for Installation Design and Installation of Large Lead Storage Batteries for Generating Stations and Substations.\index{IEEE!484}
    \item \textbf{IEEE 485}. Recommended Practice for Sizing Lead-Acid Batteries for stationary Applications.\index{IEEE!485}
    \item \textbf{IEEE 946}. Recommended Practice for the Design of Safety-Related DC Auxiliary Power Systems for Nuclear Power Generating Stations.\index{IEEE!946}
    \item \textbf{IEEE 1106}. Recommended Practice for Installation, Maintenance, Testing and Replacement of Vented Nickel-Cadmium Batteries for Stationary Applications.\index{IEEE!1106}
    \item \textbf{IEEE 1115}. Recommended Practice for Sizinig Nickel-Cadmium Batteries for Stationary Applications.\index{IEEE!1115}
    \item \textbf{IEEE 1115a}. Recommended Practice for Sizinig Nickel-Cadmium Batteries for Stationary Applications, Amendment 1: Additional Discussion on Sizing Margins.\index{IEEE!1115@1115a}
    \item \textbf{IEEE 1184}. Guide for Batteries for Uninterruptible Power Supply Systems.\index{IEEE!1184}
    \item \textbf{IEEE 1375}. Guide for the Protection of Stationary Battery Systems.\index{IEEE!1375}
    \item \textbf{IEEE 1491}. Guide for Selection and Use of Battery Monitoring Equipment in Stationary Applications.\index{IEEE!1491}
    \item \textbf{IEEE C37.14}. Low-Voltage DC Power Circuit Breakers Used in Enclosures.\index{IEEE!C37.014@C37.14}
\end{dinglist}


\subsection*{Cables}
\begin{dinglist}{'167}
    \item \textbf{IEEE 525}. Guide for the Design and Installation of Cable Systems in Substations.\index{IEEE!525}
\end{dinglist}

\subsection*{Centrals el\`{e}ctriques i subestacions}
\begin{dinglist}{'167}
    \item \textbf{IEEE 141 (Red Book)}. Recommended Practice for Electric Power Distribution for Industrial Plants.\index{IEEE!141@141 (Red Book)}
    \item \textbf{IEEE 339 (Brown Book)}. Recommended Practice for Industrial and Commercial Power Systems Analysis.\index{IEEE!339@339 (Brown Book)}
    \item \textbf{IEEE 446 (Orange Book)}. Recommended Practice for Emergency and Standby Power Systems for Industrial and Commerical Applications.\index{IEEE!446@446  (Orange Book)}
    \item \textbf{IEEE 493 (Gold Book)}. Recommended Practice for the Design of Reliable Industrial and Commercial Power Systems \index{IEEE!493@493  (Gold Book)}
    \item \textbf{IEEE 666}. Design Guide for Electric Power Service Systems for Generating Stations.\index{IEEE!666}
\end{dinglist}


\subsection*{Equips nuclears i classe 1E -- Criteris} \index{classe 1E}
\begin{dinglist}{'167}
    \item \textbf{IEEE 279}. Criteria for Protection Systems for Nuclear Power Generating Stations.\index{IEEE!279}
    \item \textbf{IEEE 308}. Criteria for Class 1E Power Systems for Nuclear Power Generating Stations.\index{IEEE!308}
    \item \textbf{IEEE 379}. Application of the Single-Failure Criterion to Nuclear Power Generating Station Safety Systems.\index{IEEE!379}
    \item \textbf{IEEE 384}. Criteria for Independence of Class 1E Equipment and Circuits.\index{IEEE!384}
        Stations.\index{IEEE!308}
    \item \textbf{IEEE 603}. Criteria for Safety Systems for Nuclear Power Generating Stations.\index{IEEE!603}
    \item \textbf{IEEE 741}. Criteria for the Protection of Class 1E Power Systems.\index{IEEE!741}
\end{dinglist}


\subsection*{Equips nuclears i classe 1E -- Disseny, insta{\l.l}aci\'{o} i proves}
\begin{dinglist}{'167}
    \item \textbf{IEEE 336}. Installation, Inspection and Testing Requirements for Class 1E Instrumentation and Electric Equipment at Nuclear Power Generating Stations.\index{IEEE!336}
     \item \textbf{IEEE 381}. Criteria for Type Tests of Class 1E Modules Used in Nuclear Power Generating Stations.\index{IEEE!381}
    \item \textbf{IEEE 383}. Standard for Type Test of Class 1E Electric Cables, Field splices and Connections for Nuclear Power Generating Stations.\index{IEEE!383}
    \item \textbf{IEEE 577}. Requirements for Reliability Analysis in Design and operation of Safety Systems for Nuclear Power Generating Stations.\index{IEEE!577}
    \item \textbf{IEEE 622}. Recommended Practice for the Design and Installation of Electric Pipe Heating Systems for Nuclear Power Generating Stations.\index{IEEE!622}
    \item \textbf{IEEE 690}. Standard for the Design and Installation of Cable Systems for Class 1E Circuits in Nuclear Power Generating Stations.\index{IEEE!690}
\end{dinglist}


\subsection*{Equips nuclears i classe 1E -- Qualificaci\'{o}}
\begin{dinglist}{'167}
    \item \textbf{IEEE 323}. Standard for Qualifying Class 1E Equipment for Nuclear Power Generating Stations.\index{IEEE!323}
    \item \textbf{IEEE 344}. Recommended Practices for Seismic Qualification of Class 1E Electric Equipment for Nuclear Power Generating Stations.\index{IEEE!344}
    \item \textbf{IEEE 535}.  Standard for Qualification of Class 1E Lead Storage Batteries for Nuclear Power Generating Stations. \index{IEEE!535}
    \item \textbf{IEEE 638}. Standard for Qualification of Class 1E Transformers for Nuclear Generating Stations.\index{IEEE!638}
    \item \textbf{IEEE 650}. Standard for Qualification of Class 1E Static Battery Chargers and Inverters for Nuclear Power Generating Stations.\index{IEEE!650}
    \item \textbf{IEEE C37.105}. Standard for Qualifying Class 1E Protective Relays and Auxiliaries for Nuclear Power Generating Stations.\index{IEEE!C37.105}
\end{dinglist}


\subsection*{Generadors diesel}
\begin{dinglist}{'167}
    \item \textbf{IEEE 387}. Criteria for Diesel-Generator Units Applied as Standby Power Supplies for Nuclear Power Generation Stations.\index{IEEE!387}
\end{dinglist}


\subsection*{Generadors el\`{e}ctrics}
\begin{dinglist}{'167}
    \item \textbf{IEEE 421.1}. Definitions for Excitation Systems for Synchronous Machines.\index{IEEE!421@421.1}
    \item \textbf{IEEE 421.2}. Guide for Identification, Testing, and Evaluation of the Dynamic Performance of Excitation Control Systems.\index{IEEE!421@421.2}
    \item \textbf{IEEE 421.3}. Standard for High-Potential Test Requirements for Excitation Systems for Synchronous Machines.\index{IEEE!421@421.3}
    \item \textbf{IEEE 421.4}. Guide for the Preparation of Excitation System Specifications.\index{IEEE!421@421.4}
    \item \textbf{IEEE 421.5}. Recommended Practice for Excitation System Models for Power System Stability Studies.\index{IEEE!421@421.5}
    \item \textbf{IEEE C37.101}. Guide for Generator Ground Protection.\index{IEEE!C37.101}
    \item \textbf{IEEE C37.102}. Guide for AC Generator Protection.\index{IEEE!C37.102}
    \item \textbf{IEEE C50.13}. Standard for Large Turbine Generators.\index{IEEE!C50.13}
\end{dinglist}


\subsection*{Interruptors d'alta tensi\'{o}}
\begin{dinglist}{'167}
    \item \textbf{IEEE C37.010}. Application Guide for AC High-Voltage Circuit Breakers Rated on a Symmetrical Current Basis.\index{IEEE!C37.010}
    \item \textbf{IEEE C37.011}. Application Guide for Transient Recovery Voltage for AC High-Voltage Circuit Breakers Rated on a Symmetrical Current Basis.\index{IEEE!C37.011}
    \item \textbf{IEEE C37.012}. Application Guide for Capacitance Current Switching for AC High-Voltage Circuit Breakers Rated on a Symmetrical Current Basis.\index{IEEE!C37.012}
    \item \textbf{IEEE C37.013}. AC High-Voltage Generator Circuit Breakers Rated on a Symmetrical Current Basis.\index{IEEE!C37.013}
    \item \textbf{IEEE C37.04}. Rating Structure for AC High-Voltage Circuit Breakers.\index{IEEE!C37.004@C37.04}
    \item \textbf{IEEE C37.06}. AC High-Voltage Circuit Breakers Rated on a Symmetrical Current Basis -- Preferred Ratings and Related Required Capabilities.\index{IEEE!C37.006@C37.06}
    \item \textbf{IEEE C37.09}. Test Procedure for AC High-Voltage Circuit Breakers Rated on a Symmetrical Current Basis.\index{IEEE!C37.009@C37.09}
    \item \textbf{IEEE C37.10}. Guide for Diagnostics and Failure Investigation of Power Circuit Breakers.\index{IEEE!C37.010@C37.10}
    \item \textbf{IEEE C37.11}. Requirements for Electrical Control for AC High-Voltage Circuit Breakers Rated on a Simmetrical Current Basis.\index{IEEE!C37.011@C37.11}
    \item \textbf{IEEE C37.12}. AC High-Voltage Circuit Breakers Rated on a Symmetrical Current Basis -- Specifications Guide.\index{IEEE!C37.012@C37.12}
\end{dinglist}


\subsection*{Interruptors de baixa tensi\'{o}}
\begin{dinglist}{'167}
\item \textbf{IEEE 1015 (Blue Book)}. Recommended Practice for Applying Low-Voltage Circuit Breakers Used in Industrial and Commercial Power Systems \index{IEEE!1015@1015 (Blue Book)}
     \item \textbf{IEEE C37.13}. Standard for Low-Voltage AC Power Circuit Breakers Used in Enclosures. \index{IEEE!C37.013@C37.13}
    \item \textbf{IEEE C37.20.1}. Metal-Enclosed Low-Voltage Power Circuit Breaker Switchgear.\index{IEEE!C37.020@C37.20.1}
    \item \textbf{IEEE C37.50}. Low-Voltage AC Power Circuit Breakers Used in Enclosures -- Test Procedures.\index{IEEE!C37.050@C37.50}
    \item \textbf{IEEE C37.51}. Metal-Enclosed Low-Voltage AC Power-Circuit-Breaker Switchgear Assemblies -- Conformance Test Procadures.\index{IEEE!C37.051@C37.51}
\end{dinglist}


\subsection*{Malles i connexions a terra}
\begin{dinglist}{'167}
    \item \textbf{IEEE 80}. Guide for Safety in AC Substaion Grounding.\index{IEEE!080@80}
    \item \textbf{IEEE 142 (Green Book)}. Recommended Practice for Grounding of Industrial and Commercial Power Systems.\index{IEEE!142@142 (Green Book)}
\end{dinglist}

\subsection*{Motors el\`{e}ctrics}
\begin{dinglist}{'167}
    \item \textbf{IEEE 288}. Guide for Induction Motor Protection.\index{IEEE!288}
    \item \textbf{IEEE C37.96}. Guide for AC Motor Protection.\index{IEEE!C37.096@C37.96}
\end{dinglist}


\subsection*{Penetracions el\`{e}ctriques}
\begin{dinglist}{'167}
    \item \textbf{IEEE 317}. Standard for Electric Penetration Assemblies in Containment Structures for Nuclear Power Generating Stations.\index{IEEE!317}
\end{dinglist}


\subsection*{Proteccions el\`{e}ctriques}
\begin{dinglist}{'167}
    \item \textbf{IEEE 242 (Buff Book)}. Recommended Practice for Protection and Coordination of Industrial and Commercial Power Systems.\index{IEEE!242@242 (Buff Book)}
    \item \textbf{IEEE C37.2}. Electrical Power System Device Function Numbers and Contact Designations.\index{IEEE!C37.002@C37.2}
    \item \textbf{IEEE C37.16}. Low-Voltage Power Circuit Breakers and AC Power Circuit Protectors.\index{IEEE!C37.016@C37.16}
    \item \textbf{IEEE C37.17}. Trip Devices for AC and General Purpose DC Low Voltage Power Circuit Breakers.\index{IEEE!C37.017@C37.17}
    \item \textbf{IEEE C37.97}. Guide for Protective Relay Applications to Power System Buses.\index{IEEE!C37.097@C37.97}
    \item \textbf{IEEE C37.99}. Guide for the Protection of Shunt Capacitor Banks.\index{IEEE!C37.099@C37.99}
    \item \textbf{IEEE C37.106}. Guide for Abnormal Frequency Protection for Power Generating Plants.\index{IEEE!C37.106}
    \item \textbf{IEEE C37.112}. Inverse-Time Characteristic Equations for Overcurrent Relays.\index{IEEE!C37.112}
    \item \textbf{IEEE C37.113}. Guide for Protective Relay Applications to Transmission Lines.\index{IEEE!C37.113}
    \item \textbf{IEEE C37.119}. Guide for Breaker Failure Protection of Power Circuit Breakers.\index{IEEE!C37.119}
\end{dinglist}


\subsection*{Rel\`{e}s}
\begin{dinglist}{'167}
    \item \textbf{IEEE C37.90}. Relays and Relay system Associated with electric Power Apparatus.\index{IEEE!C37.090@C37.90}
\end{dinglist}


\subsection*{Representaci\'{o} i simbologia}
\begin{dinglist}{'167}
    \item \textbf{IEEE 260.1}. Letter Symbols for Units of Measurement (SI Units, Customary Inch-Pound Units, and Certain Other Units).\index{IEEE!260}
    \item \textbf{IEEE 315}. Graphic Symbols for Electrical and Electronics Diagrams.\index{IEEE!315}
    \item \textbf{IEEE 315A}. Supplement to Graphic Symbols for Electrical and Electronics Diagrams.\index{IEEE!315@315A}
\end{dinglist}

\subsection*{Sistemes digitals}
\begin{dinglist}{'167}
    \item \textbf{IEEE 7-4.3.2}. Criteria for Digital Computers in Safety Systems of Nuclear Power Generating Stations.\index{IEEE!007@7-4.3.2}
\end{dinglist}


\subsection*{Soroll el\`{e}ctric}
\begin{dinglist}{'167}
    \item \textbf{IEEE 518}. Guide for the Installation of Electrical Equipment to Minimize Electrical Noise Inputs to Controllers from External Sources.\index{IEEE!518}
\end{dinglist}

\subsection*{Taulers de control}
\begin{dinglist}{'167}
    \item \textbf{IEEE C37.21}. Control Switchboards.\index{IEEE!C37.021@C37.21}
\end{dinglist}


\subsection*{Transformadors de mesura i protecci\'{o}}
\begin{dinglist}{'167}
    \item \textbf{IEEE C37.110}. Guide for the Application of Current Transformers Used for Protective Relaying Purposes.\index{IEEE!C37.110}
    \item \textbf{IEEE C57.13}. Requirements for Instrument Transformers.\index{IEEE!C57.13}
\end{dinglist}


\subsection*{Transformadors de pot\`{e}ncia}
\begin{dinglist}{'167}
    \item \textbf{IEEE C37.91}. Guide for Protecting Power Transformers.\index{IEEE!C37.091@C37.91}
    \item \textbf{IEEE C57.12.00}. General Requirements for Liquid-Immersed Distribution, Power, and Regulating Transformers.\index{IEEE!C57.12.00}
    \item \textbf{IEEE C57.12.01}. General Requirements for Dry-Type Distribution and Power Transformers, Including Those with Solid-Cast and/or Resin Encapsulated Windings.\index{IEEE!C57.12.01}
\end{dinglist}



\subsection*{V\`{a}lvules motoritzades}
\begin{dinglist}{'167}
    \item \textbf{IEEE 1290}. Guide for Motor Operated Valve (MOV) Motor Application, Protection, Control, and Testing in Nuclear Power Generating Stations.\index{IEEE!1290}
\end{dinglist}

