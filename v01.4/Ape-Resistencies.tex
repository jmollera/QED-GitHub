\chapter{Resist\`{e}ncies}

\section{Codificaci\'{o} en colors} \index{resist\`{e}ncies!codificaci\'{o} en colors}

La codificaci\'{o} en colors del valor \`{o}hmic de les resist\`{e}ncies
consisteix en tres o quatre bandes de colors, m\'{e}s una banda de color
addicional, una mica separada, per codificar la toler\`{a}ncia.

En el cas de tres bandes, les dues primeres defineixen els dos d\'{\i}gits que formen el valor
base, i la tercera el valor  multiplicatiu; en el cas de quatre bandes, les tres primeres
defineixen els tres d\'{\i}gits que formen el valor base, i la quarta el valor multiplicatiu.

En la taula seg\"{u}ent es presenta la codificaci\'{o} en colors de les resist\`{e}ncies.

\begin{table}[htb]
   \caption{Codificaci\'{o} en colors de les resist\`{e}ncies}
   \[ \begin{array}{llccc>{\hspace{3em}}l>{\hspace{3em}}l}
   \toprule[1pt]
   \multicolumn{2}{c}{\text{Color}} & \text{1r d\'{\i}git} & \text{2n d\'{\i}git} & \text{3r d\'{\i}git} & \multicolumn{1}{c}{\text{
   Multiplicador}} & \multicolumn{1}{c}{\text{Toler\`{a}ncia }[\pm\;\%]}
   \\
   \midrule
           &\text{---}     &  \text{---} &  \text{---} &  \text{---} &  \text{---} & 20 \\
   \textcolor[rgb]{0.83,0.83,0.83}{\blacksquare} &\text{plata}   &  \text{---} &  \text{---} &  \text{---} &  10^{-2} & 10 \\
   \textcolor[rgb]{0.80,0.82,0.17}{\blacksquare} &\text{or}      &  \text{---} &  \text{---} &  \text{---} &  10^{-1} & 5 \\
   \blacksquare &\text{negre}   &  \text{---}   &  0   &  0   &  1   & \text{---} \\
   \textcolor[rgb]{0.57,0.00,0.00}{\blacksquare} &\text{marr\'{o}}   &  1   &  1   &  1   &  10   & 1 \\
   \textcolor[rgb]{1.00,0.00,0.00}{\blacksquare} &\text{vermell} &  2   &  2   &  2   &  10^2   & 2 \\
   \textcolor[rgb]{1.00,0.55,0.09}{\blacksquare} &\text{taronja} &  3   &  3   &  3   &  10^3   & \text{---} \\
   \textcolor[rgb]{1.00,1.00,0.00}{\blacksquare} &\text{groc}    &  4   &  4   &  4   &  10^4   & \text{---} \\
   \textcolor[rgb]{0.00,1.00,0.00}{\blacksquare} &\text{verd}    &  5   &  5   &  5   &  10^5   & 0{,}5 \\
   \textcolor[rgb]{0.00,0.00,1.00}{\blacksquare} &\text{blau}    &  6   &  6   &  6   &  10^6   & 0{,}25 \\
   \textcolor[rgb]{0.68,0.31,0.68}{\blacksquare} &\text{violeta} &  7   &  7   &  7   &  10^7   & 0{,}1 \\
   \textcolor[rgb]{0.48,0.48,0.48}{\blacksquare} &\text{gris}    &  8   &  8   &  8   &  \text{---}   & 0{,}05 \\
   \square &\text{blanc}   &  9   &  9   &  9   &  \text{---}   & \text{---} \\
   \bottomrule[1pt]
   \end{array}   \]
\end{table}

Fa temps existien tamb\'{e} resist\`{e}ncies de toler\`{a}ncia $\pm50\unit{\%}$,
per\`{o} avui dia ja no se'n fabriquen.

Les resist\`{e}ncies de toler\`{a}ncia $\pm20\unit{\%}$ s'utilitzen molt poc, i alguns fabricants
ni tan sols les subministren.

\begin{exemple}
   Es tracta d'obtenir els valors \`{o}hmics i les toler\`{a}ncies de les dues resist\`{e}ncies seg\"{u}ents,
   definides pels colors:
\begin{enumerate}
   \renewcommand{\labelenumi}{\alph{enumi})}
   \item  \begin{minipage}{1.8cm} \PSforPDF{
        %PsTricks content-type (pstricks.sty package needed)
        %Add \usepackage{pstricks} in the preamble of your LaTeX file
        \psset{xunit=1mm,yunit=1mm,runit=1mm}
        \psset{linewidth=0.3,dotsep=1,hatchwidth=0.3,hatchsep=1.5,shadowsize=1}
        \psset{dotsize=0.7 2.5,dotscale=1 1,fillcolor=black}
        \begin{pspicture}(0,0)(17,5)
        \psline(0.5,2.5)(3,2.5)
        \newrgbcolor{userLineColour}{0.2 0.2 0}
        \psline[linecolor=userLineColour](14,2.5)(16.5,2.5) \rput(8.5,2.5){}
        \newrgbcolor{userLineColour}{0.8 0.8 0.8}
        \newrgbcolor{userFillColour}{0.8 0.8 0.8}
        \pspolygon[linecolor=userLineColour,fillcolor=userFillColour,fillstyle=solid](3,4.5)(14,4.5)
        (14,4.5)(14,0.5) (14,0.5)(3,0.5) (3,0.5)(3,4.5) \rput(23,9.5){}
        \newrgbcolor{userLineColour}{0.8 0.8 0}
        \newrgbcolor{userFillColour}{0.8 0.8 0}
        \pspolygon[linecolor=userLineColour,fillcolor=userFillColour,fillstyle=solid](12,4.5)(12.5,4.5)
        (12.5,4.5)(12.5,0.5) (12.5,0.5)(12,0.5) (12,0.5)(12,4.5)
        \pspolygon[linecolor=yellow,fillcolor=yellow,fillstyle=solid](7.5,4.5)(8,4.5)
        (8,4.5)(8,0.5) (8,0.5)(7.5,0.5) (7.5,0.5)(7.5,4.5)
        \pspolygon[linecolor=gray,fillcolor=gray,fillstyle=solid](6,4.5)(6.5,4.5)
        (6.5,4.5)(6.5,0.5) (6.5,0.5)(6,0.5) (6,0.5)(6,4.5)
        \pspolygon[linecolor=blue,fillcolor=blue,fillstyle=solid](4.5,4.5)(5,4.5)
        (5,4.5)(5,0.5) (5,0.5)(4.5,0.5) (4.5,0.5)(4.5,4.5)
        \end{pspicture}
        } \end{minipage} (Blau-Gris-Groc Or)
   \item  \begin{minipage}{1.8cm} \PSforPDF{
        %PsTricks content-type (pstricks.sty package needed)
        %Add \usepackage{pstricks} in the preamble of your LaTeX file
        \psset{xunit=1mm,yunit=1mm,runit=1mm}
        \psset{linewidth=0.3,dotsep=1,hatchwidth=0.3,hatchsep=1.5,shadowsize=1}
        \psset{dotsize=0.7 2.5,dotscale=1 1,fillcolor=black}
        \begin{pspicture}(0,0)(17,5)
        \psline(0.5,2.5)(3,2.5) \psline(14,2.5)(16.5,2.5) \rput(8.5,2.5){}
        \newrgbcolor{userLineColour}{0.8 0.8 0.8}
        \newrgbcolor{userFillColour}{0.8 0.8 0.8}
        \pspolygon[linecolor=userLineColour,fillcolor=userFillColour,fillstyle=solid](3,4.5)(14,4.5)
        (14,4.5)(14,0.5) (14,0.5)(3,0.5) (3,0.5)(3,4.5) \rput(23,9.5){}
        \newrgbcolor{userLineColour}{0.6 0.4 0}
        \newrgbcolor{userFillColour}{0.6 0.4 0}
        \pspolygon[linecolor=userLineColour,fillcolor=userFillColour,fillstyle=solid](12,4.5)(12.5,4.5)
        (12.5,4.5)(12.5,0.5) (12.5,0.5)(12,0.5) (12,0.5)(12,4.5)
        \pspolygon[fillstyle=solid](9,4.5)(9.5,4.5) (9.5,4.5)(9.5,0.5)
        (9.5,0.5)(9,0.5) (9,0.5)(9,4.5)
        \pspolygon[linecolor=yellow,fillcolor=yellow,fillstyle=solid](7.5,4.5)(8,4.5)
        (8,4.5)(8,0.5) (8,0.5)(7.5,0.5) (7.5,0.5)(7.5,4.5)
        \pspolygon[linecolor=red,fillcolor=red,fillstyle=solid](6,4.5)(6.5,4.5)
        (6.5,4.5)(6.5,0.5) (6.5,0.5)(6,0.5) (6,0.5)(6,4.5)
        \newrgbcolor{userLineColour}{1 0.6 0}
        \newrgbcolor{userFillColour}{1 0.6 0}
        \pspolygon[linecolor=userLineColour,fillcolor=userFillColour,fillstyle=solid](4.5,4.5)(5,4.5)
        (5,4.5)(5,0.5) (5,0.5)(4.5,0.5) (4.5,0.5)(4.5,4.5)
        \end{pspicture}
        } \end{minipage} (Taronja-Vermell-Groc-Negre Marr\'{o})
\end{enumerate}
\begin{align*}
   \intertext{En el primer cas tenim:}
   \text{Blau-Gris-Groc Or}\,  &\rightarrow\,
   \begin{cases}
      \text{Resist\`{e}ncia} &:\; 68\cdot10^4\unit{\ohm} = 680\unit{k\ohm} \\
      \text{Toler\`{a}ncia}  &:\; \pm5\unit{\%}
   \end{cases} \\
   \intertext{i en el segon:}
   \text{Taronja-Vermell-Groc-Negre Marr\'{o}}\,  &\rightarrow\,
   \begin{cases}
      \text{Resist\`{e}ncia} &:\; 324\cdot1\unit{\ohm} = 324\unit{\ohm} \\
      \text{Toler\`{a}ncia}  &:\; \pm1\unit{\%}
   \end{cases}
\end{align*}
\end{exemple}

\section{Valors est\`{a}ndard} \index{resist\`{e}ncies!valors est\`{a}ndard}

Els valors de les resist\`{e}ncies que es fabriquen estan estandarditzats, i depenent
del valor de la toler\`{a}ncia, el ventall de valors possibles \'{e}s m\'{e}s o menys ample.

En les taules seg\"{u}ents es llisten els valors est\`{a}ndard de les resist\`{e}ncies, segons la seva
toler\`{a}ncia.

Els valors que es donen estan normalitzats per a la d\`{e}cada compresa
entre 100\unit{\ohm} i 1000\unit{\ohm}. Els valors d'altres d\`{e}cades
s'obtenen simplement, multiplicant o dividint els valors de les
taules per les corresponents  pot\`{e}ncies de 10.

\begin{table}[htb]
   \caption{Valors \`{o}hmics est\`{a}ndard de les resist\`{e}ncies de toler\`{a}ncia $\pm20\unit{\%}$}
   \[ \begin{array}{cccccc}
   \toprule[1pt]
   100 & 150 & 220 & 330 & 470 &  680  \\
   \bottomrule[1pt]
   \end{array}   \]
\end{table}

\begin{table}[htb]
   \caption{Valors \`{o}hmics est\`{a}ndard de les resist\`{e}ncies de toler\`{a}ncia $\pm10\unit{\%}$}
   \[ \begin{array}{cccccccccccc}
   \toprule[1pt]
   100 & 120 & 150 & 180 & 220 & 270 & 330 & 390 & 470 & 560 & 680 & 820 \\
   \bottomrule[1pt]
   \end{array}   \]
\end{table}

\begin{table}[htb]
   \caption{Valors \`{o}hmics est\`{a}ndard de les resist\`{e}ncies de toler\`{a}ncia $\pm5\unit{\%}$}
   \[ \begin{array}{cccccccccccc}
   \toprule[1pt]
   100 & 110 & 120 & 130 & 150 & 160 & 180 & 200 & 220 & 240 & 270 & 300 \\
   330 & 360 & 390 & 430 & 470 & 510 & 560 & 620 & 680 & 750 & 820 & 910 \\
   \bottomrule[1pt]
   \end{array}   \]
\end{table}

\begin{table}[htb]
   \caption{Valors \`{o}hmics est\`{a}ndard de les resist\`{e}ncies de toler\`{a}ncia $\pm2\unit{\%}$}
   \[ \begin{array}{cccccccccccc}
   \toprule[1pt]
   100 & 105 & 110 & 115 & 121 & 127 & 133 & 140 & 147 & 154 & 162 & 169 \\
   178 & 187 & 196 & 205 & 215 & 226 & 237 & 249 & 261 & 274 & 287 & 301 \\
   316 & 332 & 348 & 365 & 383 & 402 & 422 & 442 & 464 & 487 & 511 & 536 \\
   562 & 590 & 619 & 649 & 681 & 715 & 750 & 787 & 825 & 866 & 909 & 953 \\
   \bottomrule[1pt]
   \end{array}   \]
\end{table}

\begin{table}[htb]
   \caption{Valors \`{o}hmics est\`{a}ndard de les resist\`{e}ncies de toler\`{a}ncia $\pm1\unit{\%}$}
   \[ \begin{array}{cccccccccccc}
   \toprule[1pt]
   100 & 102 & 105 & 107 & 110 & 113 & 115 & 118 & 121 & 124 & 127 & 130 \\
   133 & 137 & 140 & 143 & 147 & 150 & 154 & 158 & 162 & 165 & 169 & 174 \\
   178 & 182 & 187 & 191 & 196 & 200 & 205 & 210 & 215 & 221 & 226 & 232 \\
   237 & 243 & 249 & 255 & 261 & 267 & 274 & 280 & 287 & 294 & 301 & 309 \\
   316 & 324 & 332 & 340 & 348 & 357 & 365 & 374 & 383 & 392 & 402 & 412 \\
   422 & 432 & 442 & 453 & 464 & 475 & 487 & 499 & 511 & 523 & 536 & 549 \\
   562 & 576 & 590 & 604 & 619 & 634 & 649 & 665 & 681 & 698 & 715 & 732 \\
   750 & 768 & 787 & 806 & 825 & 845 & 866 & 887 & 909 & 931 & 953 & 976 \\
   \bottomrule[1pt]
   \end{array}   \]
\end{table}


\begin{table}[htb]
   \caption{Valors \`{o}hmics est\`{a}ndard de les resist\`{e}ncies de toler\`{a}ncia $\leq\pm0{,}5\unit{\%}$}
   \[ \begin{array}{cccccccccccc}
   \toprule[1pt]
   100 & 101 & 102 & 104 & 105 & 106 & 107 & 109 & 110 & 111 & 113 & 114 \\
   115 & 117 & 118 & 120 & 121 & 123 & 124 & 126 & 127 & 129 & 130 & 132 \\
   133 & 135 & 137 & 138 & 140 & 142 & 143 & 145 & 147 & 149 & 150 & 152 \\
   154 & 156 & 158 & 160 & 162 & 164 & 165 & 167 & 169 & 172 & 174 & 176 \\
   178 & 180 & 182 & 184 & 187 & 189 & 191 & 193 & 196 & 198 & 200 & 203 \\
   205 & 208 & 210 & 213 & 215 & 218 & 221 & 223 & 226 & 229 & 232 & 234 \\
   237 & 240 & 243 & 246 & 249 & 252 & 255 & 258 & 261 & 264 & 267 & 271 \\
   274 & 277 & 280 & 284 & 287 & 291 & 294 & 298 & 301 & 305 & 309 & 312 \\
   316 & 320 & 324 & 328 & 332 & 336 & 340 & 344 & 348 & 352 & 357 & 361 \\
   365 & 370 & 374 & 379 & 383 & 388 & 392 & 397 & 402 & 407 & 412 & 417 \\
   422 & 427 & 432 & 437 & 442 & 448 & 453 & 459 & 464 & 470 & 475 & 481 \\
   487 & 493 & 499 & 505 & 511 & 517 & 523 & 530 & 536 & 542 & 549 & 556 \\
   562 & 569 & 576 & 583 & 590 & 597 & 604 & 612 & 619 & 626 & 634 & 642 \\
   649 & 657 & 665 & 673 & 681 & 690 & 698 & 706 & 715 & 723 & 732 & 741 \\
   750 & 759 & 768 & 777 & 787 & 796 & 806 & 816 & 825 & 835 & 845 & 856 \\
   866 & 876 & 887 & 898 & 909 & 920 & 931 & 942 & 953 & 965 & 976 & 988 \\

   \bottomrule[1pt]
   \end{array}   \]
\end{table}
