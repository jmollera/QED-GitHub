\chapter{Escales Logar\'{\i}tmiques} \index{escales logar\'{\i}tmiques}

En diferents camps de l'electrot\`{e}cnia, \'{e}s usual trobar-se gr\`{a}fics amb escales
logar\'{\i}tmiques.

Un exemple clar, s\'{o}n els gr\`{a}fics d'actuaci\'{o} dels interruptors magnetot\`{e}rmics o dels
fusibles, on les seves corbes caracter\'{\i}stiques intensitat--temps estan representades en
una escala logar\'{\i}tmica--logar\'{\i}tmica o lineal--logar\'{\i}tmica.

En aquests casos, es presenta freq\"{u}entment la necessitat de determinar amb exactitud un
punt de la corba, que no coincideix amb cap de les l\'{\i}nies divis\`{o}ries del gr\`{a}fic. Atenent a
la Figura \vref{pic:escala log}, es tractaria de determinar el valor $x$ dins de la d\`{e}cada
$10^N$ a $10^{N+1}$.

\begin{figure}[htb]
\vspace{5mm} \centering \PSforPDF{
    %Created by jPicEdt 1.x
    %PsTricks format (pstricks.sty needed)
    %Sat Sep 11 16:54:36 CEST 2004
    \psset{xunit=1mm,yunit=1mm,runit=1mm}
    \begin{pspicture}(0,0)(150.00,40.00)
    \psline[linewidth=0.25,linecolor=black]{-}(0.00,29.00)(150.00,29.00)
    \psline[linewidth=0.25,linecolor=black]{-}(5.00,34.00)(5.00,24.00)
    \psline[linewidth=0.25,linecolor=black]{-}(47.00,34.00)(47.00,24.00)
    \psline[linewidth=0.25,linecolor=black]{-}(72.00,34.00)(72.00,24.00)
    \psline[linewidth=0.25,linecolor=black]{-}(89.00,34.00)(89.00,24.00)
    \psline[linewidth=0.25,linecolor=black]{-}(103.00,34.00)(103.00,24.00)
    \psline[linewidth=0.25,linecolor=black]{-}(113.00,34.00)(113.00,24.00)
    \psline[linewidth=0.25,linecolor=black]{-}(123.00,34.00)(123.00,24.00)
    \psline[linewidth=0.25,linecolor=black]{-}(131.00,34.00)(131.00,24.00)
    \psline[linewidth=0.25,linecolor=black]{-}(138.00,34.00)(138.00,24.00)
    \psline[linewidth=0.25,linecolor=black]{-}(145.00,34.00)(145.00,24.00)
    \pscircle[linewidth=0.25,linecolor=black,fillcolor=black,fillstyle=solid](33.00,29.00){1.00}
    \rput[b](33.00,36.00){$x$} \rput[b](145.00,36.00){$10^{N+1}$}
    \rput[b](5.00,36.00){$10^N$}
    \psline[linewidth=0.25,linecolor=black,linestyle=dashed,dash=1.00
    1.00]{-}(5.00,22.85)(5.00,0.00)
    \psline[linewidth=0.25,linecolor=black,linestyle=dashed,dash=1.00
    1.00]{-}(33.00,27.00)(33.00,11.42)
    \psline[linewidth=0.25,linecolor=black,linestyle=dashed,dash=1.00
    1.00]{-}(145.00,22.85)(145.00,0.00)
    \psline[linewidth=0.25,linecolor=black]{<->}(5.00,2.50)(145.00,2.50)
    \rput[b](19.00,15.00){$D$} \rput[b](77.00,4.00){$L$}
    \psline[linewidth=0.25,linecolor=black]{<->}(5.00,13.00)(33.00,13.00)
    \psline[linewidth=0.25,linecolor=black]{->}(33.00,34.00)(33.00,31.00)
\end{pspicture}
}
\caption{Escala logar\'{\i}tmica} \label{pic:escala log}
\end{figure}

Si mesurem amb un regle la dist\`{a}ncia $D$ des de l'inici de la d\`{e}cada fins al punt $x$, i
la longitud total $L$ de la d\`{e}cada, el valor $x$ buscat ve donat per l'expressi\'{o}:
\begin{equation}
    x = 10^{\left(N+\frac{D}{L}\right)}
\end{equation}

Si estem interessats en el problema invers, \'{e}s a dir, en  trobar la dist\`{a}ncia $D$
corresponent a un valor conegut $x$ dins de la d\`{e}cada $10^N$ a $10^{N+1}$, podem emprar
l'expressi\'{o}:
\begin{equation}
    D = L(\log x - N) = L \log\frac{x}{10^N}
\end{equation}

\begin{exemple}
Es tracta de trobar el valor $x$ dins de la d\`{e}cada 100 a 1000, corresponent a una
dist\`{a}ncia $D=11\unit{mm}$; la longitud total de la d\`{e}cada \'{e}s $L=56\unit{mm}$.

En aquest cas tenim $N=2$, i per tant:
\[
    x = 10^{\left(2+\frac{11\unit{mm}}{56\unit{mm}}\right)}= 157{,}19
\]
\end{exemple}

\begin{exemple}
Es tracta de trobar la dist\`{a}ncia $D$ a la qual hem de dibuixar el valor $x=5$, dins de la
d\`{e}cada 1 a 10; la longitud total de la d\`{e}cada \'{e}s $L=56\unit{mm}$.

En aquest cas tenim $N=0$, i per tant:
\[
    D = 56\unit{mm} \cdot (\log 5 - 0)  = 39{,}1\unit{mm}
\]
\end{exemple}
