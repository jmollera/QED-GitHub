\chapter{C\`{a}lcul Num\`{e}ric} \index{c\`{a}lcul num\`{e}ric}

\section{Interpolaci\'{o} mitjan\c{c}ant polinomis de Lagrange}\index{interpolaci\'{o}}\index{polinomis de Lagrange}

Es descriuen en aquesta secci\'{o} els polinomis de Lagrange, i el seu \'{u}s en la interpolaci\'{o} de dades.

Un polinomi d'interpolaci\'{o} de Lagrange $P(x)$, \'{e}s un polinomi de grau $n-1$, que passa exactament per $n$ punts:
$(x_1, y_1), (x_2, y_2), \dots, (x_n, y_n)$. Aquest polinomi ve donat per l'expressi\'{o}:
\begin{equation}
  P(x) = \sum_{i=1}^{n}  y_i L_i(x) \label{eq:poly_lag_1}
\end{equation}

On $L_i(x)$ s\'{o}n les anomenades funcions de Lagrange, calculades segons l'expressi\'{o}:
\begin{equation}
  L_i(x) = \prod_{\substack{k=1 \\ k\neq i}}^{n} \frac{x-x_k}{x_i-x_k} \label{eq:poly_lag_2}
\end{equation}

Es donen a continuaci\'{o} les f\'{o}rmules expl\'{\i}cites per a $n = 2$, $n=3$ i $n=4$:

\begin{dinglist}{'167}
    \item \textbf{Interpolaci\'{o} lineal} $\boldsymbol{(n=2)}$ \index{interpolaci\'{o}!lineal}

    \begin{equation}\label{eq:interp_lin}
      P(x) = \frac{x-x_2}{x_1-x_2}\, y_1 + \frac{x-x_1}{x_2-x_1}\, y_2 = y_1 + \frac{x-x_1}{x_2-x_1} (y_2-y_1)
    \end{equation}

    \item \textbf{Interpolaci\'{o} quadr\`{a}tica} $\boldsymbol{(n=3)}$ \index{interpolaci\'{o}!quadr\`{a}tica}

    \begin{equation}
      P(x) = \frac{(x-x_2)(x-x_3)}{(x_1-x_2)(x_1-x_3)}\, y_1 + \frac{(x-x_1)(x-x_3)}{(x_2-x_1)(x_2-x_3)}\, y_2 +
      \frac{(x-x_1)(x-x_2)}{(x_3-x_1)(x_3-x_2)}\, y_3
    \end{equation}

    \item \textbf{Interpolaci\'{o} c\'{u}bica} $\boldsymbol{(n=4)}$ \index{interpolaci\'{o}!c\'{u}bica}

    \begin{equation}\begin{split}\label{eq:interp_cub}
      P(x) &= \frac{(x-x_2)(x-x_3)(x-x_4)}{(x_1-x_2)(x_1-x_3)(x_1-x_4)}\, y_1 +
              \frac{(x-x_1)(x-x_3)(x-x_4)}{(x_2-x_1)(x_2-x_3)(x_2-x_4)}\, y_2 + {} \\[1.5ex]
           &+ \frac{(x-x_1)(x-x_2)(x-x_4)}{(x_3-x_1)(x_3-x_2)(x_3-x_4)}\, y_3 +
              \frac{(x-x_1)(x-x_2)(x-x_3)}{(x_4-x_1)(x_4-x_2)(x_4-x_3)}\, y_4
    \end{split}\end{equation}
\end{dinglist}


L'error en la interpolaci\'{o} dependr\`{a} molt del tipus de dades que tinguem i del grau del polinomi que utilitzem. Si els punts que volem interpolar estan molt junts o si la seva gr\`{a}fica \'{e}s suau, n'hi haur\`{a} prou amb una interpolaci\'{o} lineal. D'altra banda, si els punts estan molt separats, o la seva gr\`{a}fica dista molt de ser lineal, ser\`{a} millor emprar polinomis de grau superior.


\begin{exemple}[Exemple d'interpolaci\'{o} lineal i c\'{u}bica]
    En la taula seg\"{u}ent hi ha quatre punts de la funci\'{o} $y = \sin x$, al voltant de $x=\frac{\pi}{2}$. Es tracta de trobar per interpolaci\'{o} lineal i c\'{u}bica, el valor de $\sin \frac{\pi}{2}$.
    \vspace{-8mm}
    \begin{center}
        \[\begin{array}{ccc}
           \toprule[1pt]
              \text{punt} & x  & y \\
           \midrule
              1 & \num{1,2} & \num{0,9320} \\
              2 & \num{1,4} & \num{0,9854} \\
              3 & \num{1,6} & \num{0,9996} \\
              4 & \num{1,8} & \num{0,9738} \\
           \bottomrule[1pt]
        \end{array} \]
    \end{center}

    Fem primer la interpolaci\'{o} lineal a $x= \frac{\pi}{2}$, utilitzant l'equaci\'{o} \eqref{eq:interp_lin} amb els punts 2 i 3; fent els c\`{a}lculs obtenim:
    \[ P\left(\tfrac{\pi}{2}\right) = \num{0,9975} \]

    A continuaci\'{o} fem la interpolaci\'{o} c\'{u}bica a $x= \frac{\pi}{2}$, utilitzant l'equaci\'{o} \eqref{eq:interp_cub} amb els punts 1, 2, 3 i 4; fent els c\`{a}lculs obtenim:
    \[ P\left(\tfrac{\pi}{2}\right) = \num{1,0000} \]

    Com era d'esperar, la interpolaci\'{o} c\'{u}bica d\'{o}na un valor m\'{e}s exacte.
\end{exemple}


\section{Integraci\'{o}}\index{integraci\'{o}}

Es descriuen en aquesta secci\'{o} diversos m\`{e}todes d'integraci\'{o} num\`{e}rica de funcions, ja sigui coneixent nom\'{e}s una s\`{e}rie de punts de la funci\'{o}: $(x_1, y_1), (x_2, y_2), \dots ,(x_n, y_n)$, o ja sigui coneixent-ne l'expressi\'{o} expl\'{\i}cita: $y=f(x)$. La integraci\'{o} de la funci\'{o} entre $x_1$ i $x_n$, ens donar\`{a} l'area $A$ existent entre la funci\'{o} i l'eix d'abscises.

 \begin{equation}
    A = \int_{x_1}^{x_n} f(x) \diff x
 \end{equation}

\subsection{Regla dels trapezis}\index{integraci\'{o}!regla dels trapezis}

Aquest m\`{e}tode serveix per a qualsevol nombre de punts $n$ de la funci\'{o} que es vol integrar, amb $n \geq 2$.

L'area $A$ entre el punt inicial $(x_1, y_1)$ i el punt final $(x_n, y_n)$, ve donada per l'equaci\'{o}:

 \begin{equation}
    A = \sum_{i=1}^{n-1} \frac{y_i + y_{i+1}}{2} (x_{i+1}-x_i)
 \end{equation}

En el cas que els $n$ punts estiguin separats uniformement una dist\`{a}ncia $h = (x_2-x_1) = (x_3-x_2) = \dots = (x_n-x_{n-1})$, la regla dels trapezis esdev\'{e}:

 \begin{equation}\label{eq:trap}
    A = \frac{h}{2} \left( y_1 + y_n + 2 \sum_{i=2}^{n-1} y_i \right)
 \end{equation}

Si coneixem l'expressi\'{o} expl\'{\i}cita $y=f(x)$ de la funci\'{o} que volem integrar, entre els punts $x=a$ i $x=b$, poden fixar el nombre punts $n$ que volen utilitzar, i llavors el valor $h$ queda definit per l'equaci\'{o}:
\begin{equation}\label{eq:trap_1}
    h = \frac{b-a}{n-1}\\
\end{equation}

Aix\'{\i} mateix, Les abscises $x_1, x_2, x_3,x_4, \dotsc , x_n$ que haurem d'utilitzar, s\'{o}n:
\begin{align}\label{eq:trap_2}
    x_1 &= a \notag\\
    x_2 &= a + h \notag\\
    x_3 &= a + 2 h \notag\\
    x_4 &= a + 3 h\\
    {} &\hspace{1.5ex}\vdots {} \notag\\
    x_n &= a + (n-1)h =b \notag
\end{align}

L'odre de l'error d'integraci\'{o} de la regla dels trapezis, \'{e}s: $0(h^2)$. \'{E}s a dir, una divisi\'{o} de $h$ per 2, ocasiona una divisi\'{o} per 4 de l'error.

\subsection{Regla de Simpson 1/3}\index{integraci\'{o}!regla de Simpson 1/3}

Aquest m\`{e}tode serveix per a un nombre senar de punts $n$ de la funci\'{o} que es vol integrar, amb $n \geq 3$. A m\'{e}s, els punts han d'estar separats uniformement una dist\`{a}ncia $h = (x_2-x_1) = (x_3-x_2) = \dots = (x_n-x_{n-1})$.

L'area $A$ entre el punt inicial $(x_1, y_1)$ i el punt final $(x_n, y_n)$, ve donada per l'equaci\'{o}:

 \begin{equation}\label{eq:simp_1_3}
   A =  \frac{h}{3} \sum_{i=1,3,5\dots}^{n-2} \hspace{-0.5em} \big( y_i + 4 y_{i+1} + y_{i+2} \big) =
   \frac{h}{3} \left( y_1 + y_n + 4 \hspace{-0.5em} \sum_{i=2,4,6\dots}^{n-1} \hspace{-0.5em} y_i +
   2 \hspace{-0.5em} \sum_{j=3,5,7\dots}^{n-2} \hspace{-0.5em} y_j \right)
 \end{equation}

Les equacions \eqref{eq:trap_1} i \eqref{eq:trap_2}, descrites en la regla dels trapezis, tamb\'{e} es poden utilitzar en aquest cas, quan  coneixem l'expressi\'{o} expl\'{\i}cita $y=f(x)$ de la funci\'{o} que volem integrar, entre els punts $x=a$ i $x=b$, i volem fixar el nombre de punts $n$.

L'odre de l'error d'integraci\'{o} de la regla de Simpson 1/3, \'{e}s: $0(h^4)$. \'{E}s a dir, una divisi\'{o} de $h$ per 2, ocasiona una divisi\'{o} per 16 de l'error.

 \subsection{Regla de Simpson 3/8}\index{integraci\'{o}!regla de Simpson 3/8}

Aquest m\`{e}tode serveix per a un nombre parell de punts $n$ de la funci\'{o} que es vol integrar, amb $n \geq 4$. A m\'{e}s, els punts han d'estar separats uniformement una dist\`{a}ncia $h = (x_2-x_1) = (x_3-x_2) = \dots = (x_n-x_{n-1})$.

L'area $A$ entre el punt inicial $(x_1, y_1)$ i el punt final $(x_n, y_n)$, ve donada per l'equaci\'{o}:

 \begin{equation}\label{eq:simp_3_8}
   A =  \frac{3h}{8} \sum_{i=1}^{n} \big( y_i + 3 y_{i+1} + 3 y_{i+2} + y_{i+3}\big)
 \end{equation}

 Les equacions \eqref{eq:trap_1} i \eqref{eq:trap_2}, descrites en la regla dels trapezis, tamb\'{e} es poden utilitzar en aquest cas, quan  coneixem l'expressi\'{o} expl\'{\i}cita $y=f(x)$ de la funci\'{o} que volem integrar, entre els punts $x=a$ i $x=b$, i volem fixar el nombre de punts $n$.

 L'odre de l'error d'integraci\'{o} de la regla de Simpson 3/8, \'{e}s: $0(h^4)$. En ser del mateix ordre que el de la regla de Simpson 1/3, la regla de Simpson 3/8 nom\'{e}s s'usa quan $n$ \'{e}s parell, per avaluar la integral per a $x_1$, $x_2$, $x_3$ i $x_4$, utilitzant la regla de Simpson 1/3 per avaluar la resta de la integral per a  $x_4, x_5, \dotsc , x_n$.

\begin{exemple}[Integraci\'{o} num\`{e}rica d'una funci\'{o}]
    Es tracta de calcular num\`{e}ricament la integral: $\int_1^2 \frac{1}{x} \diff x$, utilitzant les regles del trapezis i de Simpson.

    Calculem primer el valor exacte d'aquesta integral, per tal de poder-lo comparar amb els que obtindrem  num\`{e}ricament:
    \[
      \int_1^2 \frac{1}{x} \diff x = \ln 2 = \num{0,6931}
    \]

    Si utilitzem sis punts ($n=6$), tindrem a partir de l'equaci\'{o} \eqref{eq:trap_1}, una dist\`{a}ncia $h$ entre punts de:
    \[
        h = \frac{2-1}{6-1} = \num{0,2}
    \]

    Els valors $x_i$ i $y_i$ que utilitzarem, s\'{o}n:
    \vspace{-8mm}
    \begin{center}
        \[\begin{array}{ccc}
           \toprule[1pt]
              i & x_i  & y_i = 1/x_i \\
           \midrule
              1 & \num{1,0} & \num{1,0000} \\
              2 & \num{1,2} & \num{0,8333} \\
              3 & \num{1,4} & \num{0,7143} \\
              4 & \num{1,6} & \num{0.6250} \\
              5 & \num{1,8} & \num{0.5556} \\
              6 & \num{2,0} & \num{0,5000} \\
           \bottomrule[1pt]
        \end{array} \]
    \end{center}

    Utilitzant la regla dels trapezis, equaci\'{o} \eqref{eq:trap},tenim:
    \[
        \int_1^2 \frac{1}{x} \diff x \approx \frac{\num{0,2}}{2} \big(\num{1,0000}+\num{0,5000} +2 \times(\num{0,8333}+ \num{0,7143} +
        \num{0,6250} + \num{0,5556}) \big) = \num{0,6956}
    \]

    Utilitzant la regla de Simpson 3/8 entre $x_1$ i $x_4$, equaci\'{o} \eqref{eq:simp_3_8}, i la  regla de Simpson 1/3 entre $x_4$ i $x_6$, equaci\'{o} \eqref{eq:simp_1_3}, tenim:
    \[\begin{split}
        \int_1^2 \frac{1}{x} \diff x &\approx \frac{3\times\num{0,2}}{8} \big(\num{1,0000}+3\times\num{0,8333} +3\times \num{0,7143} +
        \num{0,6250} \big) +{}\\[0.5em]
        {}&+ \frac{\num{0,2}}{3} \big(\num{0,6250} +4\times \num{0,5556} + \num{0,5000} \big)
        = \num{0,6932}
    \end{split}\]

    Com era d'esperar, l'aplicaci\'{o} conjunta de les regles de Simpson 3/8 i 1/3, d\'{o}na un resultat m\'{e}s prec\'{\i}s que la regla dels trapezis.
\end{exemple}


\section{Soluci\'{o} de funcions no lineals}\index{funcions no lineals!soluci\'{o}}\label{sec:func-no-lin}

Es descriuen en aquesta secci\'{o} dos m\`{e}todes de resoluci\'{o} de funcions no lineals, \'{e}s a dir, es vol resoldre l'equaci\'{o}: \begin{equation}
   f(x) = 0 \end{equation}

Els m\`{e}todes descrits s\'{o}n el de Newton i el de la recta secant. Ambd\'{o}s m\`{e}todes s\'{o}n iteratius i tenen una converg\`{e}ncia cap a la soluci\'{o} for\c{c}a r\`{a}pida. El m\`{e}tode de Newton requereix un valor inicial aproximat de la soluci\'{o}, i el m\`{e}tode de la recta secant en requereix dos.

El millor m\`{e}tode per obtenir valors inicials aproximats de la soluci\'{o}, \'{e}s dibuixar la funci\'{o} i localitzar-ne visualment el punt on talla l'eix d'abscises. Si els valors inicials aproximats utilitzats per iniciar la iteraci\'{o} s\'{o}n massa lluny de la soluci\'{o} real, pot ser que el proc\'{e}s iteratiu no convergeixi.


\subsection{M\`{e}tode de Newton}\index{funcions no lineals!m\`{e}tode de Newton}

Aquest m\`{e}tode, que requereix el c\`{a}lcul de la funci\'{o} derivada $f'(x)$, s'i{\l.l}ustra en la Figura \vref{pic:metode-newton}.

\begin{center}
    \input{Imatges/Ape-Calcul-Numeric-Newton.pdf_tex}
    \captionof{figure}{M\`{e}tode de Newton}
    \label{pic:metode-newton}
\end{center}


A partir d'un punt $x_{i-1}$, es calcula $f(x_{i-1})$ i es tra\c{c}a la recta tangent a la funci\'{o} $f(x)$ en aquest punt, per a la qual cosa cal con\`{e}ixer la funci\'{o} derivada $f'(x)$. A continuaci\'{o} es calcula la intersecci\'{o} d'aquesta recta amb l'eix abscises, obtenint el nou punt $x_i$. El proc\'{e}s es repeteix calculant $f(x_i)$, tra\c{c}ant una nova recta tangent en aquest punt, i trobant la nova intersecci\'{o} amb l'eix abscises; seguint aquest proc\'{e}s ens aproparem cada vegada m\'{e}s a la soluci\'{o} $x_n$, on es compleix $f(x_n)=0$.


El proc\'{e}s iteratiu \'{e}s el seg\"{u}ent:

\begin{dingautolist}{'312}
    \item Es parteix d'un valor aproximat de la soluci\'{o}: $x_0$.

    \item   S'aplica de manera successiva l'equaci\'{o}:
            \begin{equation}\label{eq:newton}
              x_i = x_{i-1} - \frac{f(x_{i-1})}{f'(x_{i-1})} \qquad\qquad (i=1,2,3,\dots,\infty)
            \end{equation}

    \item   El proc\'{e}s s'atura quan es compleix una de les condicions seg\"{u}ents, o quan es compleixen ambdues condicions alhora:
            \begin{subequations}\begin{align}
              |x_i - x_{i-1}| &\leq \varepsilon_1 \\
              |f(x_i)| &\leq \varepsilon_2
            \end{align}\end{subequations}

            On $\varepsilon_1$ i $\varepsilon_2$ s\'{o}n dos valors petits, que es fixen en funci\'{o} de la precisi\'{o} que es vulgui assolir en la soluci\'{o}.
\end{dingautolist}



\subsection{M\`{e}tode de la recta secant}\index{funcions no lineals!m\`{e}tode la recta secant}

Aquest m\`{e}tode s'i{\l.l}ustra en la Figura \vref{pic:metode-secant}. S'utilitza enlloc del m\`{e}tode de Newton quan el c\`{a}lcul de de la funci\'{o} derivada $f'(x)$ \'{e}s molt complex, o quan no \'{e}s possible fer-ho anal\'{\i}ticament; en contrapartida, la converg\`{e}ncia cap a la soluci\'{o} real \'{e}s una
mica m\'{e}s lenta.


\begin{center}
    \input{Imatges/Ape-Calcul-Numeric-secant.pdf_tex}
    \captionof{figure}{M\`{e}tode de la recta secant}
    \label{pic:metode-secant}
\end{center}


A partir de dos punts $x_{i-2}$ i $x_{i-1}$, es calcula $f(x_{i-2})$ i $f(x_{i-1})$ i es tra\c{c}a la recta secant a la funci\'{o} $f(x)$ que passa per aquests dos punts. A continuaci\'{o} es calcula la intersecci\'{o} d'aquesta recta amb l'eix abscises, obtenint el nou punt $x_i$. El proc\'{e}s es repeteix calculant $f(x_i)$, tra\c{c}ant una nova recta secant que passi per aquest punt i l'anterior, i trobant la nova intersecci\'{o} amb l'eix abscises; seguint aquest proc\'{e}s ens aproparem cada vegada m\'{e}s a la soluci\'{o} $x_n$, on es compleix $f(x_n)=0$.

El proc\'{e}s iteratiu \'{e}s el seg\"{u}ent:

\begin{dingautolist}{'312}
    \item Es parteix de dos valors aproximats de la soluci\'{o}: $x_0$ i $x_1$.

    \item   S'aplica de manera successiva l'equaci\'{o}:
            \begin{equation}\label{eq:secant-1}
              x_i = x_{i-1} - \frac{f(x_{i-1})}{g'(x_{i-1})} \qquad\qquad (i=2,3,4,\dots,\infty)
            \end{equation}

            On:
            \begin{equation}\label{eq:secant-2}
              g'(x_{i-1}) = \frac{f(x_{i-1}) - f(x_{i-2}) } {x_{i-1} - x_{i-2}} \qquad (i=2,3,4,\dots,\infty)
            \end{equation}

    \item   El proc\'{e}s s'atura quan es compleix una de les condicions seg\"{u}ents, o quan es compleixen ambdues condicions alhora:
            \begin{subequations}\begin{align}
              |x_i - x_{i-1}| &\leq \varepsilon_1 \\
              |f(x_i)| &\leq \varepsilon_2
            \end{align}\end{subequations}

            On $\varepsilon_1$ i $\varepsilon_2$ s\'{o}n dos valors petits, que es fixen en funci\'{o} de la precisi\'{o} que es vulgui assolir en la soluci\'{o}.
\end{dingautolist}

\begin{exemple}[Soluci\'{o} d'una funci\'{o} no lineal]
    Utilitzant la funci\'{o} $i(t)$ obtinguda en exemple \vref{ex:cc-RL}, es tracte de calcular el punt proper a $t = \SI{20}{ms}$, on es compleix $i(t)=0$. Per tal d'adoptar la nomenclatura d'aquesta secci\'{o}, substitu\"{\i}m $i(t)$ i $t$, per $f(x)$ i $x$ respectivament; l'equaci\'{o} que volem resoldre \'{e}s doncs:
    \[
        f(x) = \num{35953,6865}\sin(\num{314,1593}\,x-\num{1,5136}) + \num{35894,8169}\,\eu^{-18 x} = 0
    \]

    Per tal de poder utilitzar el m\`{e}tode de Newton, comencem calculant la funci\'{o} derivada:
    \begin{align*}
        f'(x) &= \num{314,1593}\times\num{35953,6865}\cos(\num{314,1593}\,x-\num{1,5136}) -18\times \num{35894,8169}\,\eu^{-18 x}= \\
        {} &= \num{11295184,9833}\cos(\num{314,1593}\,x-\num{1,5136}) - \num{646106,7042}\,\eu^{-18 x}
    \end{align*}

    Observant la gr\`{a}fica de l'exemple \vref{ex:cc-RL}, prenem com a  aproximaci\'{o} inicial de la soluci\'{o}, el valor: $x_0 = \num{0,015}$.

    A continuaci\'{o}, utilitzant l'equaci\'{o} \eqref{eq:newton}, creem la taula seg\"{u}ent:

    \[ \begin{array}{ccccc}
   \toprule[1pt]
        i & x_i  & |x_i - x_{i-1}| & f(x_i) & f'(x_i) \\
   \midrule
        0 &	\num{0,01500000000} &	{}             & \num{2,53461E+04}	& \num{-1,17699E+07} \\
        1 &	\num{0,01715345649} &	\num{2,15E-03} & \num{2,28349E+03}	& \num{-8,86324E+06} \\	
        2 &	\num{0,01741109271} &	\num{2,58E-04} & \num{8,14612E+01}	& \num{-8,22205E+06} \\	
        3 &	\num{0,01742100037} &	\num{9,91E-06} & \num{1,27243E-01}	& \num{-8,19635E+06} \\	
        4 &	\num{0,01742101589} &	\num{1,55E-08} & \num{3,13004E-07}	& \num{-8,19631E+06} \\	
        5 &	\num{0,01742101589} &	0              & 0               	& \num{-8,19631E+06} \\	
   \bottomrule[1pt]
   \end{array}   \]

Despr\'{e}s de cinc iteracions trobem la soluci\'{o} buscada: $x=\num{0,01742101589}$

Fem ara el mateix c\`{a}lcul utilitzant el m\`{e}tode de la recta secant.

Prenem com a  aproximacions inicials de la soluci\'{o}, els valors: $x_0 = \num{0,015}$ i $x_1 = \num{0,016}$.

 A continuaci\'{o}, utilitzant les equacions \eqref{eq:secant-1} i \eqref{eq:secant-2}, creem la taula seg\"{u}ent:


    \[ \begin{array}{ccccc}
   \toprule[1pt]
        i & x_i  & |x_i - x_{i-1}| & f(x_i) & g'(x_i) \\
   \midrule
       0	&  \num{0,01500000000} & {}             & \num{2,53461E+04} & {}                   \\
       1	&  \num{0,01600000000} & \num{1,00E-03} & \num{1,38657E+04} & \num{-1,1480350E+07} \\
       2	&  \num{0,01720777675} & \num{1,21E-03} & \num{1,80558E+03} & \num{-9,9853905E+06} \\
       3	&  \num{0,01738859870} & \num{1,81E-04} & \num{2,67062E+02} & \num{-8,5084579E+06} \\
       4	&  \num{0,01741998651} & \num{3,14E-05} & \num{8,43851E+00} & \num{-8,2396113E+06} \\
       5	&  \num{0,01742101065} & \num{1,02E-06} & \num{4,29711E-02} & \num{-8,1976530E+06} \\
       6	&  \num{0,01742101589} & \num{5,24E-09} & \num{7,00743E-06} & \num{-8,1963162E+06} \\
       7	&  \num{0,01742101589} & 0              & 0                 & \num{-8,1963238E+06} \\
   \bottomrule[1pt]
   \end{array}   \]

Despr\'{e}s de set iteracions trobem la soluci\'{o} buscada: $x=\num{0,01742101589}$


\end{exemple} 