\chapter{C\`{a}lcul Num\`{e}ric} \index{c\`{a}lcul num\`{e}ric}

\section{Interpolaci\'{o} mitjan\c{c}ant polinomis de Lagrange}\index{interpolaci\'{o}}\index{polinomis de Lagrange}

Es descriuen en aquesta secci\'{o} els polinomis de Lagrange, i el seu \'{u}s en la interpolaci\'{o} de dades.

Un polinomi d'interpolaci\'{o} de Lagrange $P(x)$, \'{e}s un polinomi de grau $n-1$, que passa exactament per $n$ punts:
$(x_1, y_1), (x_2, y_2), \dots, (x_n, y_n)$. Aquest polinomi ve donat per l'expressi\'{o}:
\begin{equation}
  P(x) = \sum_{i=1}^{n} P_i(x)
\end{equation}

On:
\begin{equation}
  P_i(x) = \prod_{\substack{k=1 \\ k\neq i}}^{n} \frac{x-x_k}{x_i-x_k}\, y_i
\end{equation}

Es donen a continuaci\'{o} les f\'{o}rmules expl\'{\i}cites per a $n = 2$, $n=3$ i $n=4$:

\begin{dinglist}{'167}
    \item \textbf{Interpolaci\'{o} lineal} $\boldsymbol{(n=2)}$ \index{interpolaci\'{o}!lineal}

    \begin{equation}\label{eq:interp_lin}
      P(x) = \frac{x-x_2}{x_1-x_2}\, y_1 + \frac{x-x_1}{x_2-x_1}\, y_2 = y_1 + \frac{x-x_1}{x_2-x_1} (y_2-y_1)
    \end{equation}

    \item \textbf{Interpolaci\'{o} quadr\`{a}tica} $\boldsymbol{(n=3)}$ \index{interpolaci\'{o}!quadr\`{a}tica}

    \begin{equation}
      P(x) = \frac{(x-x_2)(x-x_3)}{(x_1-x_2)(x_1-x_3)}\, y_1 + \frac{(x-x_1)(x-x_3)}{(x_2-x_1)(x_2-x_3)}\, y_2 +
      \frac{(x-x_1)(x-x_2)}{(x_3-x_1)(x_3-x_2)}\, y_3
    \end{equation}

    \item \textbf{Interpolaci\'{o} c\'{u}bica} $\boldsymbol{(n=4)}$ \index{interpolaci\'{o}!c\'{u}bica}

    \begin{equation}\begin{split}\label{eq:interp_cub}
      P(x) &= \frac{(x-x_2)(x-x_3)(x-x_4)}{(x_1-x_2)(x_1-x_3)(x_1-x_4)}\, y_1 +
              \frac{(x-x_1)(x-x_3)(x-x_4)}{(x_2-x_1)(x_2-x_3)(x_2-x_4)}\, y_2 + {} \\[1.5ex]
           &+ \frac{(x-x_1)(x-x_2)(x-x_4)}{(x_3-x_1)(x_3-x_2)(x_3-x_4)}\, y_3 +
              \frac{(x-x_1)(x-x_2)(x-x_3)}{(x_4-x_1)(x_4-x_2)(x_4-x_3)}\, y_4
    \end{split}\end{equation}
\end{dinglist}


L'error en la interpolaci\'{o} dependr\`{a} molt del tipus de dades que tinguem i del grau del polinomi que utilitzem. Si els punts que volem interpolar estan molt junts o si la seva gr\`{a}fica \'{e}s suau, n'hi haur\`{a} prou amb una interpolaci\'{o} lineal. D'altra banda, si els punts estan molt separats, o la seva gr\`{a}fica dista molt de ser lineal, ser\`{a} millor emprar polinomis de grau superior.


\begin{exemple}[Exemple d'interpolaci\'{o} lineal i c\'{u}bica]
    En la taula seg\"{u}ent hi ha quatre punts de la funci\'{o} $y = \sin x$, al voltant de $x=\frac{\pi}{2}$. Es tracta de trobar per interpolaci\'{o} lineal i c\'{u}bica, el valor de $\sin \frac{\pi}{2}$.
    \vspace{-8mm}
    \begin{center}
        \[\begin{array}{ccc}
           \toprule[1pt]
              \text{punt} & x  & y \\
           \midrule
              1 & \num{1,2} & \num{0,9320} \\
              2 & \num{1,4} & \num{0,9854} \\
              3 & \num{1,6} & \num{0,9996} \\
              4 & \num{1,8} & \num{0,9738} \\
           \bottomrule[1pt]
        \end{array} \]
    \end{center}

    Fem primer la interpolaci\'{o} lineal a $x= \frac{\pi}{2}$, utilitzant l'equaci\'{o} \eqref{eq:interp_lin} amb els punts 2 i 3; fent els c\`{a}lculs obtenim:
    \[ P\left(\tfrac{\pi}{2}\right) = \num{0,9975} \]

    A continuaci\'{o} fem la interpolaci\'{o} c\'{u}bica a $x= \frac{\pi}{2}$, utilitzant l'equaci\'{o} \eqref{eq:interp_cub} amb els punts 1, 2, 3 i 4; fent els c\`{a}lculs obtenim:
    \[ P\left(\tfrac{\pi}{2}\right) = \num{0,9999} \]

    Com era d'esperar, la interpolaci\'{o} c\'{u}bica d\'{o}na un valor m\'{e}s exacte.
\end{exemple}


\section{Integraci\'{o}}\index{integraci\'{o}}

Es descriuen en aquesta secci\'{o} diversos m\`{e}todes d'integraci\'{o} num\`{e}rica de funcions, ja sigui coneixent nom\'{e}s una s\`{e}rie de punts de la funci\'{o}: $(x_1, y_1), (x_2, y_2), \dots ,(x_n, y_n)$, o ja sigui coneixent-ne l'expressi\'{o} expl\'{\i}cita: $y=f(x)$. La integraci\'{o} de la funci\'{o} entre $x_1$ i $x_n$, ens donar\`{a} l'area $A$ existent entre la funci\'{o} i l'eix d'abscises.

 \begin{equation}
    A = \int_{x_1}^{x_n} f(x) \diff x
 \end{equation}

\subsection{Regla dels trapezis}\index{integraci\'{o}!regla dels trapezis}

Aquest m\`{e}tode serveix per a qualsevol nombre de punts $n$ de la funci\'{o} que es vol integrar, amb $n \geq 2$.

L'area $A$ entre el punt inicial $(x_1, y_1)$ i el punt final $(x_n, y_n)$, ve donada per l'equaci\'{o}:

 \begin{equation}
    A = \sum_{i=1}^{n-1} \frac{y_i + y_{i+1}}{2} (x_{i+1}-x_i)
 \end{equation}

En el cas que els $n$ punts estiguin separats uniformement una dist\`{a}ncia $h = (x_2-x_1) = (x_3-x_2) = \dots = (x_n-x_{n-1})$, la regla dels trapezis esdev\'{e}:

 \begin{equation}\label{eq:trap}
    A = \frac{h}{2} \left( y_1 + y_n + 2 \sum_{i=2}^{n-1} y_i \right)
 \end{equation}

Si coneixem l'expressi\'{o} expl\'{\i}cita $y=f(x)$ de la funci\'{o} que volem integrar, entre els punts $x=a$ i $x=b$, poden fixar el nombre punts $n$ que volen utilitzar, i llavors el valor $h$ queda definit per l'equaci\'{o}:
\begin{equation}\label{eq:trap_1}
    h = \frac{b-a}{n-1}\\
\end{equation}

Aix\'{\i} mateix, Les abscises $x_1, x_2, x_3,x_4, \dotsc , x_n$ que haurem d'utilitzar, s\'{o}n:
\begin{align}\label{eq:trap_2}
    x_1 &= a \notag\\
    x_2 &= a + h \notag\\
    x_3 &= a + 2 h \notag\\
    x_4 &= a + 3 h\\
    {} &\hspace{1.5ex}\vdots {} \notag\\
    x_n &= a + (n-1)h =b \notag
\end{align}

L'odre de l'error d'integraci\'{o} de la regla dels trapezis, \'{e}s: $0(h^2)$. \'{E}s a dir, una divisi\'{o} de $h$ per 2, ocasiona una divisi\'{o} per 4 de l'error.

\subsection{Regla de Simpson 1/3}\index{integraci\'{o}!regla de Simpson 1/3}

Aquest m\`{e}tode serveix per a un nombre senar de punts $n$ de la funci\'{o} que es vol integrar, amb $n \geq 3$. A m\'{e}s, els punts han d'estar separats uniformement una dist\`{a}ncia $h = (x_2-x_1) = (x_3-x_2) = \dots = (x_n-x_{n-1})$.

L'area $A$ entre el punt inicial $(x_1, y_1)$ i el punt final $(x_n, y_n)$, ve donada per l'equaci\'{o}:

 \begin{equation}\label{eq:simp_1_3}
   A =  \frac{h}{3} \sum_{i=1,3,5\dots}^{n-2} \hspace{-0.5em} \big( y_i + 4 y_{i+1} + y_{i+2} \big) =
   \frac{h}{3} \left( y_1 + y_n + 4 \hspace{-0.5em} \sum_{i=2,4,6\dots}^{n-1} \hspace{-0.5em} y_i +
   2 \hspace{-0.5em} \sum_{j=3,5,7\dots}^{n-2} \hspace{-0.5em} y_j \right)
 \end{equation}

Les equacions \eqref{eq:trap_1} i \eqref{eq:trap_2}, descrites en la regla dels trapezis, tamb\'{e} es poden utilitzar en aquest cas, quan  coneixem l'expressi\'{o} expl\'{\i}cita $y=f(x)$ de la funci\'{o} que volem integrar, entre els punts $x=a$ i $x=b$, i volem fixar el nombre de punts $n$.

L'odre de l'error d'integraci\'{o} de la regla de Simpson 1/3, \'{e}s: $0(h^4)$. \'{E}s a dir, una divisi\'{o} de $h$ per 2, ocasiona una divisi\'{o} per 16 de l'error.

 \subsection{Regla de Simpson 3/8}\index{integraci\'{o}!regla de Simpson 3/8}

Aquest m\`{e}tode serveix per a un nombre parell de punts $n$ de la funci\'{o} que es vol integrar, amb $n \geq 4$. A m\'{e}s, els punts han d'estar separats uniformement una dist\`{a}ncia $h = (x_2-x_1) = (x_3-x_2) = \dots = (x_n-x_{n-1})$.

L'area $A$ entre el punt inicial $(x_1, y_1)$ i el punt final $(x_n, y_n)$, ve donada per l'equaci\'{o}:

 \begin{equation}\label{eq:simp_3_8}
   A =  \frac{3h}{8} \sum_{i=1}^{n} \big( y_i + 3 y_{i+1} + 3 y_{i+2} + y_{i+3}\big)
 \end{equation}

 Les equacions \eqref{eq:trap_1} i \eqref{eq:trap_2}, descrites en la regla dels trapezis, tamb\'{e} es poden utilitzar en aquest cas, quan  coneixem l'expressi\'{o} expl\'{\i}cita $y=f(x)$ de la funci\'{o} que volem integrar, entre els punts $x=a$ i $x=b$, i volem fixar el nombre de punts $n$.

 L'odre de l'error d'integraci\'{o} de la regla de Simpson 3/8, \'{e}s: $0(h^4)$. En ser del mateix ordre que el de la regla de Simpson 1/3, la regla de Simpson 3/8 nom\'{e}s s'usa quan $n$ \'{e}s parell, per avaluar la integral per a $x_1$, $x_2$, $x_3$ i $x_4$, utilitzant la regla de Simpson 1/3 per avaluar la resta de la integral per a  $x_4, x_5, \dotsc , x_n$.

\begin{exemple}[Integraci\'{o} num\`{e}rica d'una funci\'{o}]
    Es tracta de calcular num\`{e}ricament la integral: $\int_1^2 \frac{1}{x} \diff x$, utilitzant les regles del trapezis i de Simpson.

    Calculem primer el valor exacte d'aquesta integral, per tal de poder-lo comparar amb els que obtindrem  num\`{e}ricament:
    \[
      \int_1^2 \frac{1}{x} \diff x = \ln 2 = \num{0,6931}
    \]

    Si utilitzem sis punts ($n=6$), tindrem a partir de l'equaci\'{o} \eqref{eq:trap_1}, una dist\`{a}ncia $h$ entre punts de:
    \[
        h = \frac{2-1}{6-1} = \num{0,2}
    \]

    Els valors $x_i$ i $y_i$ que utilitzarem, s\'{o}n:
    \vspace{-8mm}
    \begin{center}
        \[\begin{array}{ccc}
           \toprule[1pt]
              i & x_i  & y_i = 1/x_i \\
           \midrule
              1 & \num{1,0} & \num{1,0000} \\
              2 & \num{1,2} & \num{0,8333} \\
              3 & \num{1,4} & \num{0,7143} \\
              4 & \num{1,6} & \num{0.6250} \\
              5 & \num{1,8} & \num{0.5556} \\
              6 & \num{2,0} & \num{0,5000} \\
           \bottomrule[1pt]
        \end{array} \]
    \end{center}

    Utilitzant la regla dels trapezis, equaci\'{o} \eqref{eq:trap},tenim:
    \[
        \int_1^2 \frac{1}{x} \diff x \approx \frac{\num{0,2}}{2} \big(\num{1,0000}+\num{0,5000} +2 \times(\num{0,8333}+ \num{0,7143} +
        \num{0,6250} + \num{0,5556}) \big) = \num{0,6956}
    \]

    Utilitzant la regla de Simpson 3/8 entre $x_1$ i $x_4$, equaci\'{o} \eqref{eq:simp_3_8}, i la  regla de Simpson 1/3 entre $x_4$ i $x_6$, equaci\'{o} \eqref{eq:simp_1_3}, tenim:
    \[\begin{split}
        \int_1^2 \frac{1}{x} \diff x &\approx \frac{3\times\num{0,2}}{8} \big(\num{1,0000}+3\times\num{0,8333} +3\times \num{0,7143} +
        \num{0,6250} \big) +{}\\[0.5em]
        {}&+ \frac{\num{0,2}}{3} \big(\num{0,6250} +4\times \num{0,5556} + \num{0,5000} \big)
        = \num{0,6932}
    \end{split}\]

    Com era d'esperar, l'aplicaci\'{o} conjunta de les regles de Simpson 3/8 i 1/3, d\'{o}na un resultat m\'{e}s prec\'{\i}s que la regla dels trapezis.
\end{exemple}

