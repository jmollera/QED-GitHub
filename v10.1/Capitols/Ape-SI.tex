\chapter{Sistema Internacional d'Unitats (SI)}\label{sec:SI}\index{sistema internacional d'unitats}\index{SI}

\section{Introducció}
S'expliquen a continuació qüestions relacionades amb el sistema
internacional d'unitats (SI), el qual està definit pel BIPM «Bureau
International des Poids et Mesures». S'han utilitzat les publicacions més recents d'aquest organisme, corresponents a la 8a edició de l'any 2006, més el suplement a aquesta edició de l'any 2014; podeu trobar més informació a les següents adreces del BIPM: \href{http://www.bipm.org/}{www.bipm.org} i
\href{http://www.bipm.org/en/si/si_brochure/}{www.bipm.org/en/si/si\_brochure}.\index{BIPM}

El NIST «National Institute of Standards and Technology» també té informació referent al sistema
internacional d'unitats, a l'adreça: \href{http://www.nist.gov/pml/div684/fcdc/si-units.cfm}
{www.nist.gov/pml/div684/fcdc/si-units.cfm}.\index{NIST}

Dins de l'Estat Espanyol, el Sistema Internacional d'Unitats és d'ús oficial segons el Reial Decret 2032/2009 de 30 de desembre. Es poden descarregar versions en català, castellà i gallec, d'aquest decret a l'adreça: \href{http://www.boe.es/diario_boe/txt.php?id=BOE-A-2010-927}
{www.boe.es/diario\_boe/txt.php?id=BOE-A-2010-927}.

Els noms de totes les unitats s'escriuen tal com apareixen en el DIEC2 «Diccionari de la llengua catalana, 2a edició (2007)».

\section{Unitats fonamentals de l'SI}
\index{sistema internacional d'unitats!unitats fonamentals}

En la Taula \vref{taula:SI-fonamentals} es poden veure les unitats
fonamentals del sistema internacional d'unitats.

\begin{ThreePartTable}
\begin{TableNotes}
    \item[a] {\footnotesize La variant  «kilogram» també és correcta, segons el DIEC2.}
\end{TableNotes}
\begin{longtable}[h]{llc}
   \caption{\label{taula:SI-fonamentals} Unitats fonamentals de l'SI}\\
   \toprule[1pt]
    Magnitud & Unitat & Símbol \\
   \midrule
   \endfirsthead
   \caption[]{Unitats fonamentals de l'SI (\emph{ve de la pàgina anterior})}\\
   \toprule[1pt]
    Magnitud & Unitat & Símbol \\
   \midrule
   \endhead
   \midrule
   \multicolumn{2}{r}{\sffamily\bfseries\color{NavyBlue}(\emph{continua a la pàgina següent})}
   \endfoot
   \insertTableNotes
   \endlastfoot
   longitud & metre & m \\
   massa & quilogram\tnote{a} & kg \\
   temps & segon & s\\
   intensitat de corrent elèctric & ampere & A \\
   temperatura termodinàmica & kelvin & K\\
   quantitat de matèria & mol & mol \\
   intensitat lluminosa & candela &  cd \\
   \bottomrule[1pt]
\end{longtable}
\end{ThreePartTable}
\index{metre} \index{kilogram} \index{quilogram}\index{segon} \index{amper}
\index{kelvin} \index{mol} \index{candela} \index{longitud}
\index{massa} \index{temps} \index{intensitat!de corrent elèctric}
\index{temperatura!termodinàmica} \index{quantitat de matèria}
\index{intensitat!lluminosa} \index{m} \index{kg} \index{s}
\index{A} \index{K} \index{cd}

Es presenten a continuació de forma breu, les definicions
d'aquestes unitats fonamentals; entre parèntesis s'indica l'any
que la «Conférence Générale des Poids et Mesures» les va posar en
vigor.


\begin{list}{}
   {\setlength{\labelwidth}{22mm} \setlength{\leftmargin}{22mm} \setlength{\labelsep}{2mm}}
   \item[\textbf{metre}] És la longitud de la trajectòria recorreguda per la llum
   en el buit, durant un temps de $\frac{1}{\num{299792458}}\unit{segon}$. (1983).
   \item[\textbf{quilogram}] És la massa del prototip internacional del quilogram, fet d'un aliatge de platí-iridi i
    conservat al BIPM, a Sèvres, França. (1901).
   \item[\textbf{segon}] És la durada de \num{9192631770} períodes de la
   radiació corresponent a la transició entre els dos nivells
  hiperfins de l'estat fonamental de l'àtom de cesi-133. (1967).
   \item[\textbf{ampere}] És la intensitat d'un corrent constant,
   que mantinguda en dos conductors paraŀlels rectilinis de longitud
   infinita, de secció transversal negligible, i situats en el buit a una
   distància l'un de l'altre d'un metre, produeix entre
   aquests dos conductors  una força igual a \num{2e-7} newton per metre de longitud. (1948).
   \item[\textbf{kelvin}] És la fracció $\frac{1}{\num{273,16}}$ de la temperatura
   termodinàmica corresponent al punt triple de l'aigua. (1967).
   \item[\textbf{mol}] És la quantitat de matèria d'un sistema que conté tantes
   entitats elementals com àtoms hi ha en 0,012\unit{kg} de carboni-12. (1971).
   \item[\textbf{candela}] És la intensitat lluminosa, en una direcció determinada,
   d'una font que emet radiació monocromàtica de freqüència \num{540e12}\unit{hertz}, i
   que té una intensitat radiant en aquesta direcció de $\frac{1}{683}$ watt per estereoradiant. (1979).
\end{list}


\section{Prefixes de l'SI}
\index{sistema internacional d'unitats!prefixes}

En la Taula \vref{taula:SI-prefixes} es presenta una llista amb els
prefixes que es poden anteposar a les unitats del sistema
internacional d'unitats, per tal de formar els seus múltiples i
submúltiples.

\begin{ThreePartTable}
\begin{TableNotes}
    \item[a] {\footnotesize La variant  «kilo» també és correcta, segons el DIEC2.}
\end{TableNotes}
\begin{longtable}[h]{llccllc}
   \caption{\label{taula:SI-prefixes} Prefixes de  l'SI}\\
   \toprule[1pt]
   \multicolumn{3}{c}{Múltiples} & & \multicolumn{3}{c}{Submúltiples}\\
   \cmidrule(rl){1-3} \cmidrule(rl){5-7}
   factor & nom & símbol & & factor & nom & símbol\\
   \midrule
   \endfirsthead
   \caption[]{Prefixes de  l'SI (\emph{ve de la pàgina  anterior})}\\
   \toprule[1pt]
    \multicolumn{3}{c}{Múltiples} & & \multicolumn{3}{c}{Submúltiples}\\
   \cmidrule(rl){1-3} \cmidrule(rl){5-7}
   factor & nom & símbol & & factor & nom & símbol\\
   \midrule
   \endhead
   \midrule
   \multicolumn{7}{r}{\sffamily\bfseries\color{NavyBlue}(\emph{continua a la pàgina següent})}
   \endfoot
   \insertTableNotes
   \endlastfoot
    $10^{24}$ &  yotta & Y & & $10^{-24}$ & yocto & y \\
    $10^{21}$ &  zetta & Z & & $10^{-21}$ & zepto & z \\
    $10^{18}$ &  exa & E & & $10^{-18}$ & atto & a \\
    $10^{15}$ &  peta & P & & $10^{-15}$ & femto & f \\
    $10^{12}$ &  tera & T & & $10^{-12}$ & pico & p \\
    $10^{9}$ &  giga & G & & $10^{-9}$ & nano & n \\
    $10^{6}$ &  mega & M & & $10^{-6}$ & micro & $\micro$ \\
    $10^{3}$ &  quilo\tnote{a} & k & & $10^{-3}$ & miŀli & m \\
    $10^{2}$ &  hecto & h & & $10^{-2}$ & centi & c \\
    $10^{1}$ &  deca & da & & $10^{-1}$ & deci & d \\
   \bottomrule[1pt]
\end{longtable}
\end{ThreePartTable}
\index{yotta} \index{zetta} \index{exa} \index{peta} \index{tera} \index{giga} \index{mega}
\index{kilo} \index{quilo} \index{hecto} \index{deca} \index{deci} \index{centi} \index{mili} \index{micro}
\index{nano} \index{pico} \index{femto} \index{atto} \index{zepto} \index{yocto}



\section{Unitats derivades de l'SI amb noms i símbols propis}
\index{sistema internacional d'unitats!unitats derivades amb noms i símbols propis}

De forma convenient, s'ha donat noms i símbols propis a algunes unitats derivades de les fonamentals; en la Taula \vref{taula:SI-derivades} es mostren aquestes unitats derivades de l'SI.

\begin{ThreePartTable}
\begin{TableNotes}
    \item[a] {\footnotesize La variant «radian» també és correcta, segons el DIEC2.}
    \item[b] {\footnotesize La variant «estereoradian» també és correcta, segons el DIEC2.}
\end{TableNotes}
\begin{longtable}[h]{llclc}
   \caption{\label{taula:SI-derivades} Unitats derivades de
   l'SI amb noms i símbols propis}\\
   \toprule[1pt]
    \multirow{2}{15mm}{\rule{0mm}{6mm}Magnitud} & \multirow{2}{15mm}{\rule{0mm}{6mm}Unitat}  &
    \multirow{2}{15mm}{\rule{0mm}{6mm}Símbol}  & \multicolumn{2}{c}{Equivalència en unitats SI}\\
    \cmidrule(rl){4-5}
    &  &   & fonamentals & altres\\
   \midrule
   \endfirsthead
   \caption[]{Unitats derivades de l'SI amb noms i símbols propis (\emph{ve de la pàgina
   anterior})}\\
   \toprule[1pt]
    \multirow{2}{15mm}{\rule{0mm}{6mm}Magnitud} & \multirow{2}{15mm}{\rule{0mm}{6mm}Unitat}  &
    \multirow{2}{15mm}{\rule{0mm}{6mm}Símbol}  & \multicolumn{2}{c}{Equivalència en unitats SI}\\
    \cmidrule(rl){4-5}
    &  &  & fonamentals & altres\\
   \midrule
   \endhead
   \midrule
   \multicolumn{5}{r}{\sffamily\bfseries\color{NavyBlue}(\emph{continua a la pàgina següent})}
   \endfoot
   \insertTableNotes
   \endlastfoot
   angle pla & radiant\tnote{a} & rad   & \si{m/m} & 1\\
   angle sòlid & estereoradiant\tnote{b} & sr & \si{m^2/m^2}  & 1 \\
   freqüència & hertz & Hz & \si{s^{-1}} & --- \\
   força & newton & N & \si{kg.m.s^{-2}} & --- \\
   pressió & pascal & Pa  & \si{kg.m^{-1}.s^{-2}} & \si{N/m^2} \\
   energia, treball & joule & J & \si{kg.m^2.s^{-2}} & \si{N.m}\\
   potència & watt & W & \si{kg.m^2.s^{-3}}  & \si{J/s}\\
   càrrega elèctrica & coulomb & C  & \si{A.s} &  ---\\
   potencial elèctric & volt & V & \si{kg.m^2.s^{-3}.A^{-1}}  & \si{W/A}\\
   capacitat elèctrica & farad & F   & \si{kg^{-1}.m^{-2}.s^4.A^2}& \si{C/V}\\
   resistència elèctrica & ohm &  \si{\ohm}  & \si{kg.m^2.s^{-3}.A^{-2}} & \si{V/A}\\
   conductància elèctrica & siemens &  S  & \si{kg^{-1}.m^{-2}.s^3.A^2} & \si{A/V}\\
   flux magnètic & weber &  Wb  & \si{kg.m^2.s^{-2}.A^{-1}} & \si{V.s}\\
   densitat de flux magnètic & tesla &  T  & \si{kg.s^{-2}.A^{-1}} & \si{Wb/m^2}\\
   inductància & henry &  H  & \si{kg.m^2.s^{-2}.A^{-2}} & \si{Wb/A}\\
   temperatura Celsius & grau Celsius &  \celsius & \si{K} & --- \\
   flux lluminós & lumen & lm  & \si{cd.sr}& ---\\
   iŀluminació & lux & lx & \si{cd.sr.m^{-2}} & \si{lm/m^2} \\
   activitat  d'un radionúclid & becquerel & Bq& \si{s^{-1}} & --- \\
   dosi absorbida & gray & Gy  & \si{m^2.s^{-2}}& \si{J/kg}\\
   dosi equivalent & sievert & Sv  & \si{m^2.s^{-2}}& \si{J/kg}\\
   activitat catalítica & katal & kat & \si{mol.s^{-1}} & ---\\
   \bottomrule[1pt]
\end{longtable}
\end{ThreePartTable}
\index{radiant} \index{radian}  \index{estereoradiant}  \index{estereoradian} \index{hertz} \index{newton}
\index{pascal} \index{joule} \index{watt} \index{coulomb}
\index{volt} \index{farad} \index{ohm} \index{siemens} \index{weber}
\index{tesla} \index{henry} \index{lumen} \index{lux}
\index{becquerel} \index{gray} \index{sievert} \index{grau Celsius}\index{katal}
\index{angle pla}  \index{angle sòlid} \index{freq\"{u}ència}
\index{força} \index{pressió} \index{energia} \index{potència}
\index{carrega electrica@càrrega elèctrica} \index{potencial
elèctric} \index{capacitat} \index{resistència} \index{conductància}
\index{flux!magnètic} \index{densitat!de flux magnètic}
\index{inductància} \index{temperatura!Celsius} \index{flux!lluminós} \index{iluminacio@iŀluminació}\index{activitat!d'un radionúclid}
\index{dosi!absorbida}  \index{dosi!equivalent}\index{activitat!catalítica}
\index{rad} \index{sr} \index{Hz} \index{N}
\index{Pa} \index{J} \index{W} \index{C} \index{V} \index{F}
\index{$\Omega$} \index{S} \index{Wb} \index{T} \index{H}
\index{\celsius} \index{lm} \index{lx} \index{Bq} \index{Gy}
\index{Sv}\index{kat}

\section{Altres unitats derivades de l'SI}
\index{sistema internacional d'unitats!altres unitats derivades}

Les unitats fonamentals i les unitats amb noms i símbols propis poden combinar-se entre si per expressar noves unitats derivades.

 en la Taula \vref{taula:SI-derivades-exemples} es mostren alguns exemples d'aquestes combinacions.

\begin{longtable}[h]{lll}
   \caption{\label{taula:SI-derivades-exemples} Exemples d'altres unitats derivades de
   l'SI}\\
   \toprule[1pt]
    Magnitud &  Unitats & Equivalència en unitats fonamentals SI\\
   \midrule
   \endfirsthead
   \caption[]{Exemples d'altres unitats derivades de l'SI (\emph{ve de la pàgina
   anterior})}\\
   \toprule[1pt]
    Magnitud &  Unitats & Equivalència en unitats fonamentals SI\\
   \midrule
   \endhead
   \midrule
   \multicolumn{3}{r}{\sffamily\bfseries\color{NavyBlue}(\emph{continua a la pàgina següent})}
   \endfoot
   \endlastfoot
   viscositat dinàmica &  \si{Pa.s}& \si{kg.m^{-1}.s^{-1}} \\
   moment d'una força & \si{N.m} & \si{kg.m^2.s^{-2}} \\
   tensió superficial &  \si{N/m} &   \si{kg.s^{-2}} \\
   velocitat angular & \si{rad/s} & \si{m.m^{-1}.s^{-1}} = \si{s^{-1}} \\
   acceleració angular & \si{rad/s^2} & \si{m.m^{-1}.s^{-2}} = \si{s^{-2}} \\
   densitat de flux de calor & \si{W/m^2} & \si{kg.s^{-3}} \\
   entropia & \si{J/K} & \si{kg.m^2.s^{-2}.K^{-1}} \\
   entropia específica & \si{J/(kg.K)} &\si{m^2.s^{-2}.K^{-1}} \\
   energia específica & \si{J/kg} & \si{m^2.s^{-2}} \\
   conductivitat tèrmica & \si{W/(m.K)} & \si{kg.m.s^{-3}.K^{-1}} \\
   densitat d'energia & \si{J/m^3} & \si{kg.m^{-1}.s^{-2}} \\
   intensitat de camp elèctric & \si{V/m}& \si{kg.m.s^{-3}.A^{-1}}  \\
   densitat de càrrega elèctrica & \si{C/m^3} & \si{A.s.m^{-3}} \\
   densitat de flux elèctric & \si{C/m^2} & \si{A.s.m^{-2}}\\
   permitivitat &  \si{F/m}& \si{kg^{-1}.m^{-3}.s^4.A^2} \\
   permeabilitat &  \si{H/m} & \si{kg.m.s^{-2}.A^{-2}} \\
   energia molar & \si{J/mol} & \si{kg.m^2.s^{-2}.mol^{-1}} \\
   entropia molar& \si{J/(mol.K)} & \si{kg.m^2.s^{-2}.mol^{-1}.K^{-1}} \\
   exposició (raigs x i $\gammaup$) & \si{C/kg} & \si{A.s.kg^{-1}} \\
   tassa de dosi absorbida & \si{Gy/s} & \si{m^2.s^{-3}}\\
   intensitat radiant & \si{W/sr} & \si{kg.m^2.s^{-3}.sr^{-1}} \\
   radiància & \si{W/(sr.m^2)} & \si{kg.s^{-3}.sr^{-1}} \\
   concentració d'activitat catalítica &  \si{kat/m^3} & \si{mol.s^{-1}.m^{-3}}\\
    \bottomrule[1pt]
\end{longtable}
\index{viscositat!dinàmica}\index{moment d'una força}\index{tensió superficial}
\index{velocitat angular}\index{acceleració!angular}\index{densitat!de flux de calor}\index{entropia}\index{entropia!específica}\index{conductivitat termica@conductivitat tèrmica}\index{densitat!d'energia}\index{intensitat!de camp electric@intensitat de camp elèctric}
\index{densitat!de càrrega elèctrica}\index{densitat!de flux elèctric}
\index{permitivitat}\index{permeabilitat}\index{energia!molar}\index{entropia!molar}
\index{exposició}\index{tassa de dosi absorbida}\index{intensitat!radiant}\index{radiancia@radiància}
\index{concentracio d'activitat catalitica@concentració d'activitat catalítica}



\section{Unitats fora de l'SI}\label{sec:unitats-fora-SI}
\index{sistema internacional d'unitats!unitats fora de l'SI}

Hi ha una sèrie d'unitats que no formem part de l'SI però que són d'ús comú en el camp científic, tècnic o comercial, i que són usades freqüentment. En les taules següents es recullen algunes d'aquestes unitats.

En la Taula \vref{taula:SI-altres-acceptades} es mostren les unitats fora de l'SI, l'ús de les quals s'accepta en conjunció amb el Sistema Internacional d'Unitats, ja que són presents en la vida diària i s'espera que el seu ús continuï de forma indefinida. Cadascuna d'aquestes unitats té una definició exacte en termes d'unitats de l'SI.

\begin{ThreePartTable}
\begin{TableNotes}
    \item[a] {\footnotesize El símbol «L» es va adoptar posteriorment al símbol «l» per evitar la possible confusió entre la lletra ela minúscula i  el número 1.}
    \item[b] {\footnotesize En el països de parla anglesa aquesta unitat és coneguda com a «tona mètrica».}
    \item[c] {\footnotesize La «unitat astronòmica» va ser redefinida l'any 2012 en la 28a Assemblea General de la Unió Astronòmica Internacional (\href{http://www.iau.org/}{www.iau.org}), passant a ser un valor exacte.}
\end{TableNotes}
\begin{longtable}[h]{llcl}
   \caption{\label{taula:SI-altres-acceptades} Unitats fora de l'SI acceptades per a ser usades amb l'SI  }\\
   \toprule[1pt]
    Magnitud & Unitat &  Símbol & Valor en unitats SI\\
   \midrule
   \endfirsthead
   \caption[]{Unitats fora de l'SI acceptades per ser usades amb l'SI (\emph{ve de la pàgina
   anterior})}\\
   \toprule[1pt]
    Magnitud & Unitat &  Símbol & Valor en unitats SI\\
   \midrule
   \endhead
   \midrule
   \multicolumn{4}{r}{\sffamily\bfseries\color{NavyBlue}(\emph{continua a la pàgina següent})}
   \endfoot
   \insertTableNotes
   \endlastfoot
   temps & minut &  \si{min}& $1\unit{min} = 60\unit{s}$ \\
   temps & hora & \si{h} & $1\unit{h} = 60\unit{min} = 3600\unit{s}$ \\
   temps & dia & \si{d} & $1\unit{d} = 24\unit{h} = 86400\unit{s}$\\
   angle pla & grau &  \si{\degree} &   $\ang{1} = (\piup/180)\unit{rad}$ \\
   angle pla & minut & \si{\arcminute} & $\ang{;1;} = (1/60)\degree = (\piup/10800)\unit{rad}$ \\
   angle pla & segon & \si{\arcsecond} & $\ang{;;1} = (1/60)' = (\piup/648000)\unit{rad}$ \\
   àrea & hectàrea & \si{ha} & $1\unit{ha} = 1\unit{hm^2} = 10^4\unit{m^2}$\\
   volum & litre\tnote{a} &  \si{l},\unit{L} & $1\unit{l} = 1\unit{L} = 1\unit{dm^3} = 10^{-3}\unit{m^3}$ \\
   massa & tona\tnote{b} & \unit{t} & $1\unit{t} =1000\unit{kg}$\\
   longitud & unitat astronòmica\tnote{c} &  \unit{au} &  $1\unit{au} =  \SI{149597870700}{m}$ \\
   \bottomrule[1pt]
\end{longtable}
\end{ThreePartTable}
\index{minut}\index{hora}\index{dia}\index{grau}\index{segon}\index{litre}\index{tona}\index{hectarea@hectàrea}
\index{min}\index{h}\index{d}\index{$\degree$}\index{$'$}\index{$''$}\index{l}\index{L}\index{ha}\index{t}
\index{unitat astronomica@unitat astronòmica}\index{au}\index{temps}\index{angle pla}\index{area@àrea}
\index{volum}\index{massa}\index{longitud}

En la Taula \vref{taula:SI-altres-experimentals} es mostren les unitats fora de l'SI, el valor de les quals s'obté de forma experimental (les xifres entre parèntesi representen l'error absolut del valor -- vegeu la secció \ref{err_abs_rel}); els valors que hi apareixen són els recomanats
l'any 2014 pel «Committee on Data for Science and Technology» (CODATA). \index{CODATA}

\begin{ThreePartTable}
\begin{TableNotes}
    \item[a] {\footnotesize El «dalton» i la «unitat de massa atòmica unificada» són dos noms alternatius d'una mateixa unitat.}
\end{TableNotes}
\begin{longtable}[h]{llcl}
   \caption{\label{taula:SI-altres-experimentals} Unitats fora de l'SI obtingudes de forma experimental }\\
   \toprule[1pt]
    Magnitud & Unitat &  Símbol & Valor en unitats SI\\
   \midrule
   \endfirsthead
   \caption[]{Unitats fora de l'SI obtingudes de forma experimental (\emph{ve de la pàgina
   anterior})}\\
   \toprule[1pt]
    Magnitud & Unitat &  Símbol & Valor en unitats SI\\
   \midrule
   \endhead
   \midrule
   \multicolumn{4}{r}{\sffamily\bfseries\color{NavyBlue}(\emph{continua a la pàgina següent})}
   \endfoot
   \insertTableNotes
   \endlastfoot
   energia & electró-volt & \unit{eV} & $1\unit{eV} = \SI{1,6021766208(98)e-19}{J}$ \\
   massa & dalton\tnote{a} & Da & $1\unit{Da} = \SI{1,660539040(20)e-27}{kg}$\\
   massa & unitat de massa atòmica unificada\tnote{a} & u & $1\unit{u} =
    \SI{1,660539040(20)e-27}{kg}$  \\
\bottomrule[1pt]
\end{longtable}
\end{ThreePartTable}
\index{electró-volt}\index{unitat de massa atomica unificada@unitat de massa atòmica unificada}\index{dalton}\index{eV}\index{u}\index{Da}\index{energia}\index{massa}


En la Taula \vref{taula:SI-altres} es mostren altres unitats fora de l'SI utilitzades en diversos camps. Algunes d'aquestes unitats estan relacionades amb l'antic sistema CGS (centímetre--gram--segon).\index{CGS}

\begin{ThreePartTable}
\begin{TableNotes}
    \item[a] {\footnotesize No i ha acord internacional pel símbol de la «milla nàutica», a més d'«M» també s'utilitza «NM», «Nm» i «nmi».}
    \item[b] {\footnotesize No i ha acord internacional pel símbol del  «nus», però el símbol «kn» és àmpliament usat.}
    \item[c] {\footnotesize Aquestes unitats adimensionals s'utilitzen per expressar logaritmes de relacions entre quantitats. Per exemple, $n\unit{Np}$ fa referència a una relació del tipus $ln\frac{A_2}{A_1}= n$, i  $ m \unit{dB} =\frac{m}{10}\unit{B}$  fa referència a una relació del tipus $\log\frac{A_2}{A_1} =\frac{m}{10}$.}
\end{TableNotes}
\begin{longtable}[h]{llcl}
   \caption{\label{taula:SI-altres} Altres unitats fora de l'SI}\\
   \toprule[1pt]
    Magnitud & Unitat &  Símbol & Valor en unitats SI\\
   \midrule
   \endfirsthead
   \caption[]{Altres unitats fora de l'SI (\emph{ve de la pàgina
   anterior})}\\
   \toprule[1pt]
    Magnitud & Unitat &  Símbol & Valor en unitats SI\\
   \midrule
   \endhead
   \midrule
   \multicolumn{4}{r}{\sffamily\bfseries\color{NavyBlue}(\emph{continua a la pàgina següent})}
   \endfoot
   \insertTableNotes
   \endlastfoot
    pressió & bar & \si{bar} & $1\unit{bar} = 100\unit{kPa}$ \\
    pressió & miŀlímetre de mercuri & \si{mmHg} & $1\unit{mmHg} \approx \SI{133,322}{Pa}$ \\
    longitud & àngstrom («ångström») & $\si{\angstrom}$ & $1\unit{\angstrom} = \SI{e-10}{m}$\\
    distància & milla nàutica\tnote{a} &  \si{M} & $1\unit{M} = 1852\unit{m}$ \\
    àrea & barn & \si{b} &  $1\unit{b} = \SI{e-28}{m^2}$\\
    velocitat & nus\tnote{b} & \si{kn} & $1\unit{kn} = 1\unit{M/h} = \frac{1852}{3600}\unit{m/s}$ \\
    logaritme d'una relació & neper, bel, decibel\tnote{c} & \si{Np}, \si{B}, \si{dB} & 1\\
    energia & erg & \si{erg} & $1\unit{erg} = \SI{e-7}{J} $ \\
    força & dina & \si{dyn} & $1\unit{dyn} = \SI{e-5}{N}$ \\
    viscositat dinàmica & poise & \si{P} & $1\unit{P} = \SI{1}{dyn.s/cm^2} = \SI{0,1}{Pa.s}$ \\
    viscositat cinemàtica & stokes & \si{St} & $1\unit{St} = \SI{1}{cm^2/s} = \SI{e-4}{m^2/s}$ \\
    luminància & stilb & \si{sb} & $1\unit{sb} = \SI{1}{cd/cm^2} = \SI{e4}{cd/m^2}$ \\
    iŀluminació & fot & \si{ph} & $1\unit{ph} = \SI{1}{cd.sr/cm^2} = \SI{e4}{lx}$ \\
    acceleració & gal & \si{Gal} & $1\unit{Gal} = \SI{1}{cm/s^2} = \SI{e-2}{m/s^2}$ \\
    flux magnètic & maxwell & \si{Mx} & $1\unit{Mx} = \SI{e-8}{Wb}$ \\
    densitat de flux magnètic & gauss & \si{G} & $1\unit{G} = \SI{e-4}{T}$ \\
    camp magnètic & oersted & \si{Oe} & $1\unit{Oe} = \frac{1000}{4\piup}\unit{A/m}$ \\
\bottomrule[1pt]
\end{longtable}
\end{ThreePartTable}
\index{bar}\index{milimetre de mercuri@miŀlímetre de mercuri}\index{angstrom@àngström}
\index{milla nautica@milla nàutica}\index{barn}\index{nus}\index{neper}\index{bel}\index{decibel}
\index{erg}\index{dina}\index{poise}\index{stokes}\index{stilib}\index{fot}\index{gal}\index{maxwell}
\index{gauss}\index{oersted}\index{mmHg}\index{A@$\angstrom$}\index{M}\index{NM}\index{Nm}\index{nmi}\index{b}
\index{kn}\index{Np}\index{B}\index{dB}\index{dyn}\index{P}\index{St}\index{sb}\index{ph}\index{Gal}
\index{Mx}\index{G}\index{Oe}\index{pressió}\index{longitud}\index{distància}\index{area@àrea}\index{velocitat}
\index{energia}\index{força}\index{viscositat!dinàmica}\index{viscositat!cinemàtica}\index{luminància}
\index{iluminacio@iŀluminació}\index{luminància}\index{acceleració}\index{flux!magnètic}
\index{densitat!de flux magnètic}\index{camp magnètic}


Finalment, en la Taula \vref{taula:SI-altres-NIST} es mostren quatre unitats fora de l'SI acceptades addicionalment pel NIST «National Institute of Standards and Technology», ja que s'han utilitzat tradicionalment en els Estats Units d'Amèrica. Aquest organisme desaconsella, no obstant, continuar usant aquestes unitats, i recomana en canvi l'ús de les unitats equivalents de l'SI.\index{NIST}

\begin{ThreePartTable}
\begin{TableNotes}
    \item[a] {\footnotesize Quan hi hagi perill de confusió amb el símbol del radiant, es podrà  utilitzar el símbol «rd» enlloc del símbol  «rad».}
\end{TableNotes}
\begin{longtable}[h]{llcl}
   \caption{\label{taula:SI-altres-NIST} Altres unitats fora de l'SI acceptades pel NIST}\\
   \toprule[1pt]
    Magnitud & Unitat &  Símbol & Valor en unitats SI\\
   \midrule
   \endfirsthead
   \caption[]{Altres unitats fora de l'SI acceptades pel NIST (\emph{ve de la pàgina anterior})}\\
   \toprule[1pt]
    Magnitud & Unitat &  Símbol & Valor en unitats SI\\
   \midrule
   \endhead
   \midrule
   \multicolumn{4}{r}{\sffamily\bfseries\color{NavyBlue}(\emph{continua a la pàgina següent})}
   \endfoot
   \insertTableNotes
   \endlastfoot
    activitat d’un radionúclid & curie &  \si{Ci} & \SI{1}{Ci} = \SI{3,7e10}{Bq} \\
    dosi absorbida & rad & rad\tnote{a}  & \SI{1}{rad} = \SI{e-2}{Gy}\\
    dosi equivalent & rem & rem &  \SI{1}{rem} = \SI{e-2}{Sv} \\
    exposició (raigs x i $\gammaup$) & roentgen & \si{R} & \SI{1}{R} = \SI{2,58e-4}{C/kg} \\
\bottomrule[1pt]
\end{longtable}
\end{ThreePartTable}
\index{curie}\index{roentgen}\index{rad}\index{rem}\index{Ci}\index{R}
\index{activitat!d'un radionúclid}\index{dosi!absorbida}\index{dosi!equivalent}\index{exposició}


\section{Unitats definides en la norma CEI 60027}\label{sec:unitats-cei}
\index{sistema internacional d'unitats!unitats definides en la norma CEI 60027}

La norma CEI 60027 «Letter symbols to be used in electrical technology» adopta totes les unitats definides per l'SI, però en defineix algunes d'addicionals.

\subsection{Unitats de potència elèctrica}

Tot i que la potència es mesura en watt, la norma CEI 60027-1 «Letter symbols to be used in electrical technology -- Part 1: General», defineix noms i símbols d'unitats diferenciats per a les potències activa, reactiva i aparent.\index{CEI!60027-1}

En la Taula \vref{taula:P-Q-S} es mostren aquestes unitats de potència elèctrica.

\begin{longtable}[h]{llcl}
   \caption{\label{taula:P-Q-S} Unitats de potència elèctrica}\\
   \toprule[1pt]
    Magnitud & Unitat &  Símbol & Valor en unitats SI \\
   \midrule
   \endfirsthead
   \caption[]{Unitats de potència elèctrica (\emph{ve de la pàgina anterior})}\\
   \toprule[1pt]
    Magnitud & Unitat &  Símbol & Valor en unitats SI \\
   \midrule
   \endhead
   \midrule
   \multicolumn{4}{r}{\sffamily\bfseries\color{NavyBlue}(\emph{continua a la pàgina següent})}
   \endfoot
   \endlastfoot
   potència activa & watt &  \si{W}& $1\unit{W} = 1\unit{W}$  \\
   potència reactiva & var &  \si{var}& $1\unit{var} = 1\unit{W}$  \\
   potència aparent & voltampere &  \si{VA}& $1\unit{VA} = 1\unit{W}$  \\
   \bottomrule[1pt]
\end{longtable}
\index{potència!activa}\index{potència!reactiva}\index{potència!aparent}
\index{W}\index{VA}\index{var}\index{voltampere}\index{watt}



\subsection{Unitats informàtiques i prefixes de potències binàries}

La norma CEI 60027-2 «Letter symbols to be used in electrical technology -- Part 2: Telecommunications and electronics», defineix símbols d'unitats informàtiques i prefixes de potències binàries que cal usar amb aquestes unitats.\index{CEI!60027-2}

En la Taula \vref{taula:simbol-inform} es mostren el símbols d'unitats informàtiques.
\begin{longtable}[h]{>{\hspace{5mm}}cc}
   \caption{\label{taula:simbol-inform} Unitats informàtiques}\\
   \toprule[1pt]
    Nom & Símbol \\
   \midrule
   \endfirsthead
   \caption[]{Unitats informàtiques (\emph{ve de la pàgina anterior})}\\
   \toprule[1pt]
    Nom & Símbol \\
   \midrule
   \endhead
   \midrule
   \multicolumn{2}{r}{\sffamily\bfseries\color{NavyBlue}(\emph{continua a la pàgina següent})}
   \endfoot
   \endlastfoot
   bit & bit    \\
   octet, byte & B   \\
   \bottomrule[1pt]
\end{longtable}
\index{bit}\index{byte}\index{B}\index{octet}

En la Taula \vref{taula:prefix-inform} es mostren els prefixes de potències binàries.
\begin{longtable}[h]{lcl}
   \caption{\label{taula:prefix-inform} Prefixes de potències binàries}\\
   \toprule[1pt]
    Nom & Símbol  & Factor \\
   \midrule
   \endfirsthead
   \caption[]{Prefixes de potències binàries (\emph{ve de la pàgina anterior})}\\
   \toprule[1pt]
    Nom & Símbol  & Factor \\
   \midrule
   \endhead
   \midrule
   \multicolumn{3}{r}{\sffamily\bfseries\color{NavyBlue}(\emph{continua a la pàgina següent})}
   \endfoot
   \endlastfoot
   yobi & Yi   & $2^{80} \approx \num{1,2089e24}$ \\
   zebi & Zi   & $2^{70} \approx \num{1,1806e21}$ \\
   exbi & Ei   & $2^{60} \approx \num{1,1529e18}$ \\
   pebi & Pi   & $2^{50} \approx \num{1,1259e15}$ \\
   tebi & Ti   & $2^{40} \approx \num{1,0995e12}$ \\
   gibi & Gi   & $2^{30} \approx \num{1,0737e9}$  \\
   mebi & Mi   & $2^{20} \approx \num{1,0486e6}$ \\
   kibi & Ki   & $2^{10} = 1024$  \\
   \bottomrule[1pt]
\end{longtable}
\index{kibi} \index{Ki} \index{mebi} \index{Mi} \index{gibi}  \index{Gi} \index{tebi} \index{Ti}
\index{pebi} \index{Pi} \index{exbi} \index{Ei}

Utilitzant aquests prefixes podem escriure per exemple:
\[1\unit{MiB} =2^{20}\unit{B} = \SI{1048576}{B}\]

El prefix «M» de l'SI indica, en canvi, un altre valor:
\[1\unit{MB} =10^6\unit{B} = \SI{1000000}{B}\]



\subsection{Unitats relatives a màquines elèctriques rotatòries}

Hi ha dues magnituds que poden expressar el mateix fenomen físic de rotació d'un cos; una és la freqüència $f$, expressada en herz, que indica el nombre de cicles (voltes o revolucions senceres) per unitat de temps, i l'altra és la velocitat angular $\omega$, expressada en radiant per segon, que indica l'angle girat per unitat de temps. La relació entre ambdues magnituds és:
\begin{equation}
  \omega = 2 \piup f
\end{equation}

Quan es tracta de motors o d'altres màquines elèctriques rotatòries, la norma CEI 60027-1 «Letter symbols to be used in electrical technology -- Part 1: General», estableix «r» com el símbol internacional de «revolució»; \index{CEI!60027-1} d'aquesta manera es fa innecessari utilitzar altres formes que varien segons l'idioma, com per exemple «rev/min» o «rpm» (revolucions per minut) en català, o «tr/min» (tours par minute) en francès.

En la Taula \vref{taula:rev-min} es poden veure aquestes unitats relatives a màquines elèctriques rotatòries.

\begin{longtable}[h]{llcl}
   \caption{\label{taula:rev-min} Unitats relatives a màquines elèctriques rotatòries}\\
   \toprule[1pt]
    Magnitud & Unitat &  Símbol & Valor en unitats SI \\
   \midrule
   \endfirsthead
   \caption[]{Unitats relatives a màquines elèctriques rotatòries (\emph{ve de la pàgina anterior})}\\
   \toprule[1pt]
    Magnitud & Unitat &  Símbol & Valor en unitats SI \\
   \midrule
   \endhead
   \midrule
   \multicolumn{4}{r}{\sffamily\bfseries\color{NavyBlue}(\emph{continua a la pàgina següent})}
   \endfoot
   \endlastfoot
   angle pla & revolució &  \si{r} & $1\unit{r} = 2\piup\unit{rad}$  \\
   velocitat angular & revolució per segon &  \si{r/s}& $1\unit{r/s} = 2\piup\unit{rad/s}$  \\
   velocitat angular & revolució per minut &  \si{r/min}& $1\unit{r/min} = \frac{2\piup}{60}\unit{rad/s}$  \\
   \bottomrule[1pt]
\end{longtable}
\index{revolució}\index{revolució per segon}\index{revolució per minut}
\index{r}\index{angle pla}\index{velocitat angular}


\section{Normes d'escriptura}\label{sec:normes-escript}
\index{sistema internacional d'unitats!normes d'escriptura}

Es presenten a continuació algunes normes aplicables a l'escriptura
de les unitats del sistema internacional d'unitats.

Després de cadascuna de les explicacions es donen exemples d'escriptures correctes, precedides pel símbol \textcolor{Green}\faCheckSquare{}, exemples d'escriptures  incorrectes,  precedides pel símbol \textcolor{Red}\faTimesCircle{}, i exemples d'escriptures correctes però no recomanades, precedides pel símbol
\textcolor{Blue}\faQuestionCircle{}.

\begin{itemize}

\item El prefix utilitzat per simbolitzar 1000 és la lletra «k» (minúscula).  La lletra «K» (majúscula) és el símbol del  kelvin; cal tenir en compte que «\degree K»  no és correcte. En canvi, el símbol del grau Celsius és «\celsius», ja que la lletra «C» sola és el símbol del coulomb.

\textcolor{Green}\faCheckSquare{} 6,9  kV

\textcolor{Red}\faTimesCircle{} 6,9 KV

\textcolor{Green}\faCheckSquare{} $\SI{100}{\celsius} = \SI{373,15}{K}$

\textcolor{Red}\faTimesCircle{} $\SI{100}{C} = \SI{373,15}{\degree K}$

\item Els símbols de les unitats no han d'anar seguits d'un punt, llevat que es trobin al final d'una oració, ja que no són
abreviatures.

\textcolor{Green}\faCheckSquare{} 25 V

\textcolor{Red}\faTimesCircle{} 25 V.

\textcolor{Green}\faCheckSquare{}  40 A

\textcolor{Red}\faTimesCircle{}  40 A.


\item Els símbols de les unitats no canvien de forma en el plural, no han
d'utilitzar-se abreviatures ni han d'afegir-se o suprimir-se
lletres.

\textcolor{Green}\faCheckSquare{} 150 kg

\textcolor{Red}\faTimesCircle{} 150 Kgs

\textcolor{Green}\faCheckSquare{} 25 m

\textcolor{Red}\faTimesCircle{} 25 mts

\textcolor{Green}\faCheckSquare{} 33\unit{cm^3}

\textcolor{Red}\faTimesCircle{} 33 cc

\textcolor{Green}\faCheckSquare{} 20 s

\textcolor{Red}\faTimesCircle{} 20 seg

\textcolor{Green}\faCheckSquare{} 80 km/h

\textcolor{Red}\faTimesCircle{} 80 kph

\textcolor{Green}\faCheckSquare{} 1500 r/min

\textcolor{Red}\faTimesCircle{} 1500 rpm

\item No han de barrejar-se noms i símbols d'unitats.

\textcolor{Green}\faCheckSquare{} 4 rad/s

\textcolor{Red}\faTimesCircle{} 4 rad/segon

\textcolor{Green}\faCheckSquare{} 4 radiant per segon

\textcolor{Red}\faTimesCircle{} 4 radiant/s

\textcolor{Green}\faCheckSquare{} 100 km/h

\textcolor{Red}\faTimesCircle{} 100 km/hora

\textcolor{Green}\faCheckSquare{} 100 quilòmetre per hora

\textcolor{Red}\faTimesCircle{} 100 quilòmetre/h


\item Els símbols de les unitats s'escriuen a la dreta dels valors
numèrics, separats per un espai en blanc.

\textcolor{Green}\faCheckSquare{} 25 V

\textcolor{Red}\faTimesCircle{} 25V

\textcolor{Green}\faCheckSquare{} 40 \celsius

\textcolor{Red}\faTimesCircle{} 40\celsius

\textcolor{Green}\faCheckSquare{} 20 nF

\textcolor{Red}\faTimesCircle{} 20nF


 L'única excepció al punt anterior és la mesura d'angles en graus, minuts i segons; en aquest cas s'escriu el valor i la unitat tot junt.

\textcolor{Green}\faCheckSquare{} \ang{45}

\textcolor{Red}\faTimesCircle{} \ang[number-angle-product = \,]{45}

\textcolor{Green}\faCheckSquare{} \ang{15;32;8}

\textcolor{Red}\faTimesCircle{} \ang[number-angle-product = \,]{15;32;8}

\item En el cas de símbols d'unitats derivades formats pel producte
d'altres unitats, el producte s'indicarà mitjançant un punt volat o
un espai en blanc.

\textcolor{Green}\faCheckSquare{} 24 N$\cdot$m

\textcolor{Red}\faTimesCircle{} 24 N--m

\textcolor{Green}\faCheckSquare{} 24 N\,m

\textcolor{Red}\faTimesCircle{} 24 Nm

Quan s'utilitza un espai en blanc cal tenir en compte  l'ordre en què s'escriuen
les unitats, ja que algunes combinacions poden crear confusió i
és millor evitar-les, per exemple: 24\unit{N\,m} (24 newton metre) i
24\unit{m\,N} (24~metre newton) són expressions equivalents, però
aquesta darrera forma d'escriptura pot ser confosa amb 24\unit{mN} (24 miŀlinewton).

\item En el cas de símbols d'unitats derivades formats per la divisió
d'altres unitats, la divisió s'indicarà mitjançant una línia
inclinada o horitzontal, o mitjançant potències negatives.

\textcolor{Green}\faCheckSquare{} 100\unit{m\,/s}

\textcolor{Green}\faCheckSquare{} \SI{100}{m.s^{-1}}

\textcolor{Green}\faCheckSquare{} 100\unit{\dfrac{m}{s}}

\textcolor{Red}\faTimesCircle{} 100\unit{m\div s}

En el cas anterior, quan s'utilitza la línia inclinada i hi ha més
d'una unitat en el denominador, aquestes unitats s'han d'escriure
entre parèntesis.

\textcolor{Green}\faCheckSquare{} \SI{5}{m.kg/(s^3.A)}

\textcolor{Red}\faTimesCircle{} \SI{5}{m.kg/s^3.A}

\textcolor{Red}\faTimesCircle{} \SI{5}{m.kg/s^3/A}


\item No ha de deixar-se cap espai en blanc entre el símbol d'un prefix i
el símbol d'una unitat.

\textcolor{Green}\faCheckSquare{} 12 mm

\textcolor{Red}\faTimesCircle{} 12 m\,m

\textcolor{Green}\faCheckSquare{}  3 GHz

\textcolor{Red}\faTimesCircle{}  3 G\,Hz


\item El grup format pel símbol d'un prefix i el símbol d'una unitat
esdevé un nou símbol inseparable (formant un múltiple o submúltiple
de la unitat), i pot ser pujat a una potència positiva o negativa i
combinat amb altres símbols.

\textcolor{Green}\faCheckSquare{} 20\unit{km^2}

\textcolor{Red}\faTimesCircle{} 20\unit{(km)^2}

\textcolor{Green}\faCheckSquare{}  12\unit{kg\,/mm^2}

\textcolor{Red}\faTimesCircle{}  12\unit{kg\,/(mm)^2}


\item Només es permet un prefix davant d'una unitat.

\textcolor{Green}\faCheckSquare{} 8\unit{nm}

\textcolor{Red}\faTimesCircle{} 8\unit{m\micro m}


\item No es permeten prefixes aïllats.

\textcolor{Green}\faCheckSquare{} El nombre de partícules es de $5\times 10^6 /\unit{m^3}$

\textcolor{Red}\faTimesCircle{} El nombre de partícules es de $5 \unit{M\,/m^3}$


\item En el cas dels símbols d'unitats derivades formades per la divisió
d'altres unitats, l'ús de prefixes en el numerador i denominador de
forma simultània pot causar confusió, i és preferible per tant,
utilitzar una alta combinació d'unitats on només el numerador o el
denominador tinguin prefix.

\textcolor{Green}\faCheckSquare{} 10\unit{MV/m}

\textcolor{Blue}\faQuestionCircle{}  10\unit{kV/mm}


\item De forma anàloga, el mateix és aplicable als símbols d'unitats
derivades formades pel producte d'altres unitats.

\textcolor{Green}\faCheckSquare{} \SI{10}{kV.s}

\textcolor{Blue}\faQuestionCircle{}  \SI{10}{MV.ms}


\item Els noms de les unitats de l'SI s'escriuen en minúscula, excepte en
el cas de «grau Celsius», i a l'inici d'una oració.

\textcolor{Green}\faCheckSquare{} 10 newton

\textcolor{Red}\faTimesCircle{} 10 Newton

\textcolor{Green}\faCheckSquare{}  100 watt

\textcolor{Red}\faTimesCircle{} 100 Watt

\textcolor{Green}\faCheckSquare{}  24 volt

\textcolor{Red}\faTimesCircle{} 24 Volt

\textcolor{Green}\faCheckSquare{}  20 grau Celsius

\textcolor{Red}\faTimesCircle{} 20 grau celsius


\item Les unitats que tenen noms provinents de noms propis s'han
d'escriure tal com apareixen en les taules
\vref{taula:SI-fonamentals}, \vref{taula:SI-derivades}, \vref{taula:SI-altres-acceptades}, \vref{taula:SI-altres-experimentals} i \vref{taula:SI-altres}, i no s'han
de traduir.

\textcolor{Green}\faCheckSquare{} 50 newton

\textcolor{Red}\faTimesCircle{}  50 neuton

\textcolor{Green}\faCheckSquare{} 300 joule

\textcolor{Red}\faTimesCircle{}  300 juls

\textcolor{Green}\faCheckSquare{} $10^{-6}$ farad

\textcolor{Red}\faTimesCircle{}  $10^{-6}$ faradis


 \item Quan el nom d'una unitat
conté un prefix, ambdues parts s'han d'escriure juntes sense cap espai o element d'unió.

\textcolor{Green}\faCheckSquare{} 1 miŀligram

\textcolor{Red}\faTimesCircle{} 1 miŀli gram

\textcolor{Red}\faTimesCircle{} 1 miŀli-gram

\textcolor{Green}\faCheckSquare{}  980 hectopascal

\textcolor{Red}\faTimesCircle{} 980 hecto pascal

\textcolor{Red}\faTimesCircle{} 980 hecto-pascal


\item En el cas  d'unitats derivades que s'expressen amb divisions o
productes, s'utilitza la preposició «per» entre dos noms d'unitats
per indicar-ne la divisió, i no s'utilitza cap paraula per indicar-ne el
producte.

\textcolor{Green}\faCheckSquare{} 1\unit{m\,/s} -- 1 metre per segon

\textcolor{Red}\faTimesCircle{}  1\unit{m\,/s} -- 1 metre dividit per segon

 \textcolor{Green}\faCheckSquare{} \SI{20}{\ohm.m} -- 20 ohm metre

\textcolor{Red}\faTimesCircle{}   \SI{20}{\ohm.m} -- 20 ohm  per metre


\item El valor d'una quantitat ha d'expressar-se  utilitzant únicament una
unitat.

\textcolor{Green}\faCheckSquare{} 10,234 m

\textcolor{Red}\faTimesCircle{}  10 m 23 cm 4 mm


\item Quan s'expressa el valor d'una quantitat, és incorrecte afegir
lletres o altres símbols a la unitat; qualsevol informació
addicional necessària ha d'afegir-se a la quantitat.

\textcolor{Green}\faCheckSquare{} U\ped{rms} = 220 V

\textcolor{Red}\faTimesCircle{}  U = 220\unit{V\ped{rms}}

\textcolor{Green}\faCheckSquare{}  I\ped{max} = 36 kA

\textcolor{Red}\faTimesCircle{}   I = 36\unit{kA\ped{max}}


\item El separador decimal entre la part entera i decimal d'un valor por ser el punt o la coma. L'ús de l'un o l'altre varia segons el país. Si el valor està comprès entre -1 i +1, és obligatori escriure un zero davant del separador decimal.

\textcolor{Green}\faCheckSquare{} 0,25 A

\textcolor{Red}\faTimesCircle{}  ,25 A

\textcolor{Green}\faCheckSquare{} 0.25 A

\textcolor{Red}\faTimesCircle{}  .25 A


\item Quan un valor té moltes xifres, les xifres poden dividir-se en grups de tres mitjançant un espai curt per tal de millorar-ne la llegibilitat. No s'han d'utilitzar punts o comes per separar aquests grups de tres xifres.

\textcolor{Green}\faCheckSquare{} \SI{43279,16829}{kg}

\textcolor{Green}\faCheckSquare{} \SI[group-separator =]{43279,16829}{kg}

\textcolor{Red}\faTimesCircle{}  \SI[group-separator = .]{43279,16829}{kg}

\end{itemize}

El document  «Guide for the Use of the International System of Units (SI)», publicat pel NIST,  fa a més les recomanacions següents:

\begin{itemize}

\item Quan s'indiquen valors de magnituds amb les seves desviacions,
s'indiquen intervals o s'expressen diversos valors numèrics, les
unitats han de ser presents en cadascun dels valors o s'han d'usar
parèntesis si es vol posar les unitats només al final.

\textcolor{Green}\faCheckSquare{} \SI[separate-uncertainty, multi-part-units = repeat]{63,2(1)}{m}

\textcolor{Green}\faCheckSquare{} \SI[separate-uncertainty]{63,2(1)}{m}

\textcolor{Red}\faTimesCircle{} \SI[separate-uncertainty, multi-part-units = single]{63,2(1)}{m}

\textcolor{Red}\faTimesCircle{}  $\SI{63,2}{m} \pm \num{0,1}$


\textcolor{Green}\faCheckSquare{} \SIrange{4}{20}{mA}

\textcolor{Green}\faCheckSquare{} \SIrange[range-units = brackets]{4}{20}{mA}

\textcolor{Red}\faTimesCircle{} \SIrange[range-units = single]{4}{20}{mA}


\textcolor{Green}\faCheckSquare{} \SI{800 x 600 x 300}{mm}

\textcolor{Green}\faCheckSquare{} \SI[product-units = brackets]{800 x 600 x 300}{mm}

\textcolor{Red}\faTimesCircle{} \SI[product-units = single]{800 x 600 x 300}{mm}


\textcolor{Green}\faCheckSquare{} 127\unit{s} + 3\unit{s} = 130\unit{s}

\textcolor{Green}\faCheckSquare{}  (127 + 3)\unit{s} = 130\unit{s}

\textcolor{Red}\faTimesCircle{}  127 + 3\unit{s} = 130\unit{s}


\textcolor{Green}\faCheckSquare{} \SI[separate-uncertainty, multi-part-units = repeat]{70(5)}{\%}

\textcolor{Green}\faCheckSquare{} \SI[separate-uncertainty]{70(5)}{\%}

\textcolor{Red}\faTimesCircle{} \SI[separate-uncertainty, multi-part-units = single]{70(5)}{\%}


\textcolor{Green}\faCheckSquare{} $240 \times (1 \pm 10\unit{\%})\unit{V}$

\textcolor{Red}\faTimesCircle{}  $240\unit{V} \pm 10\unit{\%}$


\item Cal evitar la utilització de «ppm» (parts per milió) i «ppb» (parts per bilió). El cas «ppb» és especialment problemàtic, ja que 1 bilió equival a $10^{12}$ a Europa continental i a altres països, mentre que 1 bilió equival a $10^{9}$ a la Gran Bretanya, als Estats Units d'Amèrica i a altres països.

\textcolor{Green}\faCheckSquare{} La concentració d'àcid en aigua és de 25\unit{\micro L/L}

\textcolor{Red}\faTimesCircle{} La concentració d'àcid en aigua és de 25 ppm

\end{itemize}


\section{Factors de conversió d'unitats}
\index{sistema internacional d'unitats!factors de conversió}
Donat que la quantitat d'unitats existents és enorme, tenint en compte tant les que pertanyen l'SI com les que no, es dóna en aquest apartat l'adreça de la pàgina web del NIST «National Institute of Standards and Technology», on hi ha recollits un bon nombre de factors de conversió d'unitats que són rellevants en el món de la ciència i l'enginyeria.

La pàgina web en  qüestió, \href{http://www.nist.gov/pml/pubs/sp811/appenb.cfm}{www.nist.gov/pml/pubs/sp811/appenb.cfm}, correspon a l'apèndix B de la publicació «NIST Guide to the SI».

Dins d'aquesta pàgina web, l'enllaç \href{http://www.nist.gov/pml/pubs/sp811/appenb8.cfm}{B.8} ens porta a una llista de factors de conversió ordenada alfabèticament, i l'enllaç  \href{http://www.nist.gov/pml/pubs/sp811/appenb9.cfm}{B.9} ens porta a la mateixa llista de factors de conversió, però ordenada per categories. 