\chapter*{Historial}
\addcontentsline{toc}{chapter}{Historial}

Es presenta a continuació l'evolució que ha tingut aquest llibre en
les successives versions que han aparegut.

\section*{Versió 1.0 (8 de gener de 2005)}
\addcontentsline{toc}{section}{Versió 1.0}

Després de molts esforços, surt a la llum la primera versió d'aquest
llibre, format pels capítols 1, 2, 3, 4, 5, 6 i 7, i els apèndixs A,
B, C, D i E.

\section*{Versió 1.1 (8 de febrer de 2005)}
\addcontentsline{toc}{section}{Versió 1.1}

S'afegeix al llibre aquest apartat «Historial».

En l'apartat Notació, s'especifica que el mòdul d'un nombre
complex és igual a l'arrel quadrada positiva de la suma dels
quadrats de les seves parts real i imaginària.

Es modifiquen les equacions (1.51) i (1.52).

S'amplia la secció 5.5, la qual explica les diferències entre les
normatives CEI i IEEE que fan referència als
transformadors de mesura i protecció.

Es revisa tot el text fent-hi algunes petites modificacions i
correccions.

\section*{Versió 1.2 (16 d'abril de 2005)}
\addcontentsline{toc}{section}{Versió 1.2}

En l'apartat Notació, s'afegeix l'explicació de la convenció
seguida a l'hora de dibuixar les fletxes que representen les
tensions i els corrents.

S'afegeix l'apèndix F, on s'explica la designació de les classes de
refrigeració en els transformadors de potència.

\section*{Versió 1.3 (24 d'octubre de 2005)}
\addcontentsline{toc}{section}{Versió 1.3}

Els apèndixs A a F de la versió 1.2 es desplacen tres lletres cap
avall i passen a ser els apèndixs D a I respectivament.

S'afegeix un nou apèndix A dedicat a l'alfabet grec.

S'afegeix un nou apèndix B dedicat al Sistema Internacional
d'unitats (SI).

S'afegeix un nou apèndix C dedicat a les constants físiques.

En l'apartat Notació, s'amplien les definicions corresponents al
conjugat i al mòdul d'un nombre complex, i s'inclouen les
definicions de $\mcmplx{V}^*$ i $\hermit{V}$.

S'ha ampliat la secció 1.3, corresponent a la
potència complexa.

 S'ha ampliat l'exemple de la secció 3.2.

En la secció 3.3, s'ha afegit el càlcul de $R\ped{P}$ i
$\cmplx{Z}\ped{S}$.

 A l'hora de referir-se a la
relació de transformació d'un transformador, se substitueix el
símbol «$\ddot{u}$» emprat en les versions anteriors, pel símbol
«$m$».

\section*{Versió 1.4 (2 de desembre de 2005)}
\addcontentsline{toc}{section}{Versió 1.4}

Es representa correctament la figura 1.7, la qual estava
tallada per la dreta.

Es corregeix l'equació (4.9a) i l'exemple que hi
ha a continuació, el qual en fa ús.

Es revisa tot el text fent-hi algunes correccions.

\section*{Versió 2.0 (3 d'agost de 2006)}
\addcontentsline{toc}{section}{Versió 2.0}

S'ha modificat el criteri de colors utilitzat, a l'hora de ressaltar
els enllaços interns del document (equacions, pàgines, etc.) i els
enllaços externs; ara els enllaços interns són de
\textcolor{red}{color vermell} i els enllaços externs són de
\textcolor{magenta}{color magenta}. A més, tots els encapçalaments
de capítols, seccions,
 subseccions, taules  i figures, són ara de
 \textcolor{NavyBlue}{color blau}.

S'han afegit nous capítols i s'ha fet una reordenació que afecta 
diversos capítols i apèndixs, segons es detalla a continuació:
\begin{itemize}
   \item Els capítols 1 i 2  de la versió 1.4 mantenen la seva posició.
   \item S'afegeix un nou capítol 3 dedicat a les sèries de Fourier.
   \item S'afegeix un nou capítol 4 dedicat a la transformada de Laplace.
   \item El capítol 3 de la versió 1.4 es desplaça dos números cap
    avall i passa a ser el capítol 5.
   \item L'apèndix E de la versió 1.4 es converteix en el capítol 6.
   \item Els capítols 4, 5, 6 i 7  de la versió 1.4 es desplacen tres números cap
    avall i passen a ser els capítols 7, 8, 9 i 10 respectivament.
    \item L'apèndix G de la versió 1.4 es converteix en el capítol 11.
    \item Els apèndixs A, B, C i D de la versió 1.4 mantenen la seva posició.
    \item S'afegeix un nou apèndix E dedicat a les relacions trigonomètriques.
    \item L'apèndix F de la versió 1.4 manté la seva posició.
    \item Els apèndixs H i I de la versió 1.4 es desplacen una lletra cap
    amunt i passen a ser els apèndixs G i H respectivament.
\end{itemize}


 A l'hora de referir-se a la font de corrent i a l'admitància d'un circuit equivalent
 Norton, se substitueix el subíndex «Th» emprat en les versions
anteriors, pel subíndex «No».

En l'apartat Notació s'afegeixen els símbols: $\vvmathbb{N}$,
$\vvmathbb{Z}$, $\vvmathbb{Z}^+$,  $\vvmathbb{Z}^*$, $\vvmathbb{Z}^-$,
$\vvmathbb{Q}$, $\vvmathbb{R}$, $\vvmathbb{R}^+$, $\vvmathbb{R}^-$ i
$\vvmathbb{C}$.

S'ha afegit el teorema de la superposició en la secció 1.1.


S'ha afegit la bateria en la secció 1.2, com a un
dels components elementals d'un circuit elèctric.

S'ha afegit la secció 1.4, on es defineixen els
valors mitjà i eficaç, i els factors d'amplitud, de forma i
d'arrissada.

S'ha afegit la secció 1.5 dedicada als
circuits divisors de tensió i divisors de corrent.

 S'ha modificat l'equació (7.2),
i les taules 7.1 i 7.5.

S'ha afegit la secció 8.6, on s'explica com
connectar correctament transformadors de corrent i de tensió, a
aparells de mesura i de protecció.

S'ha millorat l'explicació de la secció 10.5.

S'ha reestructurat la taula B.2.

\section*{Versió 2.1 (2 de gener de 2007)}
\addcontentsline{toc}{section}{Versió 2.1}

S'adopta la compaginació moderna dels paràgrafs en tot el llibre, consistent en separar-los per una línia en blanc, i sense indentació de la primera línia de text.

S'unifica la representació de les fonts de corrent: un cercle amb una fletxa a dins.

S'afegeix una nota a peu de pàgina en la secció 1.1.1 que relaciona aquesta secció amb la secció 9.5.

Es millora l'explicació de la secció 1.6, alhora que es trasllada de lloc (en les versions anteriors formava part del capítol 5).

Es millora l'explicació de la secció 2.4.

S'afegeix una nota a peu de pàgina en la secció 5.2 que relaciona aquesta secció amb el capítol 10.

S'amplia la descripció de l'equació (7.25).

S'afegeix la secció 10.6, on s'explica com resoldre sistemes d'equacions no lineals amb els programes Mathematica®  i MATLAB®.

Es millora l'explicació de la secció E.2, modificant la figura E.1 i numerant l'equació de la llei dels sinus.

\section*{Versió 2.2 (10 de març de 2008)}
\addcontentsline{toc}{section}{Versió 2.2}

Es canvia el color dels enllaços interns i passen a ser de color negre com el text.

S'afegeixen les unitats que mancaven en alguns exemples.

En la secció 7.4.1, s'introdueixen les unitats cmil i kcmil, equivalents a les unitats CM i MCM respectivament; avui dia és més freqüent veure escrit cmil i kcmil que no pas CM i MCM.

Es revisa l'apèndix B utilitzant les publicacions de l'any 2006 del \textit{Bureau
International des Poids et Mesures} (BIPM).

Es revisa l'apèndix C utilitzant les publicacions de l'any 2006 del \textit{Committee on Data for Science and Technology} (CODATA).

\section*{Versió 3.0 (1 d'octubre de 2008)}
\addcontentsline{toc}{section}{Versió 3.0}

Els capítols 9, 10 i 11 de la versió 2.2 es desplacen un número cap
avall i passen a ser els capítols 10, 11 i 12 respectivament.

Es crea un nou capítol 9 dedicat als transformadors de potència;
l'apèndix H de la versió 2.2 desapareix com a tal i queda integrat
dins d'aquest nou capítol.


\section*{Versió 3.1 (5 de desembre de 2009)}
\addcontentsline{toc}{section}{Versió 3.1}
En l'apèndix B s'afegeixen els prefixos de potències binàries Ki, Mi, Gi, Ti, Pi i Ei.

Es revisa tot el text fent-hi algunes petites modificacions i
correccions.

\section*{Versió 3.2 (5 de gener de 2010)}
\addcontentsline{toc}{section}{Versió 3.2}
S'afegeix l'apartat Bibliografia després dels apèndixs.


\section*{Versió 4.0 (15 de febrer de 2010)}
\addcontentsline{toc}{section}{Versió 4.0}
A partir d'aquesta versió s'utilitza la font \texttt{Kp-Fonts} en la composició de tot el text. Fins ara, les fonts utilitzades eren les \texttt{Pazo Math}, \texttt{Helvetica} i \texttt{Courier}.


\section*{Versió 4.1 (27 de febrer de 2010)}
\addcontentsline{toc}{section}{Versió 4.1}
En el capítol dedicat a la transformada de Laplace, es modifiquen segons \cite{SCH} algunes definicions  i s'amplien les taules de transformades de Laplace segons \cite{SCH} i \cite{RASd}.

\section*{Versió 4.2 (12 de març de 2010)}
\addcontentsline{toc}{section}{Versió 4.2}
En el capítol dedicat a les sèries de Fourier, es completa l'equació (3.7c) i s'afegeix una taula amb les sèries de Fourier de formes d'ona usuals.

En l'apèndix dedicat a les funcions trigonomètriques, se simplifiquen les equacions (E.18) i (E.19).

\section*{Versió 4.3 (27 de novembre de 2010)}
\addcontentsline{toc}{section}{Versió 4.3}
Els apèndixs de la versió 4.2 dedicats al grau de protecció IP i  a les classes NEMA d'aïllaments tèrmics en motors, passen a formar part del capítol 12; aquest capítol canvia de nom i passa a dir-se «Normatives Diverses».

L'apèndix de la versió 4.2 dedicat a les escales logarítmiques, passa a formar part del capítol 5 dedicat a càlculs bàsics.

Es modifica l'adreça de correu electrònic de contacte amb l'autor.

\section*{Versió 4.4 (31 de març de 2011)}
\addcontentsline{toc}{section}{Versió 4.4}
En el capítol 12, s'amplia la descripció dels codis IP i IK, i s'hi afegeix el codi NEMA dedicat al grau de protecció d'equips.

Es modifica l'adreça de correu electrònic de contacte amb l'autor.

\section*{Versió 4.5 (2 de novembre de 2011)}
\addcontentsline{toc}{section}{Versió 4.5}
En l'apartat Notació, s'afegeixen diverses relacions referents a $|\cmplx{V}|$, $\arg\cmplx{V}$, $\Re\cmplx{V}$ i $\Im\cmplx{V}$.

Es modifiquen les equacions (3.7c), (D.18a) i (D.18b).

En el capítol 3, es millora l'explicació de les propietats de les sèries de Fourier.

En el capítol 12, s'afegeixen dues seccions dedicades a l'àmbit d'aplicació de diverses normes CEI i IEEE.

En l'apartat Bibliografia, s'afegeix la referència \cite{RJB}


\section*{Versió 4.6 (21 de novembre de 2011)}
\addcontentsline{toc}{section}{Versió 4.6}

En l'apèndix dedicat a l'alfabet grec, s'utilitza el \textit{Diccionari de la llengua catalana, 2a edició, 2007} (DIEC2), com la referència per escriure els noms de les lletres gregues en català. S'afegeix també el nom de les lletres gregues en francès.


\section*{Versió 5.0 (30 de gener de 2012)}
\addcontentsline{toc}{section}{Versió 5.0}

Es modifica lleugerament el nom del llibre i  passa a dir-se \textit{Qüestions Electrotècniques Diverses} en lloc de \textit{Qüestions Diverses d'Electrotècnia}, i, per tant, a parir d'ara es podrà denominar amb l'acrònim «QED» (\emph{Quod Erat Demonstrandum}).

Es canvia la tipografia dels exemples utilitzada en les versions anteriors, i passen ara a ser escrits en lletra recta en lloc d'en lletra inclinada.

El símbol $\measuredangle$ utilitzat per indicar l'argument d'un valor complex en les versions anteriors, es canvia pel símbol $\angle$ d'acord amb la norma internacional ISO 80000 \textit{Quantities and units}, la qual substitueix a l'obsoleta ISO 31.

Es canvia en tot el text el terme «vector» pel terme «fasor» quan es fa referència a magnituds sinusoidals.

S'indica en el prefaci que s'ha utilitzat la distribució MiK\TeX, la qual ofereix una implementació lliure de \LaTeX.

S'afegeix en l'apartat Notació la definició d'un fasor.

Es modifica l'equació (7.25).

S'amplia la secció 9.7.2.

S'afegeixen algunes normes en les seccions (12.6) i (12.7).

S'inclou en la taula A.1 i en l'explicació posterior, la representació gràfica $\varkappaup$ de la lletra minúscula kappa.

Es modifica en la taula B.6 el valor en unitats SI de la unitat de massa atòmica unificada.

En l'apartat Bibliografia, s'afegeixen les referències \cite{GRZ}, \cite{DUN}, \cite{REI} i \cite{TLE}.

Es revisa tot el text fent-hi algunes correccions.

\section*{Versió 5.1 (15 de febrer de 2012)}
\addcontentsline{toc}{section}{Versió 5.1}

Es millora en l'apartat Notació, la definició de l'angle $\alpha$ d'un fasor.

S'amplia la secció 1.6 dedicada als càlculs en per unitat.


\section*{Versió 5.2 (4 de maig de 2012)}
\addcontentsline{toc}{section}{Versió 5.2}

Es completa l'equació (1.75).

En el capítol 12, s'afegeix una secció dedicada als interruptors automàtics de baixa tensió segons les normes CEI.

S'afegeixen algunes normes en la secció 12.8.

\section*{Versió 5.3 (14 de juliol de 2012)}
\addcontentsline{toc}{section}{Versió 5.3}

S'amplia la secció D.2, afegint-hi la llei de les cotangents i la fórmula de Mollweide, i modificant  la figura D.1.

S'afegeixen algunes normes en la secció 12.8.

\section*{Versió 5.4 (2 de novembre de 2012)}
\addcontentsline{toc}{section}{Versió 5.4}

Es canvia de forma general el símbol «$\cdot$» pel símbol «$\times$», quan es tracta d'expressar la multiplicació de dos valors numèrics.

Es revisa l'apèndix B, sobretot en l'apartat referent a les normes d'escriptura.

Es revisa l'apèndix C utilitzant les publicacions de l'any 2010 del \textit{Committee on Data for Science and Technology} (CODATA).

\section*{Versió 5.5 (1 de desembre de 2012)}
\addcontentsline{toc}{section}{Versió 5.5}

En la secció 8.5 es referencia la norma  IEEE C57.13, en lloc de la més antiga ANSI C57.13.

En la secció 9.10 es referencia la norma  IEEE C57.12.00, en lloc de la més antiga ANSI C57.12.

Es posa al dia la secció 12.1 segons la norma IEEE C37.2, en lloc de la més antiga ANSI C37.2.

S'afegeixen algunes normes en les seccions 12.7 i 12.8.

Es modifica  la figura D.1.


\section*{Versió 6.0 (2 de gener de 2013)}
\addcontentsline{toc}{section}{Versió 6.0}

Es realitza una revisió general del text i de les figures d'aquest  llibre, utilitzant la simbologia de les normes CEI/ISO 80000  \textit{Quantities and units} i  CEI 60617  \textit{Graphical Symbols for Diagrams}.

S'amplia la secció 1.4 utilitzant les definicions de la norma CEI 60050.

Es completen les equacions (1.72) i (1.74).

Es modifica l'equació (1.79).

S'amplia la secció  3.4 utilitzant les definicions de la norma CEI 60050.

Es realitzen les modificacions següents en l'apèndix B:
\begin{itemize}
   \item  S'inclou la referència al Reial decret 2032/2009, de 30 de desembre.
   \item S'indica que les variants ortogràfiques  «kilogram»
    i «quilogram», «kilo» i «quilo», «radian» i «radiant», i
   «estereoradian» i «estereoradiant», són equivalents segons el   \textit{Diccionari de la llengua catalana, 2a edició, 2007} (DIEC2).
    \item S'escriu correctament el nom de la unitat «electró-volt». El nom  utilitzat en edicions anteriors,   «electronvolt», no apareix en el \textit{Diccionari de la llengua catalana, 2a edició, 2007} (DIEC2).
    \item S'inclou l'adreça d'Internet de l'\textit{International Earth rotation and Reference systems Service}.
     \item Es refà l'apartat dedicat a les normes d'escriptura.
\end{itemize}

Es refà la taula de l'apèndix C agrupant els valors numèrics i les seves unitats, i s'explica a continuació com obtenir els errors absoluts i relatius dels valors que hi apareixen.

\section*{Versió 6.1 (1 de febrer de 2013)}
\addcontentsline{toc}{section}{Versió 6.1}

S'afegeix un segon exemple en la secció 1.1.2, dedicat al teorema de Millman.


\section*{Versió 6.2 (11 de setembre de 2013)}
\addcontentsline{toc}{section}{Versió 6.2}

 Es revisa el capítol 8 utilitzant la norma CEI 60044 en lloc de les normes CEI 60185 i CEI 60186, les quals ja no estan en vigor.

Es crea la secció 9.11 per explicar com es formen els circuits homopolars dels transformadors de potència de dos i tres debanats.

En la secció 12.7 s'eliminen les normes CEI 60185 i CEI 60186, les quals ja no estan en vigor.

S'afegeixen els prefixos «zebi» i «yobi» a la taula B.9.

En l'apartat Bibliografia, s'afegeix la referència \cite{RASe}.

\section*{Versió 6.3 (24 de març de 2014)}
\addcontentsline{toc}{section}{Versió 6.3}

S'inclou el període $T$ en el gràfic de l'apartat Notació.

S'amplia la secció 7.4 dedicada a la capacitat tèrmica dels cables en curtcircuit.

S'amplia el capítol 8 dedicat als transformadors de mesura i protecció.

S'afegeixen algunes normes en les seccions 12.7 i 12.8.

En l'apartat Bibliografia, s'afegeixen les referències \cite{KAS} i \cite{JCD}.

Es revisa tot el text fent-hi algunes correccions. A més, es modifica la presentació de tots els exemples, emmarcant-los dins d'un rectangle.


\section*{Versió 7.0 (24 de juliol de 2014)}
\addcontentsline{toc}{section}{Versió 7.0}

En l'apartat Notació, s'afegeixen dues  relacions referents a la representació de nombres complexos en format exponencial.

Es creen i reordenen diversos capítols i seccions, segons es detalla a continuació:
\begin{itemize}
   \item El capítol 1 es queda amb les primeres cinc seccions reordenades, de les set que tenia la versió 6.3. S'afegeix a aquest capítol una sisena secció nova, dedicada als circuits R-L-C.
   \item El capítol 5 de la versió 6.3 passa a ser el capítol 2, reordenant-ne les  seccions i incorporant-hi les dues últimes seccions del capítol 1 de la versió 6.3.
   \item Els capítols 2, 3 i 4  de la versió 6.3 es desplacen un número cap avall.
   \item Es crea un nou apèndix, dedicat al càlcul numèric.
\end{itemize}

S'amplia la secció 7.5.2.

S'amplia el capítol 8 dedicat als transformadors de mesura i protecció.

S'afegeixen algunes normes en la secció 12.8.

En l'apartat Bibliografia, s'afegeixen les referències \cite{GOM}, \cite{SPK}, \cite{JDH}, \cite{EJB},  \cite{PMA}, \cite{MAI} i \cite{KNU}.

Es modifica l'adreça de correu electrònic de contacte amb l'autor.


\section*{Versió 7.1 (23 d'octubre de 2014)}
\addcontentsline{toc}{section}{Versió 7.1}

S'afegeixen algunes normes en la secció 12.8.


\section*{Versió 8.0 (9 de novembre de 2014)}
\addcontentsline{toc}{section}{Versió 8.0}

Es refan tots els dibuixos del llibre utilitzant el programa \emph{Inkscape}; aquest programa de dibuix vectorial és de distribució lliure, i pot obtenir-se a l'adreça: \href{http://www.inkscape.org/}{www.inkscape.org}. En totes les versions anteriors del llibre s'havia utilitzat el programa jPicEdt, el qual també és de distribució lliure i pot obtenir-se a l'adreça: \href{http://www.jpicedt.org/}{www.jpicedt.org}.

Es canvien en l'apartat Notació, els noms de les variables utilitzades en la definició d'un fasor.

En la secció 1.4.2, dedicada a la potència trifàsica, se substitueixen els subíndexs «$\alphaup$», «$\betaup$», «$\gammaup$» i «$\nuup$», pels subíndexs «A», «B», «C» i «N», a l'hora d'identificar les tres fases  i el neutre d'un sistema trifàsic.

En la secció 2.3.1,  dedicada als circuits divisors de tensió, es canvien els noms de les variables utilitzades.

En la secció 2.3.2,  dedicada als circuits divisors de corrent, es canvien els noms de les variables utilitzades.

En la secció 2.4, dedicada a la transformació estrella-triangle d'impedàncies, se substitueixen els subíndexs «$\alphaup$», «$\betaup$» i «$\gammaup$», pels subíndexs «A», «B» i «C», a l'hora d'identificar les tres fases d'un sistema trifàsic.

En el capítol 3, dedicat a les components simètriques, se substitueixen els superíndexs «(1)», «(2)» i «(0)», pels subíndexs «1», «2» i «0», a l'hora d'identificar les components directa, inversa i homopolar. A més, també se substitueixen els subíndexs «$\alphaup$», «$\betaup$», «$\gammaup$» i «$\nuup$», pels subíndexs «A», «B»,  «C» i «N», a l'hora d'identificar les tres fases i el neutre d'un sistema trifàsic.

S'afegeix l'equació (9.51) per tal d'explicar millor la compatibilitat entre els índexs horaris de dos transformadors.

\section*{Versió 8.1 (16 de novembre de 2014)}
\addcontentsline{toc}{section}{Versió 8.1}

Es modifica el gruix i l'estil de línia d'alguns dibuixos del llibre, per tal de fer-los més uniformes.

Es numeren les figures de les seccions 9.12.1 i 9.12.2.


\section*{Versió 8.2 (23 de novembre de 2014)}
\addcontentsline{toc}{section}{Versió 8.2}

S'afegeix la secció E.3, dedicada a la solució de funcions no lineals.

\section*{Versió 8.3 (11 de desembre de 2014)}
\addcontentsline{toc}{section}{Versió 8.3}

Es millora en l'apartat Notació, l'explicació de la definició d'un fasor.


\section*{Versió 8.4 (3 de gener de 2015)}
\addcontentsline{toc}{section}{Versió 8.4}

Es millora l'explicació de la secció 1.6.5.

S'amplia la secció 2.3, afegint-hi el cas particular de dues impedàncies.

Es crea la secció 6.3 dedicada a les potències normalitzades de les resistències.

S'expressen correctament les equacions (E.1) i (E.2).

En l'apartat Bibliografia, s'afegeixen les referències \cite{AGVS}, \cite{JSch} i \cite{RRop}.


\section*{Versió 9.0 (29 d’agost de 2016)}
\addcontentsline{toc}{section}{Versió 9.0}

Es modifiquen els estils dels textos de les capçaleres de les taules i dels peus de les figures, per tal que siguin iguals que els estils dels títols dels capítols, seccions i subseccions.

Es modifica en tot el llibre la manera de representar una variable acompanyada de les seves unitats. Les variables se separaran de les seves unitats mitjançant el símbol de divisió «/», en lloc de tancar les unitats entre «[» i  «]»; per exemple, en lloc de $S\unit{[mm^2]}$, a partir d'ara escriurem $S/\unit{mm^2}$.

Es modifica en tot el llibre la posició de les notes que fan referència a elements d'una taula, coŀlocant-les immediatament a sota de la mateixa taula en lloc de fer-ho al peu de pàgina.

Es refan totes les gràfiques de funcions del llibre utilitzant el programa \emph{gnuplot}; aquest programa de dibuix de gràfiques de funcions és de distribució lliure, i pot obtenir-se a l'adreça: \href{http://www.gnuplot.info/}{www.gnuplot.info}. En totes les versions anteriors del llibre s'havia utilitzat el paquet d'ampliació de \LaTeX{} \texttt{PSTricks}.

S'amplien algunes seccions i exemples del llibre, afegint-hi una  resolució numèrica mitjançant la calculadora \textsf{HP Prime};  aquesta calculadora disposa d'un emulador per a PC que pot descarregar-se de la pàgina de Hewlett-Packard: \href{http://www.hpprime.de/en/category/6-downloads}{www.hpprime.de/en/category/6-downloads}. Les seccions i els exemples afectats són els següents:
\begin{itemize}
  \item L'exemple 1.8  (Corrent de pic de curtcircuit).
  \item L'exemple 1.9 (Resolució de xarxes amb el mètode de les malles).
  \item L'exemple 2.3 (Resolució de circuits coneixent la potència absorbida).
  \item La secció 3.7 (Components simètriques)
  \item L'exemple 4.3 (Sèries de Fourier).
  \item L'exemple 5.4 (Transformada de Laplace).
  \item L'exemple 10.1 (Resolució de xarxes utilitzant el mètode dels nusos).
  \item La secció 11.6 (Flux de càrregues).
\end{itemize}

S'amplia la secció 1.2.2 dedicada al teorema de Millman, afegint-hi un exemple més al final.

Es millora l'explicació de la secció 1.6.5.

Es crea la secció 1.7 dedicada a la resolució de xarxes elèctriques, utilitzant el mètode de les malles.

S'amplia la secció 2.7 dedicada a les escales logarítmiques, afegint-hi al final un nou apartat, dedicat a la determinació dels paràmetres de funcions que prenen la forma d'una recta en gràfiques d'escala logarítmica-logarítmica.

Es millora l'explicació de l'exemple 3.1.

Es crea la secció 3.7 on es descriuen diversos programes de la calculadora
\textsf{HP Prime} relacionats amb les components simètriques.

S'amplia l'exemple 4.3, afegint-hi al final  una nova gràfica.

S'amplia el capítol 6, afegint-hi la codificació del coeficient de variació amb la temperatura de les resistències, i la norma CEI que defineix les sèries de resistències estàndard.

Es modifiquen les capçaleres de totes les taules del capítol 8, ja que els percentatges d'error de tensions i corrents que s'hi indicaven, estaven referits incorrectament als valors nominals dels transformadors.

Es modifiquen les equacions que fan referència a les figures 9.1 i 9.11, perquè es vegi millor la seva correspondència.

S'afegeixen algunes normes en la secció 12.7.

En l'apèndix A dedicat a l'alfabet grec, s'utilitza el  
\textit{Diccionario de la Lengua Española, 23ª
edición, 2014} (D.R.A.E.), com la referència per escriure els noms de les lletres gregues en castellà.

S'amplia l'apèndix B, afegint-hi al final una secció dedicada als factors de conversió d'unitats.

Es revisa la taula B.6 i l'apèndix C, utilitzant les publicacions de l'any 2014 del \textit{Committee on Data for Science and Technology} (CODATA).


S'amplia la secció D.2, afegint-hi les equacions de les coordenades del baricentre d'un triangle, i es modifica de manera corresponent la figura D.1.

Es modifiquen les figures E.1 i E.2.

En l'apartat Bibliografia, s'afegeixen les referències \cite{VOS}, \cite{WMF} i \cite{TRA}.


\section*{Versió 9.1 (27 de novembre de 2016)}
\addcontentsline{toc}{section}{Versió 9.1}

Es revisa tot el text fent-hi algunes  modificacions i correccions.

Es crea la figura 1.4, on es representen els paràmetres d'una funció periòdica qualsevol.

Es modifica en la secció 3.7 la funció \texttt{Triangle→Fasors}, fent-la més simple.

Es milloren les equacions (7.27) i (7.28).

Es creen les equacions (7.29) i (7.30), i un exemple de com utilitzar-les.

Es dona color a la taula D.1, per tal de distingir millor els valors positius dels negatius.

\section*{Versió 10.0 (6 de gener de 2017)}
\addcontentsline{toc}{section}{Versió 10.0}

Es modifica en tot el llibre la manera de representar el producte de dues unitats; en les versions anteriors s'havia utilitzat un punt volat, i a partir d'ara es farà servir un espai en blanc. Ambdues formes són correctes, però l'espai en blanc és la forma usada preferentment en les publicacions del  \textit{Bureau International des Poids et Mesures} (BIPM). Els apèndixs B i C són els més afectats per aquest canvi.

Es modifica en tot el llibre el símbol de la unitat  utilitzada per a la potència reactiva; en les versions anteriors s'havia emprat el símbol «VAr», i a partir d'ara es farà servir el símbol «var», ja que és el que adopta la norma CEI 80000-6.

Es modifica en tot el llibre la manera d'escriure les funcions $\Re$, $\Im$ i $\arg$, quan van seguides d'una única variable; en les versions anteriors s'havien emprat, per exemple, les formes $\Re(\cmplx{S})$, $\Im(\cmplx{S})$ i $\arg(\cmplx{S})$, i a partir d'ara es faran servir les formes $\Re\cmplx{S}$, $\Im\cmplx{S}$ i $\arg\cmplx{S}$ respectivament.


Es millora en l’apartat Notació, la definició de l’angle $\alpha$ d’un fasor.


Es completa l'exemple 1.6, calculant-hi al final les potències activa i reactiva.

Es completa la secció 1.6.1, afegint-hi les equacions corresponents a tenir el condensador carregat en l'instant inicial, a una tensió no nuŀla.

Es completa la secció 1.6.3, afegint-hi les equacions corresponents a tenir circulant per la inductància en l'instant  inicial, un corrent no nul.

Es crea l'exemple 1.7, en el qual es calcula el corrent i la tensió de càrrega i descàrrega d'un circuit R-L.

Es completa l'exemple 1.9, calculant-hi per separat els valors de $\kappa$ i $\hat{I}\ped{asim}$.


Es completa l'exemple 4.3, afegint-hi al final el càlcul del corrent $i(t)$ utilitzant les equacions de la secció 1.6.3.

Es completa l'exemple 10.1, afegint-hi al final la resolució del sistema d'equacions lineals amb la funció \texttt{simult}.


En l'apèndix A, es donen les adreces d'Internet dels diccionaris utilitzats per escriure els noms de les lletres gregues en anglès i francès. Addicionalment, es corregeix el nom en francès de la lletra $\varpiup$; el nom correcte és \textit{pi dorien}.

En l'apèndix B, es té en compte el suplement de l'any 2014 publicat pel  \textit{Bureau International des Poids et Mesures} (BIPM), el qual posa al dia la 8a edició de les seves publicacions de l'any 2006. Els canvis introduïts que afecten aquest llibre, són els següents:
\begin{itemize}
  \item Es modifica l'ordre de les unitats base, en l'expressió de les unitats derivades. Aquest canvi afecta les taules B.3 i B.4.
  \item La unitat astronòmica de longitud va ser redefinida l'any 2012 en la 28a Assemblea General de la Unió Astronòmica Internacional, passant a ser un valor exacte. Aquest canvi ocasiona que la unitat astronòmica passi de la taula B.6  a la taula  B.5.
\end{itemize}

Es crea la taula B.8 per recollir les unitats fora de l'SI acceptades addicionalment pel \textit{National Institute of Standards and Technology} (NIST).

Es crea la secció B.7 per recollir  unitats definides per la norma  CEI/ISO 80000, addicionals a les de l'SI. Algunes d'aquestes unitats estaven incloses anteriorment en la secció B.6.

En la secció B.8, s'utilitza el símbol \textcolor{Blue}\faQuestionCircle{} per indicar escriptures correctes però no recomanades.

\section*{Versió 10.1 (8 de març de 2017)}
\addcontentsline{toc}{section}{Versió 10.1}

S'afegeix una figura al final de la secció 12.1, per iŀlustar la diferència entre les funcions de protecció 50, 50TD i 51, i l'equació de les corbes característiques de la funció de protecció 51, amb els valors dels paràmetres utilitzats per les normes CEI i IEEE.

\section*{Versió 10.2 (7 de maig de 2017)}
\addcontentsline{toc}{section}{Versió 10.2}

S'afegeix al final de la secció 3.5, el càlcul del sistema de tensions fase-neutre $\cmplx{U}\ped{AG}$, $\cmplx{U}\ped{BG}$ i $\cmplx{U}\ped{CG}$, a partir del sistema de tensions fase-fase $\cmplx{U}\ped{AB}$, $\cmplx{U}\ped{BC}$ i $\cmplx{U}\ped{CA}$, i s'utilitzen les equacions obtingudes, al final de l'exemple 3.1.

Es crea el nou exemple 3.2, on es calculen les components simètriques d'un sistema de tensions desequilibrat que alimenta a una càrrega desequilibrada.

\section*{Versió 10.3 (3 de juny de 2017)}
\addcontentsline{toc}{section}{Versió 10.3}

Es millora l'explicació de l'apartat 11.5.

\section*{Versió 10.4 (28 d'agost de 2017)}
\addcontentsline{toc}{section}{Versió 10.4}

Es crea el nou apèndix F, dedicat a programes per a la calculadora \textsf{HP Prime} de Hewlett-Packard. S'eliminen els programes que en les versions anteriors estaven llistats en l'exemple 2.3 i en la secció 3.7, incorporant-se en aquest nou apèndix.

S'afegeix un nou exemple al final de la secció E.1, dedicat  a la interpolació en dues dimensions.

\section*{Versió 10.5 (11 de setembre de 2017)}
\addcontentsline{toc}{section}{Versió 10.5}

S'utilitza la font \textit{true type} \texttt{HPPrime.ttf}, per representar les tecles de la calculadora \textsf{HP Prime} en els diversos exemples d'ús d'aquesta calculadora que hi ha en el llibre.

S'inclou en l'apèndix F una imatge de la calculadora \textsf{HP Prime}.


\section*{Versió 10.6 (1 d'octubre de 2017)}
\addcontentsline{toc}{section}{Versió 10.6}

Es canvia el nom de la part II del llibre, passant a anomenar-se «Equips i Components Elèctrics».

Es modifica l'estil del text de les parts del llibre, unificant-lo amb l'estil del text dels capítols.

En l'apartat Notació, s'inclou una nota per indicar  que el valor de l'angle $\psi$ ha d'expressar-se sempre en radiant, per tal que el valor de $e^{j\psi}$ sigui correcte.

S'afegeix una nota a la taula 6.1 per indicar l'equivalència entre les unitats \unit{ppm/\degreeCelsius}  i \unit{\micro\ohm/(\ohm.\kelvin)}.

Es millora el dibuix de l'exemple 11.2.

S’afegeixen algunes normes en la secció 12.7.

En l'apartat Bibliografia, s'afegeix la referència \cite{BUR}.

Es revisa tot el text fent-hi algunes correccions.


\section*{Versió 10.7 (18 de febrer de 2018)}
\addcontentsline{toc}{section}{Versió 10.7}

Es corregeix la posició del títol de la figura 1.3, i dels títols de les taules 8.7, 12.5, 12.6 i 12.7.

Es millora la representació dels valors de la taula B.5.

\section*{Versió 10.8 (23 de maig de 2018)}
\addcontentsline{toc}{section}{Versió 10.8}

Es fan algunes correccions en el text.


\section*{Versió 11.0 (26 de juny de 2019)}
\addcontentsline{toc}{section}{Versió 11.0}

Després de la portada, s'afegeix una pàgina amb el copyright i una altra amb diverses citacions.

En el Prefaci, s'indica que s’utilitzen programes de càlcul matemàtic i la calculadora \textsf{HP Prime}
per resoldre alguns  exemples del llibre.

Es millora en l'apartat Notació la definició de  fasor, modificant-ne també el dibuix associat.

S'afegeix la taula 1.1, on es donen els valors mitjans i eficaços d'una sèrie de formes d'ona usuals.

Es crea la secció 1.4 dedicada a l'explicació de  les potències instantània, activa i reactiva.

Les seccions 1.4, 1.5, 1.6 i 1.7 de la versió 10.8 es desplacen un número cap
avall i passen a ser les seccions 1.5, 1.6, 1.7 i 1.8 respectivament.

En la secció 1.5, es millora l'explicació del factor de potència.

En les seccions 1.7.1, 1.7.2, 1.7.3 i 1.7.4, s'afegeixen gràfiques que acompanyen a les equacions de les tensions i corrents  dels circuits R-C i R-L que hi apareixen.

Es millora l'explicació de  la secció 2.7.2, dedicada
a la determinació dels paràmetres de funcions que prenen la forma d’una recta en gràfiques d’escala
logarítmica–logarítmica.

S'amplia la secció 9.8 incorporant-hi la subsecció dedicada als circuits homopolar, i afegint-hi una subsecció dedicada a les tensions i corrents de seqüència directa, inversa i homopolar.

Es crea un nou capítol 10 dedicat als motors d'inducció trifàsics.

Els capítols 10, 11 i 12 de la versió 10.8 es desplacen un número cap
avall i passen a ser els capítols 11, 12 i 13 respectivament.

S'elimina del capítol 13 la secció dedicada a les classes NEMA d’aïllaments tèrmics en motors; aquesta secció s'incorpora dins del nou capítol 10 dedicat als motors d'inducció trifàsics.

En la secció B.9, s'afegeix la referència al document \textit{The International System of Units (SI) – Conversion Factors for General Use}.

En l'apèndix F s'afegeix l'adreça de la pàgina de Hewlett-Packard des d'on pot descarregar-se un emulador per a PC de la calculadora \textsf{HP Prime}; s'afegeix a més la funció \texttt{Regla\_dels\_Trapezis}.

En l’apartat Bibliografia, s’afegeixen les referències  \cite{JFM} i \cite{WES}.

Es revisa tot el text fent-hi algunes correccions,  i incorporant-hi els canvis introduïts per l’Institut d’Estudis Catalans en la \emph{Gramàtica de la llengua catalana}
(2016) i en \emph{l’Ortografia catalana} (2017).


\section*{Versió 11.1 (7 de juliol de 2019)}
\addcontentsline{toc}{section}{Versió 11.1}

S'actualitza l'apèndix B a causa de l'entrada en vigor el 20 de maig de 2019, de la nova definició de les unitats fonamentals del Sistema Internacional d'unitats (SI).

S'actualitza l'apèndix  C amb els nous valors de les constants físiques recomanats l'any 2018 pel \textit{Committee on Data for Science and Technology} (CODATA). Aquests canvis de  valors estan  relacionats amb la nova definició de les unitats fonamentals de l'SI, indicada en el paràgraf anterior.

\section*{Versió 11.2 (25 d'agost de 2019)}
\addcontentsline{toc}{section}{Versió 11.2}

Es milloren les figures de l'exemple de la secció 8.8.

Es corregeixen referències creuades errònies en la secció B.8.

\section*{Versió 11.3 (5 de setembre de 2019)}
\addcontentsline{toc}{section}{Versió 11.3}

Es millora la presentació dels índexs general, de taules i de figures.

Es crea un nou índex d'exemples, a continuació de l'índex de figures.

\section*{Versió 11.4 (6 d'octubre de 2019)}
\addcontentsline{toc}{section}{Versió 11.4}

Es modifica l'aparença de totes les gràfiques de funcions, creades amb el programa \emph{gnuplot}.

\section*{Versió 11.5 (25 de novembre de 2019)}
\addcontentsline{toc}{section}{Versió 11.5}

Es modifica la secció 10.4.1, afegint-hi l'equació del factor de potència en funció del lliscament.

Es modifica la secció 10.4.2, afegint-hi les equacions del rendiment i de les potències aparent i reactiva en funció del lliscament, i ampliant-ne l'exemple 10.4.

S'afegeix la secció 10.4.3 dedicada al funcionament de motors alimentats amb tensió desequilibrada.

S'afegeix la secció 10.4.4 dedicada al funcionament de motors alimentats amb tensió reduïda, o que subministren una potència reduïda.

Es revisa i simplifica la secció 10.5.3.

S'estandarditza l'ús dels subíndexs «N» i  «n» en tot el llibre. A partir d'ara s'utilitzarà  «N» per indicar  «neutre», i  «n» per indicar   «nominal».


\section*{Versió 11.6 (18 de gener de 2020)}
\addcontentsline{toc}{section}{Versió 11.6}

S'amplia la secció 10.4.4, dedicada a motors que funcionen a tensió o potència reduïda.

S'amplia el contingut de l'exemple 10.11.

S'afegeixen els valors exactes dels factors de la taula B.8.


\section*{Versió 11.7 (21 de juliol de 2020)}
\addcontentsline{toc}{section}{Versió 11.7}

Es millora l'exemple 1.1.

Es modifica la figura 2.4.

En la secció B.8, es canvia el símbol \textcolor{Blue}\faQuestionCircle{} pel símbol \textcolor{Blue}\faExclamationTriangle{} a l'hora d'indicar escriptures correctes però no recomanades.

Es revisa tot el text fent-hi algunes correccions.


\section*{Versió 12.0 (9 de gener de 2021)}
\addcontentsline{toc}{section}{Versió 12.0}

Es revisa tot el text utilitzant l'eina de correcció ortogràfica i gramatical \textsf{LanguageTool}, la qual està disponible a l'adreça \href{https://www.languagetool.org}{www.languagetool.org}.


\section*{Versió 13.0 (20 de juliol de 2021)}
\addcontentsline{toc}{section}{Versió 13.0}

En l'apartat Notació, es modifiquen les definicions dels diferents conjunts de nombres i intervals, per adaptar-les a la norma internacional ISO 80000-2 \textit{Quantities and units --- Part 2: Mathematics}. Aquest canvi es fa extensiu a la resta del llibre.

Es reordenen les seccions del capítol 1.

En les figures de l'exemple 1.4, s'indiquen com a quantitats complexes les impedàncies de la inductància i del condensador.

Es millora la figura 1.9 afegint-hi la lletra $\alphaup$ que faltava  a baix a l'esquerra.

Es millora la informació representada en la figura 1.10.

Es crea el nou exemple 1.6.

Es revisa el capítol 5 per actualitzar la simbologia utilitzada en la representació de la transformada de Laplace.

Es revisa la secció 6.1 per adaptar la codificació en colors de les resistències a l'edició de l'any 2016 de la norma CEI 60062 \textit{Marking codes for resistors and capacitors}.

Es revisa la secció 6.2 per adaptar les sèries de resistències estandarditzades a l'edició de l'any 2015 de la norma CEI 60063 \textit{Preferred number series for resistors and capacitors}.

S'amplia l'exemple 6.1.

Es corregeixen les unitats dels exemples 7.2 i 7.4.

S'afegeixen les unitats que faltaven en el text i els exemples del capítol 10. 

Dins del capítol 10 es crea la nova secció 10.6 per tractar la compatibilitat entre els motors de \qty{50}{Hz} i els de \qty{60}{Hz}.

Es millora la representació i l'explicació de l'equació (10.73).

S'afegeixen les unitats que faltaven en el text i els exemples del capítol 11.

Es millora l'explicació dels paràmetres de l'equació  (13.1).

En la  secció 13.3, es canvia la norma de referència del codi IK, passant de l'antiga EN 50102 a l'actual CEI 62262 \textit{Degrees of protection provided by enclosures for
electrical equipment against external mechanical impacts (IK Code)}.

En la secció 13.6, s'afegeixen les normes CEI 60062,  CEI 60063,  CEI 60529 i  CEI 62262.

En els apèndixs  B i C, es modifiquen les unitats de la constant de Planck, passant de \unit{J.s} a \unit{J/Hz}; aquest canvi es fa d'acord amb  la publicació de l'any 2018 del \textit{Committee on Data for Science and Technology} (CODATA).

En la secció B.4, s'indica la relació numèrica entre el grau Celsius i el kelvin.

S'amplia i revisa la secció B.8, dedicada a les normes d'escriptura de quantitats i unitats.


S'explica en l'apèndix C, dedicat a les constants físiques,  el canvi introduït en les definicions de l'any 2018 d'algunes constants respecte de les definicions del 2014, quan diverses constants  van passar de tenir valors mesurats amb una certa precisió a tenir valors definits exactes, i a l'inrevés.

Es revisa l'índex alfabètic, al final del llibre, eliminant-hi entrades sobreres, afegint-ne algunes i coŀlocant-ne d'altres en la posició correcta.


El paquet d'ampliació de \LaTeX{} \texttt{siunitx}, utilitzat per escriure valors numèrics i unitats, ha sofert un gran canvi en passar de la versió 2 a la 3. Aquest canvi ha obligat a fer una  revisió total d'aquest llibre per tal de fer-lo compatible amb aquesta nova versió. 

S'unifica en tot el llibre l'ús dels operadors $\rightarrow$ (tendeix a) i $\Rightarrow$ (implica).

Seguint les indicacions de la norma internacional ISO 80000-2 \textit{Quantities and units --- Part 2: Mathematics}, quan una equació és massa llarga i ha de partir-se en dues o més línies, els símbols $=$, $+$, $-$, $\pm$, $\mp$, $\times$, $\cdot$, $/$, etc., no es repeteixen al final d'una línia i al principi de la següent, sinó que només es posen a l'inici de la  línia següent.

Es modifica l'estil d'escriptura de les paraules en idiomes diferents del català, escrivint-les en lletra inclinada en lloc de fer-ho en lletra normal i tancades entre cometes. Per tant, en lloc d'escriure, per exemple, «root mean square» ara escriurem \textit{root mean square}.

Es modifica l'estil d'escriptura dels títols de llibres i publicacions, escrivint-los en lletra inclinada independentment de l'idioma.


\section*{Versió 13.1 (20 d'agost de 2021)}
\addcontentsline{toc}{section}{Versió 13.1}

S'afegeix la nota «a» a la taula B.9.

S'amplia la secció B.8, dedicada a les normes d'escriptura de quantitats i unitats.

Es canvia el text «exacte» pel text «valor exacte» a la taula C.1.

S'afegeix una nota a peu de pàgina en la secció C.2, per precisar el significat de l'error absolut d'un valor mesurat.

Es revisa tot el text per canviar les paraules «Figura» i «Taula» per les paraules «figura» i «taula» respectivament, quan no es troben al principi d'una oració.

Es revisa tot el text per unificar l'ús dels tres tipus de guions que existeixen segons l'ortografia catalana: el guionet, el guió mitjà i el guió llarg. El guionet  «-» és el que té més usos, com ara:  unir paraules relacionades, separar seqüències numèriques, separar una paraula a final de ratlla, etc.; exemples: \textit{zig-zag, 2020-2021, fase-neutre}. El guió mitjà  «--» s'utilitza només per representar el signe matemàtic menys; exemples: $-1$, \complexnum{3-j4}. El  guió llarg  «---» té quatre usos: fer aclariments de frases anteriors, de la mateixa manera que es fa amb els parèntesis, indicar diàlegs de personatges o de persones entrevistades, introduir enumeracions, i en citacions bibliogràfiques; exemple: \textit{tots els circuits d'aquest capítol ---excepte el primer--- són de corrent continu}.


\section*{Versió 13.2 (16 d'octubre de 2021)}
\addcontentsline{toc}{section}{Versió 13.2}

S'amplia l'apartat Prefaci, fent-hi una descripció del contingut dels capítols del llibre. Es modifica l'adreça de correu electrònic de contacte amb l'autor.

S'utilitza el paquet d'ampliació de \LaTeX{} \texttt{listings} per llistar els programes de la calculadora \textsf{HP Prime}; amb aquest paquet s'aconsegueixen fàcilment  llistats amb sintaxi acolorida, la qual cosa facilita la lectura dels programes. Es crea un nou índex de llistats de programes, a continuació de l'índex d'exemples.


\section*{Versió 13.3 (5 de gener de 2022)}
\addcontentsline{toc}{section}{Versió 13.3}

Es millora l'explicació de la secció E.2 i es corregeix l'equació (E.12).


\section*{Versió 14.0 (29 de juliol de 2022)}
\addcontentsline{toc}{section}{Versió 14.0}

Es crea una part nova en el llibre ---la IV--- dedicada al llenguatge de programació Python, la qual conté els capítols 14, 15 i 16. Les parts IV i V de les versions anteriors passen a ser les parts V  i VI  respectivament. 

S'amplia l'apartat Prefaci per descriure els nous capítols 14, 15 i 16.

S'afegeix el símbol (\faPython) al final del títol dels  exemples del llibre que estan resolts mitjançant Python; aquest símbol és un enllaç que porta directament a l'apartat del capítol 16 on es resol l'exemple.

En l'apartat Notació, s'unifica el símbol que es fa servir per definir l'angle d'un fasor, utilitzant la lletra $\theta$.

Es millora l'explicació de l'exemple 1.6.

En la secció 8.7.2, es clarifica el significat de $Z\ped{ns}$ i de $S\ped{n}$ basant-se en la norma IEEE C37.110 \textit{ Guide for the Application of Current Transformers Used for Protective Relaying Purposes}.

En la secció 11.1, s'afegeix el circuit equivalent d'una font de corrent ideal.

S'afegeixen algunes normes en la secció 13.6.

En la secció B.1, es posen al dia les adreces d'internet  del \textit{Bureau International des Poids et Mesures}, i s'actualitza el Reial decret que empara l'ús del Sistema Internacional d'unitats.

A la taula D.1, es canvien els símbols < i > pels correctes $\leq$ i $\geq$ respectivament.

Es modifiquen els noms de les funcions de l'apèndix F, per tal d'igualar-les amb els noms de les funcions del mòdul Python \texttt{qed.eng\_elc} del capítol 15.

En l'apartat Bibliografia s'afegeixen les referències \cite{SUM},  \cite{RAM}, \cite{JOH}, \cite{HIL}, \cite{ZUM} i \cite{VAN}.



\section*{Versió 14.1 (14 d'agost de 2022)}
\addcontentsline{toc}{section}{Versió 14.1}

Es revisa tot el text, fent-hi petites correccions.


\section*{Versió 14.2 (2 de gener de 2023)}
\addcontentsline{toc}{section}{Versió 14.2}

Es canvia la font del text dels llistats de programes, fent-ne servir una de pas fix.

Es fa referència a  la norma internacional CEI/ISO 80000 \textit{Quantities and units}, allà on abans es feia referència  a la norma obsoleta CEI 60027 \textit{Letter symbols to be used in electrical technology}.

En l'apartat Prefaci, s'indica que totes les versions d'aquest llibre, així com els programes en Python i els de la calculadora \textsf{HP Prime} inclosos, poden trobar-se a GitHub.

En l'apartat Notació, s'unifica el símbol que es fa servir per definir el mòdul d'un fasor, utilitzant la lletra $Z$.

En l'exemple 4.3 i en la secció  12.6, s'indica l'equivalència entre els programes MATLAB® i GNU Octave.

S'afegeix la nota «a» a la taula 10.2.

Es canvien algunes equacions de la secció 10.4.4.

Es modifiquen els dibuixos dels exemples 12.1, 12.2 i 12.3, representant en els nusos flotants una xarxa externa, en lloc d'un generador.

La secció 13.1 es divideix en dues subseccions, i s'amplia amb l'explicació de l'actuació de la funció  51 quan el corrent que circula és variable en el temps. S'afegeixen les unitats que faltaven en l'equació (13.1), i en les taules 13.1 i 13.2.

S'afegeixen algunes normes en la secció 13.6.

En la secció 14.2, s'inclou la descripció dels IDE Visual Studio Code i PyCharm.

En la secció 14.3, s'inclou l'explicació de la llibreria \texttt{pandapower}.


En el capítol 15, es fan les modificacions següents al mòdul  \texttt{qed.eng\_elec}:
\begin{itemize}
	\item S'afegeix la funció \texttt{rms}, que calcula el valor eficaç d'una funció periòdica.
	\item S'afegeix la funció \texttt{average}, que calcula el valor mitjà d'una funció periòdica.
	\item S'afegeix l'excepció \texttt{NetworkNotSolvable}, associada a la classe \texttt{Network}.
	\item Es canvia el nom de la classe \texttt{Admitance} per \texttt{Admittance}.
	\item Es millora la programació d'algunes funcions sense variar-ne la funcionalitat.
\end{itemize}


En el capítol 16, s'afegeix la resolució amb Python dels exemples 5.4  i  13.1, i els exemples 12.1, 12.2 i 12.3 es resolen  fent servir addicionalment la llibreria \texttt{pandapower}.

En la taula B.2, s'inclouen els  nous prefixos de l'SI per als factors $10^{27}$,  $10^{30}$,  $10^{-27}$ i  $10^{-30}$, introduïts en la 27a \textit{Conférence Générale des Poids et Mesures} de l'any 2022.

Es millora la presentació de les taules B.8 i B.10.


\section*{Versió 14.3 (24 d'abril de 2023)}
\addcontentsline{toc}{section}{Versió 14.3}

A partir d'aquesta versió s'utilitzen els paquets d'ampliació de \LaTeX{}
\texttt{newpxtext} (\textit{New PX font package}) i \texttt{eulerpx} (\textit{The eulerpx font package}), per definir les fonts utilitzades en el llibre. Les fonts que proporcionen aquests paquets són les \texttt{Palatino}, \texttt{Helvetica} i \AmS{} \texttt{Euler}.
Fins ara, la font utilitzada era la \texttt{Kp-Fonts}.

Es revisa tot el text, fent-hi petites correccions gramaticals i estilístiques.

En l'aparat Notació, es millora l'explicació dels símbols que representen els diferents conjunts de nombres i intervals, segons la norma internacional ISO 80000-2.

S'amplia la descripció de la secció \ref{sec:acobl-mag}.

Es millora l'explicació de la secció \ref{sec:dirichlet}.

S'afegeix l'or a la taula \ref{taula:param-elc}.

Se simplifica l'explicació de les seccions \ref{sec:connex-correcta} i \ref{sec:connex-optima}.

Es crea la secció \ref{sec:python-versions} per indicar la versió de Python i de les principals llibreries utilitzades. 


En el capítol \ref{chap:python-programes}, es fan les modificacions següents al mòdul  \texttt{qed.eng\_elec}:
\begin{itemize}
	\item S'afegeixen les classes  \texttt{Motor3ph}, \texttt{Motor3phRun} i 
	\texttt{Motor3phStartUp}, dedicades a fer càlculs amb motors d'inducció trifàsics.
	\item Es modifiquen els paràmetres de creació de les classes: \texttt{Impedance}, \texttt{Admittance}, \texttt{VoltageSource}, \texttt{VoltageSourceIdeal}, \texttt{CurrentSource}, \texttt{CurrentSourceIdeal} i \texttt{ShortCircuit}.
	\item Es fan els canvis següents en la classe \texttt{Network}:
	\begin{itemize}
		\item Es modifiquen els paràmetres de crida dels mètodes \texttt{add} i \texttt{remove}.
		\item  El mètode \texttt{branch\_current} passa a denominar-se \texttt{current}; també se'n modifiquen els paràmetres de crida.
		\item S'unifiquen els dos mètodes \texttt{node\_voltage} i \texttt{branch\_voltage} en un de sol, el qual passa a denominar-se \texttt{voltage} amb uns paràmetres de crida diferents. 
	\end{itemize}
\end{itemize}

En el capítol \ref{chap:python-exemples}, es modifica:
\begin{itemize}
	\item Llistats \ref{lst:CaractMotor}, \ref{lst:MotCarregaReduida} i \ref{lst:MotTensRedSolAp}. Es modifiquen aquests programes, per fer ús de la classe \texttt{Motor3ph}.
	\item Llistat \ref{lst:MotTensRedSolEx}. S'amplia aquest programa, fent ús de la classe \texttt{Motor3ph}, per tal de calcular el temps que triga el motor a arrencar, al 100 \% i al 80 \% de la tensió nominal, i per representar l'evolució temporal del corrent d'arrencada a aquestes dues tensions.	
	\item  Llistat \ref{lst:MotTensRedIarr}. S'amplia aquest programa, fent ús de la classe \texttt{Motor3ph}, per tal de calcular el temps que triga el motor a arrencar, i per representar l'evolució temporal del  corrent d'arrencada i de la tensió en el motor.
\end{itemize}

A la secció \ref{sec:SI-intro}, s'indica que la publicació més recent del \textit{Bureau 	International des Poids et Mesures} del Sistema Internacional d'unitats, és la  9a edició de març de 2019, revisada el desembre de 2022 per afegir-hi els nous prefixos dels factors $10^{27}$,  $10^{30}$,  $10^{-27}$ i  $10^{-30}$.

Es millora l'explicació de la secció \ref{sec:int-mum}.

En l'apartat Bibliografia s'afegeix la referència \cite{ALL}.

\section*{Versió 14.4 (2 de maig de 2023)}
\addcontentsline{toc}{section}{Versió 14.4}

En la secció B.2.2, es corregeix l'explicació que segueix a la definició de l'ampere.

\section*{Versió 14.5 (18 de maig de 2023)}
\addcontentsline{toc}{section}{Versió 14.5}

A partir d'aquesta versió es fa servir la font \texttt{Josefin}, en lloc de la font \texttt{Helvetica},  per a la lletra de pal sec.

Es modifica  la manera d'escriure els valors complexos en forma polar, per tal d'adaptar-la a la forma estàndard. A partir d'ara, en lloc d'escriure, per exemple, $230_{\angle\ang{120}}$, escriurem \complexnum{230:120}.

Es milloren algunes figures i equacions de les seccions 1.6.1 i 1.6.2.

Es millora la figura 3.1.

A l'apèndix B, es modifiquen les unitats de la constant de Planck, passant de \unit{J/Hz} a \unit{J.s} ---encara que són equivalents---, ja que aquestes són les que apareixen en les publicacions  del \textit{Bureau 	International des Poids et Mesures}.

\section*{Versió 14.6 (3 de juny de 2023)}
\addcontentsline{toc}{section}{Versió 14.6}

Es canvia la manera de representar les matrius i vectors, passant a fer servir claudàtors en lloc de parèntesis. Ambdós símbols són correctes, però els parèntesis s'utilitzen preferentment en textos matemàtics, mentre que els claudàtors es fan servir majoritàriament en textos de física i enginyeria.

S'amplia la secció \ref{sec:python-versions}, per indicar les opcions per defecte utilitzades en la llibreria matplotlib.

En el llistat \ref{lst:InterpDuesDim}, es fa servir la funció \texttt{RegularGridInterpolator} del mòdul \texttt{scipy.interpolate}, en lloc de la funció \texttt{interp2d}, que es feia servir anteriorment, perquè la funció \texttt{interp2d} es considera obsoleta a partir de la versió 1.10.0 de la llibreria ScyPy, i s'eliminarà en la versió 1.12.0.