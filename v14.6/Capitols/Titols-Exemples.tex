% Títols dels exemples del Capítol Fonaments
\newcommand{\MillmanBateries}{Teorema de Millman --- Bateries en paraŀlel}
\newcommand{\MillmanTrifNeutre}{Teorema de Millman --- Càrregues trifàsiques amb corrent de neutre}
\newcommand{\MillmanTrifNoNeutre}{Teorema de Millman --- Càrregues trifàsiques sense corrent de neutre}
\newcommand{\Superposicio}{Aplicació del teorema de la superposició}
\newcommand{\FactorsCrestaFormaArr}{Càlcul de factors de cresta, de forma i d'arrissada}
\newcommand{\ValorEfValorMitja}{Valor eficaç i valor rectificat mitjà}
\newcommand{\PotSistTresFils}{Càlcul de la potència en un sistema de 3 fils}
\newcommand{\CarDescRL}{Càrrega i descàrrega d'un circuit R-L}
\newcommand{\CurtcircuitRL}{Curtcircuit en un circuit R-L}
\newcommand{\CurtcircuitPicRL}{Corrent de pic asimètric}
\newcommand{\Malles}{Aplicació del mètode de les malles}

% Títols dels exemples del Capítol Càlculs Bàsics
\newcommand{\MetodeCalculPU}{Aplicació del mètode de càlcul en per unitat}
\newcommand{\TriangleEstrella}{Transformació d'impedàncies triangle a estrella}
\newcommand{\ResCircPotAbs}{Resolució d'un circuit coneixent la potència que absorbeix}
\newcommand{\CorrentCcSecTrafo}{Corrent de curtcircuit en el secundari d'un transformador}
\newcommand{\ValorsEscalaLog}{Càlcul de valors en una escala logarítmica}
\newcommand{\ConstantsEscalaLogLog}{Càlcul de les constants n i k en una escala logarítmica-logarítmica}

% Títols dels exemples del Capítol Components Simètriques
\newcommand{\ImpedEquil}{Aplicació de les components simètriques --- Impedàncies equilibrades}
\newcommand{\ImpedDesequil}{Aplicació de les components simètriques --- Impedàncies desequilibrades}

% Títols dels exemples del Capítol Sèries de Fourier
\newcommand{\ValorMitjaEfTaxaFon}{Càlcul de valors mitjà i eficaç, i de taxa de fonamental}
\newcommand{\SerieFouTaulaFormes}{Càlcul d'una sèrie de Fourier utilitzant la taula de formes d'ona}
\newcommand{\CircuitFourier}{Resolució d'un circuit elèctric utilitzant les sèries de Fourier}

% Títols dels exemples del Capítol Transformada de Laplace
\newcommand{\CalcTransfLaplace}{Càlcul de transformades de Laplace}
\newcommand{\ResCircRC}{Resolució d'un circuit R-C}
\newcommand{\CircuitLaplace}{Resolució d'un circuit amb condicions inicial no nuŀles}
\newcommand{\CircuitLaplaceNul}{Resolució d'un circuit amb condicions inicial nuŀles}

% Títols dels exemples del Capítol Cables
\newcommand{\CaigudaDeTensio}{Càlcul de la caiguda de tensió en un sistema trifàsic}
\newcommand{\CapTermicaCable}{Càlcul de la capacitat tèrmica d'un cable}
\newcommand{\AWGammSQ}{Secció en mm² corresponent a un número AWG}
\newcommand{\mmSQaAWG}{Número AWG corresponent a una secció en mm²}

% Títols dels exemples del Capítol Transformadors de Mesura i Protecció
\newcommand{\CaractTrafoCorrent}{Determinació de les característiques d'un transformador de corrent}
\newcommand{\TrafoIeeeEqCei}{Equivalència entre transformadors IEEE i CEI}
\newcommand{\ConnexWatt}{Connexió d'un wattímetre a una instaŀlació existent}

% Títols dels exemples del Capítol Transformadors de Potència
\newcommand{\ParamTrafo}{Determinació dels paràmetres d'un transformador}
\newcommand{\ImpCircEqTrafoTresDeb}{Impedàncies del circuit equivalent d'un transformador de tres debanats}
\newcommand{\IndexHorari}{Determinació de l'índex horari d'un transformador}
\newcommand{\CCasimSecTrafo}{Curtcircuits asimètrics en el secundari d'un transformador}
\newcommand{\ConnexParalDifIndex}{Connexió en paraŀlel de transformadors amb diferent índex horari}

% Títols dels exemples del Capítol Motors d'Inducció
\newcommand{\MotorNombrePols}{Nombre de pols i lliscament d'un motor}
\newcommand{\MotorParellNom}{Parell nominal  d'un motor}
\newcommand{\MotTempsArr}{Temps d'arrencada d'un motor}
\newcommand{\CaractMotor}{Característiques de funcionament d'un motor}
\newcommand{\TensDeseqMotor}{Motor alimentat amb tensió desequilibrada}
\newcommand{\MotCarregaReduida}{Motor connectat a una càrrega  reduïda}
\newcommand{\MotTensRedSolAp}{Motor alimentat a tensió reduïda --- Solució aproximada}
\newcommand{\MotTensRedSolEx}{Motor alimentat a tensió reduïda --- Solució exacta}
\newcommand{\MotTensRedIarr}{Arrencada a tensió reduïda, deguda al   corrent d'arrencada del motor}
\newcommand{\MotorArrencNEMA}{Corrent d'arrencada  d'un motor segons NEMA MG-1}
\newcommand{\MotorTensDeseqNEMA}{Tensió d'alimentació desequilibrada en un motor segons NEMA MG-1}
\newcommand{\MotorVarParamTensFreq}{Variació dels paràmetres d'un motor amb la tensió i la freqüència}

% Titols dels exemples del Capítol Xarxes Elèctriques
\newcommand{\XarxaAmbAcobl}{Aplicació del mètode dels nusos --- Xarxa amb acoblaments magnètics}
\newcommand{\XarxaSenseAcobl}{Aplicació del mètode dels nusos --- Xarxa sense acoblaments magnètics}
\newcommand{\XarxaThevenin}{Impedància Thévenin entre dos nusos d'una xarxa}

% Títols dels exemples del Capítol Fulx de Càrregues
\newcommand{\FluxCarrXarxa}{Flux de càrrega d’una xarxa}
\newcommand{\ControlTensCond}{Control de tensió d’un nus amb condensadors}
\newcommand{\ControlTensTrafo}{Control de tensió d’un nus amb un transformador}

% Títols dels exemples del Capítol Normes Diverses
\newcommand{\ProtInvIntVar}{Actuació de la funció 51 amb un corrent variable en el temps}

% Títols dels exemples de l'Apèndix Càlcul Numèric
\newcommand{\InterpLinCub}{Interpolació lineal i cúbica}
\newcommand{\InterpDuesDim}{Interpolació en dues dimensions}
\newcommand{\IntegracioNum}{Integració numèrica d'una funció}
\newcommand{\SolFunNoLin}{Solució d'una funció no lineal}
