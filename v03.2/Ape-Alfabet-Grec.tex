\chapter{Alfabet Grec} \index{alfabet grec}

 En la Taula \vref{taula:alfabet-grec} es pot veure l'alfabet grec,
 amb els noms de les lletres en diversos idiomes.

\begin{table}[h]
   \caption{\label{taula:alfabet-grec} Alfabet grec}
   \begin{center}\begin{tabular}{ccclll}
   \toprule[1pt]
   \renewcommand*{\multirowsetup}{\centering}
   \multirow{2}{15mm}{\rule{0mm}{4.5mm}N\'{u}mero\\d'ordre} & \multicolumn{2}{c}{Lletra} &
   \multicolumn{3}{c}{Nom} \\
   \cmidrule(rl){2-3} \cmidrule(rl){4-6}
    & min\'{u}scula & maj\'{u}scula & catal\`{a} & castell\`{a} &  angl\`{e}s \\
   \midrule
   1  & $\alpha$ & A & alfa & alfa &  alpha\\
   2  & $\beta$ & B & beta & beta &  beta\\
   3  & $\gamma$ & $\Gamma$ & gamma & gamma &  gamma\\
   4  & $\delta$ & $\Delta$ & delta & delta &  delta\\
   5  & $\epsilon$, $\varepsilon$ & E & \`{e}psilon & \'{e}psilon &  epsilon\\
   6  & $\zeta$ & Z & zeta & dseda &  zeta\\
   7  & $\eta$ & H & eta & eta &  eta\\
   8  & $\theta$, $\vartheta$ & $\Theta$ & theta & zeta &  theta\\
   9  & $\iota$ & I & iota & iota &  iota\\
   10 & $\kappa$ & K & kappa & kappa &  kappa\\
   11 & $\lambda$ & $\Lambda$ & lambda & lambda &  lambda\\
   12 & $\mu$ & M & mi & mi &  mu\\
   13 & $\nu$ & N & ni & ni &  nu\\
   14 & $\xi$ & $\Xi$ & ksi, csi & xi &  xi\\
   15 & o & O & \`{o}micron & \'{o}micron &  omicron\\
   16 & $\pi$, $\varpi$ & $\Pi$ & pi & pi &  pi\\
   17 & $\rho$, $\varrho$ & P & rho, ro & ro &  rho\\
   18 & $\sigma$, $\varsigma$ & $\Sigma$ & sigma & sigma &  sigma\\
   19 & $\tau$ & T & tau & tau & tau\\
   20 & $\upsilon$ & $\Upsilon$ & \'{\i}psilon & \'{\i}psilon &  upsilon\\
   21 & $\phi$, $\varphi$ & $\Phi$ & fi & fi &  phi\\
   22 & $\chi$ & X & khi & ji &  chi\\
   23 & $\psi$ & $\Psi$ & psi & psi &  psi\\
   24 & $\omega$ & $\Omega$ & omega & omega &  omega\\
   \bottomrule[1pt]
   \end{tabular} \end{center}
\end{table}

Les dues grafies de la lletra min\'{u}scula \`{e}psilon  ($\epsilon,
\varepsilon$) s\'{o}n totalment equivalents entre s\'{\i}; el mateix passa
amb les dues grafies de les lletres min\'{u}scules theta ($\theta,
\vartheta$), rho ($\rho,\varrho$) i fi ($\phi, \varphi$).

La lletra sigma min\'{u}scula t\'{e} dues variants: $\varsigma$, escrita en
grec al final d'una paraula, i $\sigma$, escrita en grec a l'inici o
en mig d'una paraula. En els textos t\`{e}cnics i cient\'{\i}fics s'utilitza
majorit\`{a}riament la variant $\sigma$.

La variant $\varpi$ de la lletra pi, es denomina {"<}pi d\`{o}rica{">} en
catal\`{a}, {"<}pi d\'{o}rica{">} en castell\`{a} i {"<}dorian pi{">} en angl\`{e}s.

Pel que fa als noms de les lletres, alguns poden sorprendre; aix\`{o} no
\'{e}s estrany ja que algunes lletres han rebut hist\`{o}ricament noms
diversos, i fins i tot contradictoris respecte dels actuals.

Els noms anglesos de les lletres s\'{o}n els m\'{e}s uniformes, ja que no
s'ha observat cap variaci\'{o} en les diverses fonts consultades.

Els noms catalans de les lletres s\'{o}n els que apareixen en el {"<}Gran
Diccionari de la Llengua Catalana, 1999{">}. Altres noms utilitzats en
les diverses fonts consultades s\'{o}n:
\begin{multicols}{3}
\begin{list}{}
   {\setlength{\labelwidth}{16mm} \setlength{\leftmargin}{16mm} \setlength{\labelsep}{2mm}}
   \item[B, $\beta$:] vita.
   \item[Z, $\zeta$:] zita.
   \item[H, $\eta$:] ita.
   \item[$\Theta$, $\theta$:] thita.
   \item[T, $\tau$:] taf.
\end{list}
\end{multicols}

Els noms castellans de les lletres s\'{o}n els que apareixen en el
{"<}Diccionario de la Lengua Espa\~{n}ola (D.R.A.E.), 22\textordfeminine\
edici\'{o}n (2001){">}. Altres noms utilitzats en les diverses fonts
consultades s\'{o}n:
\begin{multicols}{3}
\begin{list}{}
   {\setlength{\labelwidth}{16mm} \setlength{\leftmargin}{16mm} \setlength{\labelsep}{2mm}}
   \item[Z, $\zeta$:] zeta\footnote{Aquests noms eren els que apareixien en les edicions
   del D.R.A.E anteriors a la 21a (1992)}, dseta, dzeta.
   \item[$\Theta$, $\theta$:] theta\footnotemark[1], thita.
   \item[K, $\kappa$:] cappa.
   \item[M, $\mu$:] my\footnotemark[1], mu.
   \item[N, $\nu$:] ny\footnotemark[1], nu.
   \item[O, o:] omicr\'{o}n.
   \item[P, $\rho$:] rho.
   \item[$\Upsilon$, $\upsilon$:] \'{u}psilon.
   \item[$\Phi$, $\phi$:] phi.
\end{list}
\end{multicols}
