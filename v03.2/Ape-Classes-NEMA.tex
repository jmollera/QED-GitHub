\chapter{Classes \textsf{NEMA} d'A\"{\i}llaments T\`{e}rmics en Motors}
\index{classes NEMA d'a\"{\i}llaments t\`{e}rmics} \index{NEMA}

Quan es posa en marxa un motor, la seva temperatura comen\c{c}a a pujar
per sobre de la temperatura ambient, a causa del corrent que circula
pels seus debanats.

La {"<}National Electrical Manufacturers Association{">} (\textsf{NEMA}),
defineix diverses classes d'a\"{\i}llament t\`{e}rmic, depenent de
l'increment global de temperatura perm\`{e}s respecte de la temperatura
ambient, que fixa en 40\unit{\celsius};\index{temperatura!ambient}
per a cada classe, es permet un increment addicional de temperatura
en el punt m\'{e}s calent, situat en el centre dels debanats del
motor.\index{temperatura!en el punt m\'{e}s calent}

En la Taula \vref{taula:classes-nema} es donen els valors dels increments de temperatura permesos per a cadascuna de les diferents classes d'a\"{\i}llament, partint d'una temperatura ambient de 40\unit{\celsius}.
\begin{table}[htb]
   \caption{\label{taula:classes-nema} Classes \textsf{NEMA} d'a\"{\i}llaments t\`{e}rmics en motors}
   \begin{center}\begin{tabular}{cr<{\hspace{6em}}r<{\hspace{8em}}}
   \toprule[1pt]
   Classe & \multicolumn{1}{c}{Increment global de temperatura} & \multicolumn{1}{c}{Increment addicional de temperatura} \\
   NEMA &   \multicolumn{1}{c}{sobre la temperatura ambient}  & \multicolumn{1}{c}{en el punt m\'{e}s calent} \\
   \midrule
   A & 60\unit{\celsius} & 5\unit{\celsius}   \\
   B & 80\unit{\celsius} & 10\unit{\celsius}   \\
   F & 105\unit{\celsius} & 10\unit{\celsius}   \\
   H & 125\unit{\celsius} & 15\unit{\celsius}   \\
   \bottomrule[1pt]
   \end{tabular} \end{center}
\end{table}
\index{A} \index{B} \index{F} \index{H}
