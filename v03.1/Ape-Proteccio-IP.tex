\chapter{Grau de Protecci\'{o} IP} \index{IP} \index{grau de protecci\'{o}} \index{CEI!60529}

La codificaci\'{o} {"<}International Protection{">} (\textsf{IP}), segons la
norma \textsf{CEI 60529}, s'utilitza per descriure el grau de
protecci\'{o}  proporcionat per les envolvents d'equips el\`{e}ctrics, contra
la penetraci\'{o} de cossos s\`{o}lids estranys i contra els efectes nocius
de l'aigua.

\section{Codificaci\'{o}}
 La codificaci\'{o} consisteix en les sigles \textsf{\textbf{IP}}
seguides per dues xifres, m\'{e}s una lletra addicional (opcional) i una
lletra suplement\`{a}ria (opcional); quan el grau de protecci\'{o}
corresponent a una de les dues xifres no s'utilitzi, perqu\`{e} no sigui
necessari o perqu\`{e} no sigui conegut, es reempla\c{c}ar\`{a} la xifra en
q\"{u}esti\'{o} per una \textsf{\textbf{X}}. Es defineix a continuaci\'{o} el
significat de les xifres i lletres que formen el codi \textsf{IP}:

\textbf{1a xifra}. Indica el grau de protecci\'{o} de les persones contra els contactes amb
parts en tensi\'{o}, o amb peces en moviment, i el grau de protecci\'{o} dels equips contra la
penetraci\'{o} de cossos s\`{o}lids i pols. Els valors possibles s\'{o}n els seg\"{u}ents:
\begin{multicols}{2}
\begin{list}{}
   {\setlength{\labelwidth}{4.5mm} \setlength{\leftmargin}{4.5mm} \setlength{\labelsep}{2mm}}
   \item[\textbf{0}] Sense protecci\'{o}.
   \item[\textbf{1}] Protegit contra l'entrada de cossos s\`{o}lids superiors a 50\unit{mm},
   com per exemple,   contactes involuntaris de la m\`{a}.
   \item[\textbf{2}] Protegit contra l'entrada de cossos superiors a 12,5\unit{mm}, com per exemple,
   contactes involuntaris dels dits de la m\`{a}.
   \item[\textbf{3}] Protegit contra l'entrada de cossos superiors a 2,5\unit{mm},
   com per exemple, eines o cables.
   \item[\textbf{4}] Protegit contra l'entrada de cossos superiors a 1\unit{mm}.
   \item[\textbf{5}] Protegit contra la pols. Es permet la seva entrada all\`{a} on no sigui perjudicial.
   \item[\textbf{6}] Protegit totalment contra la pols.
\end{list}
\end{multicols}

\textbf{2a xifra}. Indica el grau de protecci\'{o} dels equips contra
l'entrada d'aigua. Els valors possibles s\'{o}n els seg\"{u}ents: \pagebreak
\begin{multicols}{2}
\begin{list}{}
   {\setlength{\labelwidth}{4.5mm} \setlength{\leftmargin}{4.5mm} \setlength{\labelsep}{2mm}}
   \item[\textbf{0}] Sense protecci\'{o}.
   \item[\textbf{1}] Protegit contra la caiguda vertical d'aigua.
   \item[\textbf{2}] Protegit contra la caiguda d'aigua, fins a 15\unit{\degree} de la  vertical.
   \item[\textbf{3}] Protegit contra la caiguda d'aigua, fins a 60\unit{\degree} de la  vertical.
   \item[\textbf{4}] Protegit contra la caiguda d'aigua, en totes les direccions.
   \item[\textbf{5}] Protegit contra aigua llan\c{c}ada per m\`{a}negues.
   \item[\textbf{6}] Protegit contra aigua llan\c{c}ada per cops de mar.
   \item[\textbf{7}] Protegit contra la immersi\'{o} temporal.
   \item[\textbf{8}] Protegit contra la immersi\'{o} prolongada, o a gran pressi\'{o}.
\end{list}
\end{multicols}

\textbf{Lletra addicional (opcional)}. En alguns casos la protecci\'{o}
proporcionada per les envolvents contra l'acc\'{e}s a les parts
perilloses \'{e}s millor que la indicada per la primera xifra del codi;
en aquests casos, es pot caracteritzar aquesta protecci\'{o} amb una
lletra addicional, afegida despr\'{e}s de les dues xifres; aix\`{o} permet
tenir obertures adequades per a la ventilaci\'{o},  guardant a l'hora el
grau requerit de protecci\'{o} de les persones. Els valors possibles s\'{o}n
els seg\"{u}ents:
\begin{multicols}{2}
\begin{list}{}
   {\setlength{\labelwidth}{5mm} \setlength{\leftmargin}{5mm} \setlength{\labelsep}{2mm}}
   \item[\textbf{A}] Els  cossos estranys de di\`{a}metre superior a
   50\unit{mm},
    poden penetrar en l'envolvent, per\`{o} tan sols d'una forma volunt\`{a}ria i deliberada.
   \item[\textbf{B}] Els  cossos estranys de di\`{a}metre superior a 12,5\unit{mm}
    poden penetrar en l'envolvent, per\`{o} un dit de la m\`{a} no ha de poder entrar m\'{e}s de 80\unit{mm}, i
    ha de quedar per tant, a una dist\`{a}ncia    suficient de les parts perilloses.
   \item[\textbf{C}] Els  cossos estranys de di\`{a}metre superior a 2,5\unit{mm}
   poden penetrar en l'envolvent, per\`{o} un filferro d'acer d'aquest di\`{a}metre i 100\unit{mm}
   de longitud, ha de quedar a una dist\`{a}ncia suficient de les parts perilloses.
   \item[\textbf{D}] Els  cossos estranys de di\`{a}metre superior a 1\unit{mm}
   poden penetrar en l'envolvent, per\`{o} un filferro d'acer d'aquest di\`{a}metre i 100\unit{mm}
   de longitud, ha de quedar a una dist\`{a}ncia suficient de les parts perilloses.
\end{list}
\end{multicols}

\textbf{Lletra suplement\`{a}ria (opcional)}. El codi \textsf{IP} accepta tamb\'{e} algunes
lletres suplement\`{a}ries al final, per tal d'afegir una informaci\'{o} concreta. Els valors
possibles s\'{o}n els seg\"{u}ents:
\begin{multicols}{2}
\begin{list}{}
   {\setlength{\labelwidth}{6mm} \setlength{\leftmargin}{6mm} \setlength{\labelsep}{2mm}}
   \item[\textbf{H}] Material d'alta tensi\'{o}.
   \item[\textbf{M}] En m\`{a}quines rotatives, indica que els assajos s'han realitzat amb el
    rotor girant.
   \item[\textbf{S}] En m\`{a}quines rotatives, indica que els assajos s'han realitzat amb el
    rotor parat.
   \item[\textbf{W}] Protecci\'{o} contra la intemp\`{e}rie.
\end{list}
\end{multicols}

\section{Altres normes}
Les normes d'alguns pa\"{\i}sos, com ara B\`{e}lgica, Espanya, Fran\c{c}a i
Portugal, referents a la codificaci\'{o} \textsf{IP}, permetien afegir
una tercera xifra, despr\'{e}s de les dues primeres,  per tal de
codificar tamb\'{e} la protecci\'{o} proporcionada per les envolvents contra
els impactes mec\`{a}nics. No obstant, des de l'adopci\'{o} de la norma
\textsf{CEI 60529}\index{CEI!60529}, cap pa\'{\i}s europeu no pot tenir
un codi \textsf{IP} diferent. Havent rebutjat fins ara la
\textsf{CEI} afegir aquesta tercera xifra al codi \textsf{IP},
l'\'{u}nica soluci\'{o} possible per tal de mantenir una classificaci\'{o}
d'aquest concepte, era la creaci\'{o} d'un codi diferent. Aquest es
l'objectiu de la norma europea \textsf{EN 50102}\index{EN!50102}: el
codi \textbf{\textsf{IK}}; els valor d'aquest codi s\'{o}n:
\index{IK}\textbf{\textsf{IK00}}, \textbf{\textsf{IK01}}, ...\,,
\textbf{\textsf{IK09}}, \textbf{\textsf{IK10}}, i representen valors
creixents d'energia d'impacte que pot suportar l'envolvent.
