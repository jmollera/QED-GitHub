\chapter*{Notaci\'{o}} \addcontentsline{toc}{chapter}{Notaci\'{o}}

Es presenta a continuaci\'{o} la notaci\'{o} seguida en aquest llibre.

Cal fer notar que les variables vectorials o matricials s'escriuen
en lletra negreta inclinada,  mentre que les variables escalars
s'escriuen en lletra normal inclinada.

\begin{list}{}
{\setlength{\labelwidth}{15mm} \setlength{\leftmargin}{25mm}
\setlength{\labelsep}{10mm}}
    \item[$V$] Una variable real.
    \item[$\cmplx{V}$] Una variable complexa.
    \item[$\cmplx{V}^*$] Conjugat d'una variable complexa.
    Es compleixen les relacions:\\[1ex]
     $(\cmplx{V}_1+\cmplx{V}_2+\cdots+\cmplx{V}_n)^* = \cmplx{V}_1^* +
    \cmplx{V}_2^*+\cdots+\cmplx{V}_n^*$\\[1ex]
    $(\cmplx{V}_1 \cmplx{V}_2 \cdots \cmplx{V}_n)^* = \cmplx{V}_1^*  \cmplx{V}_2^*
    \cdots \cmplx{V}_n^*$\\[1ex]
    $(\cmplx{V}_1 / \cmplx{V}_2)^* = \cmplx{V}_1^* / \cmplx{V}_2^*$
    \item[$|\cmplx{V}|$] M\`{o}dul d'una variable complexa.
    Es compleixen les relacions:\\[1ex]
      $\cmplx{V}\cmplx{V}^* = |\cmplx{V}|^2$\\[1ex]
      $|\cmplx{V}_1 \cmplx{V}_2 \cdots \cmplx{V}_n| =
       |\cmplx{V}_1| |\cmplx{V}_2| \cdots |\cmplx{V}_n|$\\[1ex]
       $|\cmplx{V}_1 / \cmplx{V}_2| = |\cmplx{V}_1| / |\cmplx{V}_2|$\\[1ex]
      $|\cmplx{V}_1+\cmplx{V}_2+\cdots+\cmplx{V}_n| \leq
      |\cmplx{V}_1| + |\cmplx{V}_2| + \cdots  +|\cmplx{V}_n|$
    \item[$\arg(\cmplx{V})$] Argument (angle) d'una variable complexa.
    \item[$\Re(\cmplx{V})$] Part real d'una variable complexa.
    \item[$\Im(\cmplx{V})$] Part imagin\`{a}ria d'una variable complexa.
    \item[$\ju$] La unitat imagin\`{a}ria, definida com:
    $\ju\equiv\sqrt{-1}$\index{j}
    \item[$A+\ju B$] Expressi\'{o} cartesiana (part real i part
    imagin\`{a}ria) d'una variable complexa.
    \item[$Z_{\measuredangle \delta}$] Expressi\'{o} polar (m\`{o}dul i argument) d'una variable
    complexa. Les relacions entre $A, B, Z$ i $\delta$ s\'{o}n:\\[1ex]
    $Z=+\sqrt{A^2+B^2},\quad\delta=\arctan{\dfrac{B}{A}},\quad
    A=Z\cos\delta,\quad B=Z\sin\delta$
    \item[$Z\,\eu^{\ju\delta}$] Expressi\'{o} d'Euler\index{Euler} d'una variable complexa, definida com:
     $Z\,\eu^{\ju\delta} \equiv Z(\cos\delta+\ju\sin\delta)$
    \item[$\boldsymbol{V}$] Una matriu real o un vector real.
    \item[$\boldsymbol{V}^{-1}$] Matriu inversa d'una matriu real.
    \item[$\transp{\boldsymbol{V}}$] Matriu transposada d'una matriu real o vector
    transposat d'un vector real.
    \item[$\boldsymbol{V}(n)$] Element $n$-\`{e}sim d'un vector real.
    \item[$\boldsymbol{V}(m,n)$] Element de la fila $m$ i columna $n$ d'una matriu real.
    \item[$\mcmplx{V}$] Una matriu complexa o un vector complex.
    \item[$\mcmplx{V}^{-1}$] Matriu inversa d'una matriu complexa.
    \item[$\transp{\mcmplx{V}}$] Matriu transposada d'una matriu complexa o vector
    transposat d'un vector complex.
    \item[$\mcmplx{V}^*$] Matriu conjugada d'una matriu complexa o vector
    conjugat d'un vector complex.
    \item[$\hermit{V}$] Matriu conjugada transposada d'una matriu complexa o vector
    conjugat transposat d'un vector complex, definit com: $\hermit{V} \equiv
    \transp{(\mcmplx{V}^*)}$.
    \item[$\mcmplx{V}(n)$] Element $n$-\`{e}sim d'un vector complex.
    \item[$\mcmplx{V}(m,n)$] Element de la fila $m$ i columna $n$ d'una matriu complexa.
\end{list}

Pel que fa als sentits assignats a les fletxes que representen les
tensions i els corrents, en els diversos circuits el\`{e}ctrics que
apareixen en aquest llibre, s'utilitza la convenci\'{o} seg\"{u}ent:

\begin{list}{}
{\setlength{\labelwidth}{15mm} \setlength{\leftmargin}{25mm}
\setlength{\labelsep}{10mm}}
    \item[$\begin{CD} @>U>> \end{CD}$] Tensi\'{o} cont\'{\i}nua; la fletxa indica el sentit
    de la caiguda de tensi\'{o}, \'{e}s a dir, va del nus positiu al nus negatiu.
    \item[$\begin{CD} @>I>> \end{CD}$] Corrent
    continu; la fletxa indica el sentit  assignat com a positiu a la
    intensitat.
    \item[$\begin{CD} @>\cmplx{U}>> \end{CD}$] Tensi\'{o} alterna; la fletxa indica el
    sentit assignat com a positiu a la caiguda de tensi\'{o}, quan el nus d'origen de la fletxa
    t\'{e} un potencial  m\'{e}s positiu que el nus de destinaci\'{o}.
    \item[$\begin{CD} @>\cmplx{I}>> \end{CD}$] Corrent altern; la fletxa
    indica el sentit  assignat com a positiu a la
    intensitat.
\end{list}

Els s\'{\i}mbols que representes els diferents conjunts de nombres s\'{o}n:

\begin{list}{}
{\setlength{\labelwidth}{15mm} \setlength{\leftmargin}{25mm}
\setlength{\labelsep}{10mm}}
    \item[$\mathbb{Z}$] Nombres enters: $\ldots,-2,-1,0,1,2,\ldots$
    \item[$\mathbb{N}$, $\mathbb{Z}^+$] Nombres enters positius
    (naturals): $1,2,3,4\ldots$
    \item[$\mathbb{Z}^*$] Nombres enters no negatius: $0,1,2,3,4\ldots$
    \item[$\mathbb{Z}^-$] Nombres enters negatius: $-1,-2,-3,-4\ldots$
    \item[$\mathbb{Q}$] Nombres racionals.
    \item[$\mathbb{R}$] Nombres reals.
    \item[$\mathbb{R}^+$] Nombres reals positius.
    \item[$\mathbb{R}^-$] Nombres reals negatius.
    \item[$\mathbb{C}$] Nombres complexos.
\end{list}
\index{n@$\mathbb{N}$} \index{z@$\mathbb{Z}$}
\index{z@$\mathbb{Z^+}$} \index{z@$\mathbb{Z^-}$}
\index{z@$\mathbb{Z^*}$} \index{q@$\mathbb{Q}$}
\index{r@$\mathbb{R}$} \index{r@$\mathbb{R^+}$}
\index{r@$\mathbb{R^-}$} \index{c@$\mathbb{C}$}
