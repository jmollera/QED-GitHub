\chapter{Constants Físiques}\label{sec:const_fis} \index{constants físiques}

\section{Taula de valors}

En la Taula \vref{taula:Const-Fis} es pot veure una recopilació de
constants físiques; les xifres entre parèntesis que hi apareixen representen l'error absolut del valor.

Els valors de les constants d'aquesta taula són els recomanats
l'any 2014 pel «Committee on Data for Science and Technology»
(CODATA), un comitè científic de l'«International Council
for Science».

Podeu trobar  més informació a
les adreces: \href{http://www.codata.org/}{www.codata.org} i \href{http://physics.nist.gov/cuu/Constants/}{physics.nist.gov/cuu/Constants}.\index{CODATA}
\index{NIST}


\begin{longtable}{lcll}
   \caption{\label{taula:Const-Fis} Constants físiques}\\
   \toprule[1pt]
   Magnitud & Símbol & Valor & Error relatiu\\
   \midrule
   \endfirsthead
   \caption[]{Constants físiques (\emph{ve de la pàgina anterior})} \\
   \toprule[1pt]
   Magnitud & Símbol & Valor & Error relatiu\\
   \midrule
   \endhead
   \midrule
   \multicolumn{4}{r}{\sffamily\bfseries\color{NavyBlue}(\emph{continua a la pàgina següent})}
   \endfoot
   \endlastfoot
   velocitat de la llum  & $c$, $c_0$ & \SI{299792458}{m/s} & exacte\\
   en el buit & & & \\[0.5em]
   constant magnètica & $\mu_0$ & \SI{4 \piup e-7}{N/A^2} & exacte \\[0.5em]
   constant elèctrica: $1/(\mu_0 c^2)$ & $\epsilon_0$ & \SI{8,854187817\dots e-12}{F/m} & exacte \\[1em]
    impedància característica  & $Z_0$ &  \SI{376,730313461\dots}{\ohm} & exacte\\
    del buit: $\sqrt{\mu_0/\epsilon_0}=\mu_0 c$ & & &  \\[0.5em]
    atmosfera estàndard  & -- & \SI{101325}{Pa} & exacte \\[0.5em]
    acceleració de la gravetat & $g\ped{n}$ & \SI{9,80665}{m/s^2} & exacte \\
    estàndard & & & \\[0.5em]
    massa molar del carboni-12 & $M({}^{12}\mathrm{C})$ & \SI{12e-3}{kg/mol} & exacte \\[0.5em]
    constant gravitacional & $G$ &   \SI{6,67408(31) e-11}{m^3/(kg.s^2)} & \num{4,7e-5} \\
     de Newton & & & \\[0.5em]
    constant de Planck & $h$ & \SI{6,626070040(81) e-34}{J.s} & \num{1,2e-8} \\[0.5em]
    constant de Planck  & $\hbar$ & \SI{1,054571800(13) e-34}{J.s} & \num{1,2e-8} \\
    reduïda: $h/(2\piup)$ & & & \\[0.5em]
    càrrega elemental & $e$ & \SI{1,6021766208(98) e-19}{C} & \num{6,1e-9} \\[0.5em]
    massa de l'electró & $m\ped{e}$ & \SI{9,10938356(11) e-31}{kg} & \num{1,2e-8} \\[0.5em]
    massa del  protó & $m\ped{p}$ & \SI{1,672621898(21) e-27}{kg} & \num{1,2e-8} \\[0.5em]
    massa del neutró & $m\ped{n}$ & \SI{1,674927471(21) e-27}{kg} & \num{1,2e-8} \\[0.5em]
    massa del deuteri & $m\ped{d}$ & \SI{3,343583719(41) e-27}{kg} & \num{1,2e-8} \\[0.5em]
    massa del triti & $m\ped{t}$ & \SI{5,007356665(62) e-27}{kg} & \num{1,2e-8} \\[0.5em]
    massa de la partícula $\alphaup$ & $m_\alphaup$ & \SI{6,644657230(82) e-27}{kg} & \num{1,2e-8} \\[0.5em]
    radi de Bohr: $4\piup\epsilon_0\hbar^2/(m\ped{e}e^2)$ & $a_0$ & \SI{0,52917721067(12) e-10}{m} & \num{2,3e-10} \\[0.5em]
    longitud d'ona Compton:  & $\lambda\ped{C}$ & \SI{2,4263102367(11) e-12}{m} & \num{4,5e-10} \\
    $h/(m\ped{e} c)$ & & & \\[0.5em]
    número d'Avogadro & $N\ped{A}$, $L$ & \SI{6,022140857(74) e23}{mol^{-1}} & \num{1,2e-8} \\[0.5em]
    constant molar dels gasos & $R$ & \SI{8,3144598(48)}{J/(mol.K)} & \num{5,7e-7} \\[0.5em]
    constant de Faraday: $ e N\ped{A}$ & $F$ & \SI{96485,33289(59)}{C/mol} & \num{6,2e-9} \\[0.5em]
    constant de Boltzmann: & $k$ & \SI{1,38064852(79)e-23}{J/K} & \num{5,7e-7} \\
    $R/N\ped{A}$ & & & \\[0.5em]
    constant d'Stefan-Boltzmann:  & $\sigma$ & \SI{5,670367(13)e-8}{W/(m^2.K^4)} & \num{2,3e-6} \\
    $\piup^2 k^4 / (60\, \hbar^3 c^2)$ & & & \\[0.5em]
    %freqüència de la radiació de  & $\Deltaup\nu\ped{Cs}$ & \SI{9192631770}{Hz} & exacte\\
%    transició hiperfina del cesi & & & \\[0.5em]
%    eficiència lluminosa & $K\ped{cd}$ & \SI{683}{lm/W} & exacte \\[0.5em]
   \bottomrule[1pt]
\end{longtable}
\index{velocitat de la llum en el buit}  \index{constant!magnètica}
\index{constant!elèctrica} \index{impedància característica del
buit} \index{atmosfera estàndard} \index{acceleració!de la gravetat
estàndard} \index{massa!molar del carboni-12}
\index{constant!gravitacional de Newton} \index{constant!de Planck}
\index{constant!de Planck redu\"{i}da} \index{constant!de Faraday}
\index{carrega elemental@càrrega elemental} \index{massa!de
l'electró} \index{massa!del protó} \index{massa!del neutró}
\index{massa!del deuteri} \index{massa!de la partícula $\alpha$}
\index{numero d'Avogadro@número d'Avogadro} \index{constant!molar
dels gasos} \index{constant!de Boltzmann}
\index{constant!d'Stefan-Boltzmann} \index{radi de Bohr}\index{longitud d'ona Compton}
\index{$\mu_0$} \index{$\epsilon_0$} \index{c@$c$} \index{c@$c_0$}\index{atm}
\index{gn@$g\ped{n}$} \index{Z0@$Z_0$} \index{F@$F$}
\index{m@$M({}^{12}\mathrm{C})$} \index{G@$G$} \index{h@$h$}
\index{h@$\hbar$} \index{e@$e$} \index{me@$m\ped{e}$}
\index{mp@$m\ped{p}$} \index{mn@$m\ped{n}$} \index{m@$m\ped{d}$}
\index{ma@$m_\alpha$} \index{NA@$N\ped{A}$} \index{L@$L$}\index{R@$R$}
\index{k@$k$} \index{$\sigma$} \index{a0@$a_0$} \index{$\lambda\ped{C}$}

\section{Error absolut i relatiu}\label{err_abs_rel}

Tal com s'ha dit al principi, les xifres entre parèntesis indiquen l'error absolut del valor que les precedeix. El nombre de xifres entre parèntesis determina la posició decimal d'aquest error; per exemple, en el cas de la  massa de l'electró tenim:
\index{error!absolut}\index{error!relatiu}
\[
    m\ped{e} = \SI{9,10938356(11) e-31}{kg}
\]

Les dues xifres entre parèntesis, 11, determinen que la posició decimal de l'error absolut ha de correspondre a la posició de les dues últimes xifres, 56, del valor; l'error absolut $\epsilon$  és doncs:
\[
    \epsilon = \SI{0,00000011 e-31}{kg}
\]

Per tant, el valor de la massa de l'electró també es pot escriure's així:
 \[
    m\ped{e} = \SI[separate-uncertainty]{9,10938356(11) e-31}{kg}
\]

Finalment, l'error relatiu $\epsilon\ped{r}$ es calcula dividint l'error absolut pel valor sense l'error:
\[
    \epsilon\ped{r} = \frac{\SI{0,00000011 e-31}{kg}}{\SI{9,10938356 e-31}{kg}} =   \num{1,2e-8}
\]
