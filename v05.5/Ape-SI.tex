\chapter{Sistema Internacional d'Unitats (SI)} \index{sistema internacional d'unitats}

S'expliquen a continuaci\'{o} q\"{u}estions relacionades amb el sistema
internacional d'unitats (SI), el qual est\`{a} definit pel {"<}Bureau
International des Poids et Mesures{">} (\textsf{BIPM}). S'han utilitzat les publicacions m\'{e}s recents (any 2006) d'aquest organisme; podeu trobar m\'{e}s informaci\'{o} a les seg\"{u}ents adreces del \textsf{BIPM}: \href{http://www.bipm.org/}{www.bipm.org} i
\href{http://www.bipm.org/en/si/si_brochure/}{www.bipm.org/en/si/si\_brochure}.\index{BIPM}

El {"<}National Institute of Standards and Technology{">} (\textsf{NIST}), tamb\'{e} t\'{e} informaci\'{o} referent al sistema
internacional d'unitats, a l'adre\c{c}a: \href{http://www.nist.gov/pml/div684/fcdc/si-units.cfm}{www.nist.gov/pml/div684/fcdc/si-units.cfm}.\index{NIST}

\section{Unitats fonamentals de l'SI}
\index{sistema internacional d'unitats!unitats fonamentals}

En la Taula \vref{taula:SI-fonamentals} es poden veure les unitats
fonamentals del sistema internacional d'unitats.

\vspace{1cm}

\begin{longtable}[h]{llc}
   \caption{\label{taula:SI-fonamentals} Unitats fonamentals de l'SI}\\
   \toprule[1pt]
    Magnitud & Unitat & S\'{\i}mbol \\
   \midrule
   \endfirsthead
   \caption[]{Unitats fonamentals de l'SI (\emph{ve de la p\`{a}gina anterior})}\\
   \toprule[1pt]
    Magnitud & Unitat & S\'{\i}mbol \\
   \midrule
   \endhead
   \midrule
   \multicolumn{2}{r}{(\emph{continua a la p\`{a}gina seg\"{u}ent})}
   \endfoot
   \endlastfoot
   longitud & metre & m \\
   massa & kilogram\footnote{Ortogr\`{a}ficament, {"<}kilogram{">} i  {"<}quilogram{">} s\'{o}n correctes i equivalents, segons el DIEC2 {"<}Diccionari de la llengua catalana, 2a edici\'{o} (2007){">}} & kg \\
   temps & segon & s\\
   intensitat de corrent el\`{e}ctric & ampere & A \\
   temperatura termodin\`{a}mica & kelvin & K\\
   quantitat de mat\`{e}ria & mol & mol \\
   intensitat lluminosa & candela &  cd \\
   \bottomrule[1pt]
\end{longtable}
\index{metre} \index{kilogram} \index{segon} \index{amper}
\index{kelvin} \index{mol} \index{candela} \index{longitud}
\index{massa} \index{temps} \index{intensitat de corrent el\`{e}ctric}
\index{temperatura!termodin\`{a}mica} \index{quantitat de mat\`{e}ria}
\index{intensitat lluminosa} \index{m} \index{kg} \index{s}
\index{A} \index{K} \index{cd}

Es presenten a continuaci\'{o} de forma breu, les definicions
d'aquestes unitats fonamentals; entre par\`{e}ntesis s'indica l'any
que la {"<}Conf\'{e}rence G\'{e}n\'{e}rale des Poids et Mesures{">} les va posar en
vigor.
\pagebreak

\begin{list}{}
   {\setlength{\labelwidth}{22mm} \setlength{\leftmargin}{22mm} \setlength{\labelsep}{2mm}}
   \item[\textbf{metre}:] \'{E}s la longitud de la traject\`{o}ria recorreguda per la llum
   en el buit, durant una durada de 1/299792458 de segon. (1983).
   \item[\textbf{kilogram}:] \'{E}s la massa del prototip internacional del kilogram, fet d'un aliatge de plat\'{\i}-iridi i
    conservat al \textsf{BIPM}, a S\`{e}vres, Fran\c{c}a. (1901).
   \item[\textbf{segon}:] \'{E}s la durada de 9192631770 per\'{\i}odes de la
   radiaci\'{o} corresponent a la transici\'{o} entre els dos nivells
  hiperfins de l'estat fonamental de l'\`{a}tom de cesi-133. (1967).
   \item[\textbf{ampere}:] \'{E}s la intensitat d'un corrent constant,
   que mantinguda en dos conductors para{\l.l}els rectilinis de longitud
   infinita, de secci\'{o} transversal negligible, i situats a una
   dist\`{a}ncia l'un de l'altre d'un metre en el buit, produeix entre
   aquests dos conductors  una for\c{c}a igual a $2\times 10^{-7}$ newton per metre de longitud. (1948).
   \item[\textbf{kelvin}:] \'{E}s la fracci\'{o} 1/273,16 de la temperatura
   termodin\`{a}mica corresponent al punt triple de l'aigua. (1967).
   \item[\textbf{mol}:] \'{E}s la quantitat de mat\`{e}ria d'un sistema que cont\'{e} tantes
   entitats elementals com \`{a}toms hi ha en 0,012\unit{kg} de carboni-12. (1971).
   \item[\textbf{candela}:] \'{E}s la intensitat lluminosa, en una direcci\'{o} determinada,
   d'una font que emet radiaci\'{o} monocrom\`{a}tica de freq\"{u}\`{e}ncia $540\times 10^{12}\unit{hertz}$, i
   que t\'{e} una intensitat radiant en aquesta direcci\'{o} de 1/683 watt per estereoradiant. (1979).
\end{list}


\section{Prefixes de l'SI}
\index{sistema internacional d'unitats!prefixes}

En la Taula \vref{taula:SI-prefixes} es presenta una llista, amb els
prefixes que es poden anteposar a les unitats del sistema
internacional d'unitats, per tal de formar els seus m\'{u}ltiples i
subm\'{u}ltiples.


\begin{longtable}[h]{llccllc}
   \caption{\label{taula:SI-prefixes} Prefixes de  l'SI}\\
   \toprule[1pt]
   \multicolumn{3}{c}{M\'{u}ltiples} & & \multicolumn{3}{c}{Subm\'{u}ltiples}\\
   \cmidrule(rl){1-3} \cmidrule(rl){5-7}
   factor & nom & s\'{\i}mbol & & factor & nom & s\'{\i}mbol\\
   \midrule
   \endfirsthead
   \caption[]{Prefixes de  l'SI (\emph{ve de la p\`{a}gina  anterior})}\\
   \toprule[1pt]
    \multicolumn{3}{c}{M\'{u}ltiples} & & \multicolumn{3}{c}{Subm\'{u}ltiples}\\
   \cmidrule(rl){1-3} \cmidrule(rl){5-7}
   factor & nom & s\'{\i}mbol & & factor & nom & s\'{\i}mbol\\
   \midrule
   \endhead
   \midrule
   \multicolumn{7}{r}{(\emph{continua a la p\`{a}gina seg\"{u}ent})}
   \endfoot
   \endlastfoot
    $10^{24}$ &  yotta & Y & & $10^{-24}$ & yocto & y \\
    $10^{21}$ &  zetta & Z & & $10^{-21}$ & zepto & z \\
    $10^{18}$ &  exa & E & & $10^{-18}$ & atto & a \\
    $10^{15}$ &  peta & P & & $10^{-15}$ & femto & f \\
    $10^{12}$ &  tera & T & & $10^{-12}$ & pico & p \\
    $10^{9}$ &  giga & G & & $10^{-9}$ & nano & n \\
    $10^{6}$ &  mega & M & & $10^{-6}$ & micro & $\micro$ \\
    $10^{3}$ &  kilo\footnote{Ortogr\`{a}ficament, {"<}kilo{">} i  {"<}quilo{">} s\'{o}n correctes i equivalents, segons el DIEC2 {"<}Diccionari de la llengua catalana, 2a edici\'{o} (2007){">}} & k & & $10^{-3}$ & mi{\l.l}i & m \\
    $10^{2}$ &  hecto & h & & $10^{-2}$ & centi & c \\
    $10^{1}$ &  deca & da & & $10^{-1}$ & deci & d \\
   \bottomrule[1pt]
\end{longtable}
\index{yotta} \index{zetta} \index{exa} \index{peta} \index{tera} \index{giga} \index{mega}
\index{kilo} \index{hecto} \index{deca} \index{deci} \index{centi} \index{mili} \index{micro}
\index{nano} \index{pico} \index{femto} \index{atto} \index{zepto} \index{yocto}



\section{Unitats derivades de l'SI amb noms i s\'{\i}mbols propis}
\index{sistema internacional d'unitats!unitats derivades amb noms i s\'{\i}mbols propis}

De forma convenient, s'ha donat noms i s\'{\i}mbols propis a algunes unitats derivades de les fonamentals; en la Taula \vref{taula:SI-derivades} es mostren aquestes unitats derivades de l'SI.


\begin{longtable}[h]{llclc}
   \caption{\label{taula:SI-derivades} Unitats derivades de
   l'SI amb noms i s\'{\i}mbols propis}\\
   \toprule[1pt]
    \multirow{2}{15mm}{\rule{0mm}{6mm}Magnitud} & \multirow{2}{15mm}{\rule{0mm}{6mm}Unitat}  &
    \multirow{2}{15mm}{\rule{0mm}{6mm}S\'{\i}mbol}  & \multicolumn{2}{c}{Equival\`{e}ncia en unitats SI}\\
    \cmidrule(rl){4-5}
    &  &   & fonamentals & altres\\
   \midrule
   \endfirsthead
   \caption[]{Unitats derivades de l'SI amb noms i s\'{\i}mbols propis (\emph{ve de la p\`{a}gina
   anterior})}\\
   \toprule[1pt]
    \multirow{2}{15mm}{\rule{0mm}{6mm}Magnitud} & \multirow{2}{15mm}{\rule{0mm}{6mm}Unitat}  &
    \multirow{2}{15mm}{\rule{0mm}{6mm}S\'{\i}mbol}  & \multicolumn{2}{c}{Equival\`{e}ncia en unitats SI}\\
    \cmidrule(rl){4-5}
    &  &  & fonamentals & altres\\
   \midrule
   \endhead
   \midrule
   \multicolumn{5}{r}{(\emph{continua a la p\`{a}gina seg\"{u}ent})}
   \endfoot
   \endlastfoot
   angle pla & radiant & rad   & \unit{m/m} & 1\\
   angle s\`{o}lid & estereoradiant & sr & \unit{m^2/m^2}  & 1 \\
   freq\"{u}\`{e}ncia & hertz & Hz & \unit{s^{-1}} & --- \\
   for\c{c}a & newton & N & \unit{m\cdot kg\cdot s^{-2}} & --- \\
   pressi\'{o} & pascal & Pa  & \unit{m^{-1}\cdot kg\cdot s^{-2}} & \unit{N/m^2} \\
   energia, treball & joule & J & \unit{m^2\cdot kg\cdot s^{-2}} & \unit{N\cdot m}\\
   pot\`{e}ncia & watt & W & \unit{m^2\cdot kg\cdot s^{-3}}  & \unit{J/s}\\
   c\`{a}rrega el\`{e}ctrica & coulomb & C  & \unit{s\cdot A} &  ---\\
   potencial el\`{e}ctric & volt & V & \unit{m^2\cdot kg\cdot s^{-3}\cdot A^{-1}}  & \unit{W/A}\\
   capacitat el\`{e}ctrica & farad & F   & \unit{m^{-2}\cdot kg^{-1}\cdot s^4\cdot A^2}& \unit{C/V}\\
   resist\`{e}ncia el\`{e}ctrica & ohm &  \unit{\ohm}  & \unit{m^2\cdot kg\cdot s^{-3}\cdot A^{-2}} & \unit{V/A}\\
   conduct\`{a}ncia el\`{e}ctrica & siemens &  S  & \unit{m^{-2}\cdot kg^{-1}\cdot s^3\cdot A^2} & \unit{A/V}\\
   flux magn\`{e}tic & weber &  Wb  & \unit{m^2\cdot kg\cdot s^{-2}\cdot A^{-1}} & \unit{V\cdot s}\\
   densitat de flux magn\`{e}tic & tesla &  T  & \unit{kg\cdot s^{-2}\cdot A^{-1}} & \unit{Wb/m^2}\\
   induct\`{a}ncia & henry &  H  & \unit{m^2\cdot kg\cdot s^{-2}\cdot A^{-2}} & \unit{Wb/A}\\
   temperatura Celsius & grau Celsius &  \celsius & \unit{K} & --- \\
   flux llumin\'{o}s & lumen & lm  & \unit{cd}& \unit{cd\cdot sr}\\
   i{\l.l}uminaci\'{o} & lux & lx & \unit{m^{-2}\cdot cd} & \unit{lm/m^2} \\
   activitat  d'un radion\'{u}clid & becquerel & Bq& \unit{s^{-1}} & --- \\
   dosi absorbida & gray & Gy  & \unit{m^2\cdot s^{-2}}& \unit{J/kg}\\
   dosi equivalent & sievert & Sv  & \unit{m^2\cdot s^{-2}}& \unit{J/kg}\\
   activitat catal\'{\i}tica & katal & kat & \unit{s^{-1}\cdot mol} & ---\\
   \bottomrule[1pt]
\end{longtable}
\index{radiant} \index{estereoradiant} \index{hertz} \index{newton}
\index{pascal} \index{joule} \index{watt} \index{coulomb}
\index{volt} \index{farad} \index{ohm} \index{siemens} \index{weber}
\index{tesla} \index{henry} \index{lumen} \index{lux}
\index{becquerel} \index{gray} \index{sievert} \index{grau Celsius}\index{katal}
\index{angle pla}  \index{angle s\`{o}lid} \index{freq\"{u}\`{e}ncia}
\index{for\c{c}a} \index{pressi\'{o}} \index{energia} \index{pot\`{e}ncia}
\index{carrega electrica@c\`{a}rrega el\`{e}ctrica} \index{potencial
el\`{e}ctric} \index{capacitat} \index{resist\`{e}ncia} \index{conduct\`{a}ncia}
\index{flux magn\`{e}tic} \index{densitat de flux magn\`{e}tic}
\index{induct\`{a}ncia} \index{temperatura!Celsius} \index{flux
llumin\'{o}s} \index{iluminacio@i{\l.l}uminaci\'{o}} \index{activitat  d'un
radion\'{u}clid} \index{dosi absorbida}  \index{dosi equivalent}\index{activitat catalitica@activitat catal\'{\i}tica}
 \index{rad} \index{sr} \index{Hz} \index{N}
\index{Pa} \index{J} \index{W} \index{C} \index{V} \index{F}
\index{$\Omega$} \index{S} \index{Wb} \index{T} \index{H}
\index{\celsius} \index{lm} \index{lx} \index{Bq} \index{Gy}
\index{Sv}\index{kat}

\section{Altres unitats derivades de l'SI}
\index{sistema internacional d'unitats!altres unitats derivades}

Les unitats fonamentals i les unitats amb noms i s\'{\i}mbols propis poden combinar-se entre si per expressar noves unitats derivades; en la Taula \vref{taula:SI-derivades-exemples} s'en mostren alguns exemples.

\begin{longtable}[h]{lcl}
   \caption{\label{taula:SI-derivades-exemples} Exemples d'altres unitats derivades de
   l'SI}\\
   \toprule[1pt]
    Magnitud &  S\'{\i}mbol & Equival\`{e}ncia en unitats fonamentals SI\\
   \midrule
   \endfirsthead
   \caption[]{Exemples d'altres unitats derivades de l'SI (\emph{ve de la p\`{a}gina
   anterior})}\\
   \toprule[1pt]
    Magnitud &  S\'{\i}mbol & Equival\`{e}ncia en unitats fonamentals SI\\
   \midrule
   \endhead
   \midrule
   \multicolumn{3}{r}{(\emph{continua a la p\`{a}gina seg\"{u}ent})}
   \endfoot
   \endlastfoot
   viscositat din\`{a}mica &  \unit{Pa\cdot s}& \unit{m^{-1}\cdot kg\cdot s^{-1}} \\
   moment d'una for\c{c}a & \unit{N\cdot m} & \unit{m^2\cdot kg\cdot s^{-2}} \\
   tensi\'{o} superficial &  \unit{N/m} &   \unit{kg\cdot s^{-2}} \\
   velocitat angular & \unit{rad/s} & \unit{m\cdot m^{-1}\cdot s^{-1} = s^{-1}} \\
   acceleraci\'{o} angular & \unit{rad/s^2} & \unit{m\cdot m^{-1}\cdot s^{-2} = s^{-2}} \\
   densitat de flux de calor & \unit{W/m^2} & \unit{kg\cdot s^{-3}} \\
   entropia & \unit{J/K} & \unit{m^2\cdot kg\cdot s^{-2}\cdot K^{-1}} \\
   entropia espec\'{\i}fica & \unit{J/(kg\cdot K)} &\unit{m^2\cdot s^{-2}\cdot K^{-1}} \\
   energia espec\'{\i}fica & \unit{J/kg} & \unit{m^2\cdot s^{-2}} \\
   conductivitat t\`{e}rmica & \unit{W/(m\cdot K)} & \unit{m\cdot kg\cdot s^{-3}\cdot K^{-1}} \\
   densitat d'energia & \unit{J/m^3} & \unit{m^{-1}\cdot kg\cdot s^{-2}} \\
   intensitat de camp el\`{e}ctric & \unit{V/m}& \unit{m\cdot kg\cdot s^{-3}\cdot A^{-1}}  \\
   densitat de c\`{a}rrega el\`{e}ctrica & \unit{C/m^3} & \unit{m^{-3}\cdot s\cdot A} \\
   densitat de flux el\`{e}ctric & \unit{C/m^2} & \unit{m^{-2}\cdot s\cdot A }\\
   permitivitat &  \unit{F/m}& \unit{m^{-3}\cdot kg^{-1}\cdot s^4\cdot A^2} \\
   permeabilitat &  \unit{H/m} & \unit{m\cdot kg\cdot s^{-2}\cdot A^{-2}} \\
   energia molar & \unit{J/mol} & \unit{m^2\cdot kg\cdot s^{-2}\cdot mol^{-1}} \\
   entropia molar& \unit{J/(mol\cdot K)} & \unit{m^2\cdot kg\cdot s^{-2}\cdot K^{-1}\cdot mol^{-1}} \\
   exposici\'{o} (raigs x i $\gammaup$) & \unit{C/kg} & \unit{kg^{-1}\cdot s\cdot A} \\
   tassa de dosi absorbida & \unit{Gy/s} & \unit{m^2\cdot s^{-3}}\\
   intensitat radiant & \unit{W/sr} & \unit{m^4\cdot m^{-2}\cdot kg\cdot s^{-3} = m^2\cdot kg\cdot s^{-3}} \\
   radi\`{a}ncia & \unit{W/(m^2\cdot sr)} & \unit{m^2\cdot m^{-2}\cdot kg\cdot s^{-3} = kg\cdot s^{-3}} \\
   concentraci\'{o} d'activitat catal\'{\i}tica &  \unit{kat/m^3} & \unit{m^{-3}\cdot s^{-1}\cdot mol}\\
    \bottomrule[1pt]
\end{longtable}
\index{viscositat dinamica@viscositat din\`{a}mica}\index{moment d'una for\c{c}a}\index{tensi\'{o} superficial}
\index{velocitat angular}\index{acceleracio angular@acceleraci\'{o} angular}\index{densitat de flux de calor}\index{entropia}\index{entropia especifica@entropia espec\'{\i}fica}\index{conductivitat termica@conductivitat t\`{e}rmica}\index{densitat d'energia}\index{intensitat de camp electric@intensitat de camp el\`{e}ctric}\index{densitat de carrega electrica@densitat de c\`{a}rrega el\`{e}ctrica}\index{densitat de flux electric@densitat de flux el\`{e}ctric}
\index{permitivitat}\index{permeabilitat}\index{energia molar}\index{entropia molar}
\index{exposici\'{o}}\index{tassa de dosi absorbida}\index{intensitat radiant}\index{radiancia@radi\`{a}ncia}
\index{concentracio d'activitat catalitica@concentraci\'{o} d'activitat catal\'{\i}tica}



\section{Unitats i prefixes fora de l'SI}
\index{sistema internacional d'unitats!unitats fora de l'SI}

Hi ha una s\`{e}rie d'unitats que no formem part de l'SI per\`{o} que s\'{o}n d'\'{u}s com\'{u} en el camp cient\'{\i}fic, t\`{e}cnic o comercial, i que s\'{o}n usades freq\"{u}entment. En les taules seg\"{u}ents es recullen algunes d'aquestes unitats.

En la Taula \vref{taula:SI-altres-acceptades} es mostren les unitats fora de l'SI, l'\'{u}s de les quals s'accepta en conjunci\'{o} amb el Sistema Internacional d'Unitats, ja que s\'{o}n presents en la vida di\`{a}ria. S'espera que el seu \'{u}s continu\"{\i} de forma indefinida; cadascuna d'elles t\'{e} una definici\'{o} exacte en termes d'unitats de l'SI.

\begin{longtable}[h]{llcl}
   \caption{\label{taula:SI-altres-acceptades} Unitats fora de l'SI acceptades per a ser usades amb l'SI  }\\
   \toprule[1pt]
    Magnitud & Unitat &  S\'{\i}mbol & Valor en unitats SI\\
   \midrule
   \endfirsthead
   \caption[]{Unitats fora de l'SI acceptades per ser usades amb l'SI (\emph{ve de la p\`{a}gina
   anterior})}\\
   \toprule[1pt]
    Magnitud & Unitat &  S\'{\i}mbol & Valor en unitats SI\\
   \midrule
   \endhead
   \midrule
   \multicolumn{4}{r}{(\emph{continua a la p\`{a}gina seg\"{u}ent})}
   \endfoot
   \endlastfoot
   temps & minut &  \unit{min}& $1\unit{min} = 60\unit{s}$ \\
   temps & hora & \unit{h} & $1\unit{h} = 60\unit{min} = 3600\unit{s}$ \\
   temps & dia & \unit{d} & $1\unit{d} = 24\unit{h} = 86400\unit{s}$\\
   angle pla & grau &  \unit{\degree} &   $1\degree = (\piup/180)\unit{rad}$ \\
   angle pla & minut & \unit{'} & $1' = (1/60)\degree = (\piup/10800)\unit{rad}$ \\
   angle pla & segon & \unit{"} & $1" = (1/60)' = (\piup/648000)\unit{rad}$ \\
   \`{a}rea & hect\`{a}rea & \unit{ha} & $1\unit{ha} = 1\unit{hm^2} = 10^4\unit{m^2}$\\
   volum & litre\footnote{El s\'{\i}mbol {"<}L{">} es va adoptar posteriorment al s\'{\i}mbol {"<}l{">} per evitar la possible confusi\'{o} entre la lletra ela min\'{u}scula i  el n\'{u}mero 1.}& \unit{l},\unit{L} & $1\unit{l} = 1\unit{L} = 1\unit{dm^3} = 10^{-3}\unit{m^3}$ \\
   massa & tona\footnote{En el pa\"{\i}sos de parla anglesa aquesta unitat \'{e}s coneguda com a {"<}tona m\`{e}trica{">}.} & \unit{t} & $1\unit{t} =1000\unit{kg}$\\
   \bottomrule[1pt]
\end{longtable}
\index{minut}\index{hora}\index{dia}\index{grau}\index{segon}\index{litre}\index{tona}\index{hectarea@hect\`{a}rea}
\index{min}\index{h}\index{d}\index{$\degree$}\index{$'$}\index{$''$}\index{l}\index{L}\index{ha}\index{t}

En la Taula \vref{taula:SI-altres-experimentals} es mostren les unitats fora de l'SI, el valor de les quals s'obt\'{e} de forma experimental (les xifres entre par\`{e}ntesi representen la millor estimaci\'{o} coneguda actualment). Els valors de l'electronvolt i el dalton (unitat de massa at\`{o}mica unificada) s\'{o}n els recomanats
l'any 2010 pel {"<}Committee on Data for Science and Technology{">} (\textsf{CODATA}), i el valor de la unitat astron\`{o}mica \'{e}s el recomanat l'any 2010 per l'{"<}International Earth rotation and Reference systems Service{">} (\textsf{IERS}).\index{CODATA}\index{IERS}

\begin{longtable}[h]{llcl}
   \caption{\label{taula:SI-altres-experimentals} Unitats fora de l'SI obtingudes de forma experimental }\\
   \toprule[1pt]
    Magnitud & Unitat &  S\'{\i}mbol & Valor en unitats SI\\
   \midrule
   \endfirsthead
   \caption[]{Unitats fora de l'SI obtingudes de forma experimental (\emph{ve de la p\`{a}gina
   anterior})}\\
   \toprule[1pt]
    Magnitud & Unitat &  S\'{\i}mbol & Valor en unitats SI\\
   \midrule
   \endhead
   \midrule
   \multicolumn{4}{r}{(\emph{continua a la p\`{a}gina seg\"{u}ent})}
   \endfoot
   \endlastfoot
   energia & electronvolt & \unit{eV} & $1\unit{eV} = 1{,}602176565(35)\times 10^{-19}\unit{J}$ \\
   massa & dalton\footnote{El {"<}dalton{">} i la {"<}unitat de massa at\`{o}mica unificada{">} s\'{o}n dos noms alternatius d'una mateixa unitat.}& Da & $1\unit{Da} = 1{,}660538921(73)\times 10^{-27}\unit{kg}$\\
   massa & unitat de massa at\`{o}mica unificada\footnotemark[5] & u & $1\unit{u} =
    1{,}660538 921(73)\times 10^{-27}\unit{kg}$  \\
   longitud & unitat astron\`{o}mica &  \unit{ua }& $1\unit{ua} =  149597870{,}700\unit{km} \pm 3\unit{m}$ \\
\bottomrule[1pt]
\end{longtable}
\index{electronvolt}\index{unitat de massa atomica unificada@unitat de massa at\`{o}mica unificada}\index{unitat astronomica@unitat astron\`{o}mica}\index{dalton}\index{eV}\index{u}\index{ua}\index{Da}


En la Taula \vref{taula:SI-altres} es mostren altres unitats fora de l'SI utilitzades en diversos camps. Algunes d'aquestes unitats estan relacionades amb l'antic sistema CGS (cent\'{\i}metre-gram-segon).\index{CGS}

\begin{longtable}[h]{llcl}
   \caption{\label{taula:SI-altres} Altres unitats fora de l'SI}\\
   \toprule[1pt]
    Magnitud & Unitat &  S\'{\i}mbol & Valor en unitats SI\\
   \midrule
   \endfirsthead
   \caption[]{Altres unitats fora de l'SI (\emph{ve de la p\`{a}gina
   anterior})}\\
   \toprule[1pt]
    Magnitud & Unitat &  S\'{\i}mbol & Valor en unitats SI\\
   \midrule
   \endhead
   \midrule
   \multicolumn{4}{r}{(\emph{continua a la p\`{a}gina seg\"{u}ent})}
   \endfoot
   \endlastfoot
    pressi\'{o} & bar & \unit{bar} & $1\unit{bar} = 100\unit{kPa}$ \\
    pressi\'{o} & mi{\l.l}\'{\i}metre de mercuri & \unit{mmHg} & $1\unit{mmHg} \approx 133{,}322\unit{Pa}$ \\
    longitud & \`{a}ngstrom ({"<}\aa{}ngstr\"{o}m{">}) & \unit{\angs} & $1\unit{\angs} = 10^{-10}\unit{m}$\\
    dist\`{a}ncia & milla n\`{a}utica\footnote{No i ha acord internacional pel s\'{\i}mbol de la milla n\`{a}utica, a m\'{e}s d'{"<}M{">} tamb\'{e} s'utilitza {"<}NM{">}, {"<}Nm{">} i {"<}nmi{">}.} &  \unit{M} & $1\unit{M} = 1852\unit{m}$ \\
    \`{a}rea & barn & \unit{b} &  $1\unit{b} = 10^{-28}\unit{m^2}$\\
    velocitat & nus\footnote{No i ha acord internacional pel s\'{\i}mbol del nus, per\`{o} el s\'{\i}mbol {"<}kn{">} \'{e}s \`{a}mpliament usat.} & \unit{kn} & $1\unit{kn} = 1\unit{M/h} = \frac{1852}{3600}\unit{m/s}$ \\
    logaritme d'una relaci\'{o} & neper, bel, decibel\footnote{Aquestes unitats adimensionals s'utilitzen per expressar logaritmes de relacions entre quantitats. Per exemple, $n\unit{Np}$ fa refer\`{e}ncia a una relaci\'{o} del tipus $ln\frac{A_2}{A_1}= n$, i  $ m \unit{dB} =\frac{m}{10}\unit{B}$  fa refer\`{e}ncia a una relaci\'{o} del tipus $\log\frac{A_2}{A_1} =\frac{m}{10}$.} & \unit{Np},\unit{B},\unit{dB} & 1\\
    energia & erg & \unit{erg} & $1\unit{erg} = 10^{-7}\unit{J} $ \\
    for\c{c}a & dina & \unit{dyn} & $1\unit{dyn} = 10^{-5}\unit{N}$ \\
    viscositat din\`{a}mica & poise & \unit{P} & $1\unit{P} = 1\unit{dyn\cdot s/cm^2} = 0{,}1\unit{Pa\cdot s}$ \\
    viscositat cinem\`{a}tica & stokes & \unit{St} & $1\unit{St} = 1\unit{cm^2/s} = 10^{-4}\unit{m^2/s}$ \\
    lumin\`{a}ncia & stilb & \unit{sb} & $1\unit{sb} = 1\unit{cd/cm^2} = 10^4\unit{cd/m^2}$ \\
    i{\l.l}uminaci\'{o} & fot & \unit{ph} & $1\unit{ph} = 1\unit{cd\cdot sr/cm^2} = 10^4\unit{lx}$ \\
    acceleraci\'{o} & gal & \unit{Gal} & $1\unit{Gal} = 1\unit{cm/s^2} = 10^{-2}\unit{m/s^2}$ \\
    flux magn\`{e}tic & maxwell & \unit{Mx} & $1\unit{Mx} = 10^{-8}\unit{Wb}$ \\
    densitat de flux magn\`{e}tic & gauss & \unit{G} & $1\unit{G} = 10^{-4}\unit{T}$ \\
    camp magn\`{e}tic & oersted & \unit{Oe} & $1\unit{Oe} = \frac{1000}{4\piup}\unit{A/m}$ \\
\bottomrule[1pt]
\end{longtable}
\index{bar}\index{milimetre de mercuri@mi{\l.l}\'{\i}metre de mercuri}\index{angstrom@\aa{}ngstr\"{o}m}
\index{milla nautica@milla n\`{a}utica}\index{barn}\index{nus}\index{neper}\index{bel}\index{decibel}
\index{erg}\index{dina}\index{poise}\index{stokes}\index{stilib}\index{fot}\index{gal}\index{maxwell}
\index{gauss}\index{oersted}\index{mmHg}\index{A@$\angs$}\index{M}\index{NM}\index{Nm}\index{nmi}\index{b}
\index{kn}\index{Np}\index{B}\index{dB}\index{dyn}\index{P}\index{St}\index{sb}\index{ph}\index{Gal}
\index{Mx}\index{G}\index{Oe}


Finalment, en la Taula \vref{taula:simbol-inform} es mostren el s\'{\i}mbols d'unitats inform\`{a}tiques,  i en la Taula \vref{taula:prefix-inform} es mostren els prefixes de pot\`{e}ncies bin\`{a}ries que cal usar amb aquestes unitats.

Aquestes unitats i prefixes no pertanyen a l'SI, per\`{o} han estat adoptats en la norma internacional \textsf{CEI 60027-2} {"<}Letter symbols to be used in electrical technology -- Part 2: Telecommunications and electronics{">}.\index{CEI!60027-2}

\begin{longtable}[h]{>{\hspace{5mm}}cc}
   \caption{\label{taula:simbol-inform} Unitats inform\`{a}tiques}\\
   \toprule[1pt]
    Nom & S\'{\i}mbol \\
   \midrule
   \endfirsthead
   \caption[]{Unitats inform\`{a}tiques (\emph{ve de la p\`{a}gina anterior})}\\
   \toprule[1pt]
    Nom & S\'{\i}mbol \\
   \midrule
   \endhead
   \midrule
   \multicolumn{2}{r}{(\emph{continua a la p\`{a}gina seg\"{u}ent})}
   \endfoot
   \endlastfoot
   bit & bit    \\
   octet, byte & B   \\
   \bottomrule[1pt]
\end{longtable}
\index{bit}\index{byte}\index{B}

\begin{longtable}[h]{lcl}
   \caption{\label{taula:prefix-inform} Prefixes de pot\`{e}ncies bin\`{a}ries}\\
   \toprule[1pt]
    Nom & S\'{\i}mbol  & Factor \\
   \midrule
   \endfirsthead
   \caption[]{Prefixes de pot\`{e}ncies bin\`{a}ries (\emph{ve de la p\`{a}gina anterior})}\\
   \toprule[1pt]
    Nom & S\'{\i}mbol  & Factor \\
   \midrule
   \endhead
   \midrule
   \multicolumn{3}{r}{(\emph{continua a la p\`{a}gina seg\"{u}ent})}
   \endfoot
   \endlastfoot
   kibi & Ki   & $2^{10} = 1024$  \\
   mebi & Mi   & $2^{20} \approx 1{,}0486\times 10^6$ \\
   gibi & Gi   & $2^{30} \approx 1{,}0737\times 10^9$  \\
   tebi & Ti   & $2^{40} \approx 1{,}0995\times 10^{12}$ \\
   pebi & Pi   & $2^{50} \approx 1{,}1259\times 10^{15}$ \\
   exbi & Ei   & $2^{60} \approx 1{,}1529\times 10^{18}$ \\
   \bottomrule[1pt]
\end{longtable}
\index{kibi} \index{Ki} \index{mebi} \index{Mi} \index{gibi}  \index{Gi} \index{tebi} \index{Ti}
\index{pebi} \index{Pi} \index{exbi} \index{Ei}

Utilitzant aquests prefixes podem escriure per exemple:
\[1\unit{KiB} =2^{10}\unit{B} = 1024\unit{B}\]

El prefix {"<}k{">} de l'SI indica, en canvi, un altre valor:
\[1\unit{kB} =10^3\unit{B} = 1000\unit{B}\]

\section{Normes d'escriptura}
\index{sistema internacional d'unitats!normes d'escriptura}

Es presenten a continuaci\'{o} algunes normes aplicables a l'escriptura
de les unitats del sistema internacional d'unitats.

Despr\'{e}s de cadascuna de les explicacions es donen exemples correctes (precedits pel s\'{\i}mbol \textcolor{Green}{\ding{51}}) i exemples incorrectes (precedits pel s\'{\i}mbol \textcolor{Red}{\ding{55}}).

\begin{dinglist}{'167}

\item El prefix utilitzat per simbolitzar 1000 \'{e}s la lletra {"<}k{">} (min\'{u}scula).  La lletra {"<}K{">} (maj\'{u}scula) \'{e}s el s\'{\i}mbol del  kelvin; cal tenir en compte que {"<}\degree K{">}  no \'{e}s correcte. En canvi, el s\'{\i}mbol del grau Celsius \'{e}s {"<}\celsius{">}, ja que la lletra {"<}C{">} sola, \'{e}s el s\'{\i}mbol del coulomb.

\textcolor{Green}{\ding{51}} 6,9\unit{kV}, $100\unit{\celsius} = 373{,}15\unit{K}$

\textcolor{Red}{\ding{55}} 6,9\unit{KV}, 100 C = 373{,}15 \degree K

\item Els s\'{\i}mbols de les unitats no han d'anar seguits d'un punt, llevat que es trobin al final d'una oraci\'{o}, ja que no s\'{o}n
abreviatures.

\textcolor{Green}{\ding{51}} 25\unit{V}, 40\unit{A}

\textcolor{Red}{\ding{55}} 25 V.,  40\unit{A}.


\item Els s\'{\i}mbols de les unitats no canvien de forma en el plural, no han
d'utilitzar-se abreviatures ni han d'afegir-se o suprimir-se
lletres.

\textcolor{Green}{\ding{51}} 150\unit{kg}, 25\unit{m},  33\unit{cm^3}, 20\unit{s}

\textcolor{Red}{\ding{55}} 150 Kgs, 25~mts, 33 cc, 20\unit{seg}

\item No han de barrejar-se noms i s\'{\i}mbols d'unitats.

\textcolor{Green}{\ding{51}} 4 rad/s, 4 radiant per segon, 100 km/h, 100 quil\`{o}metre per hora

\textcolor{Red}{\ding{55}} 4 radiant/s, 4 rad/segon, 100 km/hora, 100 quil\`{o}metre/h

\item Els s\'{\i}mbols de les unitats s'escriuen a la dreta dels valors
num\`{e}rics, separats per un espai en blanc.

\textcolor{Green}{\ding{51}} 25\unit{V}, 40\unit{\celsius}, 20\unit{nF}

\textcolor{Red}{\ding{55}} 25V, 40\celsius, 20nF

 L'\'{u}nica excepci\'{o} \'{e}s la mesura d'angles en graus, minuts i segons; en aquest cas s'escriu el valor i la unitat tot junt.

\textcolor{Green}{\ding{51}} $15\degree{}~32'~8"$

\textcolor{Red}{\ding{55}} $15~\degree{}~32~'~8~"$

\item En el cas de s\'{\i}mbols d'unitats derivades, formats pel producte
d'altres unitats, el producte s'indicar\`{a} mitjan\c{c}ant un punt volat o
un espai en blanc.

\textcolor{Green}{\ding{51}} 24\unit{N\cdot m}, 24\unit{N\,m}

\textcolor{Red}{\ding{55}} 24 N-m, 24\unit{Nm}

Quan s'utilitza un espai en blanc, cal tenir en compte  l'ordre en qu\`{e} s'escriuen
les unitats, ja que algunes combinacions poden crear confusi\'{o} i
\'{e}s millor evitar-les, per exemple: 24\unit{N\,m} (24 newton metre) i
24\unit{m\,N} (24~metre newton) s\'{o}n expressions equivalents, per\`{o}
aquesta darrera pot ser confosa amb 24\unit{mN} (24 mi{\l.l}inewton).

\item En el cas de s\'{\i}mbols d'unitats derivades, formats per la divisi\'{o}
d'altres unitats, la divisi\'{o} s'indicar\`{a} mitjan\c{c}ant una l\'{\i}nia
inclinada o horitzontal, o mitjan\c{c}ant pot\`{e}ncies negatives.

\textcolor{Green}{\ding{51}} 100\unit{m\,/s}, 100\unit{\dfrac{m}{s}}, 100\unit{m\cdot s^{-1}}

\textcolor{Red}{\ding{55}} 100\unit{m\div s}

En el cas anterior, quan s'utilitza la l\'{\i}nia inclinada i hi ha m\'{e}s
d'una unitat en el denominador, aquestes unitats s'han d'escriure
entre par\`{e}ntesis.

\textcolor{Green}{\ding{51}} 5\unit{m\cdot kg\,/(s^3\cdot A)}

\textcolor{Red}{\ding{55}} 5\unit{m\cdot kg\,/s^3\cdot A}, 5\unit{m\cdot kg\,/s^3/A}

\item No ha de deixar-se cap espai en blanc entre el s\'{\i}mbol d'un prefixe i
el s\'{\i}mbol d'una unitat.

\textcolor{Green}{\ding{51}} 12\unit{pF}, 3\unit{GHz}

\textcolor{Red}{\ding{55}} 12\unit{p\,F}, 3\unit{G\,Hz}

\item El grup format pel s\'{\i}mbol d'un prefixe i el s\'{\i}mbol d'una unitat
esdev\'{e} un nou s\'{\i}mbol inseparable (formant un m\'{u}ltiple o subm\'{u}ltiple
de la unitat), i pot ser pujat a una pot\`{e}ncia positiva o negativa i
combinat amb altres s\'{\i}mbols.

\textcolor{Green}{\ding{51}} 20\unit{km^2}, 12\unit{kg\,/mm^2}

\textcolor{Red}{\ding{55}} 20\unit{(km)^2}, 12\unit{kg\,/(mm)^2}

\item Nom\'{e}s es permet un prefixe davant d'una unitat.

\textcolor{Green}{\ding{51}} 8\unit{nm}

\textcolor{Red}{\ding{55}} 8\unit{m\micro m}

\item No es permeten prefixes a\"{\i}llats.

\textcolor{Green}{\ding{51}} El nombre de part\'{\i}cules es de $5\times 10^6 /\unit{m^3}$

\textcolor{Red}{\ding{55}} El nombre de part\'{\i}cules es de $5 \unit{M\,/m^3}$

\item En el cas dels s\'{\i}mbols d'unitats derivades, formades per la divisi\'{o}
d'altres unitats, l'\'{u}s de prefixes en el numerador i denominador de
forma simult\`{a}nia pot causar confusi\'{o}, i \'{e}s preferible, per tant,
utilitzar una alta combinaci\'{o} d'unitats on nom\'{e}s el numerador o el
denominador tinguin prefix.

\textcolor{Green}{\ding{51}} 10\unit{MV/m}

\textcolor{Red}{\ding{55}}  10\unit{kV/mm}  (no \'{e}s incorrecte, per\`{o} el seu \'{u}s no es recomana)

\item De forma an\`{a}loga, el mateix \'{e}s aplicable als s\'{\i}mbols d'unitats
derivades formades pel producte d'altres unitats.

\textcolor{Green}{\ding{51}} 10\unit{kV\cdot s}

\textcolor{Red}{\ding{55}}  10\unit{MV\cdot ms}  (no \'{e}s incorrecte, per\`{o} el seu \'{u}s no es recomana)

\item Els noms de les unitats de l'SI s'escriuen en min\'{u}scula, excepte en
el cas de {"<}grau Celsius{">}, i a l'inici d'una oraci\'{o}.

\textcolor{Green}{\ding{51}} 10 newton, 100 watt, 24 volt, 20 grau Celsius

\textcolor{Red}{\ding{55}} 10 Newton, 100 Watt, 24 Volt, 20 grau celsius

\item Les unitats que tenen noms provinents de noms propis, s'han
d'escriure tal com apareixen en les taules
\vref{taula:SI-fonamentals}, \vref{taula:SI-derivades}, \vref{taula:SI-altres-acceptades}, \vref{taula:SI-altres-experimentals} i \vref{taula:SI-altres}, i no s'han
de traduir.

\textcolor{Green}{\ding{51}} 50 newton, 300 joule, $10^{-6}$ farad

\textcolor{Red}{\ding{55}}  50 neuton, 300 juls, $10^{-6}$ faradis

 \item Quan el nom d'una unitat
cont\'{e} un prefixe, ambdues parts s'han d'escriure juntes, sense cap espai o element d'uni\'{o}.

\textcolor{Green}{\ding{51}} 1 mi{\l.l}igram, 980 hectopascal

\textcolor{Red}{\ding{55}} 1 mi{\l.l}i gram, 1 mi{\l.l}i-gram, 980 hecto pascal, 980 hecto-pascal

\item En el cas  d'unitats derivades que s'expressen amb divisions o
productes, s'utilitza la preposici\'{o} {"<}per{">} entre dos noms d'unitats
per indicar-ne la divisi\'{o}, i no s'utilitza cap paraula per indicar-ne el
producte.

\textcolor{Green}{\ding{51}} 1 \unit{m\,/s} (1 metre per segon), 20 \unit{\ohm\cdot m} (20 ohm metre)

\textcolor{Red}{\ding{55}}  1 \unit{m\,/s} (1 metre segon), 20 \unit{\ohm\cdot m} (20 ohm multiplicat per metre)

\item El valor d'una quantitat ha d'expressar-se  utilitzant \'{u}nicament una
unitat.

\textcolor{Green}{\ding{51}} 10,234\unit{m}

\textcolor{Red}{\ding{55}}  10 m 23 cm 4 mm

\item Quan s'expressa el valor d'una quantitat, \'{e}s incorrecte afegir
lletres o altres s\'{\i}mbols a la unitat; qualsevol informaci\'{o}
addicional necess\`{a}ria ha d'afegir-se a la quantitat.

\textcolor{Green}{\ding{51}} U\ped{pp} = 1000\unit{V},  I\ped{max} = 36\unit{kA}

\textcolor{Red}{\ding{55}}  U = 1000\unit{V\ped{pp}},  I = 36\unit{kA\ped{max}}

\item El separador decimal entre la part entera i decimal d'un valor por ser el punt o la coma. L'\'{u}s de l'un o l'altra varia segons el pa\'{\i}s. Si el valor est\`{a} compr\`{e}s entre -1 i +1, \'{e}s obligatori escriure un zero davant del separador decimal.

\textcolor{Green}{\ding{51}} 0,25\unit{A}

\textcolor{Red}{\ding{55}}  ,25\unit{A}

\item Cuan un valor t\'{e} moltes xifres, les xifres poden dividir-se en grups de tres, mitjan\c{c}ant un espai curt, per tal de millorar-ne la legibilitat. No s'han d'utilitzar punts o comes per separar aquests grups de tres xifres.

\textcolor{Green}{\ding{51}} $43\,279{,}168\,29 \unit{kg}$, $43279{,}16829\unit{kg}$

\textcolor{Red}{\ding{55}}  $43.279{,}168.29 \unit{kg}$

\end{dinglist}

El document  {"<}Guide for the Use of the International System of Units (SI){">}, publicat pel (\textsf{NIST}),  fa a m\'{e}s les recomanacions seg\"{u}ents:

\begin{dinglist}{'167}

\item Quan s'indiquen valors de magnituds amb les seves desviacions,
s'indiquen intervals, o s'expressen diversos valors num\`{e}rics, les
unitats han de ser presents en cadascun dels valors, o s'han d'usar
par\`{e}ntesis si es vol posar les unitats nom\'{e}s al final.

\textcolor{Green}{\ding{51}} $63{,}2\unit{m} \pm 0{,}1\unit{m}$, $(63{,}2 \pm 0{,}1)\unit{m}$

\textcolor{Red}{\ding{55}} $63{,}2 \pm 0{,}1\unit{m}$, $63{,}2\unit{m} \pm 0{,}1$

\textcolor{Green}{\ding{51}} 0\unit{V} a 5\unit{V}, (0 a 5)\unit{V}, 4\unit{mA} a 20\unit{mA}, (4 a 20)\unit{mA}

\textcolor{Red}{\ding{55}} 0 a 5\unit{V},  4 a 20\unit{mA}

\textcolor{Green}{\ding{51}} $800\unit{mm}\times 600\unit{mm}\times 300\unit{mm}$

\textcolor{Red}{\ding{55}} $800\times 600\times 300\unit{mm}$

\textcolor{Green}{\ding{51}} 127\unit{s} + 3\unit{s} = 130\unit{s}, (127 + 3)\unit{s} = 130\unit{s}

\textcolor{Red}{\ding{55}}  127 + 3\unit{s} = 130\unit{s}

\textcolor{Green}{\ding{51}} $70\unit{\%} \pm 5\unit{\%}$, $(70 \pm 5)\unit{\%}$

\textcolor{Red}{\ding{55}}  $70 \pm 5\unit{\%}$

\textcolor{Green}{\ding{51}} $240 \times (1 \pm 10\unit{\%})\unit{V}$

\textcolor{Red}{\ding{55}}  $240\unit{V} \pm 10\unit{\%}$

\item Cal evitar l'utilitzaci\'{o} de {"<}ppm{">} (parts per mili\'{o}), {"<}ppb{">} (parts per bili\'{o}), etc.

\textcolor{Green}{\ding{51}} La concentraci\'{o} d'\`{a}cid en aigua \'{e}s de 25\unit{\micro L/L}

\textcolor{Red}{\ding{55}} La concentraci\'{o} d'\`{a}cid en aigua \'{e}s de 25 ppm

\end{dinglist}
