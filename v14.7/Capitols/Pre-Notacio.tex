\chapter*{Notació} \addcontentsline{toc}{chapter}{Notació}

\section*{Variables} \addcontentsline{toc}{section}{Variables}


En aquest llibre les variables escalars s'escriuen en lletra  de gruix normal, i  les variables vectorials i matricials s'escriuen
en lletra negreta.

\begin{list}{}
{\setlength{\labelwidth}{15mm} \setlength{\leftmargin}{20mm}
\setlength{\labelsep}{5mm}}
    \item[$j$] La unitat imaginària, definida com:
    $j\equiv\sqrt{-1}$\index{j@$j$}
    \item[$V$] Una variable real.
    \item[$\cmplx{V}$] Una variable complexa.
    \item[$\cmplx{V}^*$] Conjugat d'una variable complexa.
    Es compleixen les relacions:\\[1ex]
     $(\cmplx{V}_1 \pm \cmplx{V}_2 \pm \cdots  \pm \cmplx{V}_n)^* = \cmplx{V}_1^* \pm
    \cmplx{V}_2^*\pm\cdots\pm\cmplx{V}_n^*$\\[1ex]
    $(\cmplx{V}_1 \cmplx{V}_2 \cdots \cmplx{V}_n)^* = \cmplx{V}_1^*  \cmplx{V}_2^*
    \cdots \cmplx{V}_n^*$\\[1ex]
    $(\cmplx{V}_1 / \cmplx{V}_2)^* = \cmplx{V}_1^* / \cmplx{V}_2^*$
    \item[$|\cmplx{V}|$] Mòdul d'una variable complexa.
    Es compleixen les relacions:\\[1ex]
      $\cmplx{V}\,\cmplx{V}^* = |\cmplx{V}|^2$\\[1ex]
      $1/ \cmplx{V} = \cmplx{V}^* / \,|\cmplx{V}|^2$\\[1ex]
      $|\cmplx{V}_1 \cmplx{V}_2 \cdots \cmplx{V}_n| =
       |\cmplx{V}_1| \,|\cmplx{V}_2| \cdots |\cmplx{V}_n|$\\[1ex]
       $|\cmplx{V}_1 / \cmplx{V}_2| = |\cmplx{V}_1| \,/ \,|\cmplx{V}_2|$\\[1ex]
      $|\cmplx{V}_1+\cmplx{V}_2+\cdots+\cmplx{V}_n| \leq
      |\cmplx{V}_1| + |\cmplx{V}_2| + \cdots  +|\cmplx{V}_n|$
    \item[$\arg\cmplx{V}$] Argument (angle) d'una variable complexa.
     Es compleixen les relacions:\\[1ex]
      $\arg\cmplx{V}^* = - \arg\cmplx{V}$\\[1ex]
      $\arg(-\cmplx{V}) =  \arg\cmplx{V} + \piup$\\[1ex]
      $\arg(\cmplx{V}_1 \cmplx{V}_2 \cdots \cmplx{V}_n) = \arg\cmplx{V}_1 + \arg \cmplx{V}_2 + \cdots + \arg\cmplx{V}_n$\\[1ex]
      $\arg(\cmplx{V}_1 / \cmplx{V}_2) = \arg\cmplx{V}_1 - \arg \cmplx{V}_2$
    \item[$\Re\cmplx{V}$] Part real d'una variable complexa. Es compleix: $\Re\cmplx{V} = \dfrac{\cmplx{V} + \cmplx{V}^*}{2}$
    \item[$\Im\cmplx{V}$] Part imaginària d'una variable complexa. Es compleix: $\Im\cmplx{V} = \dfrac{\cmplx{V} - \cmplx{V}^*}{2 j}$
    \item[$A+j B$] Expressió cartesiana (part real i part
    imaginària) d'una variable complexa.
    \item[$Z\angle\theta$] Expressió polar (mòdul i argument) d'una variable
    complexa. Les relacions entre $A, B, Z$ i $\theta$ són:\footnote{Cal tenir en compte que la funció \texttt{arctan} disponible en moltes calculadores i llenguatges de programació, torna de forma  estandarditzada valors compresos entre $-\frac{\piup}{2}$ i $\frac{\piup}{2}$. En aquest cas cal sumar el valor $\piup$, quan $A$ és negatiu, a l'angle obtingut amb la funció \texttt{arctan} per tal d'obtenir l'angle en el quadrant correcte.}\\[1ex]
    $Z=+\sqrt{A^2+B^2}\qquad\quad\theta=\arctan{\dfrac{B}{A}}\qquad\quad
    A=Z\cos\theta\qquad\quad B=Z\sin\theta$
    \item[$Z\,e^{j\theta}$] Expressió d'Euler\index{Euler} d'una variable complexa, definida com:\footnote{El valor de $\theta$ ha d'expressar-se sempre en radian, per tal que el valor de $e^{j\theta}$ sigui correcte.}
     $Z\,e^{j\theta} \equiv Z(\cos\theta+j\sin\theta)$.
     Es compleixen les relacions:\\[1ex]
     $Z_1\,e^{j\theta_1} \, Z_2\,e^{j\theta_2} = Z_1 Z_2\,e^{j(\theta_1+\theta_2)}$\\[1ex]
     %$(Z_1\,e^{j\theta_1}) \,/\, (Z_2\,e^{j\theta_2}) = \dfrac{Z_1}{Z_2}\,e^{j(\theta_1-\theta_2)}$
     $\dfrac{Z_1\,e^{j\theta_1}}{Z_2\,e^{j\theta_2}} = \dfrac{Z_1}{Z_2}\,e^{j(\theta_1-\theta_2)}$
    \item[$\boldsymbol{V}$] Una matriu real o un vector real.
    \item[$\boldsymbol{V}^{-1}$] Matriu inversa d'una matriu real.
    \item[$\transpose{\boldsymbol{V}}$] Matriu transposada d'una matriu real, o vector
    transposat d'un vector real.
    \item[$\boldsymbol{V}(n)$] Element $n$-èsim d'un vector real.
    \item[$\boldsymbol{V}(m,n)$] Element de la fila $m$ i columna $n$ d'una matriu real.
    \item[$\mcmplx{V}$] Una matriu complexa o un vector complex.
    \item[$\mcmplx{V}^{-1}$] Matriu inversa d'una matriu complexa.
    \item[$\transpose{\mcmplx{V}}$] Matriu transposada d'una matriu complexa, o vector
    transposat d'un vector complex.
    \item[$\mcmplx{V}^*$] Matriu conjugada d'una matriu complexa, o vector
    conjugat d'un vector complex.
    \item[$\hermit{V}$] Matriu conjugada transposada d'una matriu complexa, o vector
    conjugat transposat d'un vector complex, definit com: $\hermit{V} \equiv
    \transpose{(\mcmplx{V}^*)}$.
    \item[$\mcmplx{V}(n)$] Element $n$-èsim d'un vector complex.
    \item[$\mcmplx{V}(m,n)$] Element de la fila $m$ i columna $n$ d'una matriu complexa.
\end{list}

Pel que fa als sentits assignats a les fletxes que representen les
tensions i els corrents en els diversos circuits elèctrics que
apareixen en aquest llibre, s'utilitza la convenció següent:

\begin{list}{}
{\setlength{\labelwidth}{15mm} \setlength{\leftmargin}{20mm}
\setlength{\labelsep}{5mm}}
    \item[$\begin{CD} @>U>> \end{CD}$] Tensió contínua: la fletxa indica el sentit
    de la caiguda de tensió, és a dir, va del nus positiu al nus negatiu.
    \item[$\begin{CD} @>I>> \end{CD}$] Corrent
    continu: la fletxa indica el sentit  assignat com a positiu al corrent.
    \item[$\begin{CD} @>\cmplx{U}>> \end{CD}$] Tensió alterna: la fletxa indica el
    sentit assignat com a positiu a la caiguda de tensió, quan el nus d'origen de la fletxa
    té un potencial  més positiu que el nus de destinació.
    \item[$\begin{CD} @>\cmplx{I}>> \end{CD}$] Corrent altern: la fletxa
    indica el sentit  assignat com a positiu al corrent.
\end{list}

\section*{Fasors} \addcontentsline{toc}{section}{Fasors}

En aquest llibre les variables complexes s'utilitzen per representar fasors. Un fasor $Z\angle\theta$ és equivalent a una funció sinusoidal variable en el temps, la qual pot expressar-se utilitzant la funció cosinus:
\[y(t)=\sqrt{2}\, Z \cos(\omega t + \theta)\]

o utilitzant la funció sinus:
\[y(t)=\sqrt{2}\, Z \sin(\omega t + \theta)\]

Quan hi ha diverses funcions sinusoidals relacionades entre si, cal utilitzar de manera uniforme la funció cosinus o la funció sinus per a totes les funcions. Les variables i paràmetres implicats són:
\begin{list}{}
{\setlength{\labelwidth}{15mm} \setlength{\leftmargin}{20mm}
\setlength{\labelsep}{5mm}}
    \item[$y(t)$] Funció sinusoidal; representa normalment una tensió o un corrent.
    \item[$t$] Temps.
    \item[$f$] Freqüència de la funció sinusoidal.
    \item[$T$] Període de la funció sinusoidal.
    \item[$\omega$] Velocitat angular de la funció sinusoidal. Es compleix: $\omega = 2 \piup f = 2 \piup/T$.
    \item[$Z$] Valor eficaç de la funció sinusoidal (vegeu la secció \vref{sec:val_mitja_ef}); els valors de pic de la funció sinusoidal  són:  $\pm\sqrt{2}\, Z$.
    \item[$\theta$] Angle inicial de la funció sinusoidal, on  $\theta=\omega \tau = -\omega (T-\tau)$; el significat del temps $\tau$  es pot veure en el gràfic que hi ha més avall.

    Quan es fa servir la funció cosinus, $\theta$ és positiu quan s'utilitza $\tau$, és a dir, el temps mesurat  des de l'origen ($t=0$) cap a l'esquerra, fins a trobar el primer valor màxim de la funció, i $\theta$ és negatiu quan s'utilitza $T-\tau$, és a dir, el temps mesurat des de l'origen cap a la dreta, fins a trobar també el primer valor màxim de la funció.

    Quan es fa servir la funció sinus, $\theta$ és positiu quan s'utilitza $\tau$, és a dir, el temps mesurat des de l'origen ($t=0$) cap a l'esquerra, fins a trobar el primer punt on la funció es fa zero (passant de valors negatius a positius), i $\theta$ és negatiu quan s'utilitza $T-\tau$, és a dir, el temps mesurat des de l'origen cap a la dreta, fins a trobar també el primer punt on la funció es fa zero (passant de valors negatius a positius).
    \item[] \input{Imatges/Not-Fasor.pdf_tex}
\end{list}
\index{fasor}

\section*{Conjunts de nombres i intervals} \addcontentsline{toc}{section}{Conjunts de nombres i intervals}

Els símbols que representen els diferents conjunts de nombres i intervals, segons la norma internacional ISO 80000-2 
\textit{Quantities and units --- Part 2: Mathematics}, són els següents:

\begin{list}{}
{\setlength{\labelwidth}{15mm} \setlength{\leftmargin}{20mm}
\setlength{\labelsep}{5mm}}
	 \item[$\mathsfb{N}$] Conjunt dels nombres naturals: $\{\,0,1,2,3,4,\ldots\,\}$. 
	 
	 L'exclusió del zero s'indica amb un asterisc: $\mathsfb{N}^* = \{ n \in \mathsfb{N} \mid n \ne 0 \}$. 
	 
	 És possible indicar altres restriccions, com per exemple:  $\mathsfb{N}_{> 5} = \{ n \in \mathsfb{N} \mid n > 5\}$. 
	 
	 També s'utilitzen els símbols $\vmathbb{N}$ i  $\vvmathbb{N}$.
	 
	 \item[$\mathsfb{Z}$] Conjunt dels nombres enters: $\{\,\ldots,-3,-2,-1,0,1,2,3,\ldots\,\}$.  
	 
	 L'exclusió del zero s'indica amb un asterisc: $\mathsfb{Z}^* = \{ n \in \mathsfb{Z} \mid n \ne 0 \}$. 
	 
	 És possible indicar altres restriccions, com per exemple:  $\mathsfb{Z}_{> -3} = \{ n \in \mathsfb{Z} \mid n > -3\}$.     
	 
	 També s'utilitzen els símbols $\vmathbb{Z}$ i $\vvmathbb{Z}$.
    
	 \item[$\mathsfb{Q}$] Conjunt dels nombres racionals. 
	 
	 L'exclusió del zero s'indica amb un asterisc: $\mathsfb{Q}^* = \{ r \in \mathsfb{Q} \mid r \ne 0 \}$. 
	 
	 És possible indicar altres restriccions, com per exemple:  $\mathsfb{Q}_{< 0} = \{ r \in \mathsfb{Q} \mid r < 0 \}$. 
	 
	 També s'utilitzen els símbols $\vmathbb{Q}$ i $\vvmathbb{Q}$.
	 
	 \item[$\mathsfb{R}$] Conjunt dels nombres reals. 
	 
	 L'exclusió del zero s'indica amb un asterisc: $\mathsfb{R}^* = \{ x \in \mathsfb{R} \mid x \ne 0 \}$. 
	 
	 És possible indicar altres restriccions, com per exemple:  $\mathsfb{R}_{> 0} = \{ x \in \mathsfb{R} \mid x > 0 \}$. 
	 
	 També s'utilitzen els símbols $\vmathbb{R}$ i $\vvmathbb{R}$.
	 
	\item[$\mathsfb{C}$] Conjunt dels nombres complexos.  
	
	L'exclusió del zero s'indica amb un asterisc: $\mathsfb{C}^* = \{ z \in \mathsfb{C} \mid z \ne 0 \}$.
	
	 També s'utilitzen els símbols $\vmathbb{C}$ i $\vvmathbb{C}$.
	 
	 \item[$\mathsfb{P}$] Conjunt dels nombres primers: $\{\,2,3,5,7,11,13,17,19,23,29,31,37,41,43,47,53,59,61,\ldots\,\}$. 
	 
	 També s'utilitzen els símbols $\vmathbb{P}$ i $\vvmathbb{P}$.
	 
	 \item[{$[a,b]$}] Interval tancat des de $a$ inclòs fins a $b$ inclòs: $[a,b] = \{x \in \mathsfb{R} \mid a \leq x \leq b\}$.
	 
	 \item[{$(a,b]$}] Interval semiobert esquerre des de $a$ exclòs fins a $b$ inclòs: $(a,b] = \{x \in \mathsfb{R} \mid a < x \leq b\}$. 
	 
	 També s'utilitza la notació $]a,b]$.
	 
	 \item[{$[a,b)$}] Interval semiobert dret des de $a$ inclòs fins a $b$ exclòs: $[a,b) = \{x \in \mathsfb{R} \mid a \leq x < b\}$. 
	 
	 També s'utilitza la notació $[a,b[$.
	 
	 \item[{$(a,b)$}] Interval obert des de $a$ exclòs fins a $b$ exclòs: $(a,b) = \{x \in \mathsfb{R} \mid a < x < b\}$. 
	 
	 També s'utilitza la notació $]a,b[$.
	 
	 \item[{$(-\infty,b]$}] Interval iŀlimitat tancat fins a $b$ inclòs: $(-\infty,b] = \{x \in \mathsfb{R} \mid x \leq b\}$.  
	 
	 També s'utilitza la notació $]-\infty,b]$.
	 
	 \item[{$(-\infty,b)$}] Interval iŀlimitat obert fins a $b$ exclòs: $(-\infty,b) = \{x \in \mathsfb{R} \mid x < b\}$. 
	 
	 També s'utilitza la notació $]-\infty,b[$.
	 
	 \item[{$[a,+\infty)$}] Interval iŀlimitat tancat des de $a$ inclòs: $[a, +\infty) = \{x \in \mathsfb{R} \mid  x \geq a\}$. 
	 
	 També s'utilitzen les notacions $[a, \infty)$, $[a, +\infty[$ i $[a, \infty[$.

	\item[{$(a,+\infty)$}] Interval iŀlimitat obert des de $a$ exclòs: $(a, +\infty) = \{x \in \mathsfb{R} \mid  x > a\}$. 
	
	També s'utilitzen les notacions $(a, \infty)$, $]a, +\infty[$ i $]a, \infty[$.
\end{list}
\index{N@$\mathsfb{N}$} \index{N@$\mathsfb{N}^*$}
\index{Z@$\mathsfb{Z}$} \index{Z@$\mathsfb{Z}^*$}
\index{Q@$\mathsfb{Q}$} \index{Q@$\mathsfb{Q}^*$}
\index{R@$\mathsfb{R}$} \index{R@$\mathsfb{R}^*$}
\index{C@$\mathsfb{C}$} \index{C@$\mathsfb{C}^*$}
\index{P@$\mathsfb{P}$}


\section*{Alfabet grec} \addcontentsline{toc}{section}{Alfabet grec}
\index{alfabet grec}

A continuació es pot veure l'alfabet grec
amb els noms de les seves lletres en diversos idiomes. Algunes lletres minúscules tenen dues grafies, totalment equivalents entre sí.

\begin{center}
\begin{threeparttable}
	\begin{tabular}{cccllll}
		\toprule[1pt]
		\renewcommand*{\multirowsetup}{\centering}
		\multirow{2}{15mm}{\rule{0mm}{4.5mm}Número\\d'ordre} & \multicolumn{2}{c}{Lletra} &
		\multicolumn{4}{c}{Nom} \\
		\cmidrule(rl){2-3} \cmidrule(rl){4-7}
		& minúscula & majúscula & català & castellà &  anglès & francès\\
		\midrule
		1  & $\alphaup$ & A & alfa & alfa &  alpha & alpha\\
		2  & $\betaup$ & B & beta & beta &  beta & bêta\\
		3  & $\gammaup$ & $\Gammaup$ & gamma & gamma &  gamma & gamma\\
		4  & $\deltaup$ & $\Deltaup$ & delta & delta &  delta & delta\\
		5  & $\epsilonup$, $\varepsilonup$ & E & èpsilon & épsilon &  epsilon & epsilon\\
		6  & $\zetaup$ & Z & zeta & dseta &  zeta & zêta\\
		7  & $\etaup$ & H & eta & eta &  eta & êta\\
		8  & $\thetaup$, $\varthetaup$ & $\Thetaup$ & theta & zeta &  theta & thêta\\
		9  & $\iotaup$ & I & iota & iota &  iota & iota\\
		10 & $\kappaup$, $\varkappaup$ & K & kappa & kappa &  kappa & kappa\\
		11 & $\lambdaup$ & $\Lambdaup$ & lambda & lambda &  lambda &lambda\\
		12 & $\muup$ & M & mi & mi &  mu & mu\\
		13 & $\nuup$ & N & ni & ni &  nu & nu\\
		14 & $\xiup$ & $\Xiup$ & ksi & xi &  xi & ksi, xi\\
		15 & o & O & òmicron & ómicron &  omicron & omicron\\
		16 & $\piup$, $\varpiup$\tnote{\color{blue}(a)} & $\Piup$ & pi & pi &  pi & pi\\
		17 & $\rhoup$, $\varrhoup$ & P & rho, ro & ro &  rho & rhô\\
		18 & $\sigmaup$, $\varsigmaup$\tnote{\color{blue}(b)} & $\Sigmaup$ & sigma & sigma &  sigma &sigma\\
		19 & $\tauup$ & T & tau & tau & tau &tau\\
		20 & $\upsilonup$ & $\Upsilonup$ & ípsilon & ípsilon &  upsilon &upsilon\\
		21 & $\phiup$, $\varphiup$ & $\Phiup$ & fi & fi &  phi & phi\\
		22 & $\chiup$ & X & khi & ji &  chi & khi\\
		23 & $\psiup$ & $\Psiup$ & psi & psi &  psi & psi\\
		24 & $\omegaup$ & $\Omegaup$ & omega & omega &  omega & oméga\\
		\bottomrule[1pt]
	\end{tabular}
	\begin{tablenotes}
		\item[\color{blue}(a)] {\footnotesize La variant $\varpiup$ de la lletra pi, que no es fa servir en textos tècnics i científics, es denomina «pi dòrica» en  català, «pi dórica» en castellà, «dorian pi» en anglès, i «pi dorien» en francès.}
		\item[\color{blue}(b)] {\footnotesize En grec, la variant $\varsigmaup$ de la lletra sigma s'escriu  al final d'una paraula, i la variant $\sigmaup$  a l'inici o en mig d'una paraula. En els textos tècnics i científics s'utilitza  la variant $\sigmaup$.}
	\end{tablenotes}
\end{threeparttable}
\end{center}



Els noms d'algunes lletres poden sorprendre, ja que n'hi ha que han rebut històricament noms
diversos, i fins i tot contradictoris respecte dels actuals.

Els noms anglesos de les lletres són els més uniformes, ja que no
s'ha observat cap variació en les diverses fonts consultades, essencialment el diccionari nord-americà Merriam-Webster\footnote{Aquest diccionari es pot consultar a l'adreça: \href{https://www.merriam-webster.com/}{https://www.merriam-webster.com}.} i els diccionaris britànics Oxford\footnote{Aquest diccionari es pot consultar a l'adreça: \href{https://www.oed.com/}{https://www.oed.com}.} i Cambridge.\footnote{Aquest diccionari es pot consultar a l'adreça: \href{https://dictionary.cambridge.org/}{https://dictionary.cambridge.org}.}

Els noms catalans de les lletres són els que apareixen en el \textit{Diccionari de la llengua catalana, 2a edició, 2007} (DIEC2).\footnote{Aquest diccionari es pot consultar a l'adreça: \href{http://dlc.iec.cat/}{https://dlc.iec.cat}. La versió en línia s'actualitza periòdicament.} Altres noms utilitzats en
les diverses fonts consultades, que actualment estan fora de la norma ortogràfica, són:
\begin{multicols}{3}
	\begin{list}{}
		{\setlength{\labelwidth}{16mm} \setlength{\leftmargin}{16mm} \setlength{\labelsep}{2mm}}
		\item[B, $\betaup :$] vita.
		\item[Z, $\zetaup :$] zita.
		\item[H, $\etaup :$] ita.
		\item[$\Thetaup$, $\thetaup :$] thita.
		\item[T, $\tauup :$] taf.
		\item[$\xiup$, $\Xiup$:] csi.\footnote{El nom «csi» apareix juntament amb «ksi» en el \textit{Gran Diccionari de la Llengua Catalana (1999)}. Aquest diccionari es pot consultar a l'adreça:  \href{https://www.diccionari.cat/}{https://www.diccionari.cat}.}
	\end{list}
\end{multicols}

Els noms castellans de les lletres són els que apareixen en el \textit{Diccionario de la Lengua Española, 23ª
	edición, 2014} (D.R.A.E.).\footnote{Aquest diccionari es pot consultar a l'adreça:  \href{https://www.rae.es/}{https://www.rae.es}. La versió en línia s'actualitza periòdicament.} Altres noms utilitzats en les diverses fonts
consultades, que actualment estan fora de la norma ortogràfica, són:
\begin{multicols}{3}
	\begin{list}{}
		{\setlength{\labelwidth}{16mm} \setlength{\leftmargin}{16mm} \setlength{\labelsep}{2mm}}
		\item[Z, $\zetaup :$] zeta,\footnote{\label{fn:zeta}Els noms «zeta», «theta», «my» i «ny» eren els que apareixien en les edicions
			del D.R.A.E anteriors a la 21a (1992).} dseda,\footnote{El nom «dseda» era el que apareixia en l'edició 22a (2001) del D.R.A.E.} dzeta.
		\item[$\Thetaup$, $\thetaup :$] theta,\footnoteref{fn:zeta} thita.
		\item[K, $\kappaup :$] cappa.
		\item[M, $\muup :$] my,\footnoteref{fn:zeta} mu.
		\item[N, $\nuup :$] ny,\footnoteref{fn:zeta} nu.
		\item[O, o :] omicrón.
		\item[P, $\rhoup :$] rho.
		\item[$\Upsilonup$, $\upsilonup :$] úpsilon.
		\item[$\Phiup$, $\phiup :$] phi.
	\end{list}
\end{multicols}

Els noms francesos de les lletres són els que apareixen en el \textit{Dictionnaire de l'Académie française, neuvième édition}.\footnote{Aquest diccionari es pot consultar a l'adreça: \href{https://www.academie-francaise.fr/le-dictionnaire/la-9e-edition}{https://www.academie-francaise.fr/le-dictionnaire/la-9e-edition}.} 