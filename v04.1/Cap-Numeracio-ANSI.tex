\chapter{Numeraci\'{o} \textsf{ANSI} de Dispositius El\`{e}ctrics} \index{numeraci\'{o} ANSI de dispositius el\`{e}ctrics}
\index{ANSI!C37.2}

Es d\'{o}na a continuaci\'{o} una llista de la numeraci\'{o} de dispositius
el\`{e}ctrics, segons la norma \textsf{ANSI C37.2}, amb una breu
explicaci\'{o} de la seva funci\'{o}.

\begin{multicols}{2}
\begin{list}{}
{\setlength{\labelwidth}{6mm} \setlength{\leftmargin}{6mm}
\setlength{\labelsep}{2mm}}

\item[\textbf{1}] \index{element principal} \textbf{Element principal}. \'{E}s un dispositiu
iniciador, com ara un commutador de control, un rel\`{e} de tensi\'{o}, un
interruptor de nivell, etc., que serveix per posar en marxa o fora
de servei un aparell, ja sigui directament, o b\'{e}  mitjan\c{c}ant altres
dispositius, com ara rel\`{e}s de protecci\'{o}, o rel\`{e}s temporitzats.

\item[\textbf{2}] \index{rel\`{e}!de tancament o arrencada, amb retard} \textbf{Rel\`{e}
de tancament o arrencada, amb retard de temps}. \'{E}s el que
proporciona un retard de temps entre les operacions d'una seq\"{u}\`{e}ncia
autom\`{a}tica, o d'un sistema de protecci\'{o}, excepte quan aquest retard
\'{e}s proporcionat espec\'{\i}ficament pels dispositius 48, 62 o 79,
descrits m\'{e}s endavant. S'utilitza principalment com a protecci\'{o} de
la discrep\`{a}ncia de pols d'un interruptor.

\item[\textbf{3}] \index{rel\`{e}!de comprovaci\'{o} o de bloqueig} \textbf{Rel\`{e} de comprovaci\'{o} o
de bloqueig}. \'{E}s el que actua en resposta a la posici\'{o} d'una s\`{e}rie
d'altres dispositius (o d'una s\`{e}rie de condicions predeterminades)
en un equip, per tal de permetre que una seq\"{u}\`{e}ncia d'operaci\'{o}
continu\"{\i}, o per tal de parar-la, o per proporcionar una prova de la
posici\'{o} d'aquests dispositius o d'aquestes condicions.

\item[\textbf{4}] \index{contactor!principal} \textbf{Contactor principal}. \'{E}s un dispositiu,
generalment governat pel dispositiu n\'{u}mero 1 i pels dispositius de perm\'{\i}s i protecci\'{o}
que calgui, que serveix per obrir i tancar els circuits de control necessaris per tal de
posar un equip en marxa, o per parar-lo.

\item[\textbf{5}] \index{dispositiu!de parada}  \textbf{Dispositiu de parada}. \'{E}s el que
t\'{e} com a funci\'{o} principal, deixar fora de servei un equip i
mantenir-lo en aquest estat; la seva actuaci\'{o} pot ser manual o
el\`{e}ctrica. Queda exclosa la funci\'{o} de bloqueig el\`{e}ctric en
situacions anormals (vegeu la funci\'{o} 86).

\item[\textbf{6}] \index{interruptor!d'arrencada} \textbf{Interruptor d'arrencada}. \'{E}s
el que t\'{e} com a funci\'{o} principal connectar una m\`{a}quina a la seva font de tensi\'{o} d'arrencada.

\item[\textbf{7}] \index{interruptor!d'\`{a}node} \textbf{Interruptor d'\`{a}node}. \'{E}s el que
s'utilitza en els circuits dels \`{a}nodes d'un rectificador de
pot\`{e}ncia, principalment per interrompre el circuit del rectificador
en  cas de produir-s'hi un arc el\`{e}ctric.

\item[\textbf{8}] \index{dispositiu!de desconnexi\'{o} de
l'energia de control} \textbf{Dispositiu de desconnexi\'{o} de l'energia
de control}. \'{E}s un element de desconnexi\'{o} (commutador de ganiveta,
interruptor de bloc o fusibles connectables) que s'utilitza per
connectar i desconnectar la font d'energia de control,  a la barra
de tensi\'{o} de control o a l'equip al qual doni servei. Es considera
que l'energia de control inclou a l'energia auxiliar que alimenta a
aparells, com ara motors petits o calefactors.

\item[\textbf{9}] \index{dispositiu!d'inversi\'{o}} \textbf{Dispositiu d'inversi\'{o}}. \'{E}s el
que s'utilitza per invertir les connexions del camp d'una m\`{a}quina, o
per realitzar qualsevol altra funci\'{o}  d'inversi\'{o}.

\item[\textbf{10}] \index{commutador!de seq\"{u}\`{e}ncia} \textbf{Commutador de seq\"{u}\`{e}ncia}. \'{E}s un
dispositiu que s'utilitza pera canviar la seq\"{u}\`{e}ncia de connexi\'{o} o
desconnexi\'{o} d'unitats, en un equip de m\'{u}ltiples unitats.

\item[\textbf{11}] \textbf{Reservat per a  futures aplicacions}.
L'\textsf{USBR}\footnote{{"<}United States Bureau of Reclamation{">}}
 li assigna la funci\'{o}: transformador de pot\`{e}ncia de control.

\item[\textbf{12}] \index{dispositiu!d'exc\'{e}s de velocitat} \textbf{Dispositiu d'exc\'{e}s de velocitat}. \'{E}s normalment un
interruptor de velocitat, de connexi\'{o} directa, que
actua quan la m\`{a}quina  s'embala.

\item[\textbf{13}] \index{dispositiu!de velocitat sincr\`{o}nica} \textbf{Dispositiu de
velocitat sincr\`{o}nica}. \'{E}s un element, com ara un interruptor de
velocitat centr\'{\i}fuga, un rel\`{e} de freq\"{u}\`{e}ncia de lliscament, un rel\`{e}
de tensi\'{o}, o qualsevol altre aparell que actua a, aproximadament, la
velocitat sincr\`{o}nica d'una m\`{a}quina.

\item[\textbf{14}] \index{dispositiu!de manca de velocitat} \textbf{Dispositiu de manca de velocitat}. \'{E}s el que actua quan la velocitat d'una m\`{a}quina baixa per sota d'un valor determinat.

\item[\textbf{15}] \index{dispositiu!igualador de velocitat o freq\"{u}\`{e}ncia}
\textbf{Dispositiu igualador de velocitat o freq\"{u}\`{e}ncia}. \'{E}s el que
actua per tal d'igualar i mantenir la velocitat o la  freq\"{u}\`{e}ncia
d'una m\`{a}quina o d'un sistema a un cert valor, aproximadament igual
al  d'una altra m\`{a}quina o sistema.

\item[\textbf{16}] \textbf{Reservat per a  futures aplicacions}.
L'\textsf{USBR}\footnotemark[1] li assigna la funci\'{o}: carregador de
bateries.

\item[\textbf{17}] \index{commutador!de \guillemotleft{}shunt\guillemotright{} o de desc\`{a}rrega} \textbf{Commutador
de  {"<}shunt{">} o de desc\`{a}rrega}. \'{E}s el que serveix per obrir i tancar
un circuit {"<}shunt{">} entre els extrems de qualsevol aparell (excepte
una resist\`{e}ncia), com ara el camp d'una m\`{a}quina, un condensador o
una react\`{a}ncia. Queden exclosos els elements que realitzen les
funcions de {"<}shunt{">} necess\`{a}ries per arrancar una m\`{a}quina, mitjan\c{c}ant
els dispositius 6, 42, o equivalents; tamb\'{e} queda exclosa la funci\'{o}
del dispositiu 73, el qual serveix per a l'operaci\'{o} de resist\`{e}ncies.

\item[\textbf{18}] \index{dispositiu!d'acceleraci\'{o} o desacceleraci\'{o}} \textbf{Dispositiu d'acceleraci\'{o} o desacceleraci\'{o}}. \'{E}s
el que s'utilitza per tancar o per causar el tancament dels circuits
que serveixen per augmentar o disminuir la velocitat d'una m\`{a}quina.

\item[\textbf{19}] \index{contactor!de transici\'{o} d'arrencada a marxa normal}
\textbf{Contactor de transici\'{o} d'arrencada a marxa normal}. La seva
funci\'{o} \'{e}s fer la transfer\`{e}ncia de les connexions de l'alimentaci\'{o}
d'arrencada, a la de marxa normal d'una m\`{a}quina.

\item[\textbf{20}] \index{valvula@v\`{a}lvula} \textbf{V\`{a}lvula}. S'assigna aquest n\'{u}mero a una v\`{a}lvula
utilitzada en un circuit de buit, d'aire, de gas, d'oli, d'aigua,
etc., quan s'acciona el\`{e}ctricament o quan t\'{e} accessoris el\`{e}ctrics,
com ara commutadors auxiliars.

\item[\textbf{21}] \index{rel\`{e}!de dist\`{a}ncia} \textbf{Rel\`{e} de dist\`{a}ncia}. \'{E}s el que actua
quan l'admit\`{a}ncia, la imped\`{a}ncia o la react\`{a}ncia d'un circuit surt fora d'un cert l\'{\i}mit.

\item[\textbf{22}] \index{interruptor!igualador} \textbf{Interruptor igualador}.  \'{E}s el
que serveix per connectar i desconnectar les connexions igualadores
o d'equilibri d'intensitat del camp d'una m\`{a}quina, o per regular
equips en una  insta{\l.l}aci\'{o} de  m\'{u}ltiples unitats.

\item[\textbf{23}] \index{dispositiu!controlador de temperatura} \textbf{Dispositiu
controlador de temperatura}. \'{E}s el que actua per tal de fer pujar la
temperatura d'un lloc o d'un aparell, quan aquesta temperatura baixa
per sota d'un cert l\'{\i}mit, o a l'inrev\'{e}s, el que actua per tal de fer
 baixar la temperatura d'un lloc o d'un aparell, quan aquesta
temperatura  puja per sobre d'un cert l\'{\i}mit. Un exemple seria un
term\`{o}stat; en canvi, un dispositiu per regular la temperatura dins
d'un marge estret, es designaria amb el n\'{u}mero 90.

\item[\textbf{24}] \textbf{Reservat per a  futures aplicacions}.
L'\textsf{USBR}\footnotemark[1] li assigna la funci\'{o}: interruptor o
contactor d'uni\'{o} de barres.

\item[\textbf{25}] \index{dispositiu!de sincronitzaci\'{o}}\index{dispositiu!de comprovaci\'{o} de
sincronisme} \textbf{Dispositiu de sincronitzaci\'{o} o de comprovaci\'{o}
de sincronisme}. \'{E}s el que actua quan dos circuits de corrent altern
s\'{o}n dins dels l\'{\i}mits desitjats de tensi\'{o}, freq\"{u}\`{e}ncia i angle de
fase, per permetre la connexi\'{o} en para{\l.l}el d'aquests dos circuits.


\item[\textbf{26}] \index{dispositiu!t\`{e}rmic} \textbf{Dispositiu t\`{e}rmic}. \'{E}s el que
actua quan la temperatura del camp {"<}shunt{">} o del bobinat esmorte\"{\i}dor
d'una m\`{a}quina, la temperatura d'una resist\`{e}ncia de limitaci\'{o} de
c\`{a}rrega, o la temperatura d'un l\'{\i}quid, etc., supera un valor
determinat. Tamb\'{e} actua si la temperatura de l'aparell protegit cau
per sota d'un valor determinat.

\item[\textbf{27}] \index{rel\`{e}!de m\'{\i}nima tensi\'{o}} \textbf{Rel\`{e} de m\'{\i}nima tensi\'{o}}. \'{E}s el que
actua quan la tensi\'{o} baixa per sota d'un l\'{\i}mit determinat.

\item[\textbf{28}] \index{detector!de flama} \textbf{Detector de flama}. La seva funci\'{o} \'{e}s
detectar l'exist\`{e}ncia de flama en el pilot o cremador principal de, per exemple, una
caldera o una turbina de gas.

\item[\textbf{29}] \index{contactor!d'a\"{\i}llament} \textbf{Contactor d'a\"{\i}llament}. \'{E}s el que
s'utilitza amb l'\'{u}nic prop\`{o}sit de desconnectar un circuit d'un
altre,  a  causa de maniobres    d'emerg\`{e}ncia,  de manteniment o de
prova.

\item[\textbf{30}] \index{rel\`{e}!anunciador} \textbf{Rel\`{e} anunciador}. \'{E}s un dispositiu de
reposici\'{o} no autom\`{a}tica, que d\'{o}na una s\`{e}rie d'indicacions visuals
individuals, de les funcions d'aparells de protecci\'{o}, i que es pot
disposar tamb\'{e} per efectuar una funci\'{o} de bloqueig.

\item[\textbf{31}] \index{dispositiu!d'excitaci\'{o} separada} \textbf{Dispositiu d'excitaci\'{o}
separada}. \'{E}s el que connecta un circuit, com ara el camp {"<}shunt{">}
d'una commutatriu, a una font d'excitaci\'{o} separada, durant el proc\'{e}s
d'arrencada. Tamb\'{e} s'utilitza per energitzar el circuit d'encesa
d'un rectificador de pot\`{e}ncia.


\item[\textbf{32}] \index{rel\`{e}!direccional de pot\`{e}ncia} \textbf{Rel\`{e} direccional de
pot\`{e}ncia}. \'{E}s el que actua quan se supera un valor determinat del
flux de pot\`{e}ncia en un sentit donat. Tamb\'{e} actua per causa d'una
inversi\'{o} de pot\`{e}ncia, originada per un arc el\`{e}ctric en el circuit
an\`{o}dic o cat\`{o}dic d'un rectificador de pot\`{e}ncia.

\item[\textbf{33}] \index{commutador!de posici\'{o}} \textbf{Commutador de posici\'{o}}. \'{E}s el que
obre o tanca un contacte, quan un dispositiu principal o una part d'un aparell que no tingui un n\'{u}mero funcional de dispositiu, arriba a una posici\'{o} determinada.

\item[\textbf{34}] \index{dispositiu!principal de seq\"{u}\`{e}ncia} \textbf{Dispositiu principal de
 seq\"{u}\`{e}ncia}. \'{E}s un element, com ara un selector de contactes m\'{u}ltiples, o com ara un
 dispositiu programable,
 que fixa la seq\"{u}\`{e}ncia d'operaci\'{o}
de dispositius principals, durant l'arrencada i la parada, o durant altres operacions
que requereixin una seq\"{u}\`{e}ncia.

\item[\textbf{35}] \index{dispositiu!per operar
escombretes} \index{dispositiu!per posar en curt circuit anells de
frec} \textbf{Dispositiu per operar escombretes o per posar en curt
circuit anells de frec}. \'{E}s el que serveix per elevar, baixar o
desviar les escombretes d'una m\`{a}quina, o per posar en curt circuit
els seus anells de frec. Tamb\'{e} serveix per fer o desfer els
contactes d'un rectificador mec\`{a}nic.

\item[\textbf{36}] \index{dispositiu!de polaritat}
\index{dispositiu!de tensi\'{o} de polaritzaci\'{o}} \textbf{Dispositiu de
polaritat o de tensi\'{o} de polaritzaci\'{o}}. \'{E}s el que acciona o permet
l'accionament d'altres dispositius, tan sols amb una polaritat
donada, o el que verifica la pres\`{e}ncia d'una tensi\'{o} de polaritzaci\'{o}
en un equip.

\item[\textbf{37}] \index{rel\`{e}!de baixa intensitat o baixa pot\`{e}ncia} \textbf{Rel\`{e} de baixa
intensitat o baixa pot\`{e}ncia}. \'{E}s el que actua quan la intensitat o la pot\`{e}ncia cauen per
sota d'un valor determinat.

\item[\textbf{38}] \index{dispositiu!protector de coixinets}
\textbf{Dispositiu protector de coixinets}. \'{E}s el que actua amb una
temperatura excessiva dels coixinets, o amb condicions mec\`{a}niques
an\`{o}males que poden derivar en una temperatura excessiva dels
coixinets.

\item[\textbf{39}] \index{detector!de condicions mec\`{a}niques}
\textbf{Detector de condicions mec\`{a}niques}. \'{E}s el que actua davant
de situacions mec\`{a}niques anormals (excepte les que tenen lloc en els
coixinets d'una m\`{a}quina, funci\'{o} 38), com ara vibraci\'{o} excessiva,
excentricitat, etc.

\item[\textbf{40}] \index{rel\`{e}!de camp} \textbf{Rel\`{e} de camp}. \'{E}s el que actua quan es
d\'{o}na un valor massa baix de la intensitat de camp d'una m\`{a}quina, o quan es d\'{o}na un valor
massa gran de la component reactiva del corrent d'armadura en una m\`{a}quina de corrent
altern, la qual cosa indica una excitaci\'{o} de camp massa baixa.

\item[\textbf{41}] \index{interruptor!de camp} \textbf{Interruptor de camp}. \'{E}s un dispositiu
que actua per tal de connectar o desconnectar l'excitaci\'{o} del camp
d'una m\`{a}quina.

\item[\textbf{42}] \index{interruptor!de marxa} \textbf{Interruptor de marxa}. \'{E}s un
dispositiu que t\'{e} per funci\'{o} principal connectar una m\`{a}quina a la
seva font de tensi\'{o} de funcionament.

\item[\textbf{43}] \index{dispositiu!de transfer\`{e}ncia} \textbf{Dispositiu de transfer\`{e}ncia}. \'{E}s
un element, accionat manualment, que efectua la transfer\`{e}ncia dels circuits de control, per tal
 de modificar el proc\'{e}s d'operaci\'{o} d'equips de connexi\'{o} o d'altres dispositius.

\item[\textbf{44}] \index{rel\`{e}!de seq\"{u}\`{e}ncia d'arrencada de grup} \textbf{Rel\`{e} de seq\"{u}\`{e}ncia
d'arrencada de grup}. \'{E}s el que actua per arrancar la seg\"{u}ent unitat
disponible, en un equip de m\'{u}ltiples unitats, quan falla o quan no
est\`{a} disponible la unitat que normalment hauria d'arrencar.

\item[\textbf{45}] \index{detector!de condiciones atmosf\`{e}riques} \textbf{Detector de condiciones
atmosf\`{e}riques}. \'{E}s el que actua davant de condicions atmosf\`{e}riques anormals, com ara fums
perillosos, gasos explosius, foc, etc.

\item[\textbf{46}] \index{rel\`{e}!de seq\"{u}\`{e}ncia negativa d'intensitat} \textbf{Rel\`{e} de
seq\"{u}\`{e}ncia negativa d'intensitat}. \'{E}s un rel\`{e} que actua quan les
intensitats polif\`{a}siques estan en seq\"{u}\`{e}ncia inversa o
desequilibrades, o quan contenen una component de seq\"{u}\`{e}ncia negativa
superior a un cert l\'{\i}mit.

\item[\textbf{47}] \index{rel\`{e}!de seq\"{u}\`{e}ncia de fase de tensi\'{o} } \textbf{Rel\`{e}
de seq\"{u}\`{e}ncia de fase de tensi\'{o}}. \'{E}s el que actua amb un valor donat
de tensi\'{o}, quan es d\'{o}na la seq\"{u}\`{e}ncia de fases desitjada.

\item[\textbf{48}] \index{rel\`{e}!de seq\"{u}\`{e}ncia incompleta} \textbf{Rel\`{e} de seq\"{u}\`{e}ncia
incompleta}. \'{E}s el que torna un equip a la seva posici\'{o} normal  i
l'enclava, si la seq\"{u}\`{e}ncia normal d'arrencada, de funcionament o de
parada no s'ha completat degudament en un interval de temps
determinat.

\item[\textbf{49}] \index{rel\`{e}!t\`{e}rmic d'una m\`{a}quina o d'un transformador}
\textbf{Rel\`{e} t\`{e}rmic d'una m\`{a}quina o d'un transformador}. \'{E}s el que
actua quan la temperatura d'un element d'una m\`{a}quina o d'un
transformador (normalment un debanat), per on circula el corrent,
supera un valor determinat.

\item[\textbf{50}] \index{rel\`{e}!instantani de sobreintensitat o de velocitat d'augment
d'intensitat} \textbf{Rel\`{e} instantani de sobreintensitat o de velocitat d'augment
d'intensitat}. \'{E}s el que actua instant\`{a}niament quan es d\'{o}na un valor excessiu de la
intensitat o de la  velocitat d'augment de la intensitat.

\item[\textbf{51}] \index{rel\`{e}!temporitzat de sobreintensitat de corrent altern}
\textbf{Rel\`{e} temporitzat de sobreintensitat de corrent altern}. \'{E}s
un rel\`{e} amb una caracter\'{\i}stica de temps inversa o definida, que
actua amb una certa temporitzaci\'{o}, quan es d\'{o}na un valor excessiu de
la intensitat.

\item[\textbf{52}] \index{interruptor!de corrent altern} \textbf{Interruptor de corrent altern}. \'{E}s
 el que s'utilitza per tancar i obrir un circuit de pot\`{e}ncia de corrent altern sota condicions
normals, de falta o d'emerg\`{e}ncia.

\item[\textbf{53}] \index{rel\`{e}!d'excitatriu o de generador de corrent continu}
\textbf{Rel\`{e} d'excitatriu o de generador de corrent continu}. \'{E}s el
que for\c{c}a la creaci\'{o} del camp d'una m\`{a}quina de corrent continu
durant l'arrencada, o el que actua quan la tensi\'{o} d'una m\`{a}quina ha
arribat a un valor determinat.

\item[\textbf{54}] \index{interruptor!d'alta velocitat, de corrent continu}
\textbf{Interruptor d'alta velocitat, de corrent continu}. \'{E}s el que
actua per tal de reduir el corrent d'un circuit principal, en un
temps inferior a 0,01\unit{s} despr\'{e}s d'haver-se produ\"{\i}t un corrent
massa elevat, o una velocitat de creixement d'aquest corrent massa
elevada.

\item[\textbf{55}] \index{rel\`{e}!de factor de pot\`{e}ncia} \textbf{Rel\`{e} de factor de pot\`{e}ncia}.
\'{E}s el que actua quan el factor de potencia en un circuit de corrent altern no arriba o
sobrepassa un valor determinat.

\item[\textbf{56}] \index{rel\`{e}!d'aplicaci\'{o} del camp} \textbf{Rel\`{e} d'aplicaci\'{o} del camp}.
\'{E}s el que s'utilitza per controlar autom\`{a}ticament l'aplicaci\'{o} de l'excitaci\'{o} de camp d'un
motor de corrent altern, en un punt predeterminat en el cicle de lliscament.

\item[\textbf{57}] \index{dispositiu!de curt circuit o de posada a terra}
\textbf{Dispositiu de curt circuit o de posada a terra}. \'{E}s el que
opera en un circuit principal per tal de curtcircuitar-lo  o
posar-lo a terra, en resposta a ordres autom\`{a}tiques o manuals.

\item[\textbf{58}] \index{rel\`{e}!de fallada de rectificador de pot\`{e}ncia} \textbf{Rel\`{e} de
fallada de rectificador de pot\`{e}ncia}. \'{E}s el que actua a causa de la
fallada d'un o m\'{e}s \`{a}nodes d'un rectificador de pot\`{e}ncia, o a causa
de la fallada d'un d\'{\i}ode a conduir o bloquejar pr\`{o}piament.

\item[\textbf{59}] \index{rel\`{e}!de sobretensi\'{o}} \textbf{Rel\`{e} de sobretensi\'{o}}. \'{E}s el que
actua quan la tensi\'{o} supera un valor determinat.

\item[\textbf{60}] \index{rel\`{e}!d'equilibri de tensi\'{o} o corrent} \textbf{Rel\`{e} d'equilibri de
tensi\'{o} o corrent}. \'{E}s el que actua amb una difer\`{e}ncia de tensi\'{o} o
corrent entre dos circuits.

\item[\textbf{61}] \textbf{Reservat per a  futures aplicacions}.

\item[\textbf{62}] \index{rel\`{e}!de parada o obertura, amb retard} \textbf{Rel\`{e} de
parada o obertura, amb retard de temps}. \'{E}s el que s'utilitza
conjuntament amb el dispositiu que inicia la parada total o la
indicaci\'{o} de parada o obertura, en una seq\"{u}\`{e}ncia autom\`{a}tica.

\item[\textbf{63}] \index{rel\`{e}!de pressi\'{o} de gas, l\'{\i}quid o buit} \textbf{Rel\`{e} de pressi\'{o}
de gas, l\'{\i}quid o buit}. \'{E}s el que actua a un valor determinat de
pressi\'{o} de l\'{\i}quid o gas, o per a una determinada velocitat de
variaci\'{o} d'aquesta pressi\'{o}.

\item[\textbf{64}] \index{rel\`{e}!de protecci\'{o} de terra} \textbf{Rel\`{e} de protecci\'{o} de terra}.
\'{E}s el que actua davant d'un defecte a terra de l'a\"{\i}llament d'una
m\`{a}quina. Aquesta funci\'{o} s'aplica nom\'{e}s a un rel\`{e} que detecti el pas
del corrent des de la carcassa  d'una m\`{a}quina a terra, o a un rel\`{e}
que detecti un terra en un circuit normalment no connectat a terra;
no s'aplica a un dispositiu connectat en el circuit secundari d'un
transformador d'intensitat, que estigui connectat en el circuit de
pot\`{e}ncia d'un sistema posat normalment a terra.

\item[\textbf{65}] \index{regulador} \textbf{Regulador}. \'{E}s un equip format per elements
el\`{e}ctrics, mec\`{a}nics o flu\'{\i}dics,  que controla el flux d'aigua,
vapor, etc.,  a una m\`{a}quina motriu, per tal d'arrancar-la, mantenir
la seva velocitat o parar-la.

\item[\textbf{66}] \index{rel\`{e}!de passos} \textbf{Rel\`{e} de passos}. \'{E}s el que actua per tal
de permetre un nombre especificat d'operacions d'un dispositiu
donat, o b\'{e}, un nombre especificat d'operacions successives amb un
interval donat de temps entre cadascuna. Tamb\'{e} pot actuar per
permetre l'energitzaci\'{o} peri\`{o}dica d'un circuit, o per accelerar una
m\`{a}quina a baixa velocitat.

\item[\textbf{67}] \index{rel\`{e}!direccional de sobreintensitat de corrent altern}
\textbf{Rel\`{e} direccional de sobreintensitat de corrent altern}. \'{E}s
el que actua a partir d'un valor determinat de circulaci\'{o}
d'intensitat  de corrent altern, en un sentit donat.

\item[\textbf{68}] \index{rel\`{e}!de bloqueig} \textbf{Rel\`{e} de bloqueig}. \'{E}s el que inicia un
senyal pilot per bloquejar o disparar, quan hi ha faltes externes en
una l\'{\i}nia de transmissi\'{o}, o en altres aparells, sota certes
condicions; pot cooperar tamb\'{e} amb altres dispositius, per tal de
bloquejar el dispar o per bloquejar el reenganxament en una condici\'{o}
de p\`{e}rdua de sincronisme.

\item[\textbf{69}] \index{dispositiu!controlador
de permissiu} \textbf{Dispositiu controlador de permissiu}. \'{E}s
generalment, un interruptor auxiliar de dues posicions, accionat
manualment, el qual permet en una posici\'{o}, el tancament d'un
interruptor o la posada en servei d'un equip, i en l'altra posici\'{o},
impedeix l'accionament de l'interruptor o de l'equip.

\item[\textbf{70}] \index{reostat@re\`{o}stat} \textbf{Re\`{o}stat}. \'{E}s un dispositiu utilitzat per
variar la resist\`{e}ncia d'un circuit, en resposta a algun m\`{e}tode de control.

\item[\textbf{71}] \index{rel\`{e}!de nivell de l\'{\i}quid o gas} \textbf{Rel\`{e} de nivell de l\'{\i}quid
o gas}. \'{E}s el que actua a partir d'un valor determinat del nivell
d'un l\'{\i}quid o d'un gas, o a partir de determinades velocitats de
variaci\'{o} d'aquests nivells.

\item[\textbf{72}] \index{interruptor!de corrent continu} \textbf{Interruptor de corrent
continu}. \'{E}s el que s'utilitza per tancar i obrir un circuit de pot\`{e}ncia de corrent continu
 sota condicions normals, de falta o d'emerg\`{e}ncia.

\item[\textbf{73}] \index{contactor!de resist\`{e}ncia de c\`{a}rrega} \textbf{Contactor de resist\`{e}ncia
 de c\`{a}rrega}. \'{E}s el que s'utilitza per posar en curt circuit o per commutar un gra\'{o} de c\`{a}rrega,
 destinat a limitar o a desviar la c\`{a}rrega, en un circuit de pot\`{e}ncia.

\item[\textbf{74}] \index{rel\`{e}!d'alarma} \textbf{Rel\`{e} d'alarma}. \'{E}s qualsevol altre rel\`{e},
diferent al dispositiu 30, que s'utilitza per actuar una alarma
visible o audible.

\item[\textbf{75}] \index{mecanisme!de canvi de posici\'{o}} \textbf{Mecanisme de canvi
de posici\'{o}}. \'{E}s el que s'utilitza per moure un dispositiu d'un equip
des d'una posici\'{o} a una altra; un  exemple, seria el mecanisme
utilitzat per canviar un interruptor entre les posicions de
connectat, desconnectat i prova.

\item[\textbf{76}] \index{rel\`{e}!de sobreintensitat de corrent continu} \textbf{Rel\`{e} de
sobreintensitat de corrent continu}. \'{E}s el que actua quan la intensitat en un circuit de
corrent continu, sobrepassa un valor determinat.

\item[\textbf{77}] \index{transmissor d'impulsos} \textbf{Transmissor d'impulsos}. \'{E}s un
 dispositiu que s'utilitza per generar o transmetre  impulsos, a trav\'{e}s d'un circuit de
telemetria o fil pilot, a un dispositiu d'indicaci\'{o} o recepci\'{o}
remot.

\item[\textbf{78}] \index{rel\`{e}!de  mesura de l'angle de fase o de protecci\'{o} de
desfase} \textbf{Rel\`{e} de  mesura de l'angle de fase o de protecci\'{o}
de desfase}. \'{E}s el que actua a partir d'un valor determinat de
l'angle de fase entre dues tensions o dues intensitats, o entre una
tensi\'{o} i una intensitat.

\item[\textbf{79}] \index{rel\`{e}!de reenganxament de corrent altern} \textbf{Rel\`{e} de
reenganxament de corrent altern}. \'{E}s el que controla el reenganxament i enclavament d'un
interruptor de corrent altern.

\item[\textbf{80}] \index{rel\`{e}!de flux de l\'{\i}quids o gasos} \textbf{Rel\`{e} de flux de l\'{\i}quids
o gasos}.
 \'{E}s el que actua a partir d'un valor determinat del flux d'un l\'{\i}quid o d'un gas, o de
 la  velocitat de variaci\'{o} d'aquest flux.

\item[\textbf{81}] \index{rel\`{e}!de freq\"{u}\`{e}ncia} \textbf{Rel\`{e} de freq\"{u}\`{e}ncia}. \'{E}s el que actua
davant una variaci\'{o} de la freq\"{u}\`{e}ncia o de la seva velocitat de variaci\'{o}.

\item[\textbf{82}] \index{rel\`{e}!de reenganxament de corrent continu} \textbf{Rel\`{e} de
reenganxament de corrent continu}. \'{E}s el que controla el tancament i el reenganxament d'un
interruptor de corrent continu, generalment responent a les condicions de c\`{a}rrega del
circuit.

\item[\textbf{83}] \index{rel\`{e}!de selecci\'{o} o transfer\`{e}ncia  del control autom\`{a}tic}
\textbf{Rel\`{e} de selecci\'{o} o transfer\`{e}ncia del control autom\`{a}tic}. \'{E}s
el que actua per tal d'escollir autom\`{a}ticament entre certes fonts
d'alimentaci\'{o} o entre certes condicions d'un equip; tamb\'{e} \'{e}s el que
efectua autom\`{a}ticament una operaci\'{o} de transfer\`{e}ncia.

\item[\textbf{84}] \index{mecanisme!d'accionament} \textbf{Mecanisme d'accionament}. \'{E}s un
mecanisme o un servo-mecanisme el\`{e}ctric complet,  d'un canviador de
preses, d'un regulador d'inducci\'{o} o de qualsevol altre aparell
similar, que no tingui n\'{u}mero de funci\'{o} propi assignat.

\item[\textbf{85}] \index{rel\`{e}!receptor d'ones portadores o fil pilot} \textbf{Rel\`{e}
receptor d'ones portadores o fil pilot}. \'{E}s un rel\`{e} actuat per un
senyal d'una ona portadora o per un fil pilot de corrent continu, provocat
per l'actuaci\'{o} d'una protecci\'{o} direccional.

\item[\textbf{86}] \index{rel\`{e}!d'enclavament} \textbf{Rel\`{e} d'enclavament}. \'{E}s un rel\`{e}
accionat el\`{e}ctricament, amb reposici\'{o} manual o el\`{e}ctrica, que actua
per parar i mantenir un equip fora de servei, quan hi ha condiciones
anormals.

\item[\textbf{87}] \index{rel\`{e}!de protecci\'{o} diferencial} \textbf{Rel\`{e} de protecci\'{o}
diferencial}. \'{E}s el que actua a partir d'una  difer\`{e}ncia quantitativa de dues intensitats
o d'algunes altres magnituds el\`{e}ctriques.

\item[\textbf{88}] \index{motor o grup moto-generador auxiliar} \textbf{Motor o grup
moto-generador auxiliar}. \'{E}s un dispositiu que s'utilitza per
accionar equips auxiliares.

\item[\textbf{89}] \index{desconnectador de l\'{\i}nia} \textbf{Desconnectador de l\'{\i}nia}. \'{E}s
un dispositiu que s'utilitza com a desconnectador o a\"{\i}llador en un
circuit de pot\`{e}ncia de corrent continu o altern, sempre que aquest
dispositiu sigui operat el\`{e}ctricament o tingui accessoris el\`{e}ctrics.

\item[\textbf{90}] \index{dispositiu!de regulaci\'{o}} \textbf{Dispositiu de regulaci\'{o}}. \'{E}s el que
actua per tal de regular una magnitud, com ara la tensi\'{o}, la intensitat, la potencia,
la velocitat, la freq\"{u}\`{e}ncia, etc., a un valor determinat.

\item[\textbf{91}] \index{rel\`{e}!direccional de tensi\'{o}} \textbf{Rel\`{e} direccional de tensi\'{o}}.
\'{E}s el que actua quan la tensi\'{o} entre els extrems oberts d'un
interruptor o contactor, sobrepassa un valor determinat, en un
sentit donat.

\item[\textbf{92}] \index{rel\`{e}!direccional de tensi\'{o} i pot\`{e}ncia} \textbf{Rel\`{e} direccional
de tensi\'{o} i pot\`{e}ncia}. \'{E}s el que permet o ocasiona la connexi\'{o} de
dos circuits, quan la difer\`{e}ncia de tensi\'{o} entre ambd\'{o}s supera un
valor determinat, en un cert sentit, i ocasiona la desconnexi\'{o} dels
dos circuits, quan la pot\`{e}ncia circulant supera un valor determinat,
en el sentit contrari.

\item[\textbf{93}] \index{contactor!de canvi del camp} \textbf{Contactor de canvi del camp}. \'{E}s el
que actua per tal de augmentar o disminuir el valor de l'excitaci\'{o}
d'una m\`{a}quina.

\item[\textbf{94}] \index{rel\`{e}!de dispar o dispar lliure} \textbf{Rel\`{e} de dispar o dispar
lliure}. \'{E}s el que actua per tal de disparar o permetre disparar un
interruptor, un contactor, etc., o per evitar un reenganxament
immediat d'un interruptor, en el cas que hagi d'obrir i l'ordre de
tancament sigui mantinguda.

\item[\textbf{95}] \textbf{Espec\'{\i}fic}.\footnote{Utilitzat en insta{\l.l}acions
individuals per a aplicacions concretes, quan cap de les funcions 1
a 94 no \'{e}s apropiada.} L'\textsf{USBR}\footnotemark[1] li assigna la
funci\'{o}: rel\`{e} o contactor de tancament.

\item[\textbf{96}] \textbf{Espec\'{\i}fic}.\footnotemark[2]

\item[\textbf{97}] \textbf{Espec\'{\i}fic}.\footnotemark[2]


\item[\textbf{98}] \textbf{Espec\'{\i}fic}.\footnotemark[2] L'\textsf{USBR}\footnotemark[1] li assigna la
funci\'{o}: rel\`{e} de p\`{e}rdua d'excitaci\'{o}.

\item[\textbf{99}] \textbf{Espec\'{\i}fic}.\footnotemark[2] L'\textsf{USBR}\footnotemark[1] li assigna la
funci\'{o}: detector d'arc el\`{e}ctric.

\end{list}
\end{multicols}
