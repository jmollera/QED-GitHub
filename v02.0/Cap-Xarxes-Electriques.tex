\chapter{Resoluci\'{o} de Xarxes El\`{e}ctriques} \label{chap:nusos}

\index{metode@m\`{e}tode dels nusos}S'explica en aquest cap\'{\i}tol el
m\`{e}tode dels nusos, per a la resoluci\'{o} de xarxes el\`{e}ctriques.

\section{Introducci\'{o}}

Quan tenim una xarxa el\`{e}ctrica amb pocs components, sempre la podem resoldre f\`{a}cilment
utilitzant les lleis de Kirchhoff. No obstant, quan la xarxa t\'{e} molts components, quan est\`{a}
 molt mallada o quan hi ha acoblaments magn\`{e}tics entre diverses branques de la xarxa, \'{e}s millor
 emprar un m\`{e}tode sistem\`{a}tic per tal de resoldre-la, com ara el m\`{e}tode del nusos.

El m\`{e}tode dels nusos serveix per resoldre xarxes, tant de corrent
continu com de corrent altern, on les c\`{a}rregues estan definides pels
seus valors d'imped\`{a}ncia o d'admit\`{a}ncia; no \'{e}s \'{u}til per tant, per
resoldre problemes de flux de c\`{a}rregues, on el que es coneix \'{e}s la
potencia absorbida per les c\`{a}rregues.

Per utilitzar aquest m\`{e}tode, les branques de la xarxa han d'estar
formades per un dels seg\"{u}ents components: \vspace{-1.5mm}
\begin{dinglist}{'121}
   \item Font de tensi\'{o} en s\`{e}rie amb una imped\`{a}ncia.
   \item Font de corrent en para{\l.l}el amb una admit\`{a}ncia (l'admit\`{a}ncia pot ser nu{\l.l}a).
   \item Imped\`{a}ncia.
   \item Admit\`{a}ncia.
   \item Acoblament magn\`{e}tic entre branques.
   \item Transformador\footnote{Cal substituir el transformador pel seu circuit equivalent, format per imped\`{a}ncies i
   admit\`{a}ncies. Vegeu la Secci\'{o} \ref{sec:trafo_reg}.}
\end{dinglist}
\vspace{-1.5mm}

No \'{e}s possible tenir branques d'imped\`{a}ncia nu{\l.l}a (curt circuits) o
branques amb fonts de tensi\'{o} ideals (sense imped\`{a}ncia s\`{e}rie).
Aquesta limitaci\'{o} es pot superar, no obstant, substituint la
imped\`{a}ncia nu{\l.l}a, per dues imped\`{a}ncies en s\`{e}rie i de valor
contrari, i introduint un nus fictici addicional en el punt d'uni\'{o}
d'aquestes dues noves imped\`{a}ncies. En la Figura
\vref{pic:branca_nula}
 es representa gr\`{a}ficament aquesta substituci\'{o}. \index{metode@m\`{e}tode dels nusos!branques d'imped\`{a}ncia
nu{\l.l}a}
\begin{figure}[htb]
\centering
    %PsTricks content-type (pstricks.sty package needed)
    %Add \usepackage{pstricks} in the preamble of your LaTeX file
    \psset{xunit=1mm,yunit=1mm,runit=1mm}
    \psset{linewidth=0.3,dotsep=1,hatchwidth=0.3,hatchsep=1.5,shadowsize=1}
    \psset{dotsize=0.7 2.5,dotscale=1 1,fillcolor=black}
    \begin{pspicture}(0,0)(135,30)
    \rput[b](94,8){$\cmplx{Z}$} \rput[b](94,26){$\cmplx{Z}$}
    \psline[linewidth=0.25](13,8)(15,8)
    \psline[linewidth=0.25](75,8)(77,8)
    \psline[linewidth=0.25](14,9)(14,7)
    \psline[linewidth=0.25](76,9)(76,7) \rput[b](19,10){$\cmplx{E}$}
    \rput[b](81,10){$\cmplx{E}$} \rput(94,5){}
    \pspolygon[linewidth=0.25](89,3)(99,3)(99,7)(89,7) \rput(94,23){}
    \pspolygon[linewidth=0.25](89,21)(99,21)(99,25)(89,25)
    \psbezier[linewidth=0.25](17,5)(17.67,6.33)(18.33,6.33)(19,5)
    \psbezier[linewidth=0.25](79,5)(79.67,6.33)(80.33,6.33)(81,5)
    \psbezier[linewidth=0.25](19,5)(19.67,3.67)(20.33,3.67)(21,5)
    \psbezier[linewidth=0.25](81,5)(81.67,3.67)(82.33,3.67)(83,5)
    \rput{0}(35,23){\psellipse[linewidth=0.25](0,0)(1,1)}
    \rput{0}(35,5){\psellipse[linewidth=0.25](0,0)(1,1)}
    \rput{0}(3,23){\psellipse[linewidth=0.25](0,0)(1,1)}
    \rput{0}(3,5){\psellipse[linewidth=0.25](0,0)(1,1)}
    \rput{0}(69,23){\psellipse[linewidth=0.25](0,0)(1,1)}
    \rput{0}(69,5){\psellipse[linewidth=0.25](0,0)(1,1)}
    \rput{0}(107,23){\psellipse[linewidth=0.25](0,0)(1,1)}
    \rput{0}(107,5){\psellipse[linewidth=0.25](0,0)(1,1)}
    \psline[linewidth=0.25](4,5)(15,5) \rput[b](120,26){$-\cmplx{Z}$}
    \rput[b](120,8){$-\cmplx{Z}$} \rput(120,23){}
    \pspolygon[linewidth=0.25](115,21)(125,21)(125,25)(115,25)
    \rput(120,5){}
    \pspolygon[linewidth=0.25](115,3)(125,3)(125,7)(115,7)
    \psline[linewidth=0.25](70,5)(77,5)
    \psline[linewidth=0.25](108,23)(115,23)
    \psline[linewidth=0.25](108,5)(115,5)
    \psline[linewidth=0.25](99,23)(106,23)
    \psline[linewidth=0.25](99,5)(106,5)
    \rput{0}(133,23){\psellipse[linewidth=0.25](0,0)(1,1)}
    \rput{0}(133,5){\psellipse[linewidth=0.25](0,0)(1,1)}
    \psline[linewidth=0.25](125,23)(132,23)
    \psline[linewidth=0.25](125,5)(132,5) \rput[b](35,25){$\beta$}
    \rput[b](35,7){$\beta$} \rput[b](133,25){$\beta$}
    \rput[b](133,7){$\beta$} \rput[b](107,25){$\gamma$}
    \rput[b](107,7){$\gamma$} \rput[b](3,25){$\alpha$}
    \rput[b](3,7){$\alpha$} \rput[b](69,25){$\alpha$}
    \rput[b](69,7){$\alpha$} \psline[linewidth=0.25](70,23)(89,23)
    \psline[linewidth=0.25](4,23)(34,23)
    \psline[linewidth=0.25](34,5)(23,5)
    \rput{0}(19,5){\psellipse[linewidth=0.25](0,0)(4,4)}
    \rput{0}(81,5){\psellipse[linewidth=0.25](0,0)(4,4)}
    \psline[linewidth=0.25](89,5)(85,5) \rput(52,23){$\equiv$}
    \rput(52,5){$\equiv$} \psline[linewidth=0.25](89,25)(99,21)
    \psline[linewidth=0.25](99,25)(89,21)
    \psline[linewidth=0.25](115,21)(125,25)
    \psline[linewidth=0.25](125,21)(115,25)
    \psline[linewidth=0.25](115,7)(125,3)
    \psline[linewidth=0.25](125,7)(115,3)
    \psline[linewidth=0.25](99,7)(89,3)
    \psline[linewidth=0.25](89,7)(99,3)
    \end{pspicture}
\caption{Substituci\'{o} de branques d'imped\`{a}ncia nu{\l.l}a}
\label{pic:branca_nula}
\end{figure}

Per ajudar-nos en l'explicaci\'{o} d'aquest m\`{e}tode de resoluci\'{o} de xarxes, farem
\'{u}s de l'exemple de la Figura \vref{pic:metode_nusos}. Els valors dels components d'aquest
circuit s\'{o}n:
\begin{align*}
   \cmplx{E}_1 &= 200_{\measuredangle 0\degree}\unit{V} & R_1 &= 10\unit{\ohm} &
   \cmplx{E}_2 &= 50_{\measuredangle 0\degree}\unit{V}  & \cmplx{X}_2 &= \ju 20\unit{\ohm} &
   \cmplx{X}_3 &= \ju 5\unit{\ohm} \\
   R_4 &= 20\unit{\ohm} & \cmplx{J}_5 &= 4_{\measuredangle 0\degree}\unit{A} &
   R_5 &= 10\unit{\ohm} & \cmplx{X}\ped{M} &= \ju 5\unit{\ohm}
\end{align*}

\begin{figure}[htb]
\vspace{-4mm} \centering
    %Created by jPicEdt 1.x
    %PsTricks format (pstricks.sty needed)
    %Sat Sep 11 15:21:24 CEST 2004
    \psset{xunit=1mm,yunit=1mm,runit=1mm}
    \begin{pspicture}(0,0)(161.00,55.00)
    \rput[b](40.00,10.50){$R_1$}
    \rput[b](32.00,50.50){$R_4$}
    \rput[b](40.00,32.50){$\cmplx{X}_2$}
    \rput[l](82.00,18.50){$R_5$}
    \rput[r](67.00,18.50){$\cmplx{J}_5$}
    \rput[r](54.00,18.50){$\cmplx{X}_3$}
    \rput[t](43.00,23.50){$\cmplx{X}\ped{M}$}
    \psline[linewidth=0.25,linecolor=black]{-}(18.00,32.50)(20.00,32.50)
    \psline[linewidth=0.25,linecolor=black]{-}(18.00,10.50)(20.00,10.50)
    \psline[linewidth=0.25,linecolor=black]{-}(19.00,33.50)(19.00,31.50)
    \psline[linewidth=0.25,linecolor=black]{-}(19.00,11.50)(19.00,9.50)
    \rput[b](24.00,34.50){$\cmplx{E}_2$}
    \rput[b](24.00,12.50){$\cmplx{E}_1$} \rput(40.00,7.50){}
    \psframe[linewidth=0.25,linecolor=black](35.00,5.50)(45.00,9.50)
    \rput(32.00,47.50){}
    \psframe[linewidth=0.25,linecolor=black](27.00,45.50)(37.00,49.50)
    \rput(79.00,18.50){}
    \psframe[linewidth=0.25,linecolor=black](77.00,13.50)(81.00,23.50)
    \psbezier[linewidth=0.25,linecolor=black]{-}(22.00,29.50)(22.67,30.83)(23.33,30.83)(24.00,29.50)
    \psbezier[linewidth=0.25,linecolor=black]{-}(22.00,7.50)(22.67,8.83)(23.33,8.83)(24.00,7.50)
    \psbezier[linewidth=0.25,linecolor=black]{-}(24.00,29.50)(24.67,28.17)(25.33,28.17)(26.00,29.50)
    \psbezier[linewidth=0.25,linecolor=black]{-}(24.00,7.50)(24.67,6.17)(25.33,6.17)(26.00,7.50)
    \pscircle[linewidth=0.25,linecolor=black](57.00,29.50){1.00}
    \pscircle[linewidth=0.25,linecolor=black](143.00,29.50){1.00}
    \pscircle[linewidth=0.25,linecolor=black](57.00,7.50){1.00}
    \pscircle[linewidth=0.25,linecolor=black](143.00,7.50){1.00}
    \pscircle[linewidth=0.25,linecolor=black](8.00,29.50){1.00}
    \pscircle[linewidth=0.25,linecolor=black](112.00,29.50){1.00}
    \psline[linewidth=0.25,linecolor=black]{-}(45.00,29.50)(56.00,29.50)
    \pscircle[linewidth=0.25,linecolor=black,fillcolor=black,fillstyle=solid](34.00,26.50){1.00}
    \pscircle[linewidth=0.25,linecolor=black,fillcolor=black,fillstyle=solid](54.00,24.50){1.00}
    \psline[linewidth=0.25,linecolor=black]{-}(57.00,28.50)(57.00,23.50)
    \psline[linewidth=0.25,linecolor=black]{-}(57.00,8.50)(57.00,13.50)
    \psline[linewidth=0.25,linecolor=black]{-}(45.00,7.50)(55.00,7.50)
    \psline[linewidth=0.25,linecolor=black]{-}(55.00,7.50)(56.00,7.50)
    \psline[linewidth=0.25,linecolor=black]{->}(143.00,28.50)(143.00,18.00)
    \psline[linewidth=0.25,linecolor=black]{->}(142.00,7.50)(127.00,7.50)
    \rput[t](143.00,5.50){$\boxed{0}$} \rput[t](57.00,5.50){$\boxed{0}$}
    \rput[bl](145.00,31.50){$\boxed{2}$}
    \rput[bl](59.00,31.50){$\boxed{2}$}
    \rput[r](110.00,29.50){$\boxed{1}$}
    \rput[r](6.00,29.50){$\boxed{1}$} \rput(71.00,18.50){}
    \psframe[linewidth=0.25,linecolor=black](68.00,14.50)(74.00,22.50)
    \psline[linewidth=0.25,linecolor=black]{>-}(128.00,29.50)(142.00,29.50)
    \rput[b](128.00,45.50){4} \rput[b](128.00,30.50){2}
    \rput[b](128.00,8.50){1} \rput[l](143.00,18.50){}
    \rput[l](157.00,18.50){} \rput[l](144.00,18.50){3}
    \rput[l](158.00,18.50){5}
    \psframe[linewidth=0.15,linecolor=black,fillcolor=black,fillstyle=solid](35.00,27.50)(45.00,31.50)
    \psframe[linewidth=0.15,linecolor=black,fillcolor=black,fillstyle=solid](55.00,13.50)(59.00,23.50)
    \psline[linewidth=0.25,linecolor=black]{-}(8.00,30.50)(8.00,47.50)(27.00,47.50)
    \psline[linewidth=0.25,linecolor=black]{-}(37.00,47.50)(57.00,47.50)(57.00,30.50)
    \psline[linewidth=0.25,linecolor=black]{-}(9.00,29.50)(20.00,29.50)
    \psline[linewidth=0.25,linecolor=black]{-}(35.00,29.50)(28.00,29.50)
    \pscircle[linewidth=0.25,linecolor=black](24.00,29.50){4.00}
    \psline[linewidth=0.25,linecolor=black]{-}(8.00,28.50)(8.00,7.50)(20.00,7.50)
    \pscircle[linewidth=0.25,linecolor=black](24.00,7.50){4.00}
    \psline[linewidth=0.25,linecolor=black]{-}(35.00,7.50)(28.00,7.50)
    \psline[linewidth=0.25,linecolor=black]{-}(71.00,22.50)(71.00,25.50)(79.00,25.50)(79.00,23.50)
    \psline[linewidth=0.25,linecolor=black]{-}(71.00,14.50)(71.00,11.50)(79.00,11.50)(79.00,13.50)
    \psline[linewidth=0.25,linecolor=black]{-}(58.00,29.50)(75.00,29.50)(75.00,25.50)
    \psline[linewidth=0.25,linecolor=black]{-}(58.00,7.50)(75.00,7.50)(75.00,11.50)
    \psbezier[linewidth=0.25,linecolor=black,linestyle=dashed,dash=1.00
    1.00]{<->}(52.00,24.50)(42.50,24.50)(37.50,24.50)(35.50,25.50)
    \psline[linewidth=0.25,linecolor=black]{>-}(128.00,44.50)(143.00,44.50)(143.00,30.50)
    \psline[linewidth=0.25,linecolor=black]{-}(129.50,44.50)(112.00,44.50)(112.00,30.50)
    \psline[linewidth=0.25,linecolor=black]{-}(113.00,29.50)(129.50,29.50)
    \psline[linewidth=0.25,linecolor=black]{-}(143.00,8.50)(143.00,19.50)
    \psline[linewidth=0.25,linecolor=black]{<-}(157.00,19.50)(157.00,7.50)(144.00,7.50)
    \psline[linewidth=0.25,linecolor=black]{-}(157.00,18.00)(157.00,29.50)(144.00,29.50)
    \psline[linewidth=0.25,linecolor=black]{-}(129.00,7.50)(112.00,7.50)(112.00,28.50)
    \psline[linewidth=0.25,linecolor=black]{->}(71.00,16.50)(71.00,20.50)
    \end{pspicture}
   \caption{Resoluci\'{o} de xarxes el\`{e}ctriques pel m\`{e}tode dels nusos} \label{pic:metode_nusos}
\end{figure}

\index{graf orientat}En primer lloc, a partir de la xarxa el\`{e}ctrica
cal representar el seu graf orientat, seguint els passos seg\"{u}ents
(Figura \vref{pic:metode_nusos}):
\begin{dingautolist}{'312}
   \item Es representen les connexions de les branques mitjan\c{c}ant l\'{\i}nies, sense dibuixar-hi cap component el\`{e}ctric.
   \item Es d\'{o}na un sentit a aquestes branques, dibuixant-hi fletxes. Aquestes fletxes representen els sentits assignats als corrents i a les difer\`{e}ncies de potencial entre nusos.
   \item Es numeren tots els nusos de forma consecutiva, comen\c{c}ant pel n\'{u}mero 0; el nus 0 s'anomena nus de potencial zero, o de refer\`{e}ncia.
\index{nus!de potencial zero}\index{nus!de refer\`{e}ncia}
   \item Es numeren totes les branques de forma consecutiva, comen\c{c}ant pel n\'{u}mero 1.
\end{dingautolist}

\index{metode@m\`{e}tode dels nusos!nombre de nusos}\index{metode@m\`{e}tode
dels nusos!nombre de branques}Es defineixen a continuaci\'{o} els dos
par\`{a}metres b\`{a}sics d'aquest m\`{e}tode, $n$ i $b$; aquests dos valors
defineixen les dimensions dels vectors i matrius que es veuran m\'{e}s
endavant:
\begin{list}{}
   {\setlength{\labelwidth}{7mm} \setlength{\leftmargin}{9mm} \setlength{\labelsep}{2mm}}
   \item[$n$:] Nombre de nusos de la xarxa, sense comptar el nus de refer\`{e}ncia.

   En el nostre exemple tenim:
   \[ n=2 \]

   \item[$b$:] Nombre de branques de la xarxa.

   En el nostre exemple tenim:
   \[ b=5 \]
\end{list}

\section{M\`{e}tode general de resoluci\'{o}}

\index{metode@m\`{e}tode dels nusos!cas general}Quan hi ha acoblaments
magn\`{e}tics entre branques de la xarxa hem d'utilitzar el m\`{e}tode
general de resoluci\'{o}, descrit a continuaci\'{o}.

En primer lloc, a partir dels valors dels components de la xarxa, formem les matrius i vectors seg\"{u}ents (es donen les seves dimensions entre claus):
\begin{list}{}
{\setlength{\labelwidth}{20mm} \setlength{\leftmargin}{22mm} \setlength{\labelsep}{2mm}}
   \item[$\boldsymbol{A}\{n\times b\}$:] \index{matriu!d'incid\`{e}ncia de nusos $\boldsymbol{A}$}Matriu d'incid\`{e}ncia de nusos. Cada columna representa una branca, en ordre creixent d'esquerra a dreta, i cada fila representa un nus (sense comptar el de refer\`{e}ncia) en ordre creixent, de dalt a baix. Cada branca del graf orientat omple la columna corresponent de la matriu $\boldsymbol{A}$ amb els valors 0, 1 \'{o} -1 segons el criteri seg\"{u}ent:
   \begin{list}{}
   {\setlength{\labelwidth}{7mm} \setlength{\leftmargin}{9mm} \setlength{\labelsep}{2mm}}
      \item[1:]  si la branca surt del nus.
      \item[-1:] si la branca va a para al nus.
      \item[0:]  si la branca ni surt ni va a parar al nus.
   \end{list}
   Els termes {"<}surt{">} i {"<}va a parar{">} s'han d'entendre segons les fletxes dibuixades en les branques del graf orientat. Les connexions al nus de refer\`{e}ncia, no apareixen en la matriu $\boldsymbol{A}$.

   En el nostre exemple tenim:
   \[
      \boldsymbol{A} = \left(\begin{array}{rrrrr} -1 & 1  & 0 &  1 & 0 \\  0 & -1 & 1 & -1 & -1
                   \end{array} \right)
   \]

   \item[$\mcmplx{Z}\ped{B}\{b\times b\}$:] \index{matriu!d'imped\`{a}ncies de branca $\mcmplx{Z}\ped{B}$}Matriu d'imped\`{a}ncies de branca. Els elements de la diagonal estan formats per les imped\`{a}ncies de les respectives branques, i els elements de fora de la diagonal estan formats per les imped\`{a}ncies dels acoblaments magn\`{e}tics entre cada parell de branques.

   Els acoblaments magn\`{e}tics poden ser positius o negatius, depenent
    de la posici\'{o} dels punts hom\`{o}legs de les induct\`{a}ncies i del sentit
    de les branques del graf orientat. L'acoblament \'{e}s positiu quan les
    fletxes de les dues branques acoblades es dirigeixen cap els seus punts
    hom\`{o}legs respectius, o quan les dues fletxes se n'allunyen; en canvi,
    l'acoblament \'{e}s negatiu quan una de les fletxes de les dues branques
    acoblades es dirigeix cap el seu punt hom\`{o}leg i l'altra se n'allunya.
    \index{acoblament magn\`{e}tic}

   En el nostre exemple tenim:
   \[
      \mcmplx{Z}\ped{B} = \begin{pmatrix}
            10 & 0 & 0 & 0 & 0 \\
            0 & \ju 20 & \ju 5 & 0 & 0 \\
            0 & \ju 5 & \ju 5 & 0 & 0 \\
            0 & 0 & 0 & 20 & 0 \\
            0 & 0 & 0 & 0 & 10
      \end{pmatrix}\unit{\ohm}
   \]

   \item[$\mcmplx{E}'\ped{B}\{b\}$:] \index{vector!de forces electromotrius $\mcmplx{E}'\ped{B}$}Vector columna de forces electromotrius de branca. Els elements d'aquest vector estan formats per les forces electromotrius de les fonts de tensi\'{o} de les respectives branques.

El signe de cada for\c{c}a electromotriu \'{e}s positiu, si el seu sentit coincideix amb el sentit de la fletxa de la branca corresponent del graf orientat, i negatiu en cas contrari.

   En el nostre exemple tenim:
   \[
      \mcmplx{E}'\ped{B} = \begin{pmatrix} 200 \\ -50 \\ 0 \\ 0 \\ 0 \end{pmatrix}\unit{V}
   \]

   \item[$\mcmplx{J}'\ped{B}\{b\}$:] \index{vector!d'intensivitats de branca $\mcmplx{J}'\ped{B}$}Vector columna d'intensivitats de branca. Els elements d'aquest vector estan formats per les intensivitats de les fonts de corrent de les respectives branques.

El signe de cada intensivitat \'{e}s positiu, si el seu sentit coincideix amb el sentit de la fletxa de la branca corresponent del graf orientat, i negatiu en cas contrari.

   En el nostre exemple tenim:
   \[
      \mcmplx{J}'\ped{B} = \begin{pmatrix} 0 \\ 0 \\ 0 \\ 0 \\ 4 \end{pmatrix}\unit{A}
   \]

\end{list}

A partir de les dades anteriors, formem ara les diverses matrius i
vectors que ens permetran resoldre la xarxa, aix\`{o} \'{e}s, trobar les
tensions de les branques i els corrents que hi circulen. Aquestes
matrius i vectors s\'{o}n (es donen les seves dimensions entre claus):

\begin{list}{}
{\setlength{\labelwidth}{20mm} \setlength{\leftmargin}{22mm} \setlength{\labelsep}{2mm}}
   \item[$\mcmplx{Y}\ped{B}\{b\times b\}$:] \index{matriu!d'admit\`{a}ncies de branca $\mcmplx{Y}\ped{B}$}Matriu d'admit\`{a}ncies de branca. Est\`{a} definida per la relaci\'{o} seg\"{u}ent:
   \begin{equation}
      \mcmplx{Y}\ped{B} = \mcmplx{Z}\ped{B}^{-1}
   \end{equation}

   En el nostre exemple tenim:
   \[
      \mcmplx{Y}\ped{B} = \begin{pmatrix}
            10 & 0 & 0 & 0 & 0 \\
            0 & \ju 20 & \ju 5 & 0 & 0 \\
            0 & \ju 5 & \ju 5 & 0 & 0 \\
            0 & 0 & 0 & 20 & 0 \\
            0 & 0 & 0 & 0 & 10
      \end{pmatrix} ^{-1} =
      \begin{pmatrix}
            \frac{1}{10} & 0 & 0 & 0 & 0 \\
            0 & -\ju\frac{1}{15} & \ju\frac{1}{15} & 0 & 0 \\
            0 & \ju\frac{1}{15} & -\ju\frac{4}{15} & 0 & 0 \\
            0 & 0 & 0 & \frac{1}{20} & 0 \\
            0 & 0 & 0 & 0 & \frac{1}{10}
      \end{pmatrix}\unit{S}
   \]

   \item[$\mcmplx{J}\ped{B}\{b\}$:] \index{vector!d'intensivitats equivalents de branca $\mcmplx{J}\ped{B}$}Vector columna d'intensivitats equivalents de branca. Est\`{a} definit per la relaci\'{o} seg\"{u}ent:
   \begin{equation}
      \mcmplx{J}\ped{B} = \mcmplx{J}'\ped{B}  + \mcmplx{Y}\ped{B} \mcmplx{E}'\ped{B}
   \end{equation}

   En el nostre exemple tenim:
   \[
      \mcmplx{J}\ped{B} =
      \begin{pmatrix} 0 \\ 0 \\ 0 \\ 0 \\ 4 \end{pmatrix} +
      \begin{pmatrix}
            \frac{1}{10} & 0 & 0 & 0 & 0 \\
            0 & -\ju\frac{1}{15} & \ju\frac{1}{15} & 0 & 0 \\
            0 & \ju\frac{1}{15} & -\ju\frac{4}{15} & 0 & 0 \\
            0 & 0 & 0 & \frac{1}{20} & 0 \\
            0 & 0 & 0 & 0 & \frac{1}{10}
      \end{pmatrix} \cdot
      \begin{pmatrix} 200 \\ -50 \\ 0 \\ 0 \\ 0 \end{pmatrix} =
      \begin{pmatrix} 20 \\ \ju\frac{10}{3} \\ -\ju\frac{10}{3} \\ 0 \\ 4 \end{pmatrix}
     \unit{A}
   \]

   \item[$\mcmplx{Y}\ped{N}\{n\times n\}$:] \index{matriu!d'admit\`{a}ncies de nus $\mcmplx{Y}\ped{N}$}Matriu d'admit\`{a}ncies de nus. Est\`{a} definida per la relaci\'{o} seg\"{u}ent:
   \begin{equation}
      \mcmplx{Y}\ped{N} = \boldsymbol{A} \mcmplx{Y}\ped{B}
      \transp{\boldsymbol{A}}
   \end{equation}

   En el nostre exemple tenim:
   \[ \begin{split}
      \mcmplx{Y}\ped{N} &=
      \left(\begin{array}{rrrrr} -1 & 1  & 0 &  1 & 0 \\  0 & -1 & 1 & -1 & -1
      \end{array}\right) \cdot
      \begin{pmatrix}
            \frac{1}{10} & 0 & 0 & 0 & 0 \\
            0 & -\ju\frac{1}{15} & \ju\frac{1}{15} & 0 & 0 \\
            0 & \ju\frac{1}{15} & -\ju\frac{4}{15} & 0 & 0 \\
            0 & 0 & 0 & \frac{1}{20} & 0 \\
            0 & 0 & 0 & 0 & \frac{1}{10}
      \end{pmatrix} \cdot
      \left(\begin{array}{rr} -1 & 0 \\ 1  & -1 \\  0 & 1 \\ 1 & -1 \\ 0 & -1
      \end{array}\right) = \\
       &=
      \frac{1}{60} \cdot \begin{pmatrix}
            9 - \ju 4 & -3 + \ju 8 \\
            3 + \ju 8 & 9 - \ju 28
      \end{pmatrix}\unit{S}
   \end{split}  \label{eq:yn}
   \]

   \item[$\mcmplx{J}\ped{N}\{n\}$:] \index{vector!d'intensivitats de nus $\mcmplx{J}\ped{N}$}Vector columna d'intensivitats de nus. Est\`{a} definit per la relaci\'{o} seg\"{u}ent:
   \begin{equation}
      \mcmplx{J}\ped{N} = - \boldsymbol{A} \mcmplx{J}\ped{B}
   \end{equation}

   En el nostre exemple tenim:
   \[
      \mcmplx{J}\ped{N} = -
      \left(\begin{array}{rrrrr} -1 & 1  & 0 &  1 & 0 \\  0 & -1 & 1 & -1 & -1
      \end{array}\right) \cdot
      \begin{pmatrix} 20 \\ \ju\frac{10}{3} \\ -\ju\frac{10}{3} \\ 0 \\ 4 \end{pmatrix}
      =
      \frac{1}{3} \cdot \begin{pmatrix}
            60 - \ju 10 \\
            12 + \ju 20
      \end{pmatrix}\unit{A}
   \]

   \item[$\mcmplx{V}\ped{N}\{n\}$:] \index{vector!de potencials de nus $\mcmplx{V}\ped{N}$}Vector columna de potencials de nus. Est\`{a} definit per la relaci\'{o} seg\"{u}ent:
   \begin{equation}
      \mcmplx{Y}\ped{N} \mcmplx{V}\ped{N} = \mcmplx{J}\ped{N} \quad\rightarrow\quad
      \mcmplx{V}\ped{N} = \mcmplx{Y}\ped{N}^{-1} \mcmplx{J}\ped{N} \label{eq:vn}
   \end{equation}

   Els elements d'aquest vector s\'{o}n els potencials de cada nus de la xarxa respecte del nus de refer\`{e}ncia.

   En el nostre exemple tenim:
   \[
      \mcmplx{V}\ped{N} = 60 \cdot
      \begin{pmatrix}
            9 - \ju 4 & -3 + \ju 8 \\
            3 + \ju 8 & 9 - \ju 28
      \end{pmatrix} ^{-1} \cdot
      \frac{1}{3} \cdot \begin{pmatrix}
            60 - \ju 10 \\
            12 + \ju 20
      \end{pmatrix}
      =
      \frac{1}{101} \cdot \begin{pmatrix}
            15430 + \ju 2295 \\
            3390 + \ju 2085
      \end{pmatrix}\unit{V}
      \label{eq:vn_exemp}
   \]

   \item[$\mcmplx{U}\ped{B}\{b\}$:] \index{vector!de tensions de branca $\mcmplx{U}\ped{B}$}Vector columna de tensions de branca. Est\`{a} definit per la relaci\'{o} seg\"{u}ent:
   \begin{equation}
      \mcmplx{U}\ped{B} = \transp{\boldsymbol{A}} \mcmplx{V}\ped{N} \label{eq:ur}
   \end{equation}

   Aquesta \'{e}s la soluci\'{o} buscada pel que fa a les tensions de les branques de la xarxa.

   En el nostre exemple tenim:
   \[
      \mcmplx{U}\ped{B} =
      \left(\begin{array}{rr} -1 & 0 \\ 1  & -1 \\  0 & 1 \\ 1 & -1 \\ 0 & -1
      \end{array}\right) \cdot
      \frac{1}{101} \cdot \begin{pmatrix}
            15430 + \ju 2295 \\
            3390 + \ju 2085
      \end{pmatrix} =
      \frac{1}{101} \cdot \begin{pmatrix}
            -15430 - \ju 2295 \\
            12040 + \ju 210  \\
            3390 + \ju 2085 \\
            12040 + \ju 210  \\
            -3390 - \ju 2085
      \end{pmatrix}\unit{V}
   \]

   \item[$\mcmplx{I}\ped{B}\{b\}$:] \index{vector!de corrents de branca $\mcmplx{I}\ped{B}$} Vector columna de corrents de branca. Est\`{a} definit per la relaci\'{o} seg\"{u}ent:
   \begin{equation}
      \mcmplx{I}\ped{B} = \mcmplx{Y}\ped{B} \mcmplx{U}\ped{B} + \mcmplx{J}\ped{B} \label{eq:ir}
   \end{equation}

   Aquesta \'{e}s la soluci\'{o} buscada pel que fa als corrents de les branques de la xarxa.

   En el nostre exemple tenim:
\end{list}
\[
   \mcmplx{I}\ped{B} =
   \begin{pmatrix}
         \frac{1}{10} & 0 & 0 & 0 & 0 \\
         0 & -\ju\frac{1}{15} & \ju\frac{1}{15} & 0 & 0 \\
         0 & \ju\frac{1}{15} & -\ju\frac{4}{15} & 0 & 0 \\
         0 & 0 & 0 & \frac{1}{20} & 0 \\
         0 & 0 & 0 & 0 & \frac{1}{10}
   \end{pmatrix} \cdot
   \frac{1}{101} \cdot \begin{pmatrix}
         -15430 - \ju 2295 \\
         12040 + \ju 210  \\
         3390 + \ju 2085 \\
         12040 + \ju 210  \\
         -3390 - \ju 2085
   \end{pmatrix}
   + \begin{pmatrix} 20 \\ \ju\frac{10}{3} \\ -\ju\frac{10}{3} \\ 0 \\ 4 \end{pmatrix} =
   \frac{1}{202} \cdot \begin{pmatrix}
         954 - \ju 459 \\
      -250 - \ju 480  \\
      1084 - \ju 876 \\
         1204 + \ju 21  \\
         130 - \ju 417
   \end{pmatrix}\unit{A}
\]

Es resumeixen finalment, els passos a seguir per tal de resoldre una
xarxa el\`{e}ctrica, mitjan\c{c}ant el m\`{e}tode dels nusos:
\begin{dingautolist}{'312}
   \item Es representa el graf orientat associat a la xarxa, i es numeren tots els seus nusos i totes les seves branques.
   \item A partir del graf orientat i dels elements de la xarxa, es formen les matrius $\boldsymbol{A}$ i $\mcmplx{Z}\ped{B}$, i els vectors $\mcmplx{E}\ped{B}'$ i $\mcmplx{J}\ped{B}'$.
   \item Es calculen les matrius $\mcmplx{Y}\ped{B}$ i $\mcmplx{Y}\ped{N}$, i els vectors $\mcmplx{J}\ped{B}$ i $\mcmplx{J}\ped{N}$.
   \item Finalment, es calculen els vectors $\mcmplx{V}\ped{N}$, $\mcmplx{U}\ped{B}$ i $\mcmplx{I}\ped{B}$.
\end{dingautolist}

\begin{exemple}
Es tracte de resoldre la xarxa seg\"{u}ent pel m\`{e}tode dels nusos; cal
tenir en compte, que a m\'{e}s de la c\`{a}rrega Q8, els generadors G1, G2 i
G3, tamb\'{e} estan units a terra (nus 0 de refer\`{e}ncia), i que hi ha un
acoblament magn\`{e}tic M entre les l\'{\i}nies L5 i L6. Es calcular\`{a} tamb\'{e}
la pot\`{e}ncia cedida pels tres generadors G1, G2 i G3, la pot\`{e}ncia
absorbida per la c\`{a}rrega Q8 i la pot\`{e}ncia perduda en la resta de
components de la xarxa.
\begin{figure}[htb]
\vspace{3mm} \centering
    %Created by jPicEdt 1.x
    %PsTricks format (pstricks.sty needed)
    %Sun Sep 12 14:10:35 CEST 2004
    \psset{xunit=1mm,yunit=1mm,runit=1mm}
    \begin{pspicture}(0,0)(143.00,33.00)
    \pscircle[linewidth=0.25,linecolor=black](5.50,17.00){5.00}
    \pscircle[linewidth=0.25,linecolor=black](137.50,17.00){5.00}
    \pscircle[linewidth=0.25,linecolor=black](49.50,7.00){5.00}
    \psline[linewidth=0.25,linecolor=black]{-}(10.50,17.00)(15.50,17.00)
    \psline[linewidth=0.25,linecolor=black]{-}(127.50,17.00)(132.50,17.00)
    \psline[linewidth=0.25,linecolor=black]{-}(16.50,22.00)(21.50,22.00)
    \psline[linewidth=0.25,linecolor=black]{-}(38.50,22.00)(43.50,22.00)
    \pscircle[linewidth=0.25,linecolor=black](26.50,22.00){5.10}
    \pscircle[linewidth=0.25,linecolor=black](33.50,22.00){5.00}
    \psbezier[linewidth=0.25,linecolor=black]{-}(3.50,17.00)(4.17,18.33)(4.83,18.33)(5.50,17.00)
    \psbezier[linewidth=0.25,linecolor=black]{-}(135.50,17.00)(136.17,18.33)(136.83,18.33)(137.50,17.00)
    \psbezier[linewidth=0.25,linecolor=black]{-}(47.50,7.00)(48.17,8.33)(48.83,8.33)(49.50,7.00)
    \psbezier[linewidth=0.25,linecolor=black]{-}(5.50,17.00)(6.17,15.67)(6.83,15.67)(7.50,17.00)
    \psbezier[linewidth=0.25,linecolor=black]{-}(137.50,17.00)(138.17,15.67)(138.83,15.67)(139.50,17.00)
    \psbezier[linewidth=0.25,linecolor=black]{-}(49.50,7.00)(50.17,5.67)(50.83,5.67)(51.50,7.00)
    \rput[b](5.50,23.00){G1} \rput[b](16.50,27.00){$\boxed{1}$}
    \rput[b](44.50,27.00){$\boxed{2}$}
    \rput[b](127.50,27.00){$\boxed{3}$}
    \rput[l](91.50,13.00){$\boxed{4}$} \rput[l](91.50,3.00){$\boxed{0}$}
    \rput[b](85.50,24.00){L6} \rput[bl](83.50,19.00){M}
    \rput[t](66.50,19.00){L5} \rput[t](109.50,19.00){L4}
    \rput[b](137.50,23.00){G3} \rput[r](43.50,7.00){G2}
    \rput[r](82.50,8.00){Q8} \rput[t](29.50,16.00){T7}
    \psline[linewidth=0.25,linecolor=black]{-}(44.50,17.00)(49.50,17.00)
    \psline[linewidth=0.25,linecolor=black]{-}(49.50,17.00)(49.50,12.00)
    \psline[linewidth=0.50,linecolor=black]{-}(86.00,9.00)(89.00,9.00)
    \psline[linewidth=0.50,linecolor=black]{-}(86.00,8.00)(89.00,8.00)
    \psline[linewidth=0.25,linecolor=black]{->}(83.50,13.00)(83.50,4.00)
    \psline[linewidth=0.25,linecolor=black]{-}(82.50,20.00)(82.50,14.00)
    \psline[linewidth=0.25,linecolor=black]{-}(88.50,20.00)(88.50,14.00)
    \psline[linewidth=0.25,linecolor=black]{-}(87.50,13.00)(87.50,9.00)
    \psline[linewidth=0.25,linecolor=black]{-}(87.50,8.00)(87.50,4.00)
    \psline[linewidth=0.25,linecolor=black]{-}(82.50,20.00)(44.50,20.00)
    \psline[linewidth=0.25,linecolor=black]{-}(88.50,20.00)(126.50,20.00)
    \psline[linewidth=0.25,linecolor=black]{-}(44.50,23.00)(126.50,23.00)
    \psframe[linewidth=0.15,linecolor=black,fillcolor=black,fillstyle=solid](43.50,14.00)(44.50,26.00)
    \psframe[linewidth=0.15,linecolor=black,fillcolor=black,fillstyle=solid](15.50,14.00)(16.50,26.00)
    \psframe[linewidth=0.15,linecolor=black,fillcolor=black,fillstyle=solid](126.50,14.00)(127.50,26.00)
    \psframe[linewidth=0.15,linecolor=black,fillcolor=black,fillstyle=solid](80.50,13.00)(90.50,14.00)
    \psframe[linewidth=0.15,linecolor=black,fillcolor=black,fillstyle=solid](80.50,3.00)(90.50,4.00)
    \end{pspicture}
\end{figure}

Els valors dels components d'aquesta xarxa, expressats en p.u. s\'{o}n:
\begin{align*}
   G1 &:\; \cmplx{e}_1 = 1{,}1 \qquad\qquad\;\;\, \cmplx{z}_1 = \ju 0{,}25 & L4 &:\; \cmplx{z}_4 = \ju 0{,}10 & T7 &:\; \cmplx{z}_7 = \ju 0{,}16 \quad\quad m_7=1:1\\
   G2 &:\; \cmplx{e}_2 = 1{,}05+\ju0{,}10 \quad \cmplx{z}_2 = \ju 0{,}20 & L5 &:\; \cmplx{z}_5 = \ju 0{,}405  & Q8 &:\; \cmplx{j}_8 = 2-\ju0{,}9 \quad \cmplx{z}_8 = -\ju25 \\
   G3 &:\; \cmplx{e}_3 = 1{,}08+\ju0{,}12 \quad \cmplx{z}_3 = \ju 0{,}25 & L6 &:\; \cmplx{z}_6 = \ju 0{,}50 & M &:\; \cmplx{x}_M = \ju0{,}05\; \text{(entre L5 i L6)}
\end{align*}

Comen\c{c}arem dibuixant el graf orientat associat a la xarxa, i formant la matriu $\boldsymbol{A}$.
\begin{figure}[htb]
\hfill
\begin{minipage}[c]{7cm}
    %Created by jPicEdt 1.x
    %PsTricks format (pstricks.sty needed)
    %Sun Sep 12 14:20:26 CEST 2004
    \psset{xunit=1mm,yunit=1mm,runit=1mm}
    \begin{pspicture}(0,0)(68.00,44.00)
    \pscircle[linewidth=0.25,linecolor=black](7.00,28.00){1.00}
    \pscircle[linewidth=0.25,linecolor=black](25.00,28.00){1.00}
    \pscircle[linewidth=0.25,linecolor=black](43.00,28.00){1.00}
    \pscircle[linewidth=0.25,linecolor=black](61.00,28.00){1.00}
    \psline[linewidth=0.25,linecolor=black]{-}(8.00,28.00)(24.00,28.00)
    \psline[linewidth=0.25,linecolor=black]{-}(44.00,28.00)(60.00,28.00)
    \pscircle[linewidth=0.25,linecolor=black](34.00,7.00){1.00}
    \psline[linewidth=0.25,linecolor=black]{->}(43.00,15.00)(43.00,14.93)
    \psline[linewidth=0.25,linecolor=black]{-}(43.00,26.00)(43.00,15.00)
    \psline[linewidth=0.25,linecolor=black]{-}(7.00,26.00)(7.00,26.00)
    \psline[linewidth=0.25,linecolor=black]{-}(61.00,26.00)(61.00,26.00)
    \psline[linewidth=0.25,linecolor=black]{->}(7.00,17.00)(7.00,17.07)
    \psline[linewidth=0.25,linecolor=black]{-}(7.00,16.00)(7.00,17.00)
    \psline[linewidth=0.25,linecolor=black]{->}(61.00,17.00)(61.00,17.07)
    \psline[linewidth=0.25,linecolor=black]{-}(61.00,16.00)(61.00,17.00)
    \psline[linewidth=0.25,linecolor=black]{-}(7.00,27.00)(7.00,17.00)
    \psline[linewidth=0.25,linecolor=black]{-}(61.00,27.00)(61.00,17.00)
    \psline[linewidth=0.25,linecolor=black]{->}(25.00,17.00)(25.00,17.07)
    \psline[linewidth=0.25,linecolor=black]{-}(25.00,16.00)(25.00,17.00)
    \psline[linewidth=0.25,linecolor=black]{-}(25.00,27.00)(25.00,17.00)
    \psline[linewidth=0.25,linecolor=black]{->}(16.00,28.00)(16.07,28.00)
    \psline[linewidth=0.25,linecolor=black]{-}(15.00,28.00)(16.00,28.00)
    \psline[linewidth=0.25,linecolor=black]{->}(35.00,28.00)(35.07,28.00)
    \psline[linewidth=0.25,linecolor=black]{-}(34.00,28.00)(35.00,28.00)
    \psline[linewidth=0.25,linecolor=black]{->}(52.00,28.00)(51.93,28.00)
    \psline[linewidth=0.25,linecolor=black]{-}(53.00,28.00)(52.00,28.00)
    \rput[b](15.00,29.00){7} \rput[b](34.00,29.00){5}
    \rput[b](53.00,29.00){4} \rput[b](42.00,41.00){6}
    \rput[l](8.00,16.00){1} \rput[l](26.00,16.00){2}
    \rput[l](44.00,16.00){8} \rput[l](62.00,16.00){3}
    \rput[b](43.00,30.00){$\boxed{4}$} \rput[r](5.00,28.00){$\boxed{1}$}
    \rput[l](63.00,28.00){$\boxed{3}$}
    \rput[br](24.00,30.00){$\boxed{2}$}
    \rput[t](34.00,5.00){$\boxed{0}$}
    \psline[linewidth=0.25,linecolor=black]{-}(35.00,28.00)(42.00,28.00)
    \psline[linewidth=0.25,linecolor=black]{-}(26.00,28.00)(34.00,28.00)
    \psline[linewidth=0.25,linecolor=black]{-}(43.00,26.00)(43.00,27.00)
    \psline[linewidth=0.25,linecolor=black]{->}(25.00,29.00)(25.00,40.00)(42.50,40.00)
    \psline[linewidth=0.25,linecolor=black]{-}(40.50,40.00)(61.00,40.00)(61.00,29.00)
    \psline[linewidth=0.25,linecolor=black]{-}(34.00,8.00)(43.00,11.00)(43.00,16.50)
    \psline[linewidth=0.25,linecolor=black]{-}(34.00,8.00)(25.00,11.00)(25.00,17.50)
    \psline[linewidth=0.25,linecolor=black]{-}(35.00,7.00)(61.00,7.00)(61.00,17.00)
    \psline[linewidth=0.25,linecolor=black]{-}(33.00,7.00)(7.00,7.00)(7.00,17.50)
    \end{pspicture}
\end{minipage}
\hfill
\begin{minipage}[c]{9cm}
   \[
   \boldsymbol{A} = \left( \begin{array}{rrrrrrrr}
     -1 & 0 & 0 & 0 & 0 & 0 & 1 & 0 \\
     0 & -1 & 0 & 0 & 1 & 1 & -1 & 0 \\
     0 & 0 & -1 & 1 & 0 & -1 & 0 & 0 \\
     0 & 0 & 0 & -1 & -1 & 0 & 0 & 1
   \end{array}\right)
   \]
\end{minipage}
\hfill{}
\end{figure}

Formem a continuaci\'{o} la matriu $\mcmplx{Z}\ped{B}$ i els vectors $\mcmplx{J}\ped{B}'$ i $\mcmplx{E}\ped{B}'$ (tots els valors en p.u.):
\[
   \mcmplx{Z}\ped{B} = \ju \cdot
   \begin{pmatrix}
     0{,}25 & 0 & 0 & 0 & 0 & 0 & 0 & 0 \\
     0 & 0{,}20 & 0 & 0 & 0 & 0 & 0 & 0 \\
     0 & 0 & 0{,}25 & 0 & 0 & 0 & 0 & 0 \\
     0 & 0 & 0 & 0{,}10 & 0 & 0 & 0 & 0 \\
     0 & 0 & 0 & 0 & 0{,}405 & 0{,}05 & 0 & 0 \\
     0 & 0 & 0 & 0 & 0{,}05 & 0{,}50 & 0 & 0 \\
     0 & 0 & 0 & 0 & 0 & 0 & 0{,}16 & 0 \\
     0 & 0 & 0 & 0 & 0 & 0 & 0 & -25
   \end{pmatrix}
   \;\;
   \mcmplx{J}\ped{B}' =
   \begin{pmatrix}
    0 \\
    0 \\
    0 \\
    0 \\
    0 \\
    0 \\
    0 \\
    2 - \ju 0{,}9
   \end{pmatrix}
   \;\;
   \mcmplx{E}\ped{B}' =
   \begin{pmatrix}
    1{,}1 \\
    1{,}05 + \ju 0{,}10 \\
    1{,}08 + \ju 0{,}12 \\
    0 \\
    0 \\
    0 \\
    0 \\
    0
   \end{pmatrix}
\]

Calculem ara la matriu $\mcmplx{Y}\ped{B}=\mcmplx{Z}\ped{B}^{-1}$ i el vector $\mcmplx{J}\ped{B} = \mcmplx{J}\ped{B}' + \mcmplx{Y}\ped{B} \mcmplx{E}\ped{B}'$  (tots els valors en p.u.):
\[
   \mcmplx{Y}\ped{B} = \ju \cdot
   \begin{pmatrix}
     -4 & 0 & 0 & 0 & 0 & 0 & 0 & 0 \\
     0 & -5 & 0 & 0 & 0 & 0 & 0 & 0 \\
     0 & 0 & -4 & 0 & 0 & 0 & 0 & 0 \\
     0 & 0 & 0 & -10 & 0 & 0 & 0 & 0 \\
     0 & 0 & 0 & 0 & -2{,}5 & 0{,}25 & 0 & 0 \\
     0 & 0 & 0 & 0 & 0{,}25 & -2{,}025 & 0 & 0 \\
     0 & 0 & 0 & 0 & 0 & 0 & -6{,}25 & 0 \\
     0 & 0 & 0 & 0 & 0 & 0 & 0 & 0{,}04
   \end{pmatrix}
   \qquad
   \mcmplx{J}\ped{B} =
   \begin{pmatrix}
    -\ju 4{,}4 \\
    0{,}5 - \ju 5{,}25 \\
    0{,}48 - \ju 4{,}32 \\
    0 \\
    0 \\
    0 \\
    0 \\
    2 - \ju 0{,}9
   \end{pmatrix}
\]

Continuem amb el c\`{a}lcul de la matriu $\mcmplx{Y}\ped{N} =
\boldsymbol{A} \mcmplx{Y}\ped{B} \transp{\boldsymbol{A}}$ i dels
vectors $\mcmplx{J}\ped{N} = - \boldsymbol{A} \mcmplx{J}\ped{B}$ i
$\mcmplx{V}\ped{N} = \mcmplx{Y}\ped{N}^{-1} \mcmplx{J}\ped{N}$ (tots
els valors en p.u.):
\[
   \mcmplx{Y}\ped{N} = \ju \cdot
   \begin{pmatrix}
     -10{,}25 & 6{,}25 & 0 & 0 \\
     6{,}25 & -15{,}275 & 1{,}775 & 2{,}25 \\
     0 & 1{,}775 & -16{,}025 & 10{,}25 \\
     0 & 2{,}25 & 10{,}25 & -12{,}46
   \end{pmatrix}
   \;\;
   \mcmplx{J}\ped{N} =
   \begin{pmatrix}
    -\ju 4{,}4 \\
    0{,}5 - \ju 5{,}25 \\
    0{,}48 - \ju 4{,}32 \\
    -2 + \ju 0{,}9
   \end{pmatrix}
   \;\;
   \mcmplx{V}\ped{N} =
   \left(\begin{array}{l}
    1{,}0494_{\measuredangle -1{,}4909\degree} \\
    1{,}0175_{\measuredangle -2{,}5224\degree} \\
    0{,}9727_{\measuredangle -10{,}3558\degree} \\
    0{,}9512_{\measuredangle -19{,}1752\degree}
   \end{array}\right)
\]

Finalment, calculem les tensions i els corrents de les branques,
mitjan\c{c}ant els vectors $\mcmplx{U}\ped{B} = \transp{\boldsymbol{A}}
\mcmplx{V}\ped{N}$ i $\mcmplx{I}\ped{B} =  \mcmplx{Y}\ped{B}
\mcmplx{U}\ped{B} + \mcmplx{J}\ped{B}$ (tots els valors en p.u.):
\[
   \mcmplx{U}\ped{B} =
   \left(\begin{array}{l}
     1{,}0494_{\measuredangle 178{,}5091\degree} \\
     1{,}0175_{\measuredangle 177{,}4776\degree} \\
     0{,}9727_{\measuredangle 169{,}6442\degree} \\
     0{,}1495_{\measuredangle 67{,}0039\degree} \\
     0{,}2925_{\measuredangle 66{,}2049\degree} \\
     0{,}1431_{\measuredangle 65{,}3702\degree} \\
     0{,}0370_{\measuredangle 28{,}1937\degree} \\
     0{,}9512_{\measuredangle -19{,}1752\degree}
   \end{array}\right)
   \qquad
   \mcmplx{I}\ped{B} =
   \left(\begin{array}{l}
     0{,}2312_{\measuredangle -61{,}8063\degree} \\
     0{,}7431_{\measuredangle -13{,}0406\degree} \\
     1{,}2782_{\measuredangle -22{,}6715\degree} \\
     1{,}4946_{\measuredangle -22{,}9961\degree} \\
     0{,}6955_{\measuredangle -23{,}7522\degree} \\
     0{,}2166_{\measuredangle -24{,}9115\degree} \\
     0{,}2312_{\measuredangle -61{,}8063\degree} \\
     2{,}1901_{\measuredangle -23{,}2362\degree}
   \end{array}\right)
\]

Calcularem ara la pot\`{e}ncia cedida pels tres generadors G1, G2 i G3, i la pot\`{e}ncia absorbida per la c\`{a}rrega Q8. Cal tenir en compte que per calcular les pot\`{e}ncies cedides per generadors, els vectors que representen la for\c{c}a electromotriu del generador i el corrent que travessa el generador han de tenir el mateix sentit de refer\`{e}ncia; aix\'{\i} mateix, per calcular les pot\`{e}ncies absorbides per c\`{a}rregues, els vectors que representen la caiguda de tensi\'{o} a la c\`{a}rrega i el corrent que travessa la c\`{a}rrega han de tenir el mateix sentit de refer\`{e}ncia. Amb aquestes consideracions, tenim:
\begin{alignat*}{3}
   \cmplx{s}\ped{G1} &= \cmplx{e}\ped{1} \, \mcmplx{I}\ped{B}^*(1) &&= 1{,}1 \cdot
    0{,}2312_{\measuredangle 61{,}8063\degree} &&= 0{,}1201 + \ju 0{,}2241 \unit{p.u.} \\
   \cmplx{s}\ped{G2} &= \cmplx{e}\ped{2} \, \mcmplx{I}\ped{B}^*(2) &&=
   (1{,}05 + \ju 0{,}10) \cdot 0{,}7431_{\measuredangle 13{,}0406\degree} &&=
   0{,}7433 + \ju 0{,}2484\unit{p.u.}   \\
   \cmplx{s}\ped{G3} &= \cmplx{e}\ped{3} \, \mcmplx{I}\ped{B}^*(3) &&=
   (1{,}08 + \ju 0{,}12) \cdot 1{,}2782_{\measuredangle 22{,}6715\degree} &&=
   1{,}2146 + \ju 0{,}6736\unit{p.u.}   \\
   \cmplx{s}\ped{Q8} &= \mcmplx{U}\ped{B}(8) \, \mcmplx{I}\ped{B}^*(8) &&=
   0{,}9512_{\measuredangle -19{,}1752\degree} \cdot 2{,}1901_{\measuredangle          23{,}2362\degree}     &&= 2{,}0780 + \ju 0{,}1475\unit{p.u.}
\end{alignat*}

La difer\`{e}ncia entre les pot\`{e}ncies generades i la pot\`{e}ncia absorbida, \'{e}s la pot\`{e}ncia perduda en la resta de components de la xarxa:
\[
   \cmplx{s}\ped{G1} + \cmplx{s}\ped{G2} + \cmplx{s}\ped{G3} -
   \cmplx{s}\ped{Q8} = \ju 0{,}9986\unit{p.u.}
\]
\end{exemple}


\section{M\`{e}tode particular de resoluci\'{o} sense acoblaments magn\`{e}tics}
\index{metode@m\`{e}tode dels nusos!cas particular!sense acoblaments
magn\`{e}tics}

Quan no hi ha acoblaments magn\`{e}tics entre branques de la xarxa, la matriu $\mcmplx{Y}\ped{N}\{n \times n\}$ i el vector $\mcmplx{J}\ped{N}\{n\}$ es poden formar de manera directa, a partir dels components de la xarxa i del seu graf orientat associat.

Per ajudar-nos en l'explicaci\'{o} d'aquest m\`{e}tode simplificat, farem \'{u}s
del mateix exemple de la Figura \vref{pic:metode_nusos}, per\`{o}
suposant que no hi ha acoblament magn\`{e}tic entre les branques 2 i 3
($\cmplx{X}\ped{M}=0$).

La matriu $\mcmplx{Y}\ped{N}\{n \times n\}$ i el vector $\mcmplx{J}\ped{N}\{n\}$ es formen tal com es descriu a continuaci\'{o}:

\begin{list}{}
   {\setlength{\labelwidth}{20mm} \setlength{\leftmargin}{22mm} \setlength{\labelsep}{2mm}}

   \item[$\mcmplx{Y}\ped{N}\{n \times n\}$:] \index{matriu!d'admit\`{a}ncies de nus $\mcmplx{Y}\ped{N}$}Matriu d'admit\`{a}ncies de nus. Els elements de la diagonal estan formats per la suma de les admit\`{a}ncies de les branques que incideixen en cada nus.
   Els elements de fora de la diagonal estan formats per la suma, canviada de signe, de les admit\`{a}ncies de les branques que estan connectades entre cada parella de nusos.

   En el nostre exemple tenim:
   \[
      \mcmplx{Y}\ped{N} =
      \begin{pmatrix}
            \frac{1}{20} + \frac{1}{\ju 20} +  \frac{1}{10} &
            -\left[\frac{1}{20} + \frac{1}{\ju 20}\right] \\
            -\left[\frac{1}{20} + \frac{1}{\ju 20}\right]  &
            \frac{1}{20} + \frac{1}{\ju 20} +  \frac{1}{\ju 5} + \frac{1}{10}
      \end{pmatrix} =
      \frac{1}{20} \cdot \begin{pmatrix}
            3 - \ju  & -1 + \ju \\ -1 + \ju & 3 - \ju 5
      \end{pmatrix}\unit{S}
   \]

   \item[$\mcmplx{J}\ped{N}\{n\}$:] \index{vector!d'intensivitats de nus $\mcmplx{J}\ped{N}$}Vector
d'intensivitats de nus. Cada element d'aquest vector est\`{a} format per la suma de
les intensivitats, degudes a les fonts de corrent i a les fonts de tensi\'{o}, de les
branques que incideixen en cada nus; el signe de cada intensivitat \'{e}s positiu si el
corrent va cap al nus, i negatiu si se n'allunya. Les fonts de tensi\'{o} han de
transformar-se en fonts de corrent, utilitzant l'equaci\'{o} \eqref{eq:Thevenin-Norton} de
la p\`{a}gina \pageref{eq:Thevenin-Norton}.

   En el nostre exemple tenim:
   \[
      \mcmplx{J}\ped{N} =
      \begin{pmatrix}
            \frac{50}{\ju 20} +  \frac{200}{10} \\
            - \frac{50}{\ju 20} + 4
      \end{pmatrix} =
      \frac{1}{2} \cdot \begin{pmatrix}
            40 - \ju 5 \\
            8 + \ju 5
      \end{pmatrix}\unit{A}
   \]

\end{list}

\index{vector!de potencials de nus $\mcmplx{V}\ped{N}$}Finalment,
trobem el vector de potencials de nus $\mcmplx{V}\ped{N}\{n\}$, tal
com  hem fet en l'apartat anterior, aplicant l'equaci\'{o} \eqref{eq:vn}

En el nostre exemple tenim:
\[
   \mcmplx{V}\ped{N} =
   20 \cdot \begin{pmatrix}
         3 - \ju  & -1 + \ju \\ -1 + \ju & 3 - \ju 5
   \end{pmatrix} ^{-1} \cdot
   \frac{1}{2} \cdot \begin{pmatrix}
         40 - \ju 5 \\
         8 + \ju 5
   \end{pmatrix} =
   \frac{1}{17} \cdot \begin{pmatrix}
         2450 + \ju 535 \\ 540  + \ju 545
   \end{pmatrix}\unit{V}
\]

Si volem trobar ara de forma sistem\`{a}tica, totes les tensions i tots
els corrents i  de les branques de la xarxa, haurem d'utilitzar
l'equaci\'{o} \eqref{eq:ur} de la p\`{a}gina \pageref{eq:ur} i l'equaci\'{o}
\eqref{eq:ir} de la p\`{a}gina \pageref{eq:ir}; aix\`{o} vol dir que haurem
de formar les matrius $\boldsymbol{A}$ i $\mcmplx{Y}\ped{B}$ i el
vector $\mcmplx{J}\ped{B}$. No obstant, si \'{u}nicament estem
interessats en algun corrent o en alguna tensi\'{o} de branca, podem
resoldre el problema aplicant les lleis de Kirchhoff a les branques
que ens interessin.

\begin{exemple}
A partir del circuit de la Figura \vref{pic:metode_nusos}, amb
$\cmplx{X}\ped{M}=0$, es tracta de trobar els corrents que circulen
per les branques 2 i 5.

Partint dels potencials dels nusos 1 i 2 calculats anteriorment, i
aplicant les lleis de Kirchhoff a les branques 2 i 5 tenim:
\begin{align*}
   \cmplx{I}_2 &= \frac{-\cmplx{E}_2 + [\mcmplx{V}\ped{N}(1) - \mcmplx{V}\ped{N}(2)]}
                  {\cmplx{X}_2} = \frac{-50 + \frac{2450+\ju 535 -540
                  - \ju 545}{17}} {\ju 20} = \frac{-1 - \ju 106}{34}\unit{A} \\[1.5ex]
   \cmplx{I}_5 &=  \frac{- \mcmplx{V}\ped{N}(2)}{R_5}  + \cmplx{J}_5 =
                  \frac{\frac{-540 - \ju 545}{17}}{10} + 4 =
                  \frac{28 - \ju 109}{34}\unit{A}
\end{align*}

\end{exemple}

\section{M\`{e}tode particular de resoluci\'{o} amb acoblaments magn\`{e}tics}
\index{metode@m\`{e}tode dels nusos!cas particular!amb acoblaments
magn\`{e}tics}

Quan hi ha acoblaments magn\`{e}tics entre branques de la xarxa que no
tenen cap font de tensi\'{o} o de corrent, tamb\'{e} es pot aplicar el
m\`{e}tode de resoluci\'{o} descrit en l'apartat anterior, substituint
pr\`{e}viament les dues branques acoblades per un circuit equivalent
d'admit\`{a}ncies, segons es veur\`{a} a continuaci\'{o}.

Un cop obtingut el circuit equivalent de les dues branques acoblades
magn\`{e}ticament, ja es pot formar la matriu $\mcmplx{Y}\ped{N}\{n \times n\}$ i
el vector $\mcmplx{J}\ped{N}\{n\}$, i  resoldre la xarxa, tal com s'ha fet en
l'apartat anterior.

En la Figura \vref{pic:equiv_acobl} es pot veure aquest circuit
equivalent.
\begin{figure}[htb]
\centering
    %PsTricks content-type (pstricks.sty package needed)
    %Add \usepackage{pstricks} in the preamble of your LaTeX file
    \psset{xunit=1mm,yunit=1mm,runit=1mm}
    \psset{linewidth=0.3,dotsep=1,hatchwidth=0.3,hatchsep=1.5,shadowsize=1}
    \psset{dotsize=0.7 2.5,dotscale=1 1,fillcolor=black}
    \begin{pspicture}(0,0)(153.00,45.00)
    \rput[r](86,20){$\cmplx{Y}_{\alpha\gamma}$}
    \rput[b](112,40){$\cmplx{Y}_{\alpha\beta}$}
    \rput[b](112,29){$\cmplx{Y}_{\alpha\delta}$}
    \rput[b](112,17){$\cmplx{Y}_{\gamma\beta}$}
    \rput[b](112,6){$\cmplx{Y}_{\gamma\delta}$}
    \rput[b](25,6){$\cmplx{Z}_{\gamma\delta}$}
    \rput[b](11,6){$\cmplx{I}_{\gamma\delta}$}
    \rput[b](82,6){$\cmplx{I}_{\gamma\delta}$}
    \rput[b](25,40){$\cmplx{Z}_{\alpha\beta}$}
    \rput[b](11,40){$\cmplx{I}_{\alpha\beta}$}
    \rput[b](82,40){$\cmplx{I}_{\alpha\beta}$}
    \rput[l](20,20){$\cmplx{X}\ped{M}$} \rput[b](41,39){$\beta$}
    \rput[b](151,39){$\beta$} \rput[b](41,5){$\delta$}
    \rput[b](151,5){$\delta$} \rput[b](3,39){$\alpha$}
    \rput[b](74,39){$\alpha$} \rput[b](3,5){$\gamma$}
    \rput[b](74,5){$\gamma$} \rput[l](145,20){$\cmplx{Y}_{\beta\delta}$}
    \rput(112,37){}
    \pspolygon[linewidth=0.25](107,35)(117,35)(117,39)(107,39)
    \rput(25,37){}
    \pspolygon[linewidth=0.25](20,35)(30,35)(30,39)(20,39)
    \rput(112,26){}
    \pspolygon[linewidth=0.25](107,24)(117,24)(117,28)(107,28)
    \rput(112,14){}
    \pspolygon[linewidth=0.25](107,12)(117,12)(117,16)(107,16)
    \rput(112,3){}
    \pspolygon[linewidth=0.25](107,1)(117,1)(117,5)(107,5) \rput(25,3){}
    \pspolygon[linewidth=0.25](20,1)(30,1)(30,5)(20,5) \rput(89,20){}
    \pspolygon[linewidth=0.25](87,15)(91,15)(91,25)(87,25)
    \rput(142,20){}
    \pspolygon[linewidth=0.25](140,15)(144,15)(144,25)(140,25)
    \psline[linewidth=0.25](142,25)(142,37)
    \psline[linewidth=0.25](89,25)(89,37)
    \psline[linewidth=0.25](89,15)(89,3)
    \psline[linewidth=0.25](142,15)(142,3)
    \psline[linewidth=0.25](117,3)(150,3)
    \psline[linewidth=0.25](117,37)(150,37)
    \rput{0}(151,37){\psellipse[linewidth=0.25](0,0)(1,1)}
    \rput{0}(151,3){\psellipse[linewidth=0.25](0,0)(1,1)}
    \rput{0}(74,37){\psellipse[linewidth=0.25](0,0)(1,1)}
    \rput{0}(74,3){\psellipse[linewidth=0.25](0,0)(1,1)}
    \psline[linewidth=0.25](30,37)(40,37)
    \psline[linewidth=0.25](30,3)(40,3)
    \rput{0}(3,37){\psellipse[linewidth=0.25](0,0)(1,1)}
    \rput{0}(41,37){\psellipse[linewidth=0.25](0,0)(1,1)}
    \rput{0}(3,3){\psellipse[linewidth=0.25](0,0)(1,1)}
    \rput{0}(41,3){\psellipse[linewidth=0.25](0,0)(1,1)}
    \rput{0}(19,33){\psellipse[linewidth=0.25,fillstyle=solid](0,0)(1,1)}
    \rput{0}(19,7){\psellipse[linewidth=0.25,fillstyle=solid](0,0)(1,1)}
    \psline[linewidth=0.25](4,37)(20,37)
    \psline[linewidth=0.25](4,3)(20,3)
    \psline[linewidth=0.25]{->}(7,5)(15,5)
    \psline[linewidth=0.25]{->}(78,5)(86,5)
    \psline[linewidth=0.25]{->}(7,39)(15,39)
    \psline[linewidth=0.25]{->}(78,39)(86,39)
    \psline[linewidth=0.25](107,37)(75,37)
    \psline[linewidth=0.25](107,3)(75,3)
    \rput{0}(89,37){\psellipse[linewidth=0.15,fillstyle=solid](0,0)(0.5,0.5)}
    \psline[linewidth=0.25](89,37)(98,26) (98,26)(107,26)
    \rput{0}(89,3){\psellipse[linewidth=0.25,fillstyle=solid](0,0)(0.5,0.5)}
    \psline[linewidth=0.25](89,3)(99,14) (99,14)(107,14)
    \psline[linewidth=0.25](117,26)(126.5,26) (126.5,26)(142,3)
    \rput{0}(142,37){\psellipse[linewidth=0.15,fillstyle=solid](0,0)(0.5,0.5)}
    \rput{0}(142,3){\psellipse[linewidth=0.15,fillstyle=solid](0,0)(0.5,0.5)}
    \psline[linewidth=0.25](117,14)(127,14) (127,14)(142,37)
    \psline[linewidth=0.25,linestyle=dashed,dash=1 1]{<->}(19,31)(19,9)
    \rput(56,20){$\equiv$} \psline[linewidth=0.25](20,39)(30,35)
    \psline[linewidth=0.25](30,39)(20,35)
    \psline[linewidth=0.25](20,5)(30,1)
    \psline[linewidth=0.25](30,5)(20,1)
    \psline[linewidth=0.25](87,25)(91,15)
    \psline[linewidth=0.25](91,25)(87,15)
    \psline[linewidth=0.25](107,39)(117,35)
    \psline[linewidth=0.25](117,39)(107,35)
    \psline[linewidth=0.25](107,28)(117,24)
    \psline[linewidth=0.25](117,28)(107,24)
    \psline[linewidth=0.25](107,16)(117,12)
    \psline[linewidth=0.25](117,16)(107,12)
    \psline[linewidth=0.25](107,5)(117,1)
    \psline[linewidth=0.25](117,5)(107,1)
    \psline[linewidth=0.25](140,25)(144,15)
    \psline[linewidth=0.25](144,25)(140,15)
    \end{pspicture}
   \caption{Circuit equivalent de dues branques acoblades magn\`{e}ticament} \label{pic:equiv_acobl}
\end{figure}

Els valors de les admit\`{a}ncies d'aquest circuit equivalent
s\'{o}n:\index{acoblament magn\`{e}tic!circuit equivalent}

\parbox{15cm}
{ \begin{align*}
   \cmplx{Y}_{\alpha\beta} &= \frac{\cmplx{Z}_{\gamma\delta}}{\cmplx{Z}_{\alpha\beta}\, \cmplx{Z}_{\gamma\delta}-\cmplx{X}\ped{M}^2} &
   \cmplx{Y}_{\alpha\gamma} &= \cmplx{Y}_{\beta\delta} = \frac{\cmplx{X}\ped{M}}{\cmplx{Z}_{\alpha\beta}\, \cmplx{Z}_{\gamma\delta}-\cmplx{X}\ped{M}^2} \\[1.5ex]
   \cmplx{Y}_{\gamma\delta} &= \frac{\cmplx{Z}_{\alpha\beta}}{\cmplx{Z}_{\alpha\beta}\, \cmplx{Z}_{\gamma\delta}-\cmplx{X}\ped{M}^2} &
   \cmplx{Y}_{\alpha\delta} &= \cmplx{Y}_{\gamma\beta} = \frac{-\cmplx{X}\ped{M}}{\cmplx{Z}_{\alpha\beta}\, \cmplx{Z}_{\gamma\delta}-\cmplx{X}\ped{M}^2}
\end{align*} }
\hfill
\parbox{1cm}{\begin{align}\end{align}}

Un cop hem resolt la xarxa i hem trobat els potencials dels quatre
nusos $\mcmplx{V}\ped{N}(\alpha)$, $\mcmplx{V}\ped{N}(\beta)$,
$\mcmplx{V}\ped{N}(\gamma)$ i $\mcmplx{V}\ped{N}(\delta)$, podem
trobar els dos corrents $\cmplx{I}_{\alpha\beta}$ i
$\cmplx{I}_{\gamma\delta}$, a partir de les expressions seg\"{u}ents:
\begin{subequations}
\begin{align}
    \cmplx{I}_{\alpha\beta} &=  \frac{[\mcmplx{V}\ped{N}(\alpha) - \mcmplx{V}\ped{N}(\beta)] \, \cmplx{Z}_{\gamma\delta} - [\mcmplx{V}\ped{N}(\gamma) - \mcmplx{V}\ped{N}(\delta)] \,
    \cmplx{X}\ped{M}}{\cmplx{Z}_{\alpha\beta}\,
    \cmplx{Z}_{\gamma\delta}-\cmplx{X}\ped{M}^2} \label{eq:i_ab}
    \\[1.5ex]
    \cmplx{I}_{\gamma\delta} &= \frac{[\mcmplx{V}\ped{N}(\gamma) - \mcmplx{V}\ped{N}(\delta)] \, \cmplx{Z}_{\alpha\beta} - [\mcmplx{V}\ped{N}(\alpha) - \mcmplx{V}\ped{N}(\beta)] \,
    \cmplx{X}\ped{M}}{\cmplx{Z}_{\alpha\beta}\,
    \cmplx{Z}_{\gamma\delta}-\cmplx{X}\ped{M}^2} \label{eq:i_gd}
\end{align}
\end{subequations}

El cas que hem vist fins ara, \'{e}s el m\'{e}s general de tots els
possibles, ja que suposa que els quatre nusos $\alpha$, $\beta$,
$\gamma$ i $\delta$ s\'{o}n diferents entre si. Es pot presentar el cas,
no obstant, on dos nusos siguin en realitat el mateix, en tenir les
dues branques acoblades magn\`{e}ticament, un extrem connectat al mateix
nus; en aquest cas el circuit equivalent resultant es pot derivar
del corresponent al cas general de forma senzilla.

Si suposem, per exemple, que les dues branques de la Figura
\vref{pic:equiv_acobl} estiguessin unides pels extrems de la dreta,
\'{e}s a dir $\beta\equiv\delta$, en aquest cas l'admit\`{a}ncia entre
$\alpha$ i $\gamma$ seria $\cmplx{Y}_{\alpha\gamma}$, l'admit\`{a}ncia
entre $\beta$ i $\delta$ desapareixeria, l'admit\`{a}ncia entre $\alpha$
i $\beta$ seria $\cmplx{Y}_{\alpha\beta} +
\cmplx{Y}_{\alpha\delta}$, i finalment, l'admit\`{a}ncia entre $\gamma$
i $\beta$ seria $\cmplx{Y}_{\gamma\beta} +
\cmplx{Y}_{\gamma\delta}$. Els corrents $\cmplx{I}_{\alpha\beta}$ i
$\cmplx{I}_{\gamma\delta}$, es calcularien tamb\'{e} amb les equacions
\eqref{eq:i_ab} i \eqref{eq:i_gd}, tenint en compte que
$\mcmplx{V}\ped{N}(\beta)\equiv\mcmplx{V}\ped{N}(\delta)$.

\section{Circuits equivalents Th\'{e}venin i Norton} \index{teorema!de Th\'{e}venin} \index{teorema!de Norton} \index{metode@m\`{e}tode dels nusos!circuits equivalents Th\'{e}venin i Norton}

\index{matriu!d'imped\`{a}ncies de nus $\mcmplx{Z}\ped{N}$}Per trobar el
circuit equivalent Th\'{e}venin o Norton entre dos nusos qualssevol
d'una xarxa, ens cal el vector de potencials de nus
$\mcmplx{V}\ped{N}\{n\}$, obtingut segons s'ha descrit en els
apartats anteriors, i la matriu d'imped\`{a}ncies de nus
$\mcmplx{Z}\ped{N}\{n\times n\}$; aquesta matriu est\`{a} definida per
la relaci\'{o} seg\"{u}ent:
\begin{equation}
   \mcmplx{Z}\ped{N} = \mcmplx{Y}\ped{N}^{-1}
\end{equation}

A partir del vector $\mcmplx{V}\ped{N}$ i de la matriu
$\mcmplx{Z}\ped{N}$, podem trobar la font de tensi\'{o} i la imped\`{a}ncia
Th\'{e}venin equivalents entre dos nusos qualssevol.

La tensi\'{o} Th\'{e}venin $\cmplx{E}\ped{Th}^{(\alpha,0)}$ i la imped\`{a}ncia
Th\'{e}venin $\cmplx{Z}\ped{Th}^{(\alpha,0)}$, entre  un nus qualsevol
$\alpha$ i el nus de refer\`{e}ncia 0, s'obtenen amb les equacions
seg\"{u}ents:
\begin{align}
    \cmplx{E}\ped{Th}^{(\alpha,0)} &= \mcmplx{V}\ped{N}(\alpha) \\
    \cmplx{Z}\ped{Th}^{(\alpha,0)} &= \mcmplx{Z}\ped{N}(\alpha,\alpha)
\end{align}

La tensi\'{o} Th\'{e}venin $\cmplx{E}\ped{Th}^{(\alpha,\beta)}$ i la
imped\`{a}ncia Th\'{e}venin $\cmplx{Z}\ped{Th}^{(\alpha,\beta)}$, entre dos
nusos qualssevol $\alpha$ i $\beta$, s'obtenen amb les equacions
seg\"{u}ents:
\begin{align}
    \cmplx{E}\ped{Th}^{(\alpha,\beta)} &= \mcmplx{V}\ped{N}(\alpha) - \mcmplx{V}\ped{N}(\beta) \\
    \cmplx{Z}\ped{Th}^{(\alpha,\beta)} &= \mcmplx{Z}\ped{N}(\alpha,\alpha) +
    \mcmplx{Z}\ped{N}(\beta,\beta) - \mcmplx{Z}\ped{N}(\alpha,\beta) -
    \mcmplx{Z}\ped{N}(\beta,\alpha)
\end{align}

A partir d'aquests valors podem calcular els valors del circuit Norton equivalent, utilitzant l'equaci\'{o} \eqref{eq:Thevenin-Norton} de la p\`{a}gina \pageref{eq:Thevenin-Norton}.

\begin{exemple}
Continuant amb el circuit de la Figura \vref{pic:metode_nusos}, es
tracta de trobar els circuits Th\'{e}venin i Norton equivalents de la
xarxa, entre els nusos 1 i 2.

El vector $\mcmplx{V}\ped{N}$ \'{e}s el calculat a la p\`{a}gina \pageref{eq:vn_exemp}.

Trobem a continuaci\'{o} la matriu $\mcmplx{Z}\ped{N}$, a partir de la matriu $\mcmplx{Y}\ped{N}$
calculada a la p\`{a}gina \pageref{eq:yn}:
\[
   \mcmplx{Z}\ped{N} =
   60 \cdot \begin{pmatrix}
            9 - \ju 4 & -3 + \ju 8 \\
            3 + \ju 8 & 9 - \ju 28
      \end{pmatrix} ^{-1} =
   \frac{1}{202} \cdot \begin{pmatrix}
         1445 + \ju 310 & 415 + \ju 110 \\
         415 + \ju 110 & 245 + \ju 430
   \end{pmatrix}\unit{\ohm}
\]

Els valors del circuit Th\'{e}venin equivalent que busquem s\'{o}n:
\begin{align*}
   \cmplx{E}\ped{Th}^{(1,2)} &= \frac{15430 + \ju 2295}{101} - \frac{3390 + \ju 2085}{101} =
   \frac{12040 + \ju 210}{101}\unit{V} \\[2ex]
   \cmplx{Z}\ped{Th}^{(1,2)} &= \frac{1445 + \ju 310}{202} + \frac{245 + \ju 430}{202} -
   2\cdot\frac{415 + \ju 110}{202} = \frac{430 + \ju 260}{101}\unit{\ohm}
\end{align*}

Els valors del circuit Norton equivalent que busquem s\'{o}n:
\begin{align*}
   \cmplx{J}\ped{No}^{(1,2)} &= \frac{\cmplx{E}\ped{Th}^{(1,2)}}{\cmplx{Z}\ped{Th}^{(1,2)}} =
   \frac{\frac{12040 + \ju 210}{101}\unit{V}}{\frac{430 + \ju 260}{101}\unit{\ohm}} =
   \frac{518 - \ju 301}{25}\unit{A} \\[2ex]
   \cmplx{Y}\ped{No}^{(1,2)} &= \frac{1}{\cmplx{Z}\ped{Th}^{(1,2)}} =
   \frac{1}{\frac{430 + \ju 260}{101}\unit{\ohm}} = \frac{43 - \ju 26}{250}\unit{S}
\end{align*}

\end{exemple}
