\chapter{Alfabet Grec} \index{alfabet grec}\label{sec:alf-grec}

 En la taula \vref{taula:alfabet-grec} es pot veure l'alfabet grec
 amb els noms de les seves lletres en diversos idiomes.

\begin{center}
   \captionof{table}{Alfabet grec} \label{taula:alfabet-grec}
   \begin{tabular}{cccllll}
   \toprule[1pt]
   \renewcommand*{\multirowsetup}{\centering}
   \multirow{2}{15mm}{\rule{0mm}{4.5mm}Número\\d'ordre} & \multicolumn{2}{c}{Lletra} &
   \multicolumn{4}{c}{Nom} \\
   \cmidrule(rl){2-3} \cmidrule(rl){4-7}
    & minúscula & majúscula & català & castellà &  anglès & francès\\
   \midrule
   1  & $\alphaup$ & A & alfa & alfa &  alpha & alpha\\
   2  & $\betaup$ & B & beta & beta &  beta & bêta\\
   3  & $\gammaup$ & $\Gammaup$ & gamma & gamma &  gamma & gamma\\
   4  & $\deltaup$ & $\Deltaup$ & delta & delta &  delta & delta\\
   5  & $\epsilonup$, $\varepsilonup$ & E & èpsilon & épsilon &  epsilon & epsilon\\
   6  & $\zetaup$ & Z & zeta & dseta &  zeta & zêta\\
   7  & $\etaup$ & H & eta & eta &  eta & êta\\
   8  & $\thetaup$, $\varthetaup$ & $\Thetaup$ & theta & zeta &  theta & thêta\\
   9  & $\iotaup$ & I & iota & iota &  iota & iota\\
   10 & $\kappaup$, $\varkappaup$ & K & kappa & kappa &  kappa & kappa\\
   11 & $\lambdaup$ & $\Lambdaup$ & lambda & lambda &  lambda &lambda\\
   12 & $\muup$ & M & mi & mi &  mu & mu\\
   13 & $\nuup$ & N & ni & ni &  nu & nu\\
   14 & $\xiup$ & $\Xiup$ & ksi & xi &  xi & ksi, xi\\
   15 & o & O & òmicron & ómicron &  omicron & omicron\\
   16 & $\piup$, $\varpiup$ & $\Piup$ & pi & pi &  pi & pi\\
   17 & $\rhoup$, $\varrhoup$ & P & rho, ro & ro &  rho & rhô\\
   18 & $\sigmaup$, $\varsigmaup$ & $\Sigmaup$ & sigma & sigma &  sigma &sigma\\
   19 & $\tauup$ & T & tau & tau & tau &tau\\
   20 & $\upsilonup$ & $\Upsilonup$ & ípsilon & ípsilon &  upsilon &upsilon\\
   21 & $\phiup$, $\varphiup$ & $\Phiup$ & fi & fi &  phi & phi\\
   22 & $\chiup$ & X & khi & ji &  chi & khi\\
   23 & $\psiup$ & $\Psiup$ & psi & psi &  psi & psi\\
   24 & $\omegaup$ & $\Omegaup$ & omega & omega &  omega & oméga\\
   \bottomrule[1pt]
   \end{tabular}
\end{center}
\pagebreak


Les dues grafies de la lletra minúscula èpsilon  ($\epsilonup,
\varepsilonup$) són totalment equivalents entre si; el mateix passa
amb les dues grafies de les lletres minúscules theta ($\thetaup,
\varthetaup$), kappa $(\kappaup, \varkappaup)$, rho ($\rhoup,\varrhoup$) i fi ($\phiup, \varphiup$).

La lletra sigma minúscula té dues variants: $\varsigmaup$, escrita en
grec al final d'una paraula, i $\sigmaup$, escrita en grec a l'inici o
en mig d'una paraula. En els textos tècnics i científics s'utilitza
majoritàriament la variant $\sigmaup$.

La variant $\varpiup$ de la lletra pi es denomina pi dòrica en
català, \textit{pi dórica} en castellà, \textit{dorian pi} en anglès i \textit{pi dorien} en francès.

Pel que fa als noms de les lletres, alguns poden sorprendre; això no
és estrany, ja que algunes lletres han rebut històricament noms
diversos, i fins i tot contradictoris respecte dels actuals.

Els noms anglesos de les lletres són els més uniformes, ja que no
s'ha observat cap variació en les diverses fonts consultades, essencialment el diccionari nord-americà Merriam-Webster\footnote{Aquest diccionari es pot consultar a l'adreça: \href{https://www.merriam-webster.com/}{www.merriam-webster.com}.} i els diccionaris britànics Oxford\footnote{Aquest diccionari es pot consultar a l'adreça: \href{http://www.oed.com/}{www.oed.com}.} i Cambridge.\footnote{Aquest diccionari es pot consultar a l'adreça: \href{http://dictionary.cambridge.org/}{dictionary.cambridge.org}.}

Els noms catalans de les lletres són els que apareixen en el \textit{Diccionari de la llengua catalana, 2a edició, 2007} (DIEC2).\footnote{Aquest diccionari es pot consultar a l'adreça: \href{http://dlc.iec.cat/}{dlc.iec.cat}.} Altres noms utilitzats en
les diverses fonts consultades, que actualment estan fora de la norma ortogràfica, són:
\begin{multicols}{3}
\begin{list}{}
   {\setlength{\labelwidth}{16mm} \setlength{\leftmargin}{16mm} \setlength{\labelsep}{2mm}}
   \item[B, $\betaup :$] vita.
   \item[Z, $\zetaup :$] zita.
   \item[H, $\etaup :$] ita.
   \item[$\Thetaup$, $\thetaup :$] thita.
   \item[T, $\tauup :$] taf.
   \item[$\xiup$, $\Xiup$:] csi.\footnote{El nom «csi» apareix juntament amb «ksi» en el \textit{Gran Diccionari de la Llengua Catalana (1999)}. Aquest diccionari es pot consultar a l'adreça:  \href{http://www.enciclopedia.cat/obra/diccionaris/gran-diccionari-de-la-llengua-catalana}{www.enciclopedia.cat/obra/diccionaris/gran-diccionari-de-la-llengua-catalana}.}
\end{list}
\end{multicols}

Els noms castellans de les lletres són els que apareixen en el \textit{Diccionario de la Lengua Española, 23ª
edición, 2014} (D.R.A.E.).\footnote{Aquest diccionari es pot consultar a l'adreça:  \href{http://www.rae.es/}{www.rae.es}.} Altres noms utilitzats en les diverses fonts
consultades, que actualment estan fora de la norma ortogràfica, són:
\begin{multicols}{3}
\begin{list}{}
   {\setlength{\labelwidth}{16mm} \setlength{\leftmargin}{16mm} \setlength{\labelsep}{2mm}}
   \item[Z, $\zetaup :$] zeta,\footnote{\label{fn:zeta}Els noms «zeta», «theta», «my» i «ny» eren els que apareixien en les edicions
   del D.R.A.E anteriors a la 21a (1992).} dseda,\footnote{El nom «dseda» era el que apareixia en l'edició 22a (2001) del D.R.A.E.} dzeta.
   \item[$\Thetaup$, $\thetaup :$] theta,\footnoteref{fn:zeta} thita.
   \item[K, $\kappaup :$] cappa.
   \item[M, $\muup :$] my,\footnoteref{fn:zeta} mu.
   \item[N, $\nuup :$] ny,\footnoteref{fn:zeta} nu.
   \item[O, o :] omicrón.
   \item[P, $\rhoup :$] rho.
   \item[$\Upsilonup$, $\upsilonup :$] úpsilon.
   \item[$\Phiup$, $\phiup :$] phi.
\end{list}
\end{multicols}

Els noms francesos de les lletres són els que apareixen en el \textit{Dictionnaire de l'Académie française, neuvième édition}.\footnote{Aquest diccionari es pot consultar a l'adreça: \href{http://www.academie-francaise.fr/le-dictionnaire/la-9e-edition}{www.academie-francaise.fr/le-dictionnaire/la-9e-edition}.} 