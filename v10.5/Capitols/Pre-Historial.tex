\chapter*{Historial}
\addcontentsline{toc}{chapter}{Historial}

Es presenta a continuació l'evolució que ha tingut aquest llibre en
les successives versions que han aparegut.

\section*{Versió 1.0 (8 de gener de 2005)}
\addcontentsline{toc}{section}{Versió 1.0}

Després de molts esforços, surt a la llum la primera versió d'aquest
llibre, format pels capítols 1, 2, 3, 4, 5, 6 i 7, i els apèndixs A,
B, C, D i E.

\section*{Versió 1.1 (8 de febrer de 2005)}
\addcontentsline{toc}{section}{Versió 1.1}

S'afegeix al llibre aquest apartat «Historial».

En l'apartat Notació, s'especifica que el mòdul d'un nombre
complex és igual a l'arrel quadrada \emph{positiva} de la suma dels
quadrats de les seves parts real i imaginària.

Es modifiquen les equacions (1.51) i (1.52).

S'amplia la secció corresponent a les diferències entre les
normatives CEI i IEEE, que fan referència als
transformadors de mesura i protecció (Secció 5.5).

Es revisa tot el text, fent-hi algunes petites modificacions i
correccions.

\section*{Versió 1.2 (16 d'abril de 2005)}
\addcontentsline{toc}{section}{Versió 1.2}

En l'apartat Notació, s'afegeix l'explicació de la convenció
seguida a l'hora de dibuixar les fletxes que representen les
tensions i els corrents.

S'afegeix l'apèndix F, on s'explica la designació de les classes de
refrigeració en els transformadors de potència.

\section*{Versió 1.3 (24 d'octubre de 2005)}
\addcontentsline{toc}{section}{Versió 1.3}

Els apèndixs A a F de la versió 1.2, es desplacen tres lletres cap
avall, passant a ser els apèndixs D a I respectivament.

S'afegeix un nou apèndix A dedicat a l'alfabet grec.

S'afegeix un nou apèndix B dedicat al sistema internacional
d'unitats (SI).

S'afegeix un nou apèndix C dedicat a les constants físiques.

En l'apartat Notació, s'amplien les definicions corresponents al
conjugat i al mòdul d'un nombre complex, i s'inclouen les
definicions de $\mcmplx{V}^*$ i $\hermit{V}$.

S'ha ampliat la secció 1.3, corresponent a la
potència complexa.

 S'ha ampliat l'exemple de la secció 3.2.

En la secció 3.3 s'ha afegit el càlcul de $R\ped{P}$ i
$\cmplx{Z}\ped{S}$.

 A l'hora de referir-se a la
relació de transformació d'un transformador, se substitueix el
símbol «$\ddot{u}$» emprat en les versions anteriors, pel símbol
«$m$».

\section*{Versió 1.4 (2 de desembre de 2005)}
\addcontentsline{toc}{section}{Versió 1.4}

Es representa correctament la Figura 1.7, ja que estava
tallada per la dreta.

Es corregeix l'equació (4.9a) i l'exemple que hi
ha a continuació, el qual en fa ús.

Es revisa tot el text, fent-hi algunes correccions.

\section*{Versió 2.0 (3 d'agost de 2006)}
\addcontentsline{toc}{section}{Versió 2.0}

S'ha modificat el criteri de colors utilitzat, a l'hora de ressaltar
els enllaços interns del document (equacions, pàgines, etc.) i els
enllaços externs; ara els enllaços interns són de
\textcolor{red}{color vermell} i els enllaços externs són de
\textcolor{magenta}{color magenta}. A més, tots els encapçalaments
de capítols, seccions,
 subseccions, taules  i figures, són ara de
 \textcolor{NavyBlue}{color blau}.

S'han afegit nous capítols i s'ha fet una reordenació que afecta a
diversos capítols i apèndixs, segons es detalla a continuació:
\begin{itemize}
   \item Els capítols 1 i 2  de la versió 1.4 mantenen la seva posició.
   \item S'afegeix un nou capítol 3 dedicat a les sèries de Fourier.
   \item S'afegeix un nou capítol 4 dedicat a la transformada de Laplace.
   \item El capítol 3 de la versió 1.4 es desplaça dos números cap
    avall, passant a ser el capítol 5.
   \item L'apèndix E de la versió 1.4 es converteix en el capítol 6.
   \item Els capítols 4, 5, 6 i 7  de la versió 1.4 es desplacen tres números cap
    avall, passant a ser els capítols 7, 8, 9 i 10 respectivament.
    \item L'apèndix G de la versió 1.4 es converteix en el capítol 11.
    \item Els apèndixs A, B, C i D de la versió 1.4 mantenen la seva posició.
    \item S'afegeix un nou apèndix E dedicat a les relacions trigonomètriques.
    \item L'apèndix F de la versió 1.4 manté la seva posició.
    \item Els apèndixs H i I de la versió 1.4 es desplacen una lletra cap
    amunt, passant a ser els apèndixs G i H respectivament.
\end{itemize}


 A l'hora de referir-se a la font de corrent i a l'admitància d'un circuit equivalent
 Norton, se substitueix el subíndex «Th» emprat en les versions
anteriors, pel subíndex «No».

En l'apartat Notació s'afegeixen els símbols: $\mathbb{N}$,
$\mathbb{Z}$, $\mathbb{Z}^+$,  $\mathbb{Z}^*$, $\mathbb{Z}^-$,
$\mathbb{Q}$, $\mathbb{R}$, $\mathbb{R}^+$, $\mathbb{R}^-$ i
$\mathbb{C}$.

S'ha afegit el teorema de la superposició en la secció 1.1.


S'ha afegit la bateria en la secció 1.2, com a un
dels components elementals d'un circuit elèctric.

S'ha afegit la secció 1.4, on es defineixen els
valors mitjà i eficaç, i els factors d'amplitud, de forma i
d'arrissada.

S'ha afegit la secció 1.5 dedicada als
circuits divisors de tensió i divisors de corrent.

 S'ha modificat l'equació (7.2),
i les taules 7.1 i 7.5.

S'ha afegit la secció 8.6, on s'explica com
connectar correctament transformadors de corrent i de tensió, a
aparells de mesura i de protecció.

S'ha millorat l'explicació de la secció 10.5.

S'ha reestructurat la taula B.2.

\section*{Versió 2.1 (2 de gener de 2007)}
\addcontentsline{toc}{section}{Versió 2.1}

S'adopta la compaginació moderna dels paràgrafs en tot el llibre, consistent en separar-los per una línia en blanc i en no entrar la primera línia de text.

S'unifica la representació de les fonts de corrent: un cercle amb una fletxa a dins.

S'afegeix una nota a peu de pàgina en la secció 1.1.1, relacionant aquesta secció amb la secció 9.5.

Es millora l'explicació de la secció 1.6, a l'hora que es trasllada de lloc (en les versions anteriors formava part del capítol 5).

Es millora l'explicació de la secció 2.4.

S'afegeix una nota a peu de pàgina en la secció 5.2, relacionant aquesta secció amb el capítol 10.

S'amplia la descripció de l'equació (7.25).

S'afegeix la secció 10.6, on s'explica com resoldre sistemes d'equacions no lineals amb els programes \textit{Mathematica}${}^\circledR$ i \textit{MATLAB}${}^\circledR$.

Es millora l'explicació de la secció E.2, modificant la figura E.1 i numerant l'equació de la llei dels sinus.

\section*{Versió 2.2 (10 de març de 2008)}
\addcontentsline{toc}{section}{Versió 2.2}

Es canvia el color dels enllaços interns, passant a ser de color negre com el text.

S'afegeixen les unitats que mancaven en alguns exemples.

En la secció 7.4.1, s'introdueixen les unitats cmil i kcmil, equivalents a les unitats CM i MCM respectivament; avui en dia és més freqüent veure escrit cmil i kcmil.

Es revisa l'apèndix B utilitzant les publicacions de l'any 2006 del «Bureau
International des Poids et Mesures» (BIPM).\index{BIPM}

Es revisa l'apèndix C utilitzant les publicacions de l'any 2006 del «Committee on Data for Science and Technology» (CODATA).\index{CODATA}

\section*{Versió 3.0 (1 d'octubre de 2008)}
\addcontentsline{toc}{section}{Versió 3.0}

Els capítols 9, 10 i 11 de la versió 2.2 es desplacen un número cap
avall, passant a ser els capítols 10, 11 i 12 respectivament.

Es crea un nou capítol 9 dedicat als transformadors de potència;
l'apèndix H de la versió 2.2 desapareix com a tal, quedant integrat
dins d'aquest nou capítol.


\section*{Versió 3.1 (5 de desembre de 2009)}
\addcontentsline{toc}{section}{Versió 3.1}
En l'Apèndix B s'afegeixen els prefixes de potències binàries Ki, Mi, Gi, Ti, Pi i Ei.

Es revisa tot el text, fent-hi algunes petites modificacions i
correccions.

\section*{Versió 3.2 (5 de gener de 2010)}
\addcontentsline{toc}{section}{Versió 3.2}
S'afegeix l'apartat Bibliografia després del apèndixs.


\section*{Versió 4.0 (15 de febrer de 2010)}
\addcontentsline{toc}{section}{Versió 4.0}
A partir d'aquesta versió s'utilitza la font «Kp-Fonts» en la composició de tot el text. Fins ara, les fonts utilitzades eren les «Pazo Math», «Helvetica» i «Courier».


\section*{Versió 4.1 (27 de febrer de 2010)}
\addcontentsline{toc}{section}{Versió 4.1}
En el capítol dedicat a la transformada de Laplace, es modifiquen segons \cite{SCH} algunes definicions  i s'amplien les taules de transformades de Laplace segons \cite{SCH} i \cite{RASd}.

\section*{Versió 4.2 (12 de març de 2010)}
\addcontentsline{toc}{section}{Versió 4.2}
En el capítol dedicat a les sèries de Fourier, es completa l'equació (3.7c) i s'afegeix una taula amb les sèries de Fourier de formes d'ona usuals.

En l'apèndix dedicat a les funcions trigonomètriques, se simplifiquen les equacions (E.18) i (E.19).

\section*{Versió 4.3 (27 de novembre de 2010)}
\addcontentsline{toc}{section}{Versió 4.3}
Els apèndixs de la versió 4.2 dedicats al grau de protecció IP i  a les classes NEMA d'aïllaments tèrmics en motors, passen a formar part del capítol 12; aquest capítol canvia de nom i passa a dir-se «Normatives Diverses».

L'apèndix de la versió 4.2 dedicat a les escales logarítmiques, passa a formar part del capítol 5 dedicat a càlculs bàsics.

Es modifica l'adreça de correu electrònic de contacte amb l'autor.

\section*{Versió 4.4 (31 de març de 2011)}
\addcontentsline{toc}{section}{Versió 4.4}
En el capítol 12, s'amplia la descripció dels codis IP i IK, i s'hi afegeix el codi NEMA dedicat al grau de protecció d'equips.

Es modifica l'adreça de correu electrònic de contacte amb l'autor.

\section*{Versió 4.5 (2 de novembre de 2011)}
\addcontentsline{toc}{section}{Versió 4.5}
En l'apartat Notació, s'afegeixen diverses relacions referents a $|\cmplx{V}|$, $\arg\cmplx{V}$, $\Re\cmplx{V}$ i $\Im\cmplx{V}$.

Es modifiquen les equacions (3.7c), (D.18a) i (D.18b).

En el capítol 3, es millora l'explicació de les propietats de les sèries de Fourier.

En el capítol 12, s'afegeixen dues seccions, dedicades a l'àmbit d'aplicació de diverses normes CEI i IEEE.

S'afegeix una nova entrada en l'apartat Bibliografia.


\section*{Versió 4.6 (21 de novembre de 2011)}
\addcontentsline{toc}{section}{Versió 4.6}

En l'apèndix dedicat a l'alfabet grec, s'utilitza el DIEC2 «Diccionari de la llengua catalana, 2a edició (2007)», com la referència per escriure els noms de les lletres gregues en català. S'afegeix també el nom de les lletres gregues en francès.


\section*{Versió 5.0 (30 de gener de 2012)}
\addcontentsline{toc}{section}{Versió 5.0}

Es modifica lleugerament el nom del llibre, passant a dir-se «Qüestions Electrotècniques Diverses» enlloc de «Qüestions Diverses d'Electrotècnia», i per tant a parir d'ara es podrà denominar de forma abreviada «QED» (\emph{quod erat demonstrandum}).

Es canvia la tipografia dels exemples utilitzada en les versions anteriors, passant ara a ser escrits en lletra recta enlloc de en lletra inclinada.

El símbol $\measuredangle$ utilitzat per indicar l'argument d'un valor complex en les versions anteriors, es canvia pel símbol $\angle$ d'acord amb la norma internacional ISO/IEC 80000 «Quantities and units» (la qual substitueix a l'antiga ISO 31).

Es canvia en tot el text el terme «vector» pel terme «fasor» quan es fa referència a magnituds sinusoïdals.

S'indica en el prefaci que s'ha utilitzat la distribució MiK\TeX, que ofereix una implementació lliure de \LaTeX.

S'afegeix en l'apartat Notació la definició d'un fasor.

Es modifica l'equació (7.25).

S'amplia la secció 9.7.2.

S'afegeixen algunes normes en les seccions (12.6) i (12.7).

S'inclou en la taula A.1 i en l'explicació posterior, la representació gràfica $\varkappaup$ de la lletra minúscula kappa.

Es modifica en la taula B.6 el valor en unitats SI de la unitat de massa atòmica unificada.

En l'apartat Bibliografia, s'afegeixen les referències \cite{GRZ}, \cite{DUN}, \cite{REI} i \cite{TLE}.

Es revisa tot el text, fent-hi algunes correccions.

\section*{Versió 5.1 (15 de febrer de 2012)}
\addcontentsline{toc}{section}{Versió 5.1}

Es millora en l'apartat Notació, la definició de l'angle $\alpha$ d'un fasor.

S'amplia la secció 1.6 dedicada als càlculs en per unitat.


\section*{Versió 5.2 (4 de maig de 2012)}
\addcontentsline{toc}{section}{Versió 5.2}

Es completa l'equació (1.75).

En el capítol 12 s'afegeix una secció dedicada als interruptors automàtics de baixa tensió segons les normes CEI.

S'afegeixen algunes normes en la secció 12.8.

\section*{Versió 5.3 (14 de juliol de 2012)}
\addcontentsline{toc}{section}{Versió 5.3}

S'amplia la secció D.2, afegint-hi la llei de les cotangents i la fórmula de Mollweide, i modificant  la figura D.1.

S'afegeixen algunes normes en la secció 12.8.

\section*{Versió 5.4 (2 de novembre de 2012)}
\addcontentsline{toc}{section}{Versió 5.4}

Es canvia de forma general el símbol «$\cdot$» pel símbol «$\times$», quan es tracta d'expressar la multiplicació de dos valors numèrics.

Es revisa l'apèndix B, sobretot en l'apartat referent a les normes d'escriptura.

Es revisa l'apèndix C utilitzant les publicacions de l'any 2010 del «Committee on Data for Science and Technology» (CODATA).\index{CODATA}

\section*{Versió 5.5 (1 de desembre de 2012)}
\addcontentsline{toc}{section}{Versió 5.5}

En la secció 8.5 es referencia la norma  IEEE C57.13, enlloc de la més antiga ANSI C57.13.

En la secció 9.10 es referencia la norma  IEEE C57.12.00, enlloc de la més antiga ANSI C57.12.

Es posa al dia la secció 12.1 segons la norma IEEE C37.2, enlloc de la més antiga ANSI C37.2.

S'afegeixen algunes normes en les seccions 12.7 i 12.8.

Es modifica  la figura D.1.


\section*{Versió 6.0 (2 de gener de 2013)}
\addcontentsline{toc}{section}{Versió 6.0}

Es realitza una revisió general del text i de les figures d'aquest  llibre, utilitzant la simbologia de les normes CEI 60027  «Letter symbols to be used in electrical technology» i  CEI 60617  «Graphical Symbols for Diagrams».

S'amplia la secció 1.4 utilitzant les definicions de la norma CEI 60050.

Es completen les equacions (1.72) i (1.74).

Es modifica l'equació (1.79).

S'amplia la secció  3.4 utilitzant les definicions de la norma CEI 60050.

Es realitzen les modificacions següents en l'apèndix B:
\begin{itemize}
   \item  S'inclou la referència al Reial Decret 2032/2009, de 30 de desembre.
   \item S'indica que les variants ortogràfiques  «kilogram»/«quilogram», «kilo»/«quilo», «radian»/«radiant» i
   «estereoradian»/«estereoradiant», són equivalents segons el DIEC2 «Diccionari de la llengua catalana, 2a edició (2007)»
    \item S'escriu correctament el nom de la unitat «electró-volt». El nom  utilitzat en edicions anteriors,   «electronvolt», no apareix en el DIEC2 «Diccionari de la llengua catalana, 2a edició (2007)».
    \item S'inclou l'adreça d'Internet de l'«International Earth rotation and Reference systems Service».
     \item Es refà l'apartat dedicat a les normes d'escriptura.
\end{itemize}

Es refà la taula de l'apèndix C agrupant els valors numèrics i les seves unitats, i s'explica a continuació com obtenir els errors absoluts i relatius dels valores que hi apareixen.

\section*{Versió 6.1 (1 de febrer de 2013)}
\addcontentsline{toc}{section}{Versió 6.1}

S'afegeix un segon exemple en la secció 1.1.2, dedicada al teorema de Millman.


\section*{Versió 6.2 (11 de setembre de 2013)}
\addcontentsline{toc}{section}{Versió 6.2}

 Es revisa el capítol 8 utilitzant la norma CEI 60044, enlloc de les normes CEI 60185 i CEI 60186, que ja no estan en vigor.

Es crea la secció 9.11 per explicar com es formen els circuits homopolars dels transformadors de potència de dos i tres debanats.

En la secció 12.7 s'eliminen les normes CEI 60185 i CEI 60186, que ja no estan en vigor.

S'afegeixen el prefixes «zebi» i «yobi» a la Taula B.9.

En l'apartat Bibliografia, s'afegeix la referència \cite{RASe}.

\section*{Versió 6.3 (24 de març de 2014)}
\addcontentsline{toc}{section}{Versió 6.3}

S'inclou el període $T$ en el gràfic de l'apartat Notació.

S'amplia la secció 7.4 dedicada a la capacitat tèrmica dels cables en curtcircuit.

S'amplia el capítol 8 dedicat als transformadors de mesura i protecció.

S'afegeixen algunes normes en les seccions 12.7 i 12.8.

En l'apartat Bibliografia, s'afegeix les referències \cite{KAS} i \cite{JCD}.

Es revisa tot el text, fent-hi algunes correccions. A més, es modifica la presentació de tots els exemples, emmarcant-los dins d'un rectangle.


\section*{Versió 7.0 (24 de juliol de 2014)}
\addcontentsline{toc}{section}{Versió 7.0}

En l'apartat Notació, s'afegeixen dues  relacions referents a la representació de nombres complexes en format exponencial.

Es creen i reordenen diversos capítols i seccions, segons es detalla a continuació:
\begin{itemize}
   \item El capítol 1 es queda amb les primeres cinc seccions reordenades, de les set que tenia la versió 6.3. S'afegeix a aquest capítol una sisena secció nova, dedicada als circuits R-L-C.
   \item El capítol 5 de la versió 6.3 passa a ser el capítol 2, reordenant les seves seccions i incorporant les dues últimes seccions del capítol 1 de la versió 6.3.
   \item Els capítols 2, 3 i 4  de la versió 6.3 es desplacen un número cap avall.
   \item Es crea un nou apèndix, dedicat al càlcul numèric.
\end{itemize}

S'amplia la secció 7.5.2.

S'amplia el capítol 8 dedicat als transformadors de mesura i protecció.

S'afegeixen algunes normes en la secció 12.8.

En l'apartat Bibliografia, s'afegeixen les referències \cite{GOM}, \cite{SPK}, \cite{JDH}, \cite{EJB},  \cite{PMA}, \cite{MAI} i \cite{KNU}.

Es modifica l'adreça de correu electrònic de contacte amb l'autor.


\section*{Versió 7.1 (23 d'octubre de 2014)}
\addcontentsline{toc}{section}{Versió 7.1}

S'afegeixen algunes normes en la secció 12.8.


\section*{Versió 8.0 (9 de novembre de 2014)}
\addcontentsline{toc}{section}{Versió 8.0}

Es refan tots els dibuixos del llibre utilitzant el programa \emph{Inkscape}; aquest programa de dibuix vectorial és de distribució lliure, i pot obtenir-se a l'adreça: \href{http://www.inkscape.org/}{www.inkscape.org}. En totes les versions anteriors del llibre s'ha utilitzat el programa jPicEdt, el qual també és de distribució lliure i pot obtenir-se a l'adreça: \href{http://www.jpicedt.org/}{www.jpicedt.org}.

Es canvien en l'apartat Notació, els noms de les variables utilitzades en la definició d'un fasor.

En la secció 1.4.2, dedicada a la potència trifàsica, se substitueixen els subíndexs «$\alphaup$», «$\betaup$», «$\gammaup$» i «$\nuup$», pels subíndexs «A», «B», «C» i «N», a l'hora d'identificar les tres fases  i el neutre d'un sistema trifàsic.

En la secció 2.3.1,  dedicada als circuits divisors de tensió, es canvien els noms de les variables utilitzades.

En la secció 2.3.2,  dedicada als circuits divisors de corrent, es canvien els noms de les variables utilitzades.

En la secció 2.4, dedicada a la transformació estrella $\boldsymbol{\leftrightarrow}$ triangle d'impedàncies, se substitueixen els subíndexs «$\alphaup$», «$\betaup$» i «$\gammaup$», pels subíndexs «A», «B» i «C», a l'hora d'identificar les tres fases d'un sistema trifàsic.

En el capítol 3, dedicat a les components simètriques, se substitueixen els superíndexs «(1)», «(2)» i «(0)», pels subíndexs «1», «2» i «0», a l'hora d'identificar les components directa, inversa i homopolar. A més, també se substitueixen els subíndexs «$\alphaup$», «$\betaup$», «$\gammaup$» i «$\nuup$» , pels subíndexs «A», «B»,  «C» i «N», a l'hora d'identificar les tres fases i el neutre d'un sistema trifàsic.

S'afegeix l'equació (9.51) per tal d'explicar millor la compatibilitat entre els índexs horaris de dos transformadors.

\section*{Versió 8.1 (16 de novembre de 2014)}
\addcontentsline{toc}{section}{Versió 8.1}

Es modifica el gruix i l'estil de línia d'alguns dibuixos del llibre, per tal de fer-los més uniformes.

Es numeren les figures de les seccions 9.12.1 i 9.12.2.


\section*{Versió 8.2 (23 de novembre de 2014)}
\addcontentsline{toc}{section}{Versió 8.2}

S'afegeix la secció E.3, dedicada a la solució de funcions no lineals.

\section*{Versió 8.3 (11 de desembre de 2014)}
\addcontentsline{toc}{section}{Versió 8.3}

Es millora en l'apartat Notació, l'explicació de la definició d'un fasor.


\section*{Versió 8.4 (3 de gener de 2015)}
\addcontentsline{toc}{section}{Versió 8.4}

Es millora l'explicació de la secció 1.6.5.

S'amplia la secció 2.3, afegint-hi el cas particular de dues impedàncies.

Es crea la secció 6.3 dedicada a les potencies normalitzades de les resistències.

S'expressen correctament les equacions (E.1) i (E.2).

En l'apartat Bibliografia, s'afegeixen les referències \cite{AGVS}, \cite{JSch} i \cite{RRop}.


\section*{Versió 9.0 (29 d’agost de 2016)}
\addcontentsline{toc}{section}{Versió 9.0}

Es modifiquen els estils dels textos de les capçaleres de les taules i dels peus de les figures, per tal que siguin iguals que els estils dels títols dels capítols, seccions i subseccions.

Es modifica en tot el llibre la manera de representar una variable acompanyada de les seves unitats. Les variables se separaran de les seves unitats mitjançant el símbol de divisió «/», enlloc de tancar les unitats entre «[» i  «]»; per exemple, enlloc de
$S\si{\,[mm^2]}$, a partir d'ara escriurem $S/\si{mm^2}$.

Es modifica en tot el llibre la posició de les notes que fan referència a elements d'una taula, coŀlocant-les immediatament a sota de la pròpia taula, enlloc de al peu de pàgina.

Es refan totes les gràfiques de funcions del llibre utilitzant el programa \emph{gnuplot}; aquest programa de dibuix de gràfiques de funcions és de distribució lliure, i pot obtenir-se a l'adreça: \href{http://www.gnuplot.info/}{www.gnuplot.info}. En totes les versions anteriors del llibre s'ha utilitzat el paquet d'ampliació  PSTricks.

S'amplien algunes seccions i exemples del llibre, afegint-hi una  resolució numèrica mitjançant la calculadora \emph{HP Prime};\index{HP Prime}  aquesta calculadora disposa d'un emulador per a PC, que pot descarregar-se de la pàgina de Hewlett-Packard: \href{http://www.hpprime.de/en/category/6-downloads}{www.hpprime.de/en/category/6-downloads}. Les seccions i els exemples afectats són els següents:
\begin{itemize}
  \item L'exemple 1.8  (Corrent de pic de curtcircuit).
  \item L'exemple 1.9 (Resolució de xarxes amb el mètode de les malles).
  \item L'exemple 2.3 (Resolució de circuits coneixent la potència absorbida).
  \item La secció 3.7 (Components simètriques)
  \item L'exemple 4.3 (Sèries de Fourier).
  \item L'exemple 5.4 (Transformada de Laplace).
  \item L'exemple 10.1 (Resolució de xarxes utilitzant el mètode dels nusos).
  \item La secció 11.6 (Flux de càrregues).
\end{itemize}

S'amplia la secció 1.2.2 dedicada al teorema de Millman, afegint-hi un exemple més al final.

Es millora l'explicació de la secció 1.6.5.

Es crea la secció 1.7 dedicada a la resolució de xarxes elèctriques, utilitzant el mètode de les malles.

S'amplia la secció 2.7 dedicada a les escales logarítmiques, afegint-hi al final un nou apartat, dedicat a la determinació del paràmetres de funcions que prenen la forma d'una recta en gràfiques d'escala logarítmica--logarítmica.

Es millora l'explicació de l'exemple 3.1.

Es crea la secció 3.7 on es descriuen diversos programes de la calculadora
\emph{HP Prime}\index{HP Prime} relacionats amb les components simètriques.

S'amplia l'exemple 4.3, afegint-hi al final  una nova gràfica.

S'amplia el capítol 6, afegint-hi la codificació del coeficient de variació amb la temperatura de les resistències, i la norma CEI que defineix les sèries de resistències estàndard.

Es modifiquen les capçaleres de totes les taules del capítol 8, ja que els percentatges d'error de tensions i corrents que s'hi indicaven, estaven referits incorrectament als valors nominals dels transformadors.

Es modifiquen les equacions que fan referència a les figures 9.1 i 9.11, perquè es vegi millor la seva correspondència.

S'afegeixen algunes normes en la secció 12.7.

En l'apèndix A dedicat a l'alfabet grec, s'utilitza el D.R.A.E.
«Diccionario de la Lengua Española, 23ª
edición (2014)», com la referència per escriure els noms de les lletres gregues en castellà.

S'amplia l'apèndix B, afegint-hi al final una secció dedicada al factors de conversió d'unitats.

Es revisa la taula B.6 i l'apèndix C, utilitzant les publicacions de l'any 2014 del «Committee on Data for Science and Technology» (CODATA).\index{CODATA}


S'amplia la secció D.2, afegint-hi les equacions de les coordenades del baricentre d'un triangle, i es modifica de manera corresponent la figura D.1.

Es modifiquen les figures E.1 i E.2.

En l'apartat Bibliografia, s'afegeixen les referències \cite{VOS}, \cite{WMF} i \cite{TRA}.


\section*{Versió 9.1 (27 de novembre de 2016)}
\addcontentsline{toc}{section}{Versió 9.1}

Es revisa tot el text, fent-hi algunes  modificacions i correccions.

Es crea la Figura 1.4, on s'hi representen els paràmetres d'una funció periòdica qualsevol.

Es modifica en la secció 3.7 la funció \texttt{\textbf{Triangle→Fasors}}, fent-la més simple.

Es milloren les equacions (7.27) i (7.28).

Es creen les equacions (7.29) i (7.30), i un exemple de com utilitzar-les.

Es dóna color a la Taula D.1, per tal de distingir millor els valors positius dels negatius.

\section*{Versió 10.0 (6 de gener de 2017)}
\addcontentsline{toc}{section}{Versió 10.0}

Es modifica en tot el llibre la manera de representar el producte de dues unitats; en les versions anteriors s'havia utilitzat un punt volat, i a partir d'ara es farà servir un espai en blanc. Ambdues forme són correctes, però l'espai en blanc és la forma utilitzada preferentment en les publicacions del BIPM «Bureau International des Poids et Mesures». Els apèndixs B i C són els més afectats per aquest canvi.

Es modifica en tot el llibre el símbol de la unitat  utilitzada per a la potència reactiva; en les versions anteriors s'havia emprat el símbol «VAr», i a partir d'ara es farà servir el símbol «var», ja que és el que adopta la norma CEI 60027-1.

Es modifica en tot el llibre la manera d'escriure les funcions $\Re$, $\Im$ i $\arg$, quan van seguides d'una única variable; en les versions anteriors s'havien emprat, per exemple, les formes $\Re(\cmplx{S})$, $\Im(\cmplx{S})$ i $\arg(\cmplx{S})$, i a partir d'ara es faran servir les formes $\Re\cmplx{S}$, $\Im\cmplx{S}$ i $\arg\cmplx{S}$ respectivament.


Es millora en l’apartat Notació, la definició de l’angle $\alpha$ d’un fasor.


Es completa l'exemple 1.6, calculant-hi al final les potències activa i reactiva.

Es completa la secció 1.6.1, afegint-hi les equacions corresponents a tenir el condensador carregat en l'instant inicial, a una tensió no nuŀla.

Es completa la secció 1.6.3, afegint-hi les equacions corresponents a tenir circulant per la inductància en l'instant  inicial, un corrent no nul.

Es crea l'exemple 1.7, en el qual es calcula el corrent i la tensió de càrrega i descàrrega d'un circuit R-L.

Es completa l'exemple 1.9, calculant-hi per separat els valors de $\kappa$ i $\hat{I}\ped{asim}$.


Es completa l'exemple 4.3, afegint-hi al final el càlcul del corrent $i(t)$ utilitzant les equacions de la secció 1.6.3.

Es completa l'exemple 10.1, afegint-hi al final la resolució del sistema d'equacions lineals amb la funció \texttt{\textbf{simult}}.


En l'apèndix A, es donen les adreces d'Internet dels diccionaris utilitzats per escriure els noms de les lletres gregues en anglès i francès. Addicionalment, es corregeix el nom en francès de la lletra $\varpiup$; el nom correcte és «pi dorien».

En l'apèndix B, es té en compte el suplement de l'any 2014 publicat pel BIPM «Bureau International des Poids et Mesures», que posa al dia la 8a edició de les seves publicacions de l'any 2006. Els canvis introduïts que afecten a aquest llibre, són els següents:
\begin{itemize}
  \item Es modifica l'ordre de les unitats base, en l'expressió de les unitats derivades. Això afecta  a les Taules B.3 i B.4.
  \item La unitat astronòmica de longitud va ser redefinida l'any 2012 en la 28a Assemblea General de la Unió Astronòmica Internacional, passant a ser un valor exacte. Això ocasiona que aquesta unitat passi de la Taula B.6  a la Taula  B.5.
\end{itemize}

Es crea la Taula B.8 per recollir les unitats fora de l'SI acceptades addicionalment pel NIST «National Institute of Standards and Technology».

Es crea la secció B.7 per recollir  unitats definides per la norma  CEI 60027, addicionals a les de l'SI. Algunes d'aquestes unitats estaven incloses anteriorment en la secció B.6.

En la secció B.8, s'utilitza el símbol \textcolor{Blue}\faQuestionCircle{} per indicar escriptures correctes però no recomanades.

\section*{Versió 10.1 (8 de març de 2017)}
\addcontentsline{toc}{section}{Versió 10.1}

S'afegeix al final de la secció 12.1, una figura per iŀlustar la diferència entre les funcions de protecció 50, 50TD i 51, i l'equació de les corbes característiques de la funció de protecció 51, amb els valors dels paràmetres utilitzats per les normes CEI i IEEE.

\section*{Versió 10.2 (7 de maig de 2017)}
\addcontentsline{toc}{section}{Versió 10.2}

S'afegeix al final de la secció 3.5, el càlcul del sistema de tensions fase--neutre $\cmplx{U}\ped{AG}$, $\cmplx{U}\ped{BG}$ i $\cmplx{U}\ped{CG}$, a partir del sistema de tensions fase--fase $\cmplx{U}\ped{AB}$, $\cmplx{U}\ped{BC}$ i $\cmplx{U}\ped{CA}$, i s'utilitzen les equacions obtingudes, al final de l'exemple 3.1.

Es crea el nou exemple 3.2, on es calculen les components simètriques d'un sistema de tensions desequilibrat que alimenta a una càrrega desequilibrada.

\section*{Versió 10.3 (3 de juny de 2017)}
\addcontentsline{toc}{section}{Versió 10.3}

Es millora l'explicació de l'apartat 11.5.

\section*{Versió 10.4 (28 d'agost de 2017)}
\addcontentsline{toc}{section}{Versió 10.4}

Es crea el nou apèndix F, dedicat a programes per a la calculadora \emph{HP Prime} de Hewlett-Packard. S'eliminen els programes que en les versions anteriors estaven llistats en l'exemple 2.3 i en la secció 3.7, incorporant-se en aquest nou apèndix.

S'afegeix un nou exemple al final de la secció E.1, dedicat  a la interpolació en dues dimensions.

\section*{Versió 10.5 (11 de setembre de 2017)}
\addcontentsline{toc}{section}{Versió 10.5}

S'utilitza la font «true type» HPPrime.ttf, per representar les tecles de la calculadora \emph{HP Prime} en els diversos exemples d'ús d'aquesta calculadora que hi ha en el llibre.

S'inclou en l'apèndix F una imatge de la calculadora \emph{HP Prime}. 