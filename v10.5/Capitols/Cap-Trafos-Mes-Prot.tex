\chapter{Transformadors de Mesura i Protecció}\label{sec:tr_mes_prot}
\index{transformadors de mesura i protecció}


\section{Introducció}
Es tracten en aquest capítol els transformadors de
mesura i de protecció, tant de tensió com de corrent. Aquest
tractament es fa més detalladament des del punt de vista de la norma CEI 60044, no obstant, es dedica també un
apartat a descriure la norma  IEEE C57.13, i la
relació entre ambdues.\index{CEI!60044-0@60044}\index{IEEE!C57.13}

En la Figura \vref{pic:TT_TI} es representen unes connexions
habituals d'un transformador de tensió (anomenats usualment Tt), a
la part superior, i d'un transformador de corrent (anomenats
usualment Tc), a la part inferior. Els sentits de les tensions
i corrents s'han representat tenint en compte els terminals
equivalents dels primaris i secundaris (\textsf{A--a} i \textsf{B--b}, en el cas del Tt, i \textsf{P1--S1} i \textsf{P2--S2}, en el cas del Tc).
\index{Tt}\index{Tc}

\hfill
\begin{minipage}[b]{90mm}
    \hspace{1.5cm}
    \input{Imatges/Cap-TrafosMesProt-TI-TT.pdf_tex}
    \captionof{figure}{Transformadors de tensió i de corrent}
    \label{pic:TT_TI}
\end{minipage}
\hfill
\begin{minipage}[b][70mm][t]{50mm}
   \begin{align}
      \cmplx{U}\ped{s} &= \cmplx{U}\ped{p} \frac{U\ped{ns}}{U\ped{np}}
      \\[21mm]
      \cmplx{I}\ped{s} &= \cmplx{I}\ped{p} \frac{I\ped{ns}}{I\ped{np}}
   \end{align}
\end{minipage}

Les relacions de transformació nominals d'aquests transformadors de
tensió i de corrent són respectivament $U\ped{np}\!:\!U\ped{ns}$ i
$I\ped{np}\!:\!I\ped{ns}$.

Al costat de la Figura
\vref{pic:TT_TI} es poden veure les relacions que lliguen les
tensions i corrents de primari amb les de secundari, suposant que
els transformadors són ideals.


Els transformadors de tensió es connecten a la línia principal en
derivació; el  primari està sotmès, per tant, a la tensió de la
línia. Els Tt per a connexió entre fases tenen els dos borns
primaris aïllats, mentre que els que estan previstos per ser
connectats entre fase y terra només en tenen aïllat un, ja que
l'altre es connecta directament a terra. Els transformadors de
corrent, en canvi, es connecten amb el seu primari intercalat en la
línia principal;  pel primari del Tc circula, per tant, el corrent
 de la línia.

 Com a mesura de protecció per a les persones, és usual
connectar a terra un dels dos terminals del secundari dels
transformadors. Cal recordar a més, en el cas dels transformadors de
corrent, que el secundari no ha de quedar mai en circuit obert, ja
que es produirien sobretensions que podrien malmetre el
transformador.

\section{Errors de mesura dels transformadors reals}

Atès que en realitat els transformadors que es construeixen no són
ideals, tots tenen un error en la transformació de la magnitud
primària en la secundària, tant pel que fa al mòdul com pel que fa
a l'angle.

\subsection{Error de relació}\index{transformadors
de mesura i protecció!error de relació}

Aquest és l'error de mòdul existent entre les magnituds primària i
secundaria; es denomina més específicament error de corrent en el
cas dels Tc i error de tensió en el cas dels Tt.

En el cas dels Tc, si $I\ped{p}$ i $I\ped{s}$ són els corrents que
realment circulen pel primari i pel secundari respectivament,
l'error de relació $\epsilon\ped{r}$ val:
\begin{equation}
    \epsilon\ped{r} = \frac{\frac{I\ped{np}}{I\ped{ns}} I\ped{s} - I\ped{p}} {I\ped{p}}
\end{equation}
\index{$\epsilon\ped{r}$}

En el cas dels Tt, si $U\ped{p}$ i $U\ped{s}$ són les tensions que
realment existeixen en el primari i en el secundari respectivament,
l'error de relació $\epsilon\ped{r}$ val:
\begin{equation}
    \epsilon\ped{r} = \frac{\frac{U\ped{np}}{U\ped{ns}} U\ped{s} - U\ped{p}} {U\ped{p}}
\end{equation}
\index{$\epsilon\ped{r}$}

Els errors de relació de tensió i de corrent s'expressen
normalment en tant per cent.

\subsection{Error de fase}\index{transformadors
de mesura i protecció!error de fase}

Aquest és l'error d'angle  existent entre les magnituds primària i
secundaria; aquesta definició és rigorosa únicament en el cas de
tensions o corrents sinusoïdals, on aquests valors es poden
representar mitjançant fasors. L'error de fase $\epsilon\ped{\phiup}$ es considera positiu quan la magnitud secundària avança a la primària.
\index{$\epsilon\ped{\phiup}$}

 Cal fer notar que mentre que l'error de relació
vist anteriorment afecta a qualsevol tipus d'aparell que es
connecti en el secundari, l'error de fase no afecta a aparells que
únicament mesuren el mòdul de la tensió o del corrent (amperímetre,
voltímetre, etc.), i sí afecta en canvi a aparells que mesuren
simultàniament diverses tensions o corrents (wattímetre, comptador
d'energia, sincronitzador, etc.)

Els errors de fase s'expressen en el valor de l'angle, mesurat en
minuts d'arc o en centiradiant (crad).

\subsection{Classe, càrrega i potència de precisió}\index{transformadors
de mesura i protecció!classe de precisió}\index{transformadors de
mesura i protecció!potència de precisió
($S\ped{n}$)}\index{transformadors de mesura i protecció!càrrega de
precisió ($Z\ped{ns}$)}

Les normes defineixen les anomenades «classes de precisió»,
cadascuna de les quals té assignades uns límits admissibles dels
errors de relació i de fase. Així doncs, a cada transformador
se li assigna una determinada classe de precisió en funció dels errors
de relació i de fase que presenta.

Els errors de relació i de fase que presenta un transformador no són
constants, sinó que depenen de les següents condicions:
\begin{itemize}
   \item La tensió present en el secundari, en el cas dels Tt, i el corrent que
   circula    pel secundari, en el cas dels Tc.
   \item La càrrega connectada en el secundari (en sèrie en el cas dels Tc,
   i en paraŀlel en el cas dels Tt), definida pel nombre i tipus d'aparells connectats.
   \item La freqüència de funcionament.
\end{itemize}

Per tant, la classe de precisió assignada a un Tc o a un Tt ha de
referir-se a un determinat valor de la càrrega, a la qual està
connectat el transformador. Es defineix, en conseqüència, el terme
«càrrega de precisió» com el valor de la càrrega en el secundari
(expressada en \si{\ohm}), a la qual està referida la classe de precisió
assignada; és més usual, no obstant,  utilitzar el terme «potència
de precisió», que es el valor de la càrrega (expressada com potència
aparent en VA),
 a la qual està referida la classe de precisió.

La relació entre la càrrega de precisió $Z\ped{ns}$ i la potència de
precisió $S\ped{n}$ en el cas dels Tt és:
\begin{equation}
    S\ped{n} = \frac{U\ped{ns}^2}{Z\ped{ns}}
\end{equation}

I en el cas dels Tc és:
\begin{equation}\label{eq:sn_ti}
    S\ped{n} = I\ped{ns}^2 \,Z\ped{ns}
\end{equation}


\section{Característiques dels transformadors de tensió segons la norma CEI 60044}
\index{transformadors de mesura i protecció (Tt)}\index{CEI!60044-0@60044}

\subsection{Característiques comunes dels Tt de mesura i de protecció}

Segons quina sigui la seva funció, els Tt es classifiquen en:
\begin{description}
   \item [\hspace{5mm}Transformadors de mesura:] Són els utilitzats per alimentar
            instruments de mesura (voltímetres, wattímetres, etc.),
            comptadors d'energia i altres aparells que requereixin senyal de tensió.
   \item [\hspace{5mm}Transformadors de protecció:] Són els utilitzats per
   alimentar relés de protecció.
\end{description}

Es presenten a continuació les característiques comunes als Tt de
mesura i de protecció.

\subsubsection{Tensió primària nominal ($U\ped{np}$)}
\index{transformadors de mesura i protecció (Tt)!tensió nominal
primària ($U\ped{np}$)}

És la tensió assignada al primari del transformador, a partir de la
qual es determinen les seves característiques de funcionament i
d'aïllament.

Els valors normalitzats per a transformadors connectats entres dues fases són els exposats en la norma CEI 60038.\index{CEI!60038-00@60038}

En el cas de transformadors connectats entre fase i terra, o entre el punt neutre d'un sistema i terra, els valor normalitzats de la norma CEI 60038 es dividiran per $\sqrt{3}$.


\subsubsection{Tensió secundària nominal ($U\ped{ns}$)}
\index{transformadors de mesura i protecció (Tt)!tensió nominal
secundària ($U\ped{ns}$)}

És la tensió assignada al secundari del transformador.
Els valors normalitzats són:
\begin{itemize}
    \item \SIlist{100;110}{V}, en el cas de transformadors connectats
    entre dues fases.
    \item $\dfrac{100}{\sqrt{3}}\si{\,V}$ i
        $\dfrac{110}{\sqrt{3}}\si{\,V}$, en el cas de transformadors
        connectats entre fase i terra.
    \item \SI{100}{V}, \SI{110}{V}, $\dfrac{100}{\sqrt{3}}\si{\,V}$,
    $\dfrac{110}{\sqrt{3}}\si{\,V}$, $\dfrac{100}{3}\si{\,V}$   i
    $\dfrac{110}{3}\si{\,V}$, en el cas de transformadors
    connectats en triangle obert.
\end{itemize}

\subsubsection{Relació de transformació nominal ($K\ped{n}$)}
\index{transformadors de mesura i protecció (Tt)!relació de
transformació nominal($K\ped{n}$)}

 Relació  dels dos paràmetres anteriors: $K\ped{n} = \dfrac{U\ped{np}}{U\ped{ns}}$.

 Valors usuals són: 10, 12, 15, 20, 25, 30, 40, 50, 60 i 80, i els seus múltiples decimals.

\subsubsection{Freqüència nominal ($f\ped{n}$)}
\index{transformadors de mesura i protecció (Tt)!freq\"{u}ència nominal ($f\ped{n}$)}

 És la freqüència d'operació per a la qual  està dissenyat el transformador, usualment \SI{50}{Hz} a Europa.

\subsubsection{Potència de precisió nominal ($S\ped{n}$)}
\index{transformadors de mesura i protecció (Tt)!potència de
precisió ($S\ped{n}$)}

Els valors normalitzats de la potència de precisió, per
a un factor de potència 0,8 inductiu són: \SIlist{10; 15; 25; 30; 50; 75; 100; 150;
 200; 300; 400; 500}{VA}.

 Els valors preferits són: \SIlist{10; 25; 50; 100; 200; 500}{VA}.

En el cas de transformadors trifàsics, $S\ped{n}$ és la potència per fase.

\subsubsection{Factor de tensió nominal}
\index{transformadors de mesura i protecció (Tt)!factor de tensió
nominal}

 És el factor pel qual ha de
multiplicar-se la tensió nominal primària, per tal de determinar la
tensió màxima que el Tt pot suportar durant un temps determinat,
sense sobrepassar ni l'escalfament admissible ni els límits d'error
corresponents a la seva classe de precisió. Les sobretensions poden
presentar-se en el transformador  per la fluctuació
pròpia de la xarxa on estiguin connectats o per l'efecte de curtcircuits.

Tots els  Tt han de tenir un factor de tensió nominal igual a 1,2 en permanència.

A més, per a certes connexions els Tt han de tenir addicionalment els següents factors de tensió nominals:
 \begin{itemize}
   \item 1,5 durant 30 s,  per a transformadors connectats entre fase i terra, en sistemes que tenen el neutre connectat a terra de forma efectiva (aquells que en produir-se una falta fase--terra, la sobretensió que apareix en les fases sanes no supera 1,4 vegades la tensió nominal).
   \item 1,9 durant 30 s,  per a transformadors connectats entre fase i terra, en sistemes que tenen el neutre connectat a terra de forma no efectiva (aquells que en produir-se una falta fase--terra, la sobretensió que apareix en les fases sanes  supera 1,4 vegades la tensió nominal), i on es produeix una desconnexió automàtica  en cas de faltes fase--terra.
   \item 1,9 durant 8 h,  per a transformadors connectats entre fase i terra, en sistemes que tenen el neutre aïllat o el neutre connectat a terra mitjançant un circuit ressonant, i on no es produeix una desconnexió automàtica  en cas de faltes fase--terra.
\end{itemize}

\subsubsection{Identificació dels terminals}
\index{transformadors de mesura i protecció (Tt)!identificació dels terminals}

 Les lletres «A», «B», «C» i «N» s'utilitzen per identificar els terminals primaris, i les lletres «a», «b», «c» i «n» s'utilitzen per identificar els terminals secundaris homòlegs.

 Les lletres «A», «B» i «C» s'utilitzen pels terminals connectats a les fases i la «N» pel terminal connectat a terra.

 En el cas de secundaris connectats en triangle obert, els dos terminals s'identifiquen amb les lletres «da» i «dn».

 En el cas d'un Tt amb doble secundari, els terminals del  primer secundari s'identifiquen amb les lletres  «1a», «1b», «1c» i «1n», i els del segon amb les lletres  «2a», «2b», «2c» i «2n».

 En el cas d'un Tt amb un  secundari amb preses múltiples, els terminals s'identifiquen amb les lletres  «a1», «a2», «a3», ..., «b» (o «n»).

\subsection{Característiques particulars dels Tt de mesura}

Es presenten a continuació les característiques particulars dels Tt
de mesura.

\subsubsection{Classe de precisió}
\index{transformadors de mesura i protecció (Tt)!classe de precisió}

 Els valors normalitzats són
0,1, 0,2, 0,5, 1 i 3.

En la Taula \vref{taula:errors_tt_m}
s'indiquen els límits dels errors de tensió i  de fase, per a
tensions compreses entre $\SI{80}{\percent}\,U\ped{ns}$ i
$\SI{120}{\percent}\,U\ped{ns}$, i per a càrregues compreses entre
$\SI{25}{\percent}\,S\ped{n}$ i $\SI{100}{\percent}\,S\ped{n}$, amb un factor de
potencia 0,8 inductiu.


\begin{center}
   \captionof{table}{Classes de precisió per a Tt de mesura i protecció} \label{taula:errors_tt_m}
   \begin{tabular}{cccc}
   \toprule[1pt]
   \renewcommand*{\multirowsetup}{\centering}
   \multirow{2}{17mm}{\rule{0mm}{4.5mm}Classe de\\precisió} &
   \multirow{2}{27mm}{\rule{0mm}{4.5mm}Error de tensió\\ \rule{12mm}{0mm}\si{\percent}}&
   \multicolumn{2}{c}{Error de fase} \\
   \cmidrule(rl){3-4}
    &   & minuts d'arc  & crad \\
   \midrule
   0,1 & 0,1 & 5  & 0,15 \\
   0,2 & 0,2 & 10 & 0,3 \\
   0,5 & 0,5 & 20 & 0,6 \\
   1 & 1,0 & 40 & 1,2 \\
   3 & 3,0 &  ---  & --- \\
   \bottomrule[1pt]
   \end{tabular}
\end{center}


\subsection{Característiques particulars dels Tt de protecció}

Es presenten a continuació les característiques particulars dels Tt
de protecció.

\subsubsection{Classe de precisió}
\index{transformadors de mesura i protecció (Tt)!classe de precisió}

 Els Tt de protecció, excepte aquells destinats a ser connectats en triangle obert, tenen
les mateixes classes de precisió que els Tt de mesura, i per tant
també els és aplicable la Taula \vref{taula:errors_tt_m}.

Addicionalment, els Tt de protecció pels marges de tensió compresos
entre $\SI{5}{\percent}\,U\ped{ns}$ i $\SI{80}{\percent}\,U\ped{ns}$  i entre
$\SI{120}{\percent}\,U\ped{ns}$ i el valor $U\ped{ns}$  multiplicat pel
factor de tensió nominal (per exemple $\SI{190}{\percent}\,U\ped{ns}$),
tenen assignada una altra classe de precisió; els valors
normalitzats són 3P i 6P.

Així, per exemple, un Tt amb factor de
tensió nominal 1,9 i classe de precisió 0,5 3P, té la classe de
precisió 0,5 entre $\SI{80}{\percent}\,U\ped{ns}$ i
$\SI{120}{\percent}\,U\ped{ns}$, i la classe de precisió 3P entre
$\SI{5}{\percent}\,U\ped{ns}$ i $\SI{80}{\percent}\,U\ped{ns}$ i entre
$\SI{120}{\percent}\,U\ped{ns}$ i $\SI{190}{\percent}\,U\ped{ns}$.

En la Taula \vref{taula:errors_tt_p} s'indiquen els límits dels
errors de tensió i  de fase, per a càrregues compreses entre
$\SI{25}{\percent}\,S\ped{n}$ i $\SI{100}{\percent}\,S\ped{n}$, amb un factor de
potencia 0,8 inductiu, dins dels dos marges de tensions indicats
anteriorment. Per a tensions de l'ordre del $\SI{2}{\percent}
\,U\ped{ns}$, els errors tenen un valor doble dels indicats en
aquesta taula.

\begin{center}
   \captionof{table}{Classes de precisió addicionals per a Tt de protecció}
   \label{taula:errors_tt_p}
   \begin{tabular}{cccc}
   \toprule[1pt]
   \renewcommand*{\multirowsetup}{\centering}
   \multirow{2}{17mm}{\rule{0mm}{4.5mm}Classe de\\precisió} &
   \multirow{2}{27mm}{\rule{0mm}{4.5mm}Error de tensió\\ \rule{12mm}{0mm}\si{\percent}}&
   \multicolumn{2}{c}{Error de fase} \\
   \cmidrule(rl){3-4}
    &   & minuts d'arc  & crad \\
   \midrule
   3P & 3 & 120 & 3,5 \\
   6P & 6 & 240 & 7,0 \\
   \bottomrule[1pt]
   \end{tabular}
\end{center}


La classe de precisió dels transformadors destinats a ser connectats en triangle obert serà 6P.

\section{Característiques dels transformadors de corrent segons la norma  CEI 60044}
\index{transformadors de mesura i protecció (Tc)}\index{CEI!60044-0@60044}

\subsection{Característiques comunes dels Tc de mesura i de protecció}

Segons quina sigui la seva funció, els Tc es classifiquen de forma
anàloga als Tt en:
\begin{description}
   \item [\hspace{5mm}Transformadors de mesura:] són els utilitzats per alimentar
            instruments de mesura (amperímetres, wattímetres, etc.),
            comptadors d'energia i altres aparells que requereixin senyal de corrent.
   \item [\hspace{5mm}Transformadors de protecció:] són els utilitzats per
   alimentar relés de protecció.
\end{description}

Es presenten a continuació les característiques comunes als Tc de
mesura i de protecció.


\subsubsection{Corrent primari nominal ($I\ped{np}$)}
\index{transformadors de mesura i protecció (Tc)!corrent nominal
primària ($I\ped{np}$)}

 És el corrent assignat al
primari del transformador. Els valors normalitzats
són: \SIlist{10; 12,5; 15; 20; 25;30; 40; 50; 60;75}{A}, i els
seus múltiples i submúltiples decimals.

Els valors preferits són: \SIlist{10; 15; 20; 30; 50;75}{A}, i els
seus múltiples i submúltiples decimals.


\subsubsection{Corrent secundària nominal ($I\ped{ns}$)}
\index{transformadors de mesura i protecció (Tc)!corrent nominal
secundària ($I\ped{ns}$)}

 És el corrent assignat al
secundari del transformador. Els valors normalitzats
són: \SIlist{1;2;5}{A}, essent aquest darrer valor el
preferit. En el cas de transformadors connectats en triangle, també són normalitzats els valors anteriors dividits per $ \sqrt{3}$.

\subsubsection{Relació de transformació nominal ($K\ped{n}$)}
\index{transformadors de mesura i protecció (Tc)!relació de
transformació nominal ($K\ped{n}$)}

Relació dels dos  paràmetres anteriors: $K\ped{n} = \dfrac{I\ped{np}}{I\ped{ns}}$.

\subsubsection{Freqüència nominal ($f\ped{n}$)}
\index{transformadors de mesura i protecció (Tc)!freq\"{u}ència nominal ($f\ped{n}$)}

 És la freqüència d'operació per a la qual    està dissenyat el transformador, usualment \SI{50}{Hz} a Europa.

\subsubsection{Error compost ($\epsilon\ped{c}$)}\index{transformadors
de mesura i protecció (Tc)!error compost}

Per a corrents de primari i secundari sinusoïdals, l'error compost $\epsilon\ped{c}$ es defineix  en funció dels errors de relació $\epsilon\ped{r}$ i de fase  $\epsilon\ped{\phiup}$, com:\index{$\epsilon\ped{c}$}
\begin{equation}
    \epsilon\ped{c} = \sqrt{\epsilon\ped{r}^2 +  \epsilon\ped{\phiup}^2}
\end{equation}

En aquesta expressió, els errors $\epsilon\ped{c}$ i $\epsilon\ped{r}$  estan expressats en \si{\percent}, i l'error $\epsilon\ped{\phiup}$ està expressat en crad.

En el cas general de corrents primari $i\ped{p}(t)$ i secundari $i\ped{s}(t)$ no sinusoïdals, però periòdics amb període $T$, l'error compost $\epsilon\ped{c}$ es defineix com:
\begin{equation}
    \epsilon\ped{c} = \frac{1}{I\ped{p}} \sqrt{\frac{1}{T} \int_0^T \left(K\ped{n} i\ped{s}(t) - i\ped{p}(t)\right)^2 \diff t}
\end{equation}

\subsubsection{Potència de precisió ($S\ped{n}$)}
\index{transformadors de mesura i protecció (Tc)!potència de
precisió ($S\ped{n}$)}

 Els valors normalitzats de la potència de precisió fins a \SI{30}{VA}
són: \SIlist{2,5; 5;10; 15; 30}{VA}.

Es poden escollir valors per sobre de \SI{30}{VA} segons les necessitats de cada cas.

\subsubsection{Sobrecorrents  assignats ($I\ped{th}$, $I\ped{dyn}$, $I\ped{cth}$)}
\index{transformadors de mesura i protecció (Tc)!sobrecorrents
assignats ($I\ped{th}$, $I\ped{dyn}$, $I\ped{cth}$)}

 Els Tc
tenen el primari connectat en sèrie amb una línia de potència, i per tant han
d'estar preparats per suportar curtcircuits fins que algun
interruptor desconnecti la línia on hi ha la falta; aquest
corrent es transforma en el secundari en un corrent de valor
també elevat, havent de suportar el transformador els efectes tèrmics
i dinàmics que això comporta.

Es defineix el «corrent tèrmic nominal de curta durada»
($I\ped{th}$), com el valor eficaç del  corrent primari que el
transformador pot suportar durant \SI{1}{s}, amb el debanat
secundari en curtcircuit, sense patir efectes perjudicials; es
considera que aquest temps és suficient perquè les proteccions
pertinents actuïn, eliminant el curtcircuit. En qualsevol cas, si
$I\ped{cc}$ és el corrent de curtcircuit i $t$ és la seva durada
(expressada en s), ha de complir-se: $I\ped{th}\geq
I\ped{cc}\sqrt{t}$. El valor d'aquest corrent tèrmic
acostuma a expressar-se com a un valor múltiple del corrent
nominal (per exemple: $I\ped{th}=150\,I\ped{np}$).

Es defineix el «corrent dinàmic nominal» ($I\ped{dyn}$), com el
valor de cresta del corrent tèrmic nominal de curta durada ($I\ped{th}$).
Normalment es pren el valor: $I\ped{dyn} =
\num{1,8}\sqrt{2}I\ped{th}\approx \num{2,5}I\ped{th}$. El transformador ha de
suportar les forces electrodinàmiques produïdes per aquest corrent.

Es defineix finalment el «corrent tèrmic nominal continu» ($I\ped{cth}$), com
el valor del màxim corrent que pot circular pel primari del
transformador  de forma permanent, amb el secundari connectat a la
càrrega de precisió, sense que l'escalfament del transformador surti
dels límits previstos i mantenint-se dins de la
seva classe de precisió. El valor usual és: $I\ped{cth} = I\ped{np}$. Quan es requereix un valor més elevat, els valors preferits són:
$I\ped{cth} = \num{1,2}I\ped{np}$, $I\ped{cth} = \num{1,5}I\ped{np}$ i $I\ped{cth} = 2 I\ped{np}$.

\subsection{Característiques particulars dels Tc de mesura}

Els circuits magnètics d'aquests transformadors es dissenyen de
manera que se saturin ràpidament, de manera que
sobrecorrents elevats en el primari  no repercuteixin en el secundari,
ja que els aparells que normalment s'hi connecten (amperímetres,
comptadors d'energia, etc.) no estan preparats per suportar grans
sobrecorrents.

Es presenten a continuació les característiques particulars dels Tc
de mesura.

\subsubsection{Corrent límit primari  assignat ($I\ped{PL}$)}
\index{transformadors de mesura i protecció (Tc)!corrent límit
primari  assignat ($I\ped{PL}$)}

El corrent  límit primari
és el corrent primari, a partir del qual l'error compost supera el valor del \SI{10}{\percent}, amb una càrrega igual a la càrrega de
precisió del transformador.

\subsubsection{Factor de seguretat ($F\ped{S}$) }
\index{transformadors de mesura i protecció (Tc)!factor de seguretat
($F\ped{S}$)}

 El factor de seguretat
es defineix com la relació entre el corrent límit primari
i el corrent primari nominal: $F\ped{S} = I\ped{PL} / I\ped{np}$.

En el cas d'un curtcircuit en la línia on està intercalat el
transformador, la seguretat dels aparells connectats en el secundari
del Tc és tan més gran com més petit és  $F\ped{S}$. Valors usuals
per a la majoria d'aparells són:  $\num{2,5}<F\ped{S}<10$, i per
alimentar a comptadors: $3<F\ped{S}<5$.

Cal tenir en compte que el valor de $F\ped{S}$ està lligat
 al valor de $S\ped{n}$, i que només és vàlid
quan tenim aquest consum de  potència en el secundari; per a un
valor de potència $S'$ diferent de $S\ped{n}$, tindrem un valor
$F'\ped{S}$ també diferent de  $F\ped{S}$. La relació que
lliga aquests valors, tenint en compte la resistència del debanat
secundari del transformador  $R\ped{s}$ és:

\begin{equation}\label{eq:fs}
    F\ped{S} (S\ped{n}+R\ped{s}I\ped{ns}^2) =
    F'\ped{S} (S'+R\ped{s}I\ped{ns}^2)
\end{equation}


\subsubsection{Classe de precisió}
\index{transformadors de mesura i protecció (Tc)!classe de precisió}

 Els valors normalitzats són 0,1, 0,2, 0,5, 1, 3 i 5.

En la Taula \vref{taula:errors_ti_m1}
s'indiquen els límits, per a diversos percentatges de $I\ped{ns}$, dels errors de corrent i  de fase de les classes de
precisió 0,1, 0,2, 0,5 i 1,  per a càrregues compreses entre
$\SI{25}{\percent}\,S\ped{n}$ i $\SI{100}{\percent}\,S\ped{n}$.

\begin{center}
   \captionof{table}{Classes de precisió 0,1, 0,2, 0,5 i 1 per a Tc de mesura}
   \label{taula:errors_ti_m1}
   \begin{tabular}{ccccc<{\hspace{1.5em}}cccc<{\hspace{1.5em}}cccc}
   \toprule[1pt]
   \renewcommand*{\multirowsetup}{\centering}
   \multirow{2}{17mm}{\rule{0mm}{4.5mm}Classe de\\precisió} &
   \multicolumn{4}{c}{\multirow{2}{35mm}{\rule{0mm}{4.5mm}Error de corrent\\ \rule{12mm}{0mm}\si{\percent}}} &
   \multicolumn{8}{c}{Error de fase} \\
   \cmidrule(rl){6-13}
    &  & & & & \multicolumn{4}{c}{\hspace{-1em}minuts d'arc}  &
   \multicolumn{4}{c}{crad} \\
   \midrule
    0,1 & 0,4 & 0,2 & 0,1 & 0,1 & 15 & 8 & 5 & 5 & 0,45 & 0,24 & 0,15 & 0,15 \\
    0,2 & 0,75 & 0,35 & 0,2 & 0,2 & 30 & 15 & 10 & 10  & 0,9 & 0,45 & 0,3 & 0,3 \\
    0,5 & 1,5 & 0,75 & 0,5 & 0,5 & 90 & 45 & 30 & 30 & 2,7 & 1,35 & 0,9  & 0,9 \\
    1 & 3,0 & 1,5 & 1,0 & 1,0 & 180 & 90 & 60 & 60 & 5,4 & 2,7 & 1,8 & 1,8 \\
    \midrule
    $\si{\percent} I\ped{ns}$ & 5 & 20 & 100 & 120 & 5 & 20 & 100 & 120 & 5 & 20 & 100 & 120 \\
   \bottomrule[1pt]
   \end{tabular}
\end{center}

En la Taula \vref{taula:errors_ti_m2} s'indiquen els
límits, per a diversos percentatges de $I\ped{ns}$, dels errors
de corrent de les classes de precisió 3 i 5,  per a  càrregues
compreses entre $\SI{50}{\percent}\,S\ped{n}$ i $\SI{100}{\percent}\,S\ped{n}$.

\begin{center}
   \captionof{table}{Classes de precisió 3 i 5 per a Tc de mesura} \label{taula:errors_ti_m2}
   \begin{tabular}{c>{\hspace{2em}}cc}
   \toprule[1pt]
   Classe de & \multicolumn{2}{c}{Error de corrent} \\
   %\cmidrule(rl){2-3}
   precisió &  \multicolumn{2}{c}{\hspace{0.5em}\si{\percent}} \\
   \midrule
    3 & 3 & 3 \\
    5 & 5 & 5 \\
    \midrule
    $\si{\percent} I\ped{ns}$ & 50 & 120 \\
   \bottomrule[1pt]
   \end{tabular}
\end{center}


Existeixen també els valors normalitzats 0,2 S i  0,5 S, que mantenen la precisió per a valors baixos de corrent.
En la Taula \vref{taula:errors_ti_m3}
s'indiquen els límits, per  a diversos percentatges de
$I\ped{ns}$, dels errors de corrent i  de fase d'aquestes dues classes de
precisió,  per a càrregues compreses entre
$\SI{25}{\percent}\,S\ped{n}$ i $\SI{100}{\percent}\,S\ped{n}$.

\begin{center}
    \fontsize{9pt}{11pt}\selectfont
   \captionof{table}{Classes de precisió 0,2 S i 0,5 S per a Tc de mesura} \label{taula:errors_ti_m3}
   \begin{tabular}{cccccc<{\hspace{1em}}ccccc<{\hspace{1em}}ccccc}
   \toprule[1pt]
   \renewcommand*{\multirowsetup}{\centering}
   \multirow{2}{17mm}{\rule{0mm}{4.5mm}Classe de\\precisió} &
   \multicolumn{5}{c}{\multirow{2}{35mm}{\rule{0mm}{4.5mm}\rule{5mm}{0mm}Error de corrent\\ \rule{15mm}{0mm}\si{\percent}}} &
   \multicolumn{10}{c}{Error de fase} \\
   \cmidrule(rl){7-16}
    &  & & & & &\multicolumn{5}{c}{\hspace{-1em}minuts d'arc}  &
   \multicolumn{5}{c}{crad} \\
   \midrule
    0,2 S & 0,75 & 0,35 & 0,2 & 0,2 & 0,2 & 30 & 15 & 10 & 10 & 10 & 0,9 & 0,45 & 0,3 & 0,3 & 0,3 \\
    0,5 S& 1,5 & 0,75 & 0,5 & 0,5 & 0,5 & 90 & 45 & 30 & 30 & 30  & 2,7 & 1,35 & 0,9 & 0,9& 0,9 \\
    \midrule
    $\si{\percent} I\ped{ns}$ & 1 & 5 & 20 & 100 & 120 & 1 & 5 & 20 & 100 & 120 & 1 & 5 & 20 & 100 & 120 \\
   \bottomrule[1pt]
   \end{tabular}
\end{center}

En les tres taules anteriors, es considera que el factor de
potencia és igual a 1 quan la potència subministrada pel secundari és inferior a \SI{5}{VA}, i 0,8 inductiu per a valors de potència superiors. En qualsevol cas, la potència serà sempre superior a \SI{1}{VA}.


\subsection{Característiques particulars dels Tc de protecció}

Contràriament als transformadors de mesura, els transformadors de
protecció es dissenyen de manera que no se saturin fins a  valors
de sobrecorrents primaris elevats, ja que interessa que el
secundari segueixi reflectint el que passa en el primari per a
 sobrecorrents elevats (encara que sigui amb errors més grans), per
tal que els relés de protecció connectats al transformador actuïn
als valors de sobrecorrents a què estan ajustats.

Es presenten a continuació les característiques particulars dels Tc
de protecció.

\subsubsection{Corrent límit de precisió assignat ($I\ped{LP}$)}
\index{transformadors de mesura i protecció (Tc)!corrent límit de
precisió assignat ($I\ped{LP}$)}

El corrent
límit de precisió és el corrent primari màxim, per al qual el transformador manté el límit
de l'error compost que té assignat.

\subsubsection{Factor límit de precisió ($F\ped{LP}$) }
\index{transformadors de mesura i protecció (Tc)!factor límit de
precisió ($F\ped{LP}$)}

 El factor límit de precisió
es defineix com la relació entre el corrent límit de precisió
i el corrent primari nominal: $F\ped{LP} = I\ped{LP} /I\ped{np}$.
Els valors normalitzats són: 5, 10, 15, 20 i 30.

Mentre es compleixi  $I\ped{p}<F\ped{LP} I\ped{np}$, queda garantit
que el transformador no se saturarà, i per tant el corrent
secundari seguirà reflectint amb suficient precisió el valor del
corrent primari.

Cal tenir en compte que el valor de $F\ped{LP}$ està lligat
 al valor de $S\ped{n}$, i que només és vàlid
quan tenim aquest consum de  potència en el secundari; per a un
valor de potència $S'$ diferent de $S\ped{n}$, tindrem un valor
$F'\ped{LP}$ també diferent de  $F\ped{LP}$. La relació que
lliga aquests valors, tenint en compte la resistència del debanat
secundari del transformador  $R\ped{s}$ és:
\begin{equation}\label{eq:flp}
    F\ped{LP} (S\ped{n}+R\ped{s}I\ped{ns}^2) =
    F'\ped{LP} (S'+R\ped{s}I\ped{ns}^2)
\end{equation}

Valors típics per a la resistència del debanat secundari són:
\begin{itemize}
    \item Secundaris de \SI{5}{A}: $R\ped{s} = \SIrange{0,2}{0,4}{\ohm}$
    \item Secundaris de \SI{1}{A}:  $R\ped{s} = \SIrange{1,5}{3,5}{\ohm}$
\end{itemize}

\subsubsection{Classe de precisió}
\index{transformadors de mesura i protecció (Tc)!classe de precisió}

 Els valors normalitzats són 5P i 10P.

En la Taula \vref{taula:errors_ti_p} s'indiquen els límits dels
errors de corrent i de fase,  per al corrent nominal
$I\ped{ns}$ i  la càrrega de precisió nominal $S\ped{n}$,  amb un
factor de potencia 0,8 inductiu. S'indica també l'error
compost per al corrent $I\ped{LP}$.


\begin{center}
    \captionof{table}{Classes de precisió per a Tc de protecció} \label{taula:errors_ti_p}
    \begin{tabular}{ccccc}
    \toprule[1pt]
    \renewcommand*{\multirowsetup}{\centering}
    \multirow{2}{17mm}{\rule{0mm}{4.5mm}Classe de\\precisió} &
    \multirow{2}{30mm}{\rule{0mm}{4.5mm}Error de corrent\\ \rule{14mm}{0mm}\si{\percent}} &
    \multicolumn{2}{c}{Error de fase} &
    \multirow{2}{25mm}{\rule{0mm}{4.5mm}Error compost\\ \rule{12mm}{0mm}\si{\percent}}\\
    \cmidrule(rl){3-4}
    &   & minuts d'arc  & crad & \\
    \midrule
    5P & 1 & 60 & 1,8 & 5 \\
    10P & 3 & --- & --- & 10\\
    \midrule
    $I\ped{s}$ & $I\ped{ns}$ & $I\ped{ns}$ & $I\ped{ns}$ & $I\ped{LP}$ \\
    \bottomrule[1pt]
    \end{tabular}
\end{center}

La classe de precisió i el factor límit de precisió s'expressen
sempre de forma conjunta, per exemple 5P15 s'interpreta com: classe
de precisió 5P i $F\ped{LP}=15.$


\begin{exemple}[Determinació de les característiques d'un transformador de corrent]
    Es tracta de determinar els valors de $S\ped{n}$ i $F\ped{LP}$,  per
    a un Tc destinant a alimentar  un relé de protecció i un convertidor
    de corrent de \SIrange{4}{20}{mA}. Les característiques
    dels diferents components són:
    \begin{description}
        \item [\hspace{5mm}Tc:] Classe de precisió  5P, $I\ped{ns}=\SI{5}{A}$,
        $R\ped{s}=\SI{0,3}{\ohm}$
        \item [\hspace{5mm}Relé:] $S\ped{n,rel\acute{e}}=\SI{0,25}{VA}$,
        $I\ped{n,rel\acute{e}}=\SI{5}{A}$, $I\ped{m\acute{a}x,rel\acute{e}}=
        100 I\ped{n,rel\acute{e}}$
        \item [\hspace{5mm}Convertidor:] $S\ped{n,conv}=\SI{1}{VA}$,
        $I\ped{n,conv}=\SI{5}{A}$, $I\ped{m\grave{a}x,conv}=
        80 I\ped{n,conv}$
        \item [\hspace{5mm}Cables de connexió:] Càrrega = $\SI{1,6}{VA}$
    \end{description}

    La potència total $S'$ que està connectada al secundari del transformador
    és:
    \[
        S' = \SI{0,25}{VA} + \SI{1}{VA} + \SI{1,6}{VA} = \SI{2,85}{VA}
    \]

    Prenem per calcular el factor límit de precisió  a aquesta potencia $F'\ped{LP}$, el
    corrent màxim que pot suportar el convertidor, ja que és menor que el corrent màxim que pot suportar el relé; així doncs tenim:
    \[
        F'\ped{LP} = \frac{80 I\ped{n,conv}}{I\ped{ns}} =
        \frac{80\times \SI{5}{A}}{\SI{5}{A}} = 80
    \]

    Si apliquem ara l'equació \eqref{eq:flp}, tenim:
    \begin{align*}
        F\ped{LP}\times(S\ped{n}+\SI{0,3}{\ohm} \times (\SI{5}{A})^2) &=
        80\times(\SI{2,85}{VA}+\SI{0,3}{\ohm} \times (\SI{5}{A})^2) \\
        F\ped{LP}\times(S\ped{n}+\SI{7,5}{VA}) &= \SI{828}{VA}
    \end{align*}

    Escollim a continuació el valor normalitzat $S\ped{n}=
    \SI{15}{VA}$, i calculem $F\ped{LP}$:
    \[
        F\ped{LP} = \frac{\SI{828}{VA}}{\SI{15}{VA}+\SI{7,5}{VA}}
        = \num{36,8}
    \]

    Finalment, escollim el valor normalitzat immediatament inferior al valor
    de càlcul obtingut: $F\ped{LP} = 30$, i
    recalculem el valor $F'\ped{LP}$ que tindrem realment:
    \[
    F'\ped{LP} = \frac{30\times(\SI{15}{VA} + \SI{7,5}{VA})}
    {\SI{2,85}{VA} + \SI{7,5}{VA}} = 65 < 80
    \]

    El valor resultant és per tant acceptable. Així doncs les
    característiques buscades del transformador són: $\SI{15}{VA}$ 5P30.
\end{exemple}


\section{Resum de característiques segons les normes CEI 60044}\index{CEI!60044-0@60044}

Es resumeix a continuació les característiques que apareixen en la placa de característiques dels transformador de mesura i protecció. La paraula
«classe» s'abrevia a «cl.»:

\begin{itemize}
   \item \textbf{Tt}: Tensions nominals  primària ($U\ped{np}$) i secundària ($U\ped{ns}$), freqüència nominal ($f\ped{n}$),
    potència nominal ($S\ped{n}$), factor de tensió i     designació dels terminals. Addicionalment tenim:
       \begin{itemize}
           \item \textbf{Tt de mesura}: Classe de precisió, per  exemple: cl.~0,5.
           \item \textbf{Tt de protecció}: Classes de precisió, per  exemple: cl.~0,5 3P.
        \end{itemize}
    \item \textbf{Tc}: Corrents nominals primari ($I\ped{np}$) i secundari ($I\ped{np}$), freqüència nominal ($f\ped{n}$),
     potència nominal ($S\ped{n}$),  corrents tèrmic nominal de curta durada ($I\ped{th}$), dinàmic nominal ($I\ped{dyn}$) i tèrmic nominal continu ($I\ped{cth}$), i     designació dels terminals. Addicionalment tenim:
        \begin{itemize}
           \item \textbf{Tc de mesura}: Classe de precisió i factor de seguretat, per exemple: cl.~0,5 $F\ped{S} 10$
           \item \textbf{Tc de protecció}: Classe i factor límit de precisió. Normalment s'expressen de forma conjunta, i s'omet l'abreviatura «cl.»,  per exemple: 5P15.
        \end{itemize}
\end{itemize}


\section{Característiques dels transformadors de tensió segons la norma IEEE C57.13}
\index{transformadors de mesura i protecció (Tt)}\index{IEEE!C57.13}

\subsubsection{Tensió}

El valor estàndard de la tensió de
secundari és \SI{120}{V}, amb un rang de tensions que pot anar de \SIrange{108}{132}{V}. Aquests valors es divideixen per $\sqrt{3}$ en el cas de transformadors connectats entre fase i terra.


\subsubsection{Classe i potència de precisió}

Els Tt es designen a partir dels dos
elements indicats a continuació.

\begin{dingautolist}{'312}
    \item \textbf{Classe de precisió}: Aquest concepte és equivalent
    a l'utilitzat en les normes CEI. Els valors
    normalitzats són: cl.~0,3, 0,6, 1,2 i 2,4.

    \item \textbf{Potència de precisió}: Aquest concepte és equivalent
    a l'utilitzat en les normes CEI. Els valors
    normalitzats es designen mitjançant lletres, i es poden veure en
    la Taula \vref{taula:s_ieee_tt}.

    \begin{center}
        \captionof{table}{Potències IEEE de precisió  per a Tt}\label{taula:s_ieee_tt}
        \begin{tabular}{ccc}
        \toprule[1pt]
        Lletra de & Potència de & $\cos\varphi$\\
        designació &  precisió /VA &  (inductiu)\\
        \midrule
            W & 12,5 & 0,10\\
            X & 25 & 0,70 \\
            Y & 75 & 0,85 \\
            Z & 200 & 0,85 \\
            ZZ & 400 & 0,85 \\
            M & 35 & 0,20 \\
        \bottomrule[1pt]
        \end{tabular}
    \end{center}
\end{dingautolist}

Aquests dos elements s'expressen de forma conjunta, per exemple:
1,2Y.

\subsubsection{Identificació dels terminals}

Els terminal s'identifiquen amb lletres. S'utilitza la lletra H per designar els terminals del primari, i la lletra X per designar els terminals del secundari (i també la Y, Z, U, W, V,  etc.{}, en el cas de múltiples secundaris); cada terminal estarà numerat, per exemple: H\ped{1}, H\ped{2},  X\ped{1}, X\ped{2}. Els terminal homòlegs són H\ped{1} i X\ped{1} (i també Y\ped{1}, Z\ped{1}, U\ped{1}, W\ped{1}, V\ped{1}, etc.{}, en el cas de múltiples secundaris).

En el cas de múltiples primaris, els terminals es designen amb la lletra H, numerant-los per parelles (H\ped{1}, H\ped{2}, H\ped{3}, H\ped{4}, etc.{}). Els terminals senars són terminals homòlegs.

Quan els secundaris tenen preses múltiples, el terminals s'identifiquen com X\ped{1}, X\ped{2}, X\ped{3}, etc.{}, (o Y\ped{1}, Y\ped{2}, Y\ped{3}, etc.{}, Z\ped{1}, Z\ped{2}, Z\ped{3}, etc.{}). Quan el terminal X\ped{1} no s'utilitza, el terminal utilitzat amb el menor número és l'homòleg del terminal primari; per exemple, un transformador amb un primari H\ped{1}, H\ped{2}, i un secundari  X\ped{1}, X\ped{2}, X\ped{3}, X\ped{4} i X\ped{5}, on els terminals secundaris utilitzats són els X\ped{2} i  X\ped{4}, els terminals homòlegs són H\ped{1} i X\ped{2}.


\section{Característiques dels transformadors de corrent segons la norma IEEE C57.13}
\index{transformadors de mesura i protecció (Tt)}\index{IEEE!C57.13}


\subsection{Tc de mesura}

Els Tc de mesura  es designen a partir
dels tres elements indicats a continuació.

\begin{dingautolist}{'312}
    \item \textbf{Classe de precisió}: Aquest concepte és equivalent
    a l'utilitzat en les normes CEI. Els valors
    normalitzats són: cl.~0,3, 0,6, 1,2 i 2,4.
    \item \textbf{La lletra «B»}:\index{B} És la inicial de la paraula
    «burden»  (càrrega).\index{burden@\guillemotleft burden\guillemotright}
    \item \textbf{Càrrega de precisió}: Aquest concepte és equivalent
    a l'utilitzat en les normes CEI. Els valors
    normalitzats són: $Z\ped{ns}$ = \SIlist{0,1;0,2;0,5;0,9;1,8}{\ohm}.

    La potència de precisió es pot calcular a partir del
    corrent  nominal secundari $I\ped{ns}$, utilitzant l'equació
    \eqref{eq:sn_ti}.
\end{dingautolist}

Aquests tres elements s'expressen de forma conjunta, per exemple:
0,3B0,2.

\subsection{Tc de protecció}

Els Tc de protecció es designen a
partir dels tres elements indicats a continuació.

\begin{dingautolist}{'312}
    \item \textbf{Error compost}: Indica l'error compost màxim (en tant per cent) del
    transformador, quan el corrent que circula pel
    transformador és 20 vegades el corrent nominal. Aquest concepte
     és equivalent a la classe de precisió de la norma CEI,
     amb $F\ped{LP}=20$. Només s'utilitza amb els transformadors antics (tipus «L» o «H»); en el cas del transformadors actuals
     (tipus «C», «K» o «T»), l'error és sempre el \SI{10}{\percent}, i no s'indica.

    \item \textbf{Les lletres «C», «K», «T», «L» o «H»}: La lletra «C» és la inicial de la
    paraula  «calculated» (calculada). El flux de dispersió d'aquests transformadors és negligible, i el seu error es pot calcular.\index{C}\index{calculated@\guillemotleft calculated\guillemotright}

    La lletra «K» és equivalent a la «C», però la tensió del colze de la corba d'excitació ha de ser com a mínim el \SI{70}{\percent}
    de la tensió nominal de secundari.\index{K}

    La lletra «T» és la inicial de la   paraula  «tested» (mesurada). El flux de dispersió d'aquests transformadors és apreciable, i el seu error només es pot obtenir mitjançant un assaig.\index{T}\index{tested@\guillemotleft tested\guillemotright}

    Les lletres «L» i «H» són denominacions antigues,  no utilitzades actualment. La lletra «L» és la inicial de «low leakage» (baixa
    dispersió), i la lletra «H» és la inicial de «high leakage» (alta dispersió).\index{L}\index{low leakage@\guillemotleft low leakage\guillemotright}\index{H}\index{high leakage@\guillemotleft high leakage\guillemotright}

    \item \textbf{Tensió nominal de secundari}: És la tensió màxima
    que hi pot haver en el secundari, per tal de no sobrepassar l'error compost que té
    assignat el transformador, quan el corrent que hi circula
     és 20 vegades el corrent nominal. Els valors
    normalitzats són: \SIlist{10; 50; 100; 200; 400; 800}{V}.

    La càrrega de precisió en el secundari
    $Z\ped{ns}$ i la potència de precisió $S\ped{n}$, s'obtenen a partir d'aquesta
    tensió màxima de secundari $U\ped{m\grave{a}x,s}$
    i del corrent     nominal de secundari $I\ped{ns}$, segons les equacions següents:
    \begin{align}
        Z\ped{ns} &= \frac{U\ped{m\grave{a}x,s}}{20 I\ped{ns}}\\
        S\ped{n} &= Z\ped{ns} \,I\ped{ns}^2 = \frac{U\ped{m\grave{a}x,s} I\ped{ns}}{20}
        \label{eq:sn_ti_ieee}
    \end{align}
\end{dingautolist}

Aquests dos o tres elements s'expressen de forma conjunta, per exemple:
10L200 o C400.


\begin{exemple}[Equivalència entre transformadors IEEE i CEI]
    Es tracta de trobar els transformadors equivalents, segons les normes CEI, als dos
    transformadors següents, donats segons les nomes IEEE: 0,3B0,2 i
    C50; el corrent nominal de secundari és:    $I\ped{ns}=\SI{5}{A}$.

    En el primer cas, tenim de forma directa: cl.~0,3; la potència de precisió la trobem
    aplicant l'equació \eqref{eq:sn_ti}:
    \[
        S\ped{n} =(\SI{5}{A})^2 \times \SI{0,2}{\ohm} =  \SI{5}{VA}
    \]
    Atès que 0,3 no és una classe de precisió CEI normalitzada,
    escolliríem un transformador de característiques: $\SI{5}{VA}$ cl.~0,2; caldria a més, definir el factor de
    seguretat apropiat per a la nostra aplicació, amb l'ajut de l'equació \eqref{eq:fs}.

    En el segon cas, tenim de forma directa la classe i el factor límit de
    precisió: 10P20; la potència de precisió la trobem
    aplicant l'equació \eqref{eq:sn_ti_ieee}:
    \[
        S\ped{n} = \frac{\SI{50}{V} \times\SI{5}{A}}{20} = \SI{12,5}{VA}
    \]
    Atès que \SI{12,5}{VA} no és una potència de precisió CEI normalitzada,
     escolliríem un transformador de característiques:
    \SI{15}{VA} 10P20; caldria a més, comprovar el factor límit de precisió real
    que tindrem en la nostra aplicació, utilitzant l'equació \eqref{eq:flp}.
\end{exemple}

\section{Connexió de Tc i Tt a aparells de mesura o de
protecció}\label{sec:conex_ti_tt}\index{transformadors de mesura i
protecció!connexió}

A vegades es presenta la necessitat de connectar un nou aparell de
mesura o de protecció en una instaŀlació existent, on els
transformadors de tensió i corrent ja estan muntats i connectats a
altres aparells. En aquest cas, cal parar atenció a la connexió
que ens demana el nou aparell que volem instaŀlar, per tal de no
equivocar-nos.

La connexió dels Tt a un nou aparell sol ser simple, ja que només
cal veure a quin terminal de l'aparell cal connectar cadascuna de
les tensions (fases R, S i T), i reproduir aquesta connexió en la
nostra instaŀlació.

La connexió dels Tc a un nou aparell demana una mica més
d'atenció, ja que a més de saber a  quins terminals de l'aparell hem
de connectar els corrents (de les fases R, S i T), hem de fixar-nos
en els sentits de circulació d'aquests corrents que ens demana
l'aparell, i mantenir-los quan incorporem l'aparell a la nostra
instaŀlació.

 La manera de no equivocar-se, és suposar un sentit de
circulació arbitrari del corrent  pel primari del Tc (per exemple,
de la font de tensió cap a la càrrega), i veure a continuació fent
servir els terminals homòlegs P1-S1 i P2-S2, quin és el sentit de
circulació del corrent en el secundari del Tc cap a l'aparell;
aquest sentit és el que haurem de respectar en la nostra
instaŀlació quan hi afegim el nou aparell.


\begin{exemple}[Connexió d'un wattímetre a una instaŀlació existent]
    Es representa a continuació la connexió d'un wattímetre, extreta
    d'un catàleg.

    El costat del circuit primari on es troben les càrregues, ve indicat
    per les tres línies amb una creu al mig.

    \begin{center}
        \input{Imatges/Cap-TrafosMesProt-Watt.pdf_tex}
    \end{center}

    A continuació es representa una instaŀlació existent, amb dos Tt i
    dos Tc, que alimenten a dos voltímetres i a dos amperímetres
    respectivament; les càrregues es troben a la dreta del circuit
    primari.

    \begin{center}
        \input{Imatges/Cap-TrafosMesProt-Instal.pdf_tex}
    \end{center}

    Es tracta d'afegir el nou wattímetre a aquesta
    instaŀlació.

    La connexió completa amb els dos voltímetres, els dos amperímetres i el wattímetre, es pot veure a continuació; la manera de fer-la es detalla ara pas a pas:

    \begin{dingautolist}{'312}

    \item Comencem fixant-nos en les tensions del wattímetre, i veiem que cal
    connectar-li la tensió de la fase R al terminal 1, la tensió de la
    fase S al terminal 3, i la tensió de la fase T al terminal 2.

    Per aconseguir-ho en la nostra instaŀlació, sense tocar la
    connexió dels dos voltímetres existents, només cal connectar
    el terminal «\textsf{a}» del primer Tt al terminal 1 del wattímetre (tensió de
    la fase R), el terminal «\textsf{a}» del segon Tt al terminal 3 del wattímetre
    (tensió de la fase S), i el terminal «\textsf{b}» d'un dels dos Tt
    al terminal 2 del wattímetre (tensió de la fase T).

    \item Ens fixem a continuació en els corrents del wattímetre. Si suposem
    de forma arbitrària, uns corrents pels circuits primaris dels Tc
    que vagin d'esquerra a dreta (això és, cap a les càrregues), veiem
    que aquests corrents entren pels terminals «\textsf{P1}» dels primaris dels Tc,
    i per tant surten, transformats, pels terminals «\textsf{S1}» dels secundaris
    dels Tc. Així doncs, el corrent que circula pel secundari del primer
    Tc, entra al wattímetre pel terminal 4, i en surt pel terminal 5, i
    el corrent que circula pel secundari del segon Tc, entra al
    wattímetre pel terminal 6, i en surt pel terminal 7. Aquest sentit
    de circulació dels corrents és el que hem de mantenir quan
    connectem el wattímetre a la nostra instaŀlació.

    Per aconseguir-ho en la nostra instaŀlació, sense tocar la
    connexió dels dos amperímetres existents, comencem per suposar un
    sentit dels corrents primaris idèntic al suposat anteriorment, és a
    dir cap a les càrregues (això és, d'esquerra a dreta). L'objectiu
    serà veure el sentit de circulació dels corrents de secundari
    respecte dels terminals «\textsf{S1}» dels dos Tc, ja que disposem d'un fil per
    a cadascun dels dos terminals de forma separada; no passa el mateix
    amb els dos terminals «\textsf{S2}», ja que únicament disposem d'un fil pel
    qual circula la suma dels dos corrents. Per tant, veiem que amb el
    sentit de circulació que hem adoptat, aquests corrents surten pels
    terminals «\textsf{P1}» dels primaris dels Tc, i per tant entren, transformats,
    pels terminals «\textsf{S1}» dels secundaris dels Tc.

    Aquests corrents que entren pels terminals «\textsf{S1}», i que hem de dur al
    wattímetre, seran corrents que vistos des del wattímetre, en
    sortiran; si ens fixem en l'anàlisi que hem fet en el
    circuit inicial del wattímetre, veiem que els terminal per on surten
    els corrents són el 5 i el 7. Per tant ara queda clar que hem de
    connectar el terminal «\textsf{S1}» del primer Tc, després de passar per
    l'amperímetre \textsf{A1}, al terminal 5 del wattímetre, i el
    terminal «\textsf{S1}» del segon Tc, després de passar per l'amperímetre
    \textsf{A2}, al terminal 7 del wattímetre. Finalment, només ens cal
    tancar el circuit dels corrents secundaris, unint entre si els dos
    terminals d'entrada 4  i 6, i connectant-los amb el fil comú que
    uneix els dos terminals «\textsf{S2}» dels dos Tc.

    Com es pot veure, no té cap incidència sobre la connexió quin és
    el terminal del secundari que està connectat a terra ni en el cas
    dels Tt ni en el cas dels Tc.
    \end{dingautolist}

     \begin{center}
        \input{Imatges/Cap-TrafosMesProt-Instal-Watt.pdf_tex}
    \end{center}

\end{exemple}
