\chapter{Constants Físiques}\label{sec:const_fis} \index{constants!físiques}

\section{Taula de valors}

En la taula \vref{taula:Const-Fis} es pot veure una recopilació de
constants físiques; les xifres entre parèntesis que hi apareixen representen l'error absolut del valor mesurat. Els valors de les constants d'aquesta taula són els recomanats
l'any 2022 pel \textit{Committee on Data for Science and Technology}
(CODATA), un comitè científic de l'\textit{International Council
for Science} (ISC). 


\begin{ThreePartTable}
\begin{TableNotes}
    \item[\color{blue}(a)] {\footnotesize El valor numèric en si, és l'anomenat nombre d'Avogadro.}
    \item[\color{blue}(b)] {\footnotesize $K\ped{cd}$ és l'eficàcia lluminosa d'una radiació monocromàtica de freqüència \qty{540e12}{Hz}.}
    \item[\color{blue}(c)] {\footnotesize Donada una partícula X, $m(\mathrm{X})$ és la massa atòmica de la partícula X, $M(\mathrm{X})$ és la massa molar de la partícula X, i $A\ped{r}(\mathrm{X})$ és la massa atòmica relativa de la partícula X. Es compleixen les relacions següents: $M(\mathrm{X}) = m(\mathrm{X})\, N\ped{A}$ i $ A\ped{r}(\mathrm{X}) = \frac{m(\mathrm{X})}{m\ped{u}} = \frac{M(\mathrm{X})}{M\ped{u}}$.}
    \item[\color{blue}(d)] {\footnotesize Un «electró-volt» és l'energia cinètica que adquireix un electró després de creuar una diferència de potencial d'un volt en el buit.}
    \item[\color{blue}(e)] {\footnotesize El deuteri ---també anomenat hidrogen-2 o hidrogen pesat--- és un isòtop estable de l'hidrogen, format per un protó, un neutró i un electró.}
    \item[\color{blue}(f)] {\footnotesize El triti ---també anomenat hidrogen-3--- és un isòtop inestable de l'hidrogen, format per un protó, dos neutrons i un electró.}
    \item[\color{blue}(g)] {\footnotesize La partícula $\alphaup$ és el nucli atòmic de l'heli-4, format per dos protons i dos neutrons, que s'emet a gran velocitat en la desintegració radioactiva de núclids generalment pesants.}
\end{TableNotes}
\begin{longtable}{lcll}
	\caption{\label{taula:Const-Fis} Constants físiques --- CODATA 2022}\\
	\toprule[1pt]
	Magnitud & Símbol & Valor & Error relatiu\\
	\midrule
	\endfirsthead
	\caption[]{Constants físiques --- CODATA 2022 (\emph{ve de la pàgina anterior})} \\
	\toprule[1pt]
	Magnitud & Símbol & Valor & Error relatiu\\
	\midrule
	\endhead
	\midrule
	\multicolumn{4}{r}{\sffamily\bfseries\color{NavyBlue}(\emph{continua a la pàgina següent})}
	\endfoot
	\insertTableNotes
	\endlastfoot
	freqüència de la transició & $\Deltaup\nu\ped{Cs}$ & \qty{9 192 631 770}{Hz} & valor exacte \\
	hiperfina del cesi-133 & & & \index{$\Deltaup\nu\ped{Cs}$}\\[1em]
	velocitat de la llum en el buit & $c$ & \qty{299 792 458}{m/s} & valor exacte \index{c@$c$}\\[1em]
	constant de Planck\footnote{Max  Planck, físic alemany: \href{https://en.wikipedia.org/wiki/Max_Planck}{https:/\!\!/en.wikipedia.org/wiki/Max\_Planck}.} & $h$ & \qty{6,62607015 e-34}{J/Hz} & valor exacte\index{constants!de Planck}\index{h@$h$}\\[1em]
	càrrega elemental & $e$ & \qty{1,602176634 e-19}{C} & valor exacte\index{carrega@càrrega!elemental}\index{e@$e$}\\[1em]
	constant de Boltzmann\footnote{Ludwig  Boltzmann, físic austríac: \href{https://en.wikipedia.org/wiki/Ludwig_Boltzmann}{https:/\!\!/en.wikipedia.org/wiki/Ludwig\_Boltzmann}.} & $k$ & \qty{1,380649e-23}{J/K} & valor exacte\index{constants!de Boltzmann}\index{k@$k$}\\[1em]
	constant d'Avogadro\tnote{\color{blue}(a)}\hspace{4mm}\footnote{Amedeo Avogadro, científic italià: \href{https://en.wikipedia.org/wiki/Amedeo_Avogadro}{https:/\!\!/en.wikipedia.org/wiki/Amedeo\_Avogadro}.} & $N\ped{A}$ & \qty{6,02214076 e23}{mol^{-1}} & valor exacte\index{constants!d'Avogadro}\index{NA@$N\ped{A}$}\\[1em]
	eficàcia lluminosa\tnote{\color{blue}(b)} & $K\ped{cd}$ & \qty{683}{lm/W} & valor exacte\index{eficàcia lluminosa}\index{Kcd@$K\ped{cd}$}\\[1em]
	massa atòmica relativa\tnote{\color{blue}(c)} & $A\ped{r}({}^{12}\mathrm{C})\,$ & 12 & valor exacte\\
	del carboni-12 & & &\index{massa!atòmica relativa del carboni-12}\index{Ar@$A\ped{r}({}^{12}\mathrm{C})$}\\[1em]
	constant de Planck reduïda: $\frac{h}{2\piup}$ & $\hslash$ & \qty{1,054571817\dots e-34}{J.s} & valor exacte\index{constants!de Planck redu\"{i}da}\index{h@$\hslash$}\\[.8em]
	constant d'Stefan-Boltzmann:\footnote{Josef Stefan, matemàtic i físic eslovè: \href{https://en.wikipedia.org/wiki/Josef_Stefan}{https:/\!\!/en.wikipedia.org/wiki/Josef\_Stefan}.}  & $\sigma$ & \qty{5,670374419\dots e-8}{W/(m^2.K^4)} & valor exacte\index{constants!d'Stefan-Boltzmann}\index{$\sigma$}\\ 
	$\frac{\piup^2 k^4}{60\, \hslash^3 c^2}$ & & & \\[0.8em]
	constant de Josephson:\footnote{Brian Josephson, físic britànic: \href{https://en.wikipedia.org/wiki/Brian_Josephson}{https:/\!\!/en.wikipedia.org/wiki/Brian\_Josephson}.} $\frac{2 e}{h}$ & $K\ped{J}$ & \qty{483597,8484\dots e9}{Hz/V} & valor exacte\index{constants!de Josephson}\index{KJ@$K\ped{J}$}\\[0.9em]
	electró-volt\tnote{\color{blue}(d)} & eV & \qty{1,602176634e-19}{J} & valor exacte\index{electró-volt}\index{eV}\\[0.6em]
	atmosfera estàndard  & -- & \qty{101325}{Pa} & valor exacte\index{atmosfera estàndard}\\[0.8em]
	acceleració de la gravetat & $g\ped{n}$ & \qty{9,80665}{m/s^2} & valor exacte \\
	estàndard & & &\index{acceleració!de la gravetat estàndard}\index{gn@$g\ped{n}$}\\[0.9em]
	constant molar dels gasos: $N\ped{A} k$ & $R$ & \qty{8,31446261815324}{\,J/(mol.K)} & valor exacte\index{constants!molar dels gasos}\index{R@$R$}\\[0.8em]
	constant de Faraday:\footnote{Michael Faraday, físic britànic: \href{https://en.wikipedia.org/wiki/Michael_Faraday}{https:/\!\!/en.wikipedia.org/wiki/Michael\_Faraday}.} $N\ped{A} e$ & $F$ & \qty{96485,3321233100184}{C/mol} & valor exacte\index{constants!de Faraday}\index{F@$F$}\\[0.9em]
	volum molar d'un gas ideal: $\frac{R T}{p}$  & $V\ped{m}$ & \qty{22,413969 54\dots e-3}{m^3/mol} & valor exacte\index{volum molar d'un gas ideal}\index{Vm@$V\ped{m}$}\\
	{\small$(T=\qty{273,15}{K}, p=\qty{101325}{Pa})$} & & & \\[0.8em]	
	constant de Loschmidt:\footnote{Johann Josef Loschmidt, físic i químic austríac: \href{https://en.wikipedia.org/wiki/Johann_Josef_Loschmidt}{https:/\!\!/en.wikipedia.org/wiki/Johann\_Josef\_Loschmidt}.} $\frac{N\ped{A}}{V\ped{m}}$  & $n_0$ & \qty{2.686 780 111 \dots e25}{m^{-3}} & valor exacte\index{constants!de Loschmidt}\index{n0@$n_0$}\\
	{\small$(T=\qty{273,15}{K}, p=\qty{101325}{Pa})$} & & & \\[0.8em]	
	constant de massa atòmica:\tnote{\color{blue}(c)} & $m\ped{u}$ & \qty{1,66053906892(52)e-27}{kg} & \num{3.1e-10}\\ 
	$\frac{1}{12}  m({}^{12}\mathrm{C})$ & & &\index{constants!de massa atòmica}\index{mu@$m\ped{u}$}\\[0.9em]
	constant de massa molar:\tnote{\color{blue}(c)} & $M\ped{u}$ & \qty{1,00000000105(31)e-3}{kg/mol} & \num{3.1e-10}\index{constants!de massa molar}\index{Mu@$M\ped{u}$}\\ 
	$N\ped{A} m\ped{u}$ & & & \\[0.8em]
	massa molar del carboni-12:\tnote{\color{blue}(c)} & $M({}^{12}\mathrm{C})\,$ & \qty{12,0000000126(37)e-3}{kg/mol} & \num{3,1e-10}\index{massa!molar del carboni-12}\index{M@$M({}^{12}\mathrm{C})$}\\
	$A\ped{r}({}^{12}\mathrm{C})\,M\ped{u} $  & & & \\[0.9em]
	constant gravitacional & $G$ &   \qty{6,67430(15) e-11}{m^3/(kg.s^2)} & \num{2,2e-5}\\
	de Newton\footnote{Isaac Newton, matemàtic i físic britànic: \href{https://en.wikipedia.org/wiki/Isaac_Newton}{https:/\!\!/en.wikipedia.org/wiki/Isaac\_Newton}.} & & &\index{constants!gravitacional de Newton}\index{G@$G$}\\[0.9em]
	constant de l'estructura & $\alpha$ & \num{7,2973525643(11) e-3} & \num{1,6e-10} \\
	fina: $\frac{e^2}{4\piup\epsilon_0\hslash  c}$ & & &\index{constants!de l'estructura fina}\index{$\alphaup$}\\[0.9em]
	permeabilitat magnètica & $\mu_0$ & \qty{1,25663706127(20) e-6}{N/A^2} & \num{1,6e-10}\index{permeabilitat!del buit}\index{$\mu_0$}\\ 
	del buit: $\frac{4\piup\alpha\hslash}{e^2  c}$  & & & \\[0.9em]
	permitivitat  elèctrica  & $\epsilon_0$ & \qty{8,8541878188(14) e-12}{F/m} & \num{1,6e-10}\index{permitivitat!del buit}\index{$\epsilon_0$}\\ 
	del buit: $\frac{1}{\mu_0 c^2}$ & & & \\[0.9em]
	impedància característica  & $Z_0$ &  \qty{376,730313412(59)}{\ohm} & \num{1,6e-10}\\
	del buit: $\sqrt{\frac{\mu_0}{\epsilon_0}}=\mu_0 c$ & & & \index{impedància característica del buit}\index{Z0@$Z_0$}\\[0.9em]
	massa de l'electró & $m\ped{e}$ & \qty{9,1093837139(28) e-31}{kg} & \num{3,1e-10}\index{massa!de l'electró}\index{me@$m\ped{e}$}\\[0.9em]
	massa del  protó & $m\ped{p}$ & \qty{1,67262192595(52) e-27}{kg} & \num{3,1e-10}\index{massa!del protó}\index{mp@$m\ped{p}$}\\[0.9em]
	massa del neutró & $m\ped{n}$ & \qty{1,67492750056(85) e-27}{kg} & \num{5,1e-10}\index{massa!del neutró} \index{mn@$m\ped{n}$}\\[0.9em]
	massa del deuteri\tnote{\color{blue}(e)} & $m\ped{d}$ & \qty{3,3435837768(10) e-27}{kg} & \num{3,1e-10}\index{massa!del deuteri}\index{md@$m\ped{d}$}\\[0.9em]
	massa del triti\tnote{\color{blue}(f)} & $m\ped{t}$ & \qty{5,0073567512(16) e-27}{kg} & \num{3,1e-10}\index{massa!del triti}\index{md@$m\ped{t}$}\\[0.9em]
	massa de la partícula $\alphaup$\tnote{\color{blue}(g)} & $m_\alphaup$ & \qty{6,6446573450(21) e-27}{kg} & \num{3,1e-10}\index{massa!de la partícula $\alpha$}\index{ma@$m_\alpha$}\\[0.9em]
	radi de Bohr:\footnote{Niels Bohr, físic danès: \href{https://en.wikipedia.org/wiki/Niels_Bohr}{https:/\!\!/en.wikipedia.org/wiki/Niels\_Bohr}.} $\frac{\hslash}{\alpha m\ped{e}c} = \frac{4\piup\epsilon_0\hslash^2}{m\ped{e}e^2}$ & $a_0$ & \qty{5.29177210544(82) e-11}{m} & \num{1,6e-10}\index{radi de Bohr}\index{a0@$a_0$}\\[0.9em]
	radi clàssic de l'electró: $\alpha^2 a_0$ & $r\ped{e}$ & \qty{2,817 940 3205(13)e-15}{m} & \num{4,7e-10}\index{radi clàssic de l'electró}\index{re@$r\ped{e}$} \\[0.9em]
	longitud d'ona Compton\footnote{Arthur Compton, físic americà: \href{https://en.wikipedia.org/wiki/Arthur_Compton}{https:/\!\!/en.wikipedia.org/wiki/Arthur\_Compton}.}   & $\lambdabar\ped{C}$ & \qty{3,8615926744(12) e-13}{m} & \num{3,1e-10} \\
	reduïda: $\frac{\hslash}{m\ped{e} c} = \alpha a_0$ & & &\index{longitud!d'ona Compton reduïda}\index{$\lambdabar\ped{C}$}\\[0.6em]
	massa de Planck: $\sqrt{\frac{\hslash c}{G}}$ & $m\ped{P}$ & \qty{2,176434(24)e-8}{kg} & \num{1,1e-5}\index{massa!de Planck}\index{mp/P@$m\ped{P}$}\\[0.6em]
	temperatura de Planck: $\frac{\sqrt{\frac{\hslash c^5}{G}}}{k}$ & $T\ped{P}$ & \qty{1,416784(16)e32 }{K} & \num{1,1e-5}\index{temperatura de Planck}\index{TP@$T\ped{P}$}\\[0.6em]
	longitud de Planck: $\frac{\hslash}{m\ped{P} c} = \sqrt{\frac{\hslash G}{c^3}}$ & $l\ped{P}$ & \qty{1,616255(18)e-35}{m} & \num{1,1e-5}\index{longitud!de Planck}\index{lP@$l\ped{P}$}\\[0.9em]
	temps de Planck: $\frac{l\ped{P}}{c} = \sqrt{\frac{\hslash G}{c^5}}$ & $t\ped{P}$ & \qty{5,391247(60)e-44}{s} & \num{1,1e-5} \index{temps de Planck}\index{tP@$t\ped{P}$}\\[0.6em]
	\bottomrule[1pt]
\end{longtable}
\end{ThreePartTable}

 
Els valors d'aquesta taula  han anat canviant amb els anys, en esdevenir cada vegada més precises les mesures d'aquestes constants. Els valors anteriors recomanats van ser els dels anys 1969, 1973, 1986, 1998, 2002, 2006, 2010, 2014 i 2018.   Tota la informació referent a aquestes constants es pot trobar   a
les adreces: \href{https://codata.org//}{https:/\!\!/codata.org} i \href{https://physics.nist.gov/cuu/Constants/}{https:/\!\!/physics.nist.gov/cuu/Constants}.\index{CODATA}
\index{NIST}


A partir de l'any 2018 hi va haver un canvi important en el tractament d'algunes constants respecte de les recomanades anteriorment pel CODATA l'any 2014; això es va deure al fet que les definicions de les constants de l'any 2018 es van fer d'acord amb la renovació de les definicions de les unitats de l'SI que va tenir lloc l'any 2019.\footnote{Vegeu l'apèndix \ref{sec:SI}} La qüestió més remarcable d'aquest canvi, és que algunes constants que abans tenien un valor mesurat amb una certa precisió, es van redefinir amb un valor exacte; en aquesta categoria figuren les constants de Planck, Boltzmann i Avogadro, la càrrega fonamental, l'eficàcia lluminosa, la freqüència de transició hiperfina del cesi-133 i la massa atòmica relativa del carboni-12. Aquestes redefinicions van ocasionar l'efecte contrari en altres constants que abans tenien un valor exacte i que van passar a tenir un valor mesurat amb una certa precisió; en aquesta categoria figuren, per exemple, la permeabilitat  magnètica del buit (que abans tenia el valor exacte \qty[parse-numbers = false]{4 \pi \times 10^{-7}}{N/A^2}), la permitivitat elèctrica del buit, i la impedància característica del buit.


\section{Errors absolut i relatiu}\label{sec:err_abs_rel}

Tal com s'ha dit al principi, les xifres entre parèntesis indiquen l'error absolut del valor mesurat que les precedeix.\footnote{Més exactament, el valor entre parèntesis és la incertesa estàndard del valor mesurat, segons el document JCGM 100:2008 (GUM 1995 \textit{with minor
corrections}), \textit{Evaluation of measurement data --- Guide to the expression of uncertainty in
measurement}, el qual es pot obtenir a:  \href{https://www.bipm.org/documents/20126/2071204/JCGM_100_2008_E.pdf/}{https:/\!\!/www.bipm.org/documents/20126/2071204/JCGM\_100\_2008\_E.pdf}.} El nombre de xifres entre parèntesis determina la posició decimal d'aquest error; per exemple, en el cas de la  massa de l'electró tenim:
\index{error!absolut}\index{error!relatiu}

\[
    m\ped{e} = \qty{9,109 383 7139(28) e-31}{kg}
\]

Les dues xifres entre parèntesis, 28, determinen que la posició decimal de l'error absolut ha de correspondre amb la posició de les dues últimes xifres, 39, del valor; l'error absolut $\epsilon$  és doncs:

\[
    \epsilon = \qty{0,000 000 0028 e-31}{kg}
\]

Per tant, el valor de la massa de l'electró també es podria escriure així:

 \[
    m\ped{e} = \qty[separate-uncertainty,separate-uncertainty-units = repeat]{9,109 383 7139(28) e-31}{kg}
\]

o de forma més compacta:
\[
m\ped{e} = \qty[separate-uncertainty,separate-uncertainty-units = bracket]{9,109 383 7139(28) e-31}{kg}
\]

Finalment, l'error relatiu $\epsilon\ped{r}$ es calcula dividint l'error absolut pel valor sense l'error:

\[
    \epsilon\ped{r} = \frac{\qty{0,000 000 0028 e-31}{kg}}{\qty{9,109 383 7139 e-31}{kg}} =   \num{3,1e-10}
\]
