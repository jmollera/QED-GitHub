\begin{thebibliography}{99}

    \addcontentsline{toc}{chapter}{Bibliografia}

    \bibitem{KOP} Helmut Kopka, Patrick W. Daly. \textbf{\textit{A Guide To \LaTeX}}.  Addison-Wesley, 4th edition, 2004.
    \bibitem{GRZ} George Grätzer. \textbf{\textit{More Math Into \LaTeX}}.  Springer, 4th edition, 2007.
    \bibitem{GOM} Michel Goossens, Frank Mittelbach. \textbf{\textit{The \LaTeX{} Companion}}.  Addison-Wesley, 2nd edition, 2004.
    \bibitem{GOO} Michel Goossens, Frank Mittelbach, Sebstian Rahtz, Denis Roegel, Herbert Voß. \textbf{\textit{The \LaTeX{} Graphics Companion}}.  Addison-Wesley, 2nd edition, 2008.
    \bibitem{VOS} Herbert Voß. \textbf{\textit{Typesetting tables with \LaTeX{}}}.  UIT Cambrige Ltd., 2011.
    \bibitem{SPK} Scott Pakin. \textbf{\textit{The Comprehensive \LaTeX{} Symbol List}}. CTAN.ORG.

    \bibitem{VALa} Gabriel Valiente Feruglio. \textbf{\textit{Composició de textos científics amb \LaTeX}}.  Edicions UPC, 1998.
    \bibitem{VALb} Gabriel Valiente Feruglio. \textbf{\textit{Modern Catalan Typographical Conventions}}.  TUGBoat, 16(3), 329-338, 1995.
    \bibitem{BEC} Claudio Beccari. \textbf{\textit{Typesetting mathematics for science and technology according to ISO 31/XL}}.  TUGBoat, 18(1), 39-48, 1997.
    \bibitem{WIL} J. William Howard, Jr. \textbf{\textit{Graecum est: el uso del griego en textos electrónicos de carácter científico-técnico}}.  Panace@, VI(19), 45-54, 2005.

	\bibitem{ALL} Michael Alley. \textbf{\textit{The Craft of Scientific Writing}}. Springer, 4th edition, 2018.
    \bibitem{BUR} Richard Stevens Burington. \textbf{\textit{Handbook of Mathematical Tables and Formulas}}.  McGraw-Hill, 4th edition, 1965.
    \bibitem{SCH} Joel L. Schiff. \textbf{\textit{The Laplace Transform: Theory and Applications}}.  Springer, 1999.
    \bibitem{RJB} R. J. Beerends, H. G. ter Morsche, J. C. van den Berg, E. M. van de Vrie. \textbf{\textit{Fourier and Laplace Transforms}}.  Cambridge University Press, 2003.
    \bibitem{JDH} Joe D. Hoffman. \textbf{\textit{Numerical Methods for Engineers and Scientists}}.  Marcel Dekker, Inc., 2nd edition, 2001.
    \bibitem{EJB} E. Joseph Billo. \textbf{\textit{Excel® for Engineers and Scientists --- Numerical Methods}}.  Wiley-INTERSCIENCE, 2007.
    \bibitem{AGVS} Amos Gilat, Vish Subramaniam. \textbf{\textit{Numerical Methods for Engineers and Scientists ---
        An Introduction with Applications using MATLAB®}}.  Wiley, 3rd edition, 2013.
    \bibitem{WMF} Walter Mora Flores. \textbf{\textit{Introducción a los Métodos Numéricos}}.  Instituto Tecnológico de Costa Rica, 2016.
     \bibitem{KRE} Erwin Kreyszig. \textbf{\textit{Advanced Engineering Mathematics}}.  Wiley, 10th edition, 2011.

    \bibitem{RASa} Enrique Ras. \textbf{\textit{Teoría de circuitos. Fundamentos}}.  Marcombo Boixareu Editores, 3\textordfeminine\ edición, 1977.
    \bibitem{RASb} Enrique Ras. \textbf{\textit{Transformadores. De potencia, medida y protección}}.  Marcombo Boixareu Editores, 6\textordfeminine\ edición, 1985.
    \bibitem{RASc} Enrique Ras. \textbf{\textit{Teoría de líneas eléctricas (Volumen I)}}.  Marcombo Boixareu Editores, 2\textordfeminine\ edición, 1986.
    \bibitem{RASd} Enrique Ras. \textbf{\textit{Redes eléctricas y multipolos}}.  Marcombo Boixareu Editores, 1980.
    \bibitem{RASe} Enrique Ras. \textbf{\textit{Análisis de Fourier y cálculo operacional aplicados a la electrotecnia}}.  Marcombo Boixareu Editores, 1979.

    \bibitem{COR} Felipe Córcoles López, Joaquim Pedra Durán, Miquel Salichs Vivancos. \textbf{\textit{Transformadores}}.  Edicions UPC, 2004.

 	\bibitem{SAL} Miquel Salichs Vivancos. \textbf{\textit{Teoria, Problemes i Exàmens Resolts}}.  Edicions UPC, 2009.
 
    \bibitem{TRA} Jesús Trashorras Montecelos. \textbf{\textit{Subestaciones eléctricas}}.  Paraninfo, 2015.


    \bibitem{CHA} Stephen J. Chapman. \textbf{\textit{Máquinas Eléctricas}}.  McGraw-Hill, 4\textordfeminine\ edición, 2005.
    \bibitem{FIT} A. E. Fitzgerald, Charles Kingsley Jr., Stephen D. Umans. \textbf{\textit{Electric Machinery}}.  McGraw-Hill, 6th edition, 2003.
    \bibitem{JFM} Jesús Fraile Mora. \textbf{\textit{Máquinas Eléctricas}}.  Garceta grupo editorial, 8ª edición, 2016.

    \bibitem{GRA} John J. Grainger, William D. Stevenson Jr. \textbf{\textit{Análisis de Sistemas de Potencia}}.  McGraw-Hill, 1996.
    \bibitem{HAD} Hadi Saadat. \textbf{\textit{Power System Analysis}}.  McGraw-Hill, 2nd edition, 2004.
    \bibitem{DUN} J. Duncan Glover, Thomas J. Overbye, Mulukutla S. Sarma. \textbf{\textit{Power System Analysis \& Design}}.  CENGAGE Learning, 6th edition, 2015.
    \bibitem{WES} Westinghouse Electric Corporation. \textbf{\textit{Electrical Transmission and Distribution Reference Book}}.  Westinghouse Electric Corporation, 4th edition, 1950.

    \bibitem{PMA} Paul M. Anderson. \textbf{\textit{Analysis of Faulted Power Systems}}.  Wiley-INTERSCIENCE, 1995.
    \bibitem{BLAa} J. Lewis Blackburn. \textbf{\textit{Simmetrical Components for Power Systems Engineering}}.  Marcel Dekker, Inc, 1993.
    \bibitem{BLAb} J. Lewis Blackburn, Thomas J. Domin. \textbf{\textit{Protective Relaying. Principles and Applications}}.  CRC Press, 3rd edition, 2007.
    \bibitem{REI} Donald Reimert. \textbf{\textit{Protective Relaying for Power Generation Systems}}.  CRC Press, 2006.


    \bibitem{TLE} Nasser D. Tleis. \textbf{\textit{Power Systems Modelling and Fault Analysis --- Theory and Practise}}.  ELSEVIER, 2008.
    \bibitem{KAS} Ismail Kasikci. \textbf{\textit{Short Circuits in Power Systems. A practical Guide to IEC 60909}}.  Wiley-VCH, 2nd edition, 2017.
    \bibitem{JSch} Jürgen Schlabbach. \textbf{\textit{Short-Circuit Currents}}.  The Institution of Engineering and Technology, 2005.
    \bibitem{RRop} Richard Roeper. \textbf{\textit{Corrientes de cortocircuito en redes trifásicas}}.  Siemens Aktiengesellschaft \& Marcombo Boixareu Editores, 1985.

    \bibitem{JCD} J. C. Das. \textbf{\textit{Power System Analysis --- Short-Circuit, Load Flow and Harmonics}}. Marcel Dekker, Inc., 2002.
    \bibitem{MAI} Mohamed A. Ibrahim. \textbf{\textit{Disturbance Analysis for Power Systems}}. Wiley-IEEE Press, 2012.

    \bibitem{CAP} Robert Capella. \textbf{\textit{Protecciones eléctricas en MT}}.  Publicación Técnica de Schneider 071, mayo 2003.
    \bibitem{LLO} Manuel Llorente Antón. \textbf{\textit{Líneas y cables}}.  Publicación Técnica de Schneider 073, enero 2003.
    \bibitem{PAS} Jean Pasteau. \textbf{\textit{Envolventes y grados de protección}}.  Cuaderno Técnico de Schneider 166, febrero 2001.
    \bibitem{FONa} Paola Fonti. \textbf{\textit{Transformadores de intensidad: cómo determinar sus especificaciones}}.  Cuaderno Técnico de Schneider 194, agosto 2000.
    \bibitem{FONb} Paola Fonti. \textbf{\textit{Transformadores de intensidad: errores de especificación y soluciones}}.  Cuaderno Técnico de Schneider 195, diciembre 2001.

    \bibitem{KNU} Knut Sjövall. \textbf{\textit{Instrument Transformers Application Guide, Edition 3}}.  ABB, 2009.
    
    \bibitem{SUM} Mark Summerfield. \textbf{\textit{Programming in Python 3 --- A Complete Introduction to the Python Language}}.  Addison-Wesley, 2nd edition, 2010.
    
    \bibitem{RAM} Luciano Ramalho. \textbf{\textit{Fluent Python}}.  O'Reilly, 2nd edition, 2022.
    	
    \bibitem{JOH} Robert Johansson. \textbf{\textit{Numerical
    		Python --- Scientific Computing and Data Science Applications with Numpy,
    		SciPy and Matplotlib}}.  Apress, 2nd edition, 2019.
    
    \bibitem{HIL} Christian Hill. \textbf{\textit{Learning Scientific Programming with Python}}.  Cambridge University Press, 2nd edition,  2020.
    
     \bibitem{ZUM} Felix Zumstein. \textbf{\textit{Python for Excel®}}.  O'Reilly, 2021.
        
     \bibitem{VAN} Jake VanderPlas. \textbf{\textit{Python Data Science Handbook --- Essential Tools for Working with Data}}.  O'Reilly, 2nd edition, 2022.
     
     \bibitem{MGIO} Micha Gorelick and Ian Ozsvald. \textbf{\textit{High Performance Python --- Practical Performant Programming for Humans}}.  O'Reilly, 2nd edition, 2020.

\end{thebibliography}
