\chapter{Constants F\'{\i}siques}\label{sec:const_fis} \index{constants f\'{\i}siques}

\section{Taula de valors}

En la Taula \vref{taula:Const-Fis} es pot veure una recopilaci\'{o} de
constants f\'{\i}siques; les xifres entre par\`{e}ntesis que hi apareixen representen l'error absolut del valor.

Els valors de les constants d'aquesta taula s\'{o}n els recomanats
l'any 2010 pel {"<}Committee on Data for Science and Technology{">}
(\textsf{CODATA}), un comit\`{e} cient\'{\i}fic de l'{"<}International Council
for Science{">}.

Podeu trobar  m\'{e}s informaci\'{o} a
les adreces: \href{http://www.codata.org/}{www.codata.org} i \href{http://physics.nist.gov/cuu/Constants/}{physics.nist.gov/cuu/Constants}.\index{CODATA}
\index{NIST}


\begin{longtable}{lcll}
   \caption{\label{taula:Const-Fis} Constants f\'{\i}siques}\\
   \toprule[1pt]
   Magnitud & S\'{\i}mbol & Valor & Error relatiu\\
   \midrule
   \endfirsthead
   \caption[]{Constants f\'{\i}siques (\emph{ve de la p\`{a}gina anterior})} \\
   \toprule[1pt]
   Magnitud & S\'{\i}mbol & Valor & Error relatiu\\
   \midrule
   \endhead
   \midrule
   \multicolumn{4}{r}{(\emph{continua a la p\`{a}gina seg\"{u}ent})}
   \endfoot
   \endlastfoot
   velocitat de la llum  & $c$, $c_0$ & \SI{299792458}{m/s} & exacte\\
   en el buit & & & \\[0.5em]
   constant magn\`{e}tica & $\mu_0$ & \SI{4 \piup e-7}{N/A^2} & exacte \\[0.5em]
   constant el\`{e}ctrica: $1/(\mu_0 c^2)$ & $\epsilon_0$ & \SI{8,854187817\dots e-12}{F/m} & exacte \\[1em]
    imped\`{a}ncia caracter\'{\i}stica  & $Z_0$ &  \SI{376,730313461\dots}{\ohm} & exacte\\
    del buit: $\sqrt{\mu_0/\epsilon_0}$& & &  \\[0.5em]
    atmosfera est\`{a}ndard  & -- & \SI{101325}{Pa} & exacte \\[0.5em]
    acceleraci\'{o} de la gravetat & $g\ped{n}$ & \SI{9,80665}{m/s^2} & exacte \\
    est\`{a}ndard & & & \\[0.5em]
    massa molar del carboni-12 & $M({}^{12}\mathrm{C})$ & \SI{12e-3}{kg/mol} & exacte \\[0.5em]
    constant gravitacional & $G$ &   \SI{6,67384(80) e-11}{m^3/(kg.s^2)} & \num{1,2e-4} \\
     de Newton & & & \\[0.5em]
    constant de Planck & $h$ & \SI{6,62606957(29) e-34}{J.s} & \num{4,4e-8} \\[0.5em]
    constant de Planck  & $\hbar$ & \SI{1,054571726(47) e-34}{J.s} & \num{4,4e-8} \\
    redu\"{\i}da: $h/(2\piup)$ & & & \\[0.5em]
    c\`{a}rrega elemental & $e$ & \SI{1,602176565(35) e-19}{C} & \num{2,2e-8} \\[0.5em]
    massa de l'electr\'{o} & $m\ped{e}$ & \SI{9,10938291(40) e-31}{kg} & \num{4,4e-8} \\[0.5em]
    massa del  prot\'{o} & $m\ped{p}$ & \SI{1,672621777(74) e-27}{kg} & \num{4,4e-8} \\[0.5em]
    massa del neutr\'{o} & $m\ped{n}$ & \SI{1,674927351(74) e-27}{kg} & \num{4,4e-8} \\[0.5em]
    massa del deuteri & $m\ped{d}$ & \SI{3,34358348(15) e-27}{kg} & \num{4,4e-8} \\[0.5em]
    massa del triti & $m\ped{t}$ & \SI{5,00735630(22) e-27}{kg} & \num{4,4e-8} \\[0.5em]
    massa de la part\'{\i}cula $\alphaup$ & $m_\alphaup$ & \SI{6,64465675(29) e-27}{kg} & \num{4,4e-8} \\[0.5em]
    radi de Bohr: $4\piup\epsilon_0\hbar^2/(m\ped{e}e^2)$ & $a_0$ & \SI{0,52917721092(17) e-10}{m} & \num{3,2e-10} \\[0.5em]
    longitud d'ona Compton:  & $\lambda\ped{C}$ & \SI{2,4263102389(16) e-12}{m} & \num{6,5e-10} \\
    $h/(m\ped{e} c)$ & & & \\[0.5em]
    n\'{u}mero d'Avogadro & $N\ped{A}$, $L$ & \SI{6,02214129(27) e23}{mol^{-1}} & \num{4,4e-8} \\[0.5em]
    constant molar dels gasos & $R$ & \SI{8,3144621(75)}{J/(mol.K)} & \num{9,1e-7} \\[0.5em]
    constant de Faraday: $ e N\ped{A}$ & $F$ & \SI{96485,3365(21)}{C/mol} & \num{2,2e-8} \\[0.5em]
    constant de Boltzmann: & $k$ & \SI{1,3806488(13)e-23}{J/K} & \num{9,1e-7} \\
    $R/N\ped{A}$ & & & \\[0.5em]
    constant d'Stefan-Boltzmann:  & $\sigma$ & \SI{5,670373(21)e-8}{W/(m^2.K^4)} & \num{3,6e-6} \\
    $\piup^2 k^4 / (60\, \hbar^3 c^2)$ & & & \\[0.5em]
   \bottomrule[1pt]
\end{longtable}
\index{velocitat de la llum en el buit}  \index{constant!magn\`{e}tica}
\index{constant!el\`{e}ctrica} \index{imped\`{a}ncia caracter\'{\i}stica del
buit} \index{atmosfera est\`{a}ndard} \index{acceleraci\'{o} de la gravetat
est\`{a}ndard} \index{massa!molar del carboni-12}
\index{constant!gravitacional de Newton} \index{constant!de Planck}
\index{constant!de Planck redu\"{\i}da} \index{constant!de Faraday}
\index{carrega elemental@c\`{a}rrega elemental} \index{massa!de
l'electr\'{o}} \index{massa!del prot\'{o}} \index{massa!del neutr\'{o}}
\index{massa!del deuteri} \index{massa!de la part\'{\i}cula $\alpha$}
\index{numero d'Avogadro@n\'{u}mero d'Avogadro} \index{constant!molar
dels gasos} \index{constant!de Boltzmann}
\index{constant!d'Stefan-Boltzmann} \index{radi de Bohr}\index{longitud d'ona Compton}
\index{$\mu_0$} \index{$\epsilon_0$} \index{c@$c$} \index{c@$c_0$}\index{atm}
\index{gn@$g\ped{n}$} \index{Z0@$Z_0$} \index{F@$F$}
\index{m@$M({}^{12}\mathrm{C})$} \index{G@$G$} \index{h@$h$}
\index{h@$\hbar$} \index{e@$e$} \index{me@$m\ped{e}$}
\index{mp@$m\ped{p}$} \index{mn@$m\ped{n}$} \index{m@$m\ped{d}$}
\index{ma@$m_\alpha$} \index{NA@$N\ped{A}$} \index{L@$L$}\index{R@$R$}
\index{k@$k$} \index{$\sigma$} \index{a0@$a_0$} \index{$\lambda\ped{C}$}

\section{Error absolut i relatiu}\label{err_abs_rel}

Tal com s'ha dit al principi, les xifres entre par\`{e}ntesis indiquen l'error absolut del valor que les precedeix. El nombre de xifres entre par\`{e}ntesis determina la posici\'{o} decimal d'aquest error, per exemple, en el cas de la  massa de l'electr\'{o} tenim:
\index{error!absolut}\index{error!relatiu}
\[
    m\ped{e} = \SI{9,10938291(40) e-31}{kg}
\]

Les dues xifres entre par\`{e}ntesis, 40, ens determina que la posici\'{o} decimal de l'error absolut ha de correspondre amb la posici\'{o} de les dues \'{u}ltimes xifres, 91, del valor; l'error absolut $\epsilon$  \'{e}s doncs:
\[
    \epsilon = \SI{0,00000040 e-31}{kg}
\]

Per tant, el valor de la massa de l'electr\'{o} tamb\'{e} es pot escriure's aix\'{\i}:
 \[
    m\ped{e} = \SI[separate-uncertainty]{9,10938291(40) e-31}{kg}
\]

Finalment, l'error relatiu $\epsilon\ped{r}$ es calcula dividint l'error absolut pel valor sense l'error:
\[
    \epsilon\ped{r} = \frac{\SI{0,00000040 e-31}{kg}}{\SI{9,10938291 e-31}{kg}} =   \num{4,4e-8}
\]
