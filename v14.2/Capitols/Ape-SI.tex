\chapter{Sistema Internacional d'Unitats (SI)}\label{sec:SI}\index{sistema internacional d'unitats}\index{SI}

\section{Introducció}\label{sec:SI-intro}
S'exposa a continuació el Sistema
Internacional d'unitats (SI), el qual està definit pel  \textit{Bureau
International des Poids et Mesures} (BIPM). S'ha utilitzat la publicació més recent, corresponent a la 9a edició de l'any 2019; podeu trobar més informació a les adreces: \href{https://www.bipm.org/en/home/}{https://www.bipm.org/en/home} i \href{https://www.bipm.org/en/publications/si-brochure/}{https://www.bipm.org/en/publications/si-brochure}. S'inclouen els nous prefixos per als factors $10^{27}$,  $10^{30}$,  $10^{-27}$ i  $10^{-30}$, introduïts en la 27a \textit{Conférence Générale des Poids et Mesures} de l'any 2022. \index{BIPM}

El  \textit{National Institute of Standards and Technology} (NIST) també té informació referent al Sistema
Internacional d'unitats, a l'adreça: \href{http://www.nist.gov/pml/div684/fcdc/si-units.cfm}
{www.nist.gov/pml/div684/fcdc/si-units.cfm}.\index{NIST}

Dins de l'Estat Espanyol, el Sistema Internacional d'unitats és d'ús oficial segons el Reial decret 493/2020, del 28 d'abril. Es poden descarregar versions en català, castellà i gallec d'aquest Reial decret a l'adreça: \href{https://www.boe.es/eli/es/rd/2020/04/28/493}
{https://www.boe.es/eli/es/rd/2020/04/28/493}.

Els noms de les unitats s'escriuen segons el  \textit{Diccionari de la llengua catalana, 2a edició, 2007} (DIEC2).

\section{Unitats fonamentals de l'SI}
\index{sistema internacional d'unitats!unitats fonamentals}

En la taula \vref{taula:SI-fonamentals} es poden veure les unitats
fonamentals del Sistema Internacional d'unitats.

\begin{ThreePartTable}
\begin{TableNotes}
    \item[a] {\footnotesize La variant  «kilogram» també és correcta, segons el DIEC2.}
\end{TableNotes}
\begin{longtable}[h]{llc}
   \caption{\label{taula:SI-fonamentals} Unitats fonamentals de l'SI}\\
   \toprule[1pt]
    Magnitud & Unitat & Símbol \\
   \midrule
   \endfirsthead
   \caption[]{Unitats fonamentals de l'SI (\emph{ve de la pàgina anterior})}\\
   \toprule[1pt]
    Magnitud & Unitat & Símbol \\
   \midrule
   \endhead
   \midrule
   \multicolumn{3}{r}{\sffamily\bfseries\color{NavyBlue}(\emph{continua a la pàgina següent})}
   \endfoot
   \insertTableNotes
   \endlastfoot
   temps & segon & s\\
   longitud & metre & m \\
   massa & quilogram\tnote{a} & kg \\
   intensitat de corrent elèctric & ampere & A \\
   temperatura termodinàmica & kelvin & K\\
   quantitat de matèria & mol & mol \\
   intensitat lluminosa & candela &  cd \\
   \bottomrule[1pt]
\end{longtable}
\end{ThreePartTable}
\index{metre} \index{kilogram} \index{quilogram}\index{segon} \index{amper}
\index{kelvin} \index{mol} \index{candela} \index{longitud}
\index{massa} \index{temps} \index{intensitat!de corrent elèctric}
\index{temperatura!termodinàmica} \index{quantitat de matèria}
\index{intensitat!lluminosa} \index{m} \index{kg} \index{s}
\index{A} \index{K} \index{cd}

\subsection{Definicions històriques}
\index{sistema internacional d'unitats!definicions històriques}
Es donen a continuació les definicions històriques, prèvies al 20 de maig de 2019, de les unitats fonamentals; entre parèntesis s'indica l'any que la \textit{Conférence Générale des Poids et Mesures} les va posar en vigor. Aquestes definicions són obsoletes.

\begin{list}{}
   {\setlength{\labelwidth}{22mm} \setlength{\leftmargin}{22mm} \setlength{\labelsep}{2mm}}
   \item[\textbf{segon}] És la durada de \num{9192631770} períodes de la
   radiació corresponent a la transició entre els dos nivells
  hiperfins de l'estat fonamental de l'àtom de cesi-133. (1967).
   \item[\textbf{metre}] És la longitud de la trajectòria recorreguda per la llum
   en el buit, durant un temps de $\frac{1}{\num{299792458}}\unit{\,segon}$. (1983).
   \item[\textbf{quilogram}] És la massa del prototip internacional del quilogram, fet d'un aliatge de platí-iridi i
    conservat al BIPM (Pavillon de Breteuil,  12bis Grande Rue, Sèvres, França). (1901).
   \item[\textbf{ampere}] És la intensitat d'un corrent constant,
   que mantinguda en dos conductors paraŀlels rectilinis de longitud
   infinita, de secció transversal negligible, i situats en el buit a una
   distància l'un de l'altre d'un metre, produeix entre
   aquests dos conductors  una força igual a \num{2e-7} newton per metre de longitud. (1948).
   \item[\textbf{kelvin}] És la fracció $\frac{1}{\num{273,16}}$ de la temperatura
   termodinàmica corresponent al punt triple de l'aigua. (1967).
   \item[\textbf{mol}] És la quantitat de matèria d'un sistema que conté tantes
   entitats elementals com àtoms hi ha en \qty{0,012}{kg} de carboni-12. (1971).
   \item[\textbf{candela}] És la intensitat lluminosa, en una direcció determinada,
   d'una font que emet radiació monocromàtica de freqüència \qty{540e12}{hertz}, i
   que té una intensitat radiant en aquesta direcció de $\frac{1}{683}$ watt per estereoradiant. (1979).
\end{list}

\subsection{Definicions actuals}
\index{sistema internacional d'unitats!definicions actuals}

Les definicions actuals de les unitats fonamentals de l'SI, que van entrar en vigor el 20 de maig de 2019, estan basades en set constants de la natura. En concret, l'SI es defineix com el  sistema d'unitats en el qual:
\begin{itemize}

\item La freqüència de la transició hiperfina de l'estat fonamental de l'àtom de cesi-133 no pertorbat, $\Deltaup\nu\ped{Cs}$, és igual a \qty{9 192 631 770}{Hz}.
\item La velocitat de la llum en el buit, $c$, és igual a \qty{299792458}{m/s}.
\item La constant de Planck, $h$, és igual a \qty{6,62607015 e-34}{J/Hz}.
\item La càrrega elemental, $e$, és igual a \qty{1,602176634 e-19}{C}.
\item La constant de Boltzmann, $k$, és igual a \qty{1,380649e-23}{J/K}.
\item La constant d'Avogadro, $N\ped{A}$, és igual a \qty{6,02214076 e23}{mol^{-1}}.
\item L'eficàcia lluminosa d'una radiació monocromàtica de freqüència \qty{540e12}{Hz},  $K\ped{cd}$, és igual a \qty{683}{lm/W}.
\end{itemize}
\index{velocitat de la llum en el buit}
\index{constant!de Planck}\index{eficàcia lluminosa}
\index{carrega elemental@càrrega elemental}
\index{constant!d'Avogadro} \index{constant!de Boltzmann}
\index{c@$c$}\index{h@$h$}\index{$\Deltaup\nu\ped{Cs}$}
\index{e@$e$} \index{NA@$N\ped{A}$}
\index{k@$k$}\index{Kcd@$K\ped{cd}$}


Les unitats  hertz, joule, coulomb, lumen i watt, amb els símbols respectius Hz, J, C, lm, i W, estan relacionades amb les unitats segon, metre, quilogram, ampere, kelvin, mol i candela, amb els símbols respectius s, m, kg, A, K, mol i cd, segons:  Hz = \unit{s^{-1}}, J = \unit{kg.m^2.s^{-2}}, C =\unit{A.s}, lm = \unit{cd.m^2.m^{-2}} = \unit{cd.sr}, i W = \unit{kg.m^2.s^{-3}}.

El valor numèric de cadascuna d'aquestes set constants és exacte (vegeu l'apèndix \vref{sec:const_fis}).

A partir d'aquestes constants, les definicions de les unitats fonamentals de l'SI són les següents:
\begin{list}{}
   {\setlength{\labelwidth}{22mm} \setlength{\leftmargin}{22mm} \setlength{\labelsep}{2mm}}
   \item[\textbf{segon}] El segon, amb símbol s, es defineix a partir de la constant $\Deltaup\nu\ped{Cs}$ expressada en la unitat Hz, la qual és equivalent a $\unit{s^{-1}}$:
       \[
            \qty{1}{s} = \frac{\num{9192631770}}{\Deltaup\nu\ped{Cs}}
       \]

    La conseqüència d'aquesta definició és que el segon és igual a la durada de \num{9192631770} períodes de la
   radiació corresponent a la transició entre els dos nivells
  hiperfins de l'estat fonamental de l'àtom de cesi-133 no pertorbat.

   \item[\textbf{metre}]  El metre, amb símbol m, es defineix a partir de la constant $c$ expressada en les unitats \unit{m.s^{-1}}, on el segon està definit en funció de la constant  $\Deltaup\nu\ped{Cs}$:
       \[
            \qty{1}{m} = \left(\frac{c}{\num{299792458}}\right)\unit{\,s} = \frac{\num{9192631770}}{\num{299792458}}\frac{c}{\Deltaup\nu\ped{Cs}}
       \]

    La conseqüència d'aquesta definició és que el metre és igual a la longitud de la trajectòria recorreguda per la llum
   en el buit, durant un temps de $\frac{1}{\num{299792458}}$ s.

   \item[\textbf{quilogram}] El quilogram, amb símbol kg, es defineix a partir de la constant $h$ expressada en les unitats \unit{J/Hz}, equivalents a \unit{kg.m^2.s^{-1}}, on el metre i el segon estan definits en funció de les constants $c$ i $\Deltaup\nu\ped{Cs}$:
       \[
            \qty{1}{kg} = \left(\frac{h}{\num{6,62607015 e-34}}\right)\unit{\,m^{-2}.s} = \frac{\num{299792458}^2}{\num{6,62607015 e-34}\times\num{9192631770}}\frac{h\Deltaup\nu\ped{Cs}}{c^2}
       \]

   Aquesta definició permet definir la unitat \unit{kg.m^2.s^{-1}} (unitat de les quantitats físiques acció i moment cinètic). Associada així a les definicions del segon i del metre,  la unitat de massa s'expressa en funció  de la constant de Planck.


   \item[\textbf{ampere}] L'ampere, amb símbol A, es defineix a partir de la constant $e$ expressada en la unitat C,  equivalent a \unit{A.s}, on el segon està definit en funció de la constant  $\Deltaup\nu\ped{Cs}$:
       \[
            \qty{1}{A} = \left(\frac{e}{\num{1,602176634 e-19}}\right)\unit{\,s^{-1}} = \frac{1}{\num{9192631770}\times\num{1,602176634 e-19}} \Deltaup\nu\ped{Cs} e
       \]

    La conseqüència d'aquesta definició és que l'ampere és igual al corrent elèctric corresponent al flux de \num{1,602176634 e19}  càrregues elementals per segon.

   \item[\textbf{kelvin}]  El kelvin, amb símbol K, es defineix a partir de la constant $k$ expressada en les unitats \unit{J.K^{-1}},  equivalents a \unit{kg.m^2.s^{-2}.K^{-1}},  on el quilogram, el metre i el segon estan definits en funció de les constants $h$, $c$ i $\Deltaup\nu\ped{Cs}$:
       \[
            \qty{1}{K} = \left(\frac{\num{1,380649e-23}}{k}\right)\unit{\,kg.m^2.s^{-2}} = \frac{\num{1,380649e-23}}{\num{6,62607015 e-34}\times\num{9192631770}} \frac{\Deltaup\nu\ped{Cs} h}{k}
       \]

    La conseqüència d'aquesta definició és que el kelvin és igual al canvi de la temperatura termodinàmica resultant d'un canvi d'energia tèrmica de \qty{1,380649e-23}{J}.

   \item[\textbf{mol}]  El mol, amb símbol mol, es defineix a partir de la constant $N\ped{A}$, expressada en la unitat  $\unit{mol^{-1}}$:
       \[
            \qty{1}{mol} = \frac{\num{6,02214076 e23}}{N\ped{A}}
       \]

    La conseqüència d'aquesta definició és que el mol és igual a la quantitat de matèria d'un sistema que conté
    \num{6,02214076 e23} unitats elementals.

   \item[\textbf{candela}] La candela, amb símbol cd, es defineix a partir de la constant $K\ped{cd}$ expressada en les unitats \unit{lm.W^{-1}},  equivalents a \unit{cd.sr.kg^{-1}.m^{-2}.s^3},  on el quilogram, el metre i el segon estan definits en funció de les constants $h$, $c$ i $\Deltaup\nu\ped{Cs}$:
       \[
            \qty{1}{cd} = \left(\frac{K\ped{cd}}{683}\right)\unit{\,kg.m^2.s^{-3}.sr^{-1}} = \frac{1}{\num{6,62607015 e-34}\times\num{9192631770}^2\times 683} \Deltaup\nu\ped{Cs}^2 h K\ped{cd}
       \]

    La conseqüència d'aquesta definició és que la candela és igual a la intensitat lluminosa, en una direcció determinada,   d'una font que emet radiació monocromàtica de freqüència \qty{540e12}{Hz}, i
   que té una intensitat d'energia en aquesta direcció de $\frac{1}{683}$ \unit{W/sr}.
\end{list}


\section{Prefixos de l'SI}
\index{sistema internacional d'unitats!prefixos}

En la taula \vref{taula:SI-prefixes} es presenta una llista amb els
prefixos que es poden anteposar a les unitats del Sistema
Internacional d'unitats, per tal de formar-ne els múltiples i
submúltiples.

\begin{ThreePartTable}
\begin{TableNotes}
    \item[a] {\footnotesize La variant  «kilo» també és correcta, segons el DIEC2.}
\end{TableNotes}
\begin{longtable}[h]{llccllc}
   \caption{\label{taula:SI-prefixes} Prefixos de  l'SI}\\
   \toprule[1pt]
   \multicolumn{3}{c}{Múltiples} & & \multicolumn{3}{c}{Submúltiples}\\
   \cmidrule(rl){1-3} \cmidrule(rl){5-7}
   factor & nom & símbol & & factor & nom & símbol\\
   \midrule
   \endfirsthead
   \caption[]{Prefixes de  l'SI (\emph{ve de la pàgina  anterior})}\\
   \toprule[1pt]
    \multicolumn{3}{c}{Múltiples} & & \multicolumn{3}{c}{Submúltiples}\\
   \cmidrule(rl){1-3} \cmidrule(rl){5-7}
   factor & nom & símbol & & factor & nom & símbol\\
   \midrule
   \endhead
   \midrule
   \multicolumn{7}{r}{\sffamily\bfseries\color{NavyBlue}(\emph{continua a la pàgina següent})}
   \endfoot
   \insertTableNotes
   \endlastfoot
    $10^{30}$ &  quetta & Q & & $10^{-30}$ & quecto & q \\
    $10^{27}$ &  ronna & R & & $10^{-27}$ & ronto & r \\
    $10^{24}$ &  yotta & Y & & $10^{-24}$ & yocto & y \\
    $10^{21}$ &  zetta & Z & & $10^{-21}$ & zepto & z \\
    $10^{18}$ &  exa & E & & $10^{-18}$ & atto & a \\
    $10^{15}$ &  peta & P & & $10^{-15}$ & femto & f \\
    $10^{12}$ &  tera & T & & $10^{-12}$ & pico & p \\
    $10^{9}$ &  giga & G & & $10^{-9}$ & nano & n \\
    $10^{6}$ &  mega & M & & $10^{-6}$ & micro & \unit{\micro\noop} \\
    $10^{3}$ &  quilo\tnote{a} & k & & $10^{-3}$ & miŀli & m \\
    $10^{2}$ &  hecto & h & & $10^{-2}$ & centi & c \\
    $10^{1}$ &  deca & da & & $10^{-1}$ & deci & d \\
   \bottomrule[1pt]
\end{longtable}
\end{ThreePartTable}
\index{quetta} \index{ronna}
\index{yotta} \index{zetta} \index{exa} \index{peta} \index{tera} \index{giga} \index{mega}
\index{kilo} \index{quilo} \index{hecto} \index{deca} \index{deci} \index{centi} \index{mili} \index{micro}
\index{nano} \index{pico} \index{femto} \index{atto} \index{zepto} \index{yocto} \index{ronto} \index{quecto}
\index{Q} \index{R} \index{Y} \index{Z} \index{E} \index{P} \index{T} \index{G} \index{M} \index{k} \index{h} \index{da} \index{q} \index{r} \index{y} \index{z} \index{a} \index{f} \index{p} \index{n} \index{\unit{\micro\noop}} \index{m} \index{c} \index{d}


\section{Unitats derivades de l'SI amb noms i símbols propis}
\label{sec:unitata-derv-SI}
\index{sistema internacional d'unitats!unitats derivades amb noms i símbols propis}

De forma convenient, s'ha donat noms i símbols propis a algunes unitats derivades de les fonamentals; en la taula \vref{taula:SI-derivades} es mostren aquestes unitats derivades de l'SI.

\begin{ThreePartTable}
\begin{TableNotes}
    \item[a] {\footnotesize La variant «radian» també és correcta, segons el DIEC2.}
    \item[b] {\footnotesize La variant «estereoradian» també és correcta, segons el DIEC2.}
\end{TableNotes}
\begin{longtable}[h]{llclc}
   \caption{\label{taula:SI-derivades} Unitats derivades de
   l'SI amb noms i símbols propis}\\
   \toprule[1pt]
    \multirow{2}{15mm}{\rule{0mm}{6mm}Magnitud} & \multirow{2}{15mm}{\rule{0mm}{6mm}Unitat}  &
    \multirow{2}{15mm}{\rule{0mm}{6mm}Símbol}  & \multicolumn{2}{c}{Equivalència en unitats SI}\\
    \cmidrule(rl){4-5}
    &  &   & fonamentals & altres\\
   \midrule
   \endfirsthead
   \caption[]{Unitats derivades de l'SI amb noms i símbols propis (\emph{ve de la pàgina
   anterior})}\\
   \toprule[1pt]
    \multirow{2}{15mm}{\rule{0mm}{6mm}Magnitud} & \multirow{2}{15mm}{\rule{0mm}{6mm}Unitat}  &
    \multirow{2}{15mm}{\rule{0mm}{6mm}Símbol}  & \multicolumn{2}{c}{Equivalència en unitats SI}\\
    \cmidrule(rl){4-5}
    &  &  & fonamentals & altres\\
   \midrule
   \endhead
   \midrule
   \multicolumn{5}{r}{\sffamily\bfseries\color{NavyBlue}(\emph{continua a la pàgina següent})}
   \endfoot
   \insertTableNotes
   \endlastfoot
   angle pla & radiant\tnote{a} & rad   & \unit{m/m} & 1\\
   angle sòlid & estereoradiant\tnote{b} & sr & \unit{m^2/m^2}  & 1 \\
   freqüència & hertz & Hz & \unit{s^{-1}} & --- \\
   força & newton & N & \unit{kg.m.s^{-2}} & --- \\
   pressió & pascal & Pa  & \unit{kg.m^{-1}.s^{-2}} & \unit{N/m^2} \\
   energia, treball & joule & J & \unit{kg.m^2.s^{-2}} & \unit{N.m}\\
   potència & watt & W & \unit{kg.m^2.s^{-3}}  & \unit{J/s}\\
   càrrega elèctrica & coulomb & C  & \unit{A.s} &  ---\\
   potencial elèctric & volt & V & \unit{kg.m^2.s^{-3}.A^{-1}}  & \unit{W/A}\\
   capacitat elèctrica & farad & F   & \unit{kg^{-1}.m^{-2}.s^4.A^2}& \unit{C/V}\\
   resistència elèctrica & ohm &  \unit{\ohm}  & \unit{kg.m^2.s^{-3}.A^{-2}} & \unit{V/A}\\
   conductància elèctrica & siemens &  S  & \unit{kg^{-1}.m^{-2}.s^3.A^2} & \unit{A/V}\\
   flux magnètic & weber &  Wb  & \unit{kg.m^2.s^{-2}.A^{-1}} & \unit{V.s}\\
   densitat de flux magnètic & tesla &  T  & \unit{kg.s^{-2}.A^{-1}} & \unit{Wb/m^2}\\
   inductància & henry &  H  & \unit{kg.m^2.s^{-2}.A^{-2}} & \unit{Wb/A}\\
   temperatura Celsius & grau Celsius &  \unit{\degreeCelsius} & \unit{K} & --- \\
   flux lluminós & lumen & lm  & \unit{cd.sr}& ---\\
   iŀluminació & lux & lx & \unit{cd.sr.m^{-2}} & \unit{lm/m^2} \\
   activitat  d'un radionúclid & becquerel & Bq& \unit{s^{-1}} & --- \\
   dosi absorbida & gray & Gy  & \unit{m^2.s^{-2}}& \unit{J/kg}\\
   dosi equivalent & sievert & Sv  & \unit{m^2.s^{-2}}& \unit{J/kg}\\
   activitat catalítica & katal & kat & \unit{mol.s^{-1}} & ---\\
   \bottomrule[1pt]
\end{longtable}
\end{ThreePartTable}
\index{radiant} \index{radian}  \index{estereoradiant}  \index{estereoradian} \index{hertz} \index{newton}
\index{pascal} \index{joule} \index{watt} \index{coulomb}
\index{volt} \index{farad} \index{ohm} \index{siemens} \index{weber}
\index{tesla} \index{henry} \index{lumen} \index{lux}
\index{becquerel} \index{gray} \index{sievert} \index{grau Celsius}\index{katal}
\index{angle pla}  \index{angle sòlid} \index{freq\"{u}ència}
\index{força} \index{pressió} \index{energia} \index{potència}
\index{carrega electrica@càrrega elèctrica} \index{potencial
elèctric} \index{capacitat} \index{resistència} \index{conductància}
\index{flux!magnètic} \index{densitat!de flux magnètic}
\index{inductància} \index{temperatura!Celsius} \index{flux!lluminós} \index{iluminacio@iŀluminació}\index{activitat!d'un radionúclid}
\index{dosi!absorbida}  \index{dosi!equivalent}\index{activitat!catalítica}
\index{rad} \index{sr} \index{Hz} \index{N}
\index{Pa} \index{J} \index{W} \index{C} \index{V} \index{F}
\index{\unit{\ohm}} \index{S} \index{Wb} \index{T} \index{H}
\index{\unit{\degreeCelsius}} \index{lm} \index{lx} \index{Bq} \index{Gy}
\index{Sv}\index{kat}

Pel que fa a la igualtat entre el grau Celsius i el kelvin, cal tenir en compte que aquesta igualtat es refereix a diferències o intervals de temperatures. En general s'utilitza el símbol $t$ per expressar una temperatura en grau Celsius i el símbol $T$ per expressar una temperatura en kelvin; amb aquesta convenció podem escriure:
\begin{equation}
	\Delta t = \Delta T
\end{equation}

En canvi, pel que fa a una temperatura en particular, la relació entre els valors d'aquesta temperatura expressada en grau Celsius i en kelvin, és:
\begin{equation}
	t / \unit{\degreeCelsius} = T / \unit{K} - \num{273,15}
\end{equation}


\section{Altres unitats derivades de l'SI}
\index{sistema internacional d'unitats!altres unitats derivades}

Les unitats fonamentals (vegeu la taula \vref{taula:SI-fonamentals}) i les unitats derivades amb noms i símbols propis (vegeu la taula \vref{taula:SI-derivades}) poden combinar-se entre si per tal d'expressar noves unitats derivades.

 En la taula \vref{taula:SI-derivades-exemples} es mostren alguns exemples d'aquestes combinacions.

\begin{longtable}[h]{lll}
   \caption{\label{taula:SI-derivades-exemples} Exemples d'altres unitats derivades de
   l'SI}\\
   \toprule[1pt]
    Magnitud &  Unitats & Equivalència en unitats fonamentals SI\\
   \midrule
   \endfirsthead
   \caption[]{Exemples d'altres unitats derivades de l'SI (\emph{ve de la pàgina
   anterior})}\\
   \toprule[1pt]
    Magnitud &  Unitats & Equivalència en unitats fonamentals SI\\
   \midrule
   \endhead
   \midrule
   \multicolumn{3}{r}{\sffamily\bfseries\color{NavyBlue}(\emph{continua a la pàgina següent})}
   \endfoot
   \endlastfoot
   viscositat dinàmica &  \unit{Pa.s}& \unit{kg.m^{-1}.s^{-1}} \\
   moment d'una força & \unit{N.m} & \unit{kg.m^2.s^{-2}} \\
   tensió superficial &  \unit{N/m} &   \unit{kg.s^{-2}} \\
   velocitat angular & \unit{rad/s} & \unit{m.m^{-1}.s^{-1}} = \unit{s^{-1}} \\
   acceleració angular & \unit{rad/s^2} & \unit{m.m^{-1}.s^{-2}} = \unit{s^{-2}} \\
   densitat de flux de calor & \unit{W/m^2} & \unit{kg.s^{-3}} \\
   entropia & \unit{J/K} & \unit{kg.m^2.s^{-2}.K^{-1}} \\
   entropia específica & \unit{J/(kg.K)} &\unit{m^2.s^{-2}.K^{-1}} \\
   energia específica & \unit{J/kg} & \unit{m^2.s^{-2}} \\
   conductivitat tèrmica & \unit{W/(m.K)} & \unit{kg.m.s^{-3}.K^{-1}} \\
   densitat d'energia & \unit{J/m^3} & \unit{kg.m^{-1}.s^{-2}} \\
   intensitat de camp elèctric & \unit{V/m}& \unit{kg.m.s^{-3}.A^{-1}}  \\
   densitat de càrrega elèctrica & \unit{C/m^3} & \unit{A.s.m^{-3}} \\
   densitat de flux elèctric & \unit{C/m^2} & \unit{A.s.m^{-2}}\\
   permitivitat &  \unit{F/m}& \unit{kg^{-1}.m^{-3}.s^4.A^2} \\
   permeabilitat &  \unit{H/m} & \unit{kg.m.s^{-2}.A^{-2}} \\
   energia molar & \unit{J/mol} & \unit{kg.m^2.s^{-2}.mol^{-1}} \\
   entropia molar& \unit{J/(mol.K)} & \unit{kg.m^2.s^{-2}.mol^{-1}.K^{-1}} \\
   exposició (raigs X i $\gammaup$) & \unit{C/kg} & \unit{A.s.kg^{-1}} \\
   tassa de dosi absorbida & \unit{Gy/s} & \unit{m^2.s^{-3}}\\
   intensitat radiant & \unit{W/sr} & \unit{kg.m^2.s^{-3}.sr^{-1}} \\
   radiància & \unit{W/(sr.m^2)} & \unit{kg.s^{-3}.sr^{-1}} \\
   concentració d'activitat catalítica &  \unit{kat/m^3} & \unit{mol.s^{-1}.m^{-3}}\\
    \bottomrule[1pt]
\end{longtable}
\index{viscositat!dinàmica}\index{moment d'una força}\index{tensió superficial}
\index{velocitat angular}\index{acceleració!angular}\index{densitat!de flux de calor}\index{entropia}\index{entropia!específica}\index{conductivitat termica@conductivitat tèrmica}\index{densitat!d'energia}\index{intensitat!de camp electric@intensitat de camp elèctric}
\index{densitat!de càrrega elèctrica}\index{densitat!de flux elèctric}
\index{permitivitat}\index{permeabilitat}\index{energia!molar}\index{entropia!molar}
\index{exposició}\index{tassa de dosi absorbida}\index{intensitat!radiant}\index{radiancia@radiància}
\index{concentracio d'activitat catalitica@concentració d'activitat catalítica}



\section{Unitats fora de l'SI}\label{sec:unitats-fora-SI}
\index{sistema internacional d'unitats!unitats fora de l'SI}

Hi ha una sèrie d'unitats que no formem part de l'SI, però que són d'ús comú en el camp científic, tècnic o comercial, i que són usades freqüentment. En les taules següents es recullen algunes d'aquestes unitats.

En la taula \vref{taula:SI-altres-acceptades} es mostren les unitats fora de l'SI, l'ús de les quals s'accepta en conjunció amb el Sistema Internacional d'unitats, ja que són presents en la vida diària i es preveu que el seu ús continuï de forma indefinida. 

\begin{ThreePartTable}
\begin{TableNotes}
    \item[a] {\footnotesize La «unitat astronòmica» va ser redefinida l'any 2012 en la 28a Assemblea General de la Unió Astronòmica Internacional (\href{http://www.iau.org/}{www.iau.org}), passant a ser un valor exacte.}
    \item[b] {\footnotesize L'«hectàrea» s'utilitza per expressar superfícies agràries.}
    \item[c] {\footnotesize El símbol «L» es va adoptar posteriorment al símbol «l» per evitar la  confusió entre aquesta lletra  i  el número 1.}
    \item[d] {\footnotesize En els països de parla anglesa aquesta unitat és coneguda com a «tona mètrica».}
    \item[e] {\footnotesize El «dalton» i la «unitat de massa atòmica unificada», amb símbol u,  són dos noms alternatius d'una mateixa unitat. El seu valor és el de la constant de massa atòmica  $m\ped{u}$ (vegeu la taula \vref{taula:Const-Fis}).}
    \item[f] {\footnotesize Un «electró-volt» és l'energia cinètica que adquireix un electró després de creuar una diferència de potencial d'un volt en el buit. El seu valor és exacte (vegeu la taula \vref{taula:Const-Fis}).}
    \item[g] {\footnotesize Aquestes unitats adimensionals s'utilitzen per expressar logaritmes de relacions entre quantitats. Per exemple, $n\unit{\,Np}$ fa referència a una relació del tipus $ln\frac{A_2}{A_1}= n$, i  $ m\unit{\,dB} =\frac{m}{10}\unit{\,B}$  fa referència a una relació del tipus $\log\frac{A_2}{A_1} =\frac{m}{10}$.}
\end{TableNotes}
\begin{longtable}[h]{llcl}
   \caption{\label{taula:SI-altres-acceptades} Unitats fora de l'SI acceptades per a ser usades amb l'SI  }\\
   \toprule[1pt]
    Magnitud & Unitat &  Símbol & Valor en unitats SI\\
   \midrule
   \endfirsthead
   \caption[]{Unitats fora de l'SI acceptades per ser usades amb l'SI (\emph{ve de la pàgina
   anterior})}\\
   \toprule[1pt]
    Magnitud & Unitat &  Símbol & Valor en unitats SI\\
   \midrule
   \endhead
   \midrule
   \multicolumn{4}{r}{\sffamily\bfseries\color{NavyBlue}(\emph{continua a la pàgina següent})}
   \endfoot
   \insertTableNotes
   \endlastfoot
   temps & minut &  \unit{min}& $\qty{1}{min} = \qty{60}{s}$ \\
   temps & hora & \unit{h} & $\qty{1}{h} = \qty{60}{min} = \qty{3600}{s}$ \\
   temps & dia & \unit{d} & $\qty{1}{d} = \qty{24}{h} = \qty{86400}{s}$\\
   longitud & unitat astronòmica\tnote{a} &  \unit{au} &  $\qty{1}{au} =  \qty{149597870700}{m}$ \\
   angle pla & grau &  \unit{\degree} &   $\ang{1} = \frac{\piup}{180}\unit{\,rad}$ \\[2mm]
   angle pla & minut & \unit{\arcminute} & $\ang{;1;} = \big(\frac{1}{60}\big)\raisebox{1.5mm}{\unit{\degree}} = \frac{\piup}{10800}\unit{\,rad}$ \\[2mm]
   angle pla & segon & \unit{\arcsecond} & $\ang{;;1} = \big(\frac{1}{60}\big)' = \frac{\piup}{648000}\unit{\,rad}$ \\[2mm]
   superfície & hectàrea\tnote{b} & \unit{ha} & $\qty{1}{ha} = \qty{1}{hm^2} = \qty[print-unity-mantissa = false]{e4}{m^2}$\\
   volum & litre\tnote{c} &  \unit{l},\unit{\,L} & $\qty{1}{l} = \qty{1}{L} = \qty{1}{dm^3} = \qty[print-unity-mantissa = false]{e-3}{m^3}$ \\
   massa & tona\tnote{d} & \unit{t} & $\qty{1}{t} = \qty{1000}{kg}$\\
   massa & dalton\tnote{e} & Da & $\qty{1}{Da} = \qty{1,6605390660(50)e-27}{kg}$\\
   energia & electró-volt\tnote{f} & eV & $\qty{1}{eV} = \qty{1,602176634e-19}{J}$ \\
   logaritme d'una relació & neper\tnote{g} & \unit{Np} & ---\\
   logaritme d'una relació & bel, decibel\tnote{g} &  \unit{B}, \unit{dB} & ---\\
   \bottomrule[1pt]
\end{longtable}
\end{ThreePartTable}
\index{minut}\index{hora}\index{dia}\index{grau}\index{segon}\index{litre}\index{tona}\index{hectarea@hectàrea}
\index{min}\index{h}\index{d}\index{\unit{\degree}}\index{$'$}\index{$''$}\index{l}\index{L}\index{ha}\index{t}
\index{unitat astronomica@unitat astronòmica}\index{au}\index{temps}\index{angle pla}\index{area@àrea}
\index{volum}\index{massa}\index{longitud}
\index{electró-volt}\index{unitat de massa atomica unificada@unitat de massa atòmica unificada}
\index{dalton}\index{eV}\index{u}\index{Da}\index{energia}\index{massa}
\index{Np}\index{B}\index{dB}\index{neper}\index{bel}\index{decibel}


Encara que hi ha moltes més unitats que no formen part de l'SI, que o bé són d'interès històric, o bé continuen utilitzant-se en aplicacions específiques, se'n desaconsella totalment l'ús en textos científics i tècnics moderns. Es dona, no obstant això, en la taula \vref{taula:SI-altres-NIST} quatre unitats fora de l'SI acceptades addicionalment pel NIST, ja que s'han utilitzat tradicionalment en els Estats Units d'Amèrica; aquest organisme desaconsella però continuar usant-les, i recomana l'ús de les unitats equivalents de l'SI.\index{NIST}

\begin{ThreePartTable}
\begin{TableNotes}
    \item[a] {\footnotesize Quan hi hagi perill de confusió amb el símbol del radiant, es podrà  utilitzar el símbol «rd» en lloc del símbol  «rad» per a la dosi absorbida.}
\end{TableNotes}
\begin{longtable}[h]{llcl}
   \caption{\label{taula:SI-altres-NIST} Altres unitats fora de l'SI acceptades pel NIST}\\
   \toprule[1pt]
    Magnitud & Unitat &  Símbol & Valor en unitats SI\\
   \midrule
   \endfirsthead
   \caption[]{Altres unitats fora de l'SI acceptades pel NIST (\emph{ve de la pàgina anterior})}\\
   \toprule[1pt]
    Magnitud & Unitat &  Símbol & Valor en unitats SI\\
   \midrule
   \endhead
   \midrule
   \multicolumn{4}{r}{\sffamily\bfseries\color{NavyBlue}(\emph{continua a la pàgina següent})}
   \endfoot
   \insertTableNotes
   \endlastfoot
    activitat d’un radionúclid & curie &  \unit{Ci} & \qty{1}{Ci} = \qty{3,7e10}{Bq} \\
    dosi absorbida & rad & rad\tnote{a}  & \qty{1}{rad} = \qty[print-unity-mantissa = false]{e-2}{Gy}\\
    dosi equivalent & rem & rem &  \qty{1}{rem} = \qty[print-unity-mantissa = false]{e-2}{Sv} \\
    exposició (raigs x i $\gammaup$) & roentgen & \unit{R} & \qty{1}{R} = \qty{2,58e-4}{C/kg} \\
\bottomrule[1pt]
\end{longtable}
\end{ThreePartTable}
\index{curie}\index{roentgen}\index{rad}\index{rem}\index{Ci}\index{R}
\index{activitat!d'un radionúclid}\index{dosi!absorbida}\index{dosi!equivalent}\index{exposició}


\section{Unitats definides en la norma CEI/ISO 80000}\label{sec:unitats-cei}
\index{sistema internacional d'unitats!unitats definides en la norma CEI 80000}

La norma CEI/ISO 80000 adopta totes les unitats definides per l'SI, però en defineix d'addicionals.\index{CEI!80000-00@80000}
\index{ISO!80000-00@80000}


\subsection{Unitats informàtiques i prefixos de potències binàries}

La norma CEI 80000-13 \textit{Quantities and units --- Part 13: Information science and technology}, defineix símbols d'unitats informàtiques i prefixos de potències binàries que cal usar amb aquestes unitats.\index{CEI!80000-13}

En la taula \vref{taula:simbol-inform} es mostren els símbols d'unitats informàtiques.
\begin{longtable}[h]{>{\hspace{5mm}}cc}
   \caption{\label{taula:simbol-inform} Unitats informàtiques}\\
   \toprule[1pt]
    Nom & Símbol \\
   \midrule
   \endfirsthead
   \caption[]{Unitats informàtiques (\emph{ve de la pàgina anterior})}\\
   \toprule[1pt]
    Nom & Símbol \\
   \midrule
   \endhead
   \midrule
   \multicolumn{2}{r}{\sffamily\bfseries\color{NavyBlue}(\emph{continua a la pàgina següent})}
   \endfoot
   \endlastfoot
   bit & bit    \\
   octet, byte & B   \\
   \bottomrule[1pt]
\end{longtable}
\index{bit}\index{byte}\index{B}\index{octet}

En la taula \vref{taula:prefix-inform} es mostren els prefixos de potències binàries.
\begin{longtable}[h]{lccll}
   \caption{\label{taula:prefix-inform} Prefixos de potències binàries}\\
   \toprule[1pt]
    Nom & Símbol  & Factor & Valor exacte & Valor aproximat\\
   \midrule
   \endfirsthead
   \caption[]{Prefixes de potències binàries (\emph{ve de la pàgina anterior})}\\
   \toprule[1pt]
    Nom & Símbol  & Factor \\
   \midrule
   \endhead
   \midrule
   \multicolumn{4}{r}{\sffamily\bfseries\color{NavyBlue}(\emph{continua a la pàgina següent})}
   \endfoot
   \endlastfoot
   yobi & Yi   & $2^{80}$ & \num{1208925819614629174706176} &  $\quad\num{1,209e24}$ \\
   zebi & Zi   & $2^{70}$ & \num{1180591620717411303424}&  $\quad\num{1,181e21}$ \\
   exbi & Ei   & $2^{60}$ & \num{1152921504606846976}&  $\quad\num{1,153e18}$ \\
   pebi & Pi   & $2^{50}$ & \num{1125899906842624}&  $\quad\num{1,126e15}$ \\
   tebi & Ti   & $2^{40}$ & \num{1099511627776}&  $\quad\num{1,100e12}$ \\
   gibi & Gi   & $2^{30}$ & \num{1073741824}&  $\quad\num{1,074e9}$  \\
   mebi & Mi   & $2^{20}$ & \num{1048576} &  $\quad\num{1,049e6}$ \\
   kibi & Ki   & $2^{10}$ & \num{1024} & $\quad\num{1,024e3}$  \\
   \bottomrule[1pt]
\end{longtable}
\index{kibi} \index{Ki} \index{mebi} \index{Mi} \index{gibi}  \index{Gi} \index{tebi} \index{Ti}
\index{pebi} \index{Pi} \index{exbi} \index{Ei}


Els noms dels prefixos provenen de la fusió dels noms dels prefixos de l'SI amb la paraula «binari»; per exemple, el nom «mebi» prové de «megabinari», i el nom «yobi» prové de «yottabinari». Tots els símbols comencen per una lletra amb majúscula. 

Aquests prefixos poden aplicar-se tant al bit com al byte (o octet). Utilitzant aquests prefixos, podem escriure per exemple:
\begin{align*}
	\qty{1}{Kibit} &= 2^{10}\unit{\,bit} = \qty{1024}{bit} \\
	\qty{1}{MiB} &= 2^{20}\unit{\,B} = \qty{1048576}{B}
\end{align*}


Els prefixos «k» i «M» de l'SI indiquen, en canvi, uns altres valors:
\begin{align*}
	\qty{1}{kbit} &= \qty[print-unity-mantissa = false]{e3}{bit} = \qty{1000}{bit} \\
	\qty{1}{MB} &=\qty[print-unity-mantissa = false]{e6}{B} = \qty{1000000}{B}
\end{align*}


\subsection{Unitats de potència elèctrica}

Tot i que la potència es mesura en watt, la norma CEI 80000-6 \textit{Quantities and units --- Part 6: Electromagnetism}, defineix noms i símbols d'unitats diferenciats per a les potències activa, reactiva i aparent.\index{CEI!80000-06@80000-6}

En la taula \vref{taula:P-Q-S} es mostren aquestes unitats de potència elèctrica.

\begin{ThreePartTable}
	\begin{TableNotes}
		\item[a] {\footnotesize En alguns llibres i publicacions tècniques es poden veure els símbols no estàndard «VAR» i «VAr».}
	\end{TableNotes}
\begin{longtable}[h]{llcl}
   \caption{\label{taula:P-Q-S} Unitats de potència elèctrica}\\
   \toprule[1pt]
    Magnitud & Unitat &  Símbol & Valor en unitats SI \\
   \midrule
   \endfirsthead
   \caption[]{Unitats de potència elèctrica (\emph{ve de la pàgina anterior})}\\
   \toprule[1pt]
    Magnitud & Unitat &  Símbol & Valor en unitats SI \\
   \midrule
   \endhead
   \midrule
   \multicolumn{4}{r}{\sffamily\bfseries\color{NavyBlue}(\emph{continua a la pàgina següent})}
   \endfoot
   \insertTableNotes
   \endlastfoot
   potència activa & watt &  \unit{W}& $\qty{1}{W} = \qty{1}{W}$  \\
   potència reactiva & var &  \unit{var}\tnote{a} & $\qty{1}{var} = \qty{1}{W}$  \\
   potència aparent & voltampere &  \unit{VA}& $\qty{1}{VA} = \qty{1}{W}$  \\
   \bottomrule[1pt]
\end{longtable}
\end{ThreePartTable}
\index{potència!activa}\index{potència!reactiva}\index{potència!aparent}
\index{W}\index{VA}\index{var}\index{voltampere}\index{watt}




\subsection{Unitats relatives a màquines elèctriques rotatòries}\label{sec:unit-maq-rotativ}

Hi ha dues magnituds que poden expressar el mateix fenomen físic de rotació d'un cos; una és la freqüència $f$, expressada en hertz, la qual indica el nombre de cicles --- voltes o revolucions senceres --- per unitat de temps, i l'altra és la velocitat angular $\omega$, expressada en radiant per segon, la qual indica l'angle girat per unitat de temps. La relació entre ambdues magnituds és:
\begin{equation}
  \omega = 2 \piup f
\end{equation}

Quan es tracta de motors o d'altres màquines elèctriques rotatòries, la norma ISO 80000-3 \textit{Quantities and units --- Part 3: Space and time}, indica que «r» és el símbol àmpliament utilitzat per indicar  «revolució». \index{CEI!80000-03@80000-3} Oficialment, però, no hi ha cap símbol reconegut, i aquesta norma li assigna a la revolució la unitat  adimensional igual a 1. En aquest llibre s'utilitza el símbol «r».

En la taula \vref{taula:rev-min} es poden veure aquestes unitats relatives a màquines elèctriques rotatòries.

\begin{longtable}[h]{llcl}
   \caption{\label{taula:rev-min} Unitats relatives a màquines elèctriques rotatòries}\\
   \toprule[1pt]
    Magnitud & Unitat &  Símbol & Valor en unitats SI \\
   \midrule
   \endfirsthead
   \caption[]{Unitats relatives a màquines elèctriques rotatòries (\emph{ve de la pàgina anterior})}\\
   \toprule[1pt]
    Magnitud & Unitat &  Símbol & Valor en unitats SI \\
   \midrule
   \endhead
   \midrule
   \multicolumn{4}{r}{\sffamily\bfseries\color{NavyBlue}(\emph{continua a la pàgina següent})}
   \endfoot
   \endlastfoot
   angle pla & revolució &  \unit{r} & $\qty{1}{r} = 2\piup\unit{\,rad}$  = 1 (cicle)\\
   velocitat angular & revolució per segon &  \unit{r/s}& $\qty{1}{r/s} = 2\piup\unit{\,rad/s} = \qty{1}{Hz}$  \\
   velocitat angular & revolució per minut &  \unit{r/min}& $\qty{1}{r/min} = \frac{2\piup}{60}\unit{\,rad/s} = \frac{1}{60}\unit{\,Hz}$  \\
   \bottomrule[1pt]
\end{longtable}
\index{revolució}\index{revolució per segon}\index{revolució per minut}
\index{r}\index{angle pla}\index{velocitat angular}


\section{Normes d'escriptura}\label{sec:normes-escript}
\index{sistema internacional d'unitats!normes d'escriptura}

Es presenten a continuació algunes normes aplicables a l'escriptura
de les unitats del Sistema Internacional d'unitats.

Després de cadascuna de les explicacions es donen exemples d'escriptures correctes, precedides pel símbol \textcolor{Green}\faCheckSquare{}; exemples d'escriptures  incorrectes,  precedides pel símbol \textcolor{Red}\faTimesCircle{}; i exemples d'escriptures correctes però no recomanades, precedides pel símbol
\textcolor{Blue}\faExclamationTriangle{}.

\begin{itemize}

\item El prefix utilitzat per simbolitzar 1000 és la lletra k minúscula.  La lletra K majúscula és el símbol del  kelvin; cal tenir en compte que «\unit{\degree K}»  i «grau Kelvin» no són correctes. En canvi, el símbol de «grau Celsius» és «\unit{\degreeCelsius}», ja que la lletra C majúscula sola, és el símbol del coulomb.

\textcolor{Green}\faCheckSquare{} 6,9  kV

\textcolor{Red}\faTimesCircle{} 6,9 KV

\textcolor{Green}\faCheckSquare{} $\qty{100}{\degreeCelsius} = \qty{373,15}{K}$

\textcolor{Red}\faTimesCircle{} $\qty{100}{C} = \qty{373,15}{\degree K}$

\item Els símbols de les unitats no han d'anar seguits d'un punt, llevat que es trobin al final d'una oració, ja que no són pas
abreviatures.

\textcolor{Green}\faCheckSquare{} La tensió existent de 25 V és inferior a la necessària

\textcolor{Red}\faTimesCircle{} La tensió existent de 25 V. és inferior a la necessària

\textcolor{Green}\faCheckSquare{}  Aquest interruptor és de 40 A i 2 pols

\textcolor{Red}\faTimesCircle{}  Aquest interruptor és de 40 A. i 2 pols


\item Els símbols de les unitats no canvien de forma en el plural, no han
d'utilitzar-se abreviatures ni han d'afegir-se o suprimir-se
lletres.

\textcolor{Green}\faCheckSquare{} 150 kg

\textcolor{Red}\faTimesCircle{} 150 Kgs

\textcolor{Green}\faCheckSquare{} 25 m

\textcolor{Red}\faTimesCircle{} 25 mts

\textcolor{Green}\faCheckSquare{} \qty{33}{cm^3}

\textcolor{Red}\faTimesCircle{} 33 cc

\textcolor{Green}\faCheckSquare{} 20 s

\textcolor{Red}\faTimesCircle{} 20 seg

\textcolor{Green}\faCheckSquare{} 80 km/h

\textcolor{Red}\faTimesCircle{} 80 kph

\textcolor{Green}\faCheckSquare{} 1500 r/min

\textcolor{Red}\faTimesCircle{} 1500 rpm

\item No han de barrejar-se noms i símbols d'unitats.

\textcolor{Green}\faCheckSquare{} 4 rad/s

\textcolor{Red}\faTimesCircle{} 4 rad/segon

\textcolor{Green}\faCheckSquare{} 4 radiant per segon

\textcolor{Red}\faTimesCircle{} 4 radiant/s

\textcolor{Green}\faCheckSquare{} 100 km/h

\textcolor{Red}\faTimesCircle{} 100 km/hora

\textcolor{Green}\faCheckSquare{} 100 quilòmetre per hora

\textcolor{Red}\faTimesCircle{} 100 quilòmetre/h


\item Els símbols de les unitats  poden anar precedits de valors escrits amb xifres, però no amb lletres. En canvi, els noms de les unitats poden anar precedits de valors escrits amb xifres o amb lletres. 

\textcolor{Green}\faCheckSquare{} Només cal \qty{1}{L} d'aigua

\textcolor{Red}\faTimesCircle{} Només cal un L d'aigua

\textcolor{Green}\faCheckSquare{} Només cal 1 litre d'aigua

\textcolor{Green}\faCheckSquare{} Només cal un litre d'aigua



\item Els símbols de les unitats s'escriuen a la dreta dels valors
numèrics, separats per un espai en blanc.

\textcolor{Green}\faCheckSquare{} 25 V

\textcolor{Red}\faTimesCircle{} 25V

\textcolor{Green}\faCheckSquare{} \qty{40}{\degreeCelsius}

\textcolor{Red}\faTimesCircle{} 40\unit{\degreeCelsius}

\textcolor{Red}\faTimesCircle{} 40º C

\textcolor{Green}\faCheckSquare{} 20 nF

\textcolor{Red}\faTimesCircle{} 20nF


\item  L'única excepció al punt anterior és la mesura d'angles en graus, minuts i segons; en aquest cas s'escriu el valor i la unitat tot junt.

\textcolor{Green}\faCheckSquare{} \ang{45}

\textcolor{Red}\faTimesCircle{} \ang[number-angle-product = \,]{45}

\textcolor{Green}\faCheckSquare{} \ang{15;32;8}

\textcolor{Red}\faTimesCircle{} \ang[number-angle-product = \,]{15;32;8}

\item En el cas de símbols d'unitats derivades formats pel producte
d'altres unitats, el producte s'indicarà mitjançant un punt volat o
un espai en blanc.

\textcolor{Green}\faCheckSquare{} 2000 kW$\cdot$h

\textcolor{Green}\faCheckSquare{} 2000 kW\,h

\textcolor{Red}\faTimesCircle{} 2000 kW-h

\textcolor{Red}\faTimesCircle{} 2000 kWh

\item Quan s'utilitza un espai en blanc cal tenir en compte  l'ordre en què s'escriuen
les unitats, ja que algunes combinacions poden crear confusió i
és millor evitar-les, per exemple: \qty{24}{N\,m} (24 newton metre) i
\qty{24}{m\,N} (24~metre newton) són expressions equivalents, però
aquesta darrera forma d'escriptura pot ser confosa amb \qty{24}{mN} (24~miŀlinewton).

\textcolor{Green}\faCheckSquare{} 24 N\,m

\textcolor{Blue}\faExclamationTriangle{} 24 m\,N

\item En el cas de símbols d'unitats derivades formats per la divisió
d'altres unitats, la divisió s'indicarà mitjançant una línia
inclinada o horitzontal, o mitjançant potències negatives.

\textcolor{Green}\faCheckSquare{} \qty{100}{m\,/s}

\textcolor{Green}\faCheckSquare{} \qty{100}{m.s^{-1}}

\textcolor{Green}\faCheckSquare{} $100\,\dfrac{\text{m}}{\text{s}}$

\textcolor{Red}\faTimesCircle{} $100\,\text{m}\div\text{s}$

\item En el cas anterior, quan s'utilitza la línia inclinada i hi ha més
d'una unitat en el denominador, aquestes unitats s'han d'escriure
entre parèntesis. No és correcte usar més d'una línia inclinada.

\textcolor{Green}\faCheckSquare{} \qty{5}{m.kg/(s^3.A)}

\textcolor{Red}\faTimesCircle{} \qty{5}{m.kg/s^3.A}

\textcolor{Red}\faTimesCircle{} \qty{5}{m.kg/s^3/A}


\item No ha de deixar-se cap espai en blanc entre el símbol d'un prefix i
el símbol d'una unitat.

\textcolor{Green}\faCheckSquare{} 12 mm

\textcolor{Red}\faTimesCircle{} 12 m\,m

\textcolor{Green}\faCheckSquare{}  3 GHz

\textcolor{Red}\faTimesCircle{}  3 G\,Hz


\item El grup format pel símbol d'un prefix i el símbol d'una unitat
esdevé un nou símbol inseparable (formant un múltiple o submúltiple
de la unitat), i pot ser pujat a una potència positiva o negativa i
combinat amb altres símbols.

\textcolor{Green}\faCheckSquare{} \qty{20}{km^2}

\textcolor{Red}\faTimesCircle{} \qty{20}{(km)^2}

\textcolor{Green}\faCheckSquare{}  \qty{12}{kg\,/mm^2}

\textcolor{Red}\faTimesCircle{}  \qty{12}{kg\,/(mm)^2}


\item Només es permet un prefix davant d'una unitat.

\textcolor{Green}\faCheckSquare{} \qty{8}{nm}

\textcolor{Red}\faTimesCircle{} \qty{8}{m\micro m}


\item El quilogram --- la unitat base de massa --- és per raons històriques l'única unitat base de l'SI que conté un prefix tant en el seu nom: «quilo» o «kilo», com en el seu símbol: «k». Els noms i els símbols dels múltiples i submúltiples de la unitat de massa es formen afegint els noms i els símbols dels prefixos al nom «gram» i al símbol «g» respectivament.

\textcolor{Green}\faCheckSquare{} \qty{3e-6}{kg} = \qty{3}{mg}

\textcolor{Green}\faCheckSquare{} \qty{3e-6}{quilogram} = 3 miŀligram

\textcolor{Red}\faTimesCircle{} \qty{3e-6}{kg} = \qty{3}{\micro kg}

\textcolor{Red}\faTimesCircle{} \qty{3e-6}{quilogram} = 3 microquilogram

\item No es permeten prefixos aïllats.

\textcolor{Green}\faCheckSquare{} El nombre de partícules és de \qty[per-mode = symbol]{5e6}{\per\cubic\metre}

\textcolor{Red}\faTimesCircle{} El nombre de partícules és de \qty{5}{M\,/m^3}


\item Els prefixos no poden utilitzar-se amb les magnituds  de temps de la taula \vref{taula:SI-altres-acceptades}.

\textcolor{Green}\faCheckSquare{} \qty{1000}{d}

\textcolor{Red}\faTimesCircle{}  \qty{1}{kd}

\textcolor{Green}\faCheckSquare{} \qty{800}{h}

\textcolor{Red}\faTimesCircle{}  \qty{8}{hh}


\item En el cas dels símbols d'unitats derivades formades per la divisió
d'altres unitats, l'ús de prefixos en el numerador i denominador de
forma simultània pot causar confusió, i és preferible, per tant,
utilitzar una alta combinació d'unitats on només el numerador o el
denominador tinguin prefix.

\textcolor{Green}\faCheckSquare{} \qty{10}{MV/m}

\textcolor{Blue}\faExclamationTriangle{}  \qty{10}{kV/mm}


\item De forma anàloga, el mateix és aplicable als símbols d'unitats
derivades formades pel producte d'altres unitats.

\textcolor{Green}\faCheckSquare{} \qty{10}{kV.s}

\textcolor{Blue}\faExclamationTriangle{}  \qty{10}{MV.ms}


\item Els noms de totes les unitats de l'SI s'escriuen amb minúscula, llevat que estiguin a l'inici d'una oració o en un títol tot amb majúscules. Cal tenir en compte que seguint aquesta regla, l'escriptura  correcta
del nom de la unitat amb el símbol «\unit{\degreeCelsius}» és «grau Celsius», perquè la unitat  «grau» comença per la lletra g  minúscula i el qualificatiu «Celsius» comença par la lletra C  majúscula, atès que és un nom propi.


\textcolor{Green}\faCheckSquare{} 10 newton

\textcolor{Red}\faTimesCircle{} 10 Newton

\textcolor{Green}\faCheckSquare{}  100 watt

\textcolor{Red}\faTimesCircle{} 100 Watt

\textcolor{Green}\faCheckSquare{}  24 volt

\textcolor{Red}\faTimesCircle{} 24 Volt

\textcolor{Green}\faCheckSquare{}  20 grau Celsius

\textcolor{Red}\faTimesCircle{} 20 grau celsius


\item Les unitats que tenen noms provinents de noms propis s'han
d'escriure tal com apareixen en les taules
\vref{taula:SI-fonamentals}, \vref{taula:SI-derivades}, \vref{taula:SI-altres-acceptades}, i \vref{taula:SI-altres-NIST}, i no s'han de traduir.

\textcolor{Green}\faCheckSquare{} 50 newton

\textcolor{Red}\faTimesCircle{}  50 neuton

\textcolor{Green}\faCheckSquare{} 300 joule

\textcolor{Red}\faTimesCircle{}  300 juls

\textcolor{Green}\faCheckSquare{} $10^{-6}$ farad

\textcolor{Red}\faTimesCircle{}  $10^{-6}$ faradis


 \item Quan el nom d'una unitat
conté un prefix, ambdues parts s'han d'escriure juntes sense cap espai o element d'unió.

\textcolor{Green}\faCheckSquare{} 1 miŀligram

\textcolor{Red}\faTimesCircle{} 1 miŀli gram

\textcolor{Red}\faTimesCircle{} 1 miŀli-gram

\textcolor{Green}\faCheckSquare{}  980 hectopascal

\textcolor{Red}\faTimesCircle{} 980 hecto pascal

\textcolor{Red}\faTimesCircle{} 980 hecto-pascal


\item Els noms dels prefixos s'escriuen sempre amb minúscula, llevat que es trobin al principi d'una oració.

\textcolor{Green}\faCheckSquare{} \qty{1200}{MW} són 1200 megawatt

\textcolor{Red}\faTimesCircle{} \qty{1200}{MW} són 1200 Megawatt

\textcolor{Green}\faCheckSquare{} \qty{12}{GHz} són 12 gigahertz

\textcolor{Red}\faTimesCircle{} \qty{12}{GHz}  són 12 Gigahertz


\item En el cas  d'unitats derivades que s'expressen amb divisions o
productes, s'utilitza la preposició «per» entre dos noms d'unitats
per indicar-ne la divisió, i no s'utilitza cap paraula per indicar-ne el
producte.

\textcolor{Green}\faCheckSquare{} \qty{1}{m/s} és 1 metre per segon

\textcolor{Red}\faTimesCircle{}  \qty{1}{m/s} és 1 metre segon

\textcolor{Red}\faTimesCircle{}  \qty{1}{m/s} és 1 metre dividit per segon

 \textcolor{Green}\faCheckSquare{} \qty{20}{\ohm.m} són 20 ohm metre

\textcolor{Red}\faTimesCircle{}   \qty{20}{\ohm.m} són 20 ohm  per metre

\textcolor{Red}\faTimesCircle{}   \qty{20}{\ohm.m} són 20 ohm multiplicat per metre


\item El valor d'una quantitat ha d'expressar-se  utilitzant únicament una unitat. En són una excepció les magnituds de temps i d'angle pla de la taula \vref{taula:SI-altres-acceptades}; tot i això, pel que fa a l'angle pla és preferible seguir la regla general excepte en disciplines on s'utilitzen angles molt petits, com ara la navegació, la cartografia o l'astronomia.

\textcolor{Green}\faCheckSquare{} 10,234 m

\textcolor{Green}\faCheckSquare{} 102,34 cm

\textcolor{Green}\faCheckSquare{} 10234 mm

\textcolor{Red}\faTimesCircle{}  10 m 23 cm 4 mm

\textcolor{Green}\faCheckSquare{} 50,5 kg


\textcolor{Red}\faTimesCircle{}  50 kg 500 mg

\textcolor{Green}\faCheckSquare{} 2 d 13 h 12 min

\textcolor{Red}\faTimesCircle{}   2,55 d

\textcolor{Green}\faCheckSquare{} \ang{22,20}

\textcolor{Blue}\faExclamationTriangle{} \ang{22;12}

\item Quan s'expressa el valor d'una quantitat, és incorrecte afegir
lletres o altres símbols a la unitat; qualsevol informació
addicional necessària ha d'afegir-se a la quantitat.

\textcolor{Green}\faCheckSquare{} U\ped{rms} = 220 V

\textcolor{Red}\faTimesCircle{}  U = \qty{220}{V\ped{rms}}

\textcolor{Green}\faCheckSquare{}  I\ped{max} = 36 kA

\textcolor{Red}\faTimesCircle{}   I = \qty{36}{kA\ped{max}}


\item El separador decimal entre la part entera i decimal d'un valor pot ser el punt o la coma. L'ús de l'un o l'altre varia segons el país. Si el valor està comprès entre $-1$ i $+1$, és obligatori escriure un zero davant del separador decimal.

\textcolor{Green}\faCheckSquare{} 0,25 A

\textcolor{Red}\faTimesCircle{}  ,25 A

\textcolor{Green}\faCheckSquare{} 0.25 A

\textcolor{Red}\faTimesCircle{}  .25 A


\item Quan un valor té moltes xifres, les xifres poden dividir-se en grups de tres mitjançant un espai curt per tal de millorar-ne la llegibilitat. No s'han d'utilitzar punts o comes per separar aquests grups de tres xifres.

\textcolor{Green}\faCheckSquare{} \qty{43279,16829}{kg}

\textcolor{Green}\faCheckSquare{} \qty[group-separator =]{43279,16829}{kg}

\textcolor{Red}\faTimesCircle{}  \qty[group-separator = .]{43279,16829}{kg}

\item El símbol «\%» (per cent), reconegut internacionalment, pot utilitzar-se amb l'SI. En aquest cas, és convenient posar un espai entre el valor numèric i el símbol. És també preferible utilitzar el símbol «\%» i no pas el text  «per cent».

\textcolor{Green}\faCheckSquare{} 15 \%

\textcolor{Blue}\faExclamationTriangle{} 15\%

\textcolor{Blue}\faExclamationTriangle{} 15 per cent


\item També es pot utilitzar el símbol «ppm» (parts per milió), encara que és millor utilitzar un quocient d'unitats. Es desaconsella, en canvi, l'ús de «ppb» (parts per bilió) i «ppt» (parts per trilió), ja que 1 bilió equival a $10^{12}$ i 1 trilió equival a $10^{18}$ a l'Europa continental i a altres països, mentre que 1 bilió equival a $10^{9}$ i 1 trilió equival a $10^{12}$ a la Gran Bretanya, als Estats Units d'Amèrica i a altres països.

\textcolor{Green}\faCheckSquare{} La concentració d'àcid en aigua és de \qty{25}{\micro L/L}

\textcolor{Blue}\faExclamationTriangle{} La concentració d'àcid en aigua és de 25 ppm

\end{itemize}


El document  \textit{Guide for the Use of the International System of Units (SI)}, publicat pel NIST,  fa a més les recomanacions següents:

\begin{itemize}

\item Quan s'indiquen valors de magnituds amb les seves precisions,
s'indiquen intervals o s'expressen diversos valors numèrics, les
unitats han de ser presents en cadascun dels valors o s'han d'usar
parèntesis si es vol posar les unitats només al final.

\textcolor{Green}\faCheckSquare{} \qty[separate-uncertainty, separate-uncertainty-units = repeat]{63,2(1)}{m}

\textcolor{Green}\faCheckSquare{} \qty[separate-uncertainty]{63,2(1)}{m}

\textcolor{Blue}\faExclamationTriangle{} \qty[separate-uncertainty, separate-uncertainty-units = single]{63,2(1)}{m}

\textcolor{Blue}\faExclamationTriangle{}  $\qty{63,2}{m} \pm \num{0,1}$


\textcolor{Green}\faCheckSquare{} \qtyrange{4}{20}{mA}

\textcolor{Green}\faCheckSquare{} \qtyrange[range-units = bracket]{4}{20}{mA}

\textcolor{Blue}\faExclamationTriangle{} \qtyrange[range-units = single]{4}{20}{mA}


\textcolor{Green}\faCheckSquare{} \qtyproduct{800 x 600 x 300}{mm}

\textcolor{Green}\faCheckSquare{} \qtyproduct[product-units = bracket]{800 x 600 x 300}{mm}

\textcolor{Blue}\faExclamationTriangle{} \qtyproduct[product-units = single]{800 x 600 x 300}{mm}


\textcolor{Green}\faCheckSquare{} \qty{127}{s} + \qty{3}{s} = \qty{130}{s}

\textcolor{Green}\faCheckSquare{}  (127 + 3)\unit{\,s} = \qty{130}{s}

\textcolor{Blue}\faExclamationTriangle{}  127 + \qty{3}{s} = \qty{130}{s}


\textcolor{Green}\faCheckSquare{} \qty[separate-uncertainty, separate-uncertainty-units = repeat]{70(5)}{\percent}

\textcolor{Green}\faCheckSquare{} \qty[separate-uncertainty]{70(5)}{\percent}

\textcolor{Blue}\faExclamationTriangle{} \qty[separate-uncertainty, separate-uncertainty-units = single]{70(5)}{\percent}


\textcolor{Green}\faCheckSquare{} $240 \times (1 \pm \qty{10}{\percent})\unit{\,V}$

\textcolor{Blue}\faExclamationTriangle{}  $\qty{240}{V} \pm \qty{10}{\percent}$


\item Quan s'indica un rang de valors, és preferible fer servir la preposició «a» en lloc d'un guió, ja que el guió pot confondre's amb el signe menys.

\textcolor{Green}\faCheckSquare{} La tensió pot variar en el rang \qtyrange{-24}{24}{V}

\textcolor{Green}\faCheckSquare{} La tensió pot variar en el rang \qtyrange[range-units = bracket]{-24}{24}{V}

\textcolor{Blue}\faExclamationTriangle{}  La tensió pot variar en el rang \qty{-24}{V} - \qty{24}{V}

\textcolor{Blue}\faExclamationTriangle{} La tensió pot variar en el rang \num{-24} - 24 V 

\textcolor{Blue}\faExclamationTriangle{} La tensió pot variar en el rang (\num{-24} - 24) V 

\end{itemize}


\section{Factors de conversió d'unitats}\label{sec:SI-fact-conv}
\index{sistema internacional d'unitats!factors de conversió}
Com que la quantitat d'unitats existents és enorme, tenint en compte tant les que pertanyen a l'SI com les que no, es dona en aquest apartat l'adreça de la pàgina web del NIST on hi ha recollits un bon nombre de factors de conversió d'unitats que són rellevants en el món de la ciència i l'enginyeria.

La pàgina web en  qüestió, \href{http://www.nist.gov/pml/pubs/sp811/appenb.cfm}{www.nist.gov/pml/pubs/sp811/appenb.cfm}, correspon a l'apèndix B de la publicació \textit{Guide for the Use of the International System of Units (SI)}.
Dins d'aquesta pàgina web, l'enllaç \href{http://www.nist.gov/pml/pubs/sp811/appenb8.cfm}{B.8} ens porta a una llista de factors de conversió ordenada alfabèticament, i l'enllaç  \href{http://www.nist.gov/pml/pubs/sp811/appenb9.cfm}{B.9} ens porta a la mateixa llista de factors de conversió, però ordenada per categories.

També es pot descarregar a l'enllaç: \href{https://dx.doi.org/10.6028/NIST.SP.1038}{https://dx.doi.org/10.6028/NIST.SP.1038}, de la pàgina web del NIST, el document \textit{The International System of Units (SI) --- Conversion Factors
for General Use}, publicat l'any 2006.
