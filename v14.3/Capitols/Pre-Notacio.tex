\chapter*{Notació} \addcontentsline{toc}{chapter}{Notació}

Es presenta a continuació la notació seguida en aquest llibre.

Cal fer notar que les variables escalars
s'escriuen en lletra  de gruix normal, i que les variables vectorials i matricials s'escriuen
en lletra negreta.

\begin{list}{}
{\setlength{\labelwidth}{15mm} \setlength{\leftmargin}{20mm}
\setlength{\labelsep}{5mm}}
    \item[$j$] La unitat imaginària, definida com:
    $j\equiv\sqrt{-1}$\index{j}
    \item[$V$] Una variable real.
    \item[$\cmplx{V}$] Una variable complexa.
    \item[$\cmplx{V}^*$] Conjugat d'una variable complexa.
    Es compleixen les relacions:\\[1ex]
     $(\cmplx{V}_1 \pm \cmplx{V}_2 \pm \cdots  \pm \cmplx{V}_n)^* = \cmplx{V}_1^* \pm
    \cmplx{V}_2^*\pm\cdots\pm\cmplx{V}_n^*$\\[1ex]
    $(\cmplx{V}_1 \cmplx{V}_2 \cdots \cmplx{V}_n)^* = \cmplx{V}_1^*  \cmplx{V}_2^*
    \cdots \cmplx{V}_n^*$\\[1ex]
    $(\cmplx{V}_1 / \cmplx{V}_2)^* = \cmplx{V}_1^* / \cmplx{V}_2^*$
    \item[$|\cmplx{V}|$] Mòdul d'una variable complexa.
    Es compleixen les relacions:\\[1ex]
      $\cmplx{V}\,\cmplx{V}^* = |\cmplx{V}|^2$\\[1ex]
      $1/ \cmplx{V} = \cmplx{V}^* / \,|\cmplx{V}|^2$\\[1ex]
      $|\cmplx{V}_1 \cmplx{V}_2 \cdots \cmplx{V}_n| =
       |\cmplx{V}_1| \,|\cmplx{V}_2| \cdots |\cmplx{V}_n|$\\[1ex]
       $|\cmplx{V}_1 / \cmplx{V}_2| = |\cmplx{V}_1| \,/ \,|\cmplx{V}_2|$\\[1ex]
      $|\cmplx{V}_1+\cmplx{V}_2+\cdots+\cmplx{V}_n| \leq
      |\cmplx{V}_1| + |\cmplx{V}_2| + \cdots  +|\cmplx{V}_n|$
    \item[$\arg\cmplx{V}$] Argument (angle) d'una variable complexa.
     Es compleixen les relacions:\\[1ex]
      $\arg\cmplx{V}^* = - \arg\cmplx{V}$\\[1ex]
      $\arg(-\cmplx{V}) =  \arg\cmplx{V} + \piup$\\[1ex]
      $\arg(\cmplx{V}_1 \cmplx{V}_2 \cdots \cmplx{V}_n) = \arg\cmplx{V}_1 + \arg \cmplx{V}_2 + \cdots + \arg\cmplx{V}_n$\\[1ex]
      $\arg(\cmplx{V}_1 / \cmplx{V}_2) = \arg\cmplx{V}_1 - \arg \cmplx{V}_2$
    \item[$\Re\cmplx{V}$] Part real d'una variable complexa. Es compleix: $\Re\cmplx{V} = \dfrac{\cmplx{V} + \cmplx{V}^*}{2}$
    \item[$\Im\cmplx{V}$] Part imaginària d'una variable complexa. Es compleix: $\Im\cmplx{V} = \dfrac{\cmplx{V} - \cmplx{V}^*}{2 j}$
    \item[$A+j B$] Expressió cartesiana (part real i part
    imaginària) d'una variable complexa.
    \item[$Z_{\angle \theta}$] Expressió polar (mòdul i argument) d'una variable
    complexa. Les relacions entre $A, B, Z$ i $\theta$ són:\footnote{Cal tenir en compte que la funció \texttt{arctan} disponible en moltes calculadores i llenguatges de programació, torna de forma  estandarditzada valors compresos entre $-\frac{\piup}{2}$ i $\frac{\piup}{2}$. En aquest cas cal sumar el valor $\piup$, quan $A$ és negatiu, a l'angle obtingut amb la funció \texttt{arctan} per tal d'obtenir l'angle en el quadrant correcte.}\\[1ex]
    $Z=+\sqrt{A^2+B^2}\quad\quad\theta=\arctan{\dfrac{B}{A}}\quad\quad
    A=Z\cos\theta\quad\quad B=Z\sin\theta$
    \item[$Z\,e^{j\theta}$] Expressió d'Euler\index{Euler} d'una variable complexa, definida com:\footnote{El valor de $\theta$ ha d'expressar-se sempre en radiant, per tal que el valor de $e^{j\theta}$ sigui correcte.}
     $Z\,e^{j\theta} \equiv Z(\cos\theta+j\sin\theta)$.
     Es compleixen les relacions:\\[1ex]
     $Z_1\,e^{j\theta_1} \, Z_2\,e^{j\theta_2} = Z_1 Z_2\,e^{j(\theta_1+\theta_2)}$\\[1ex]
     %$(Z_1\,e^{j\theta_1}) \,/\, (Z_2\,e^{j\theta_2}) = \dfrac{Z_1}{Z_2}\,e^{j(\theta_1-\theta_2)}$
     $\dfrac{Z_1\,e^{j\theta_1}}{Z_2\,e^{j\theta_2}} = \dfrac{Z_1}{Z_2}\,e^{j(\theta_1-\theta_2)}$
    \item[$\boldsymbol{V}$] Una matriu real o un vector real.
    \item[$\boldsymbol{V}^{-1}$] Matriu inversa d'una matriu real.
    \item[$\transpose{\boldsymbol{V}}$] Matriu transposada d'una matriu real, o vector
    transposat d'un vector real.
    \item[$\boldsymbol{V}(n)$] Element $n$-èsim d'un vector real.
    \item[$\boldsymbol{V}(m,n)$] Element de la fila $m$ i columna $n$ d'una matriu real.
    \item[$\mcmplx{V}$] Una matriu complexa o un vector complex.
    \item[$\mcmplx{V}^{-1}$] Matriu inversa d'una matriu complexa.
    \item[$\transpose{\mcmplx{V}}$] Matriu transposada d'una matriu complexa, o vector
    transposat d'un vector complex.
    \item[$\mcmplx{V}^*$] Matriu conjugada d'una matriu complexa, o vector
    conjugat d'un vector complex.
    \item[$\hermit{V}$] Matriu conjugada transposada d'una matriu complexa, o vector
    conjugat transposat d'un vector complex, definit com: $\hermit{V} \equiv
    \transpose{(\mcmplx{V}^*)}$.
    \item[$\mcmplx{V}(n)$] Element $n$-èsim d'un vector complex.
    \item[$\mcmplx{V}(m,n)$] Element de la fila $m$ i columna $n$ d'una matriu complexa.
\end{list}

Pel que fa als sentits assignats a les fletxes que representen les
tensions i els corrents en els diversos circuits elèctrics que
apareixen en aquest llibre, s'utilitza la convenció següent:

\begin{list}{}
{\setlength{\labelwidth}{15mm} \setlength{\leftmargin}{20mm}
\setlength{\labelsep}{5mm}}
    \item[$\begin{CD} @>U>> \end{CD}$] Tensió contínua: la fletxa indica el sentit
    de la caiguda de tensió, és a dir, va del nus positiu al nus negatiu.
    \item[$\begin{CD} @>I>> \end{CD}$] Corrent
    continu: la fletxa indica el sentit  assignat com a positiu al corrent.
    \item[$\begin{CD} @>\cmplx{U}>> \end{CD}$] Tensió alterna: la fletxa indica el
    sentit assignat com a positiu a la caiguda de tensió, quan el nus d'origen de la fletxa
    té un potencial  més positiu que el nus de destinació.
    \item[$\begin{CD} @>\cmplx{I}>> \end{CD}$] Corrent altern: la fletxa
    indica el sentit  assignat com a positiu al corrent.
\end{list}

\pagebreak

En aquest llibre les variables complexes s'utilitzen per representar fasors. Un fasor $Z_{\angle \theta}$ és equivalent a una funció sinusoidal variable en el temps, la qual pot expressar-se utilitzant la funció cosinus:
\[y(t)=\sqrt{2}\, Z \cos(\omega t + \theta)\]

També pot expressar-se utilitzant la funció sinus:
\[y(t)=\sqrt{2}\, Z \sin(\omega t + \theta)\]

Quan hi ha diverses funcions sinusoidals relacionades entre si, cal utilitzar de manera uniforme la funció cosinus o la funció sinus per a totes les funcions. Les variables i paràmetres implicats són:
\begin{list}{}
{\setlength{\labelwidth}{15mm} \setlength{\leftmargin}{20mm}
\setlength{\labelsep}{5mm}}
    \item[$y(t)$] Funció sinusoidal; representa normalment una tensió o un corrent.
    \item[$t$] Temps.
    \item[$f$] Freqüència de la funció sinusoidal.
    \item[$T$] Període de la funció sinusoidal.
    \item[$\omega$] Velocitat angular de la funció sinusoidal. Es compleix: $\omega = 2 \piup f = 2 \piup\,/T$.
    \item[$Z$] Valor eficaç de la funció sinusoidal (vegeu la secció \vref{sec:val_mitja_ef}); els valors de pic de la funció sinusoidal  són:  $\pm\sqrt{2}\, Z$.
    \item[$\theta$] Angle inicial de la funció sinusoidal, on  $\theta=\omega \tau$, o $\theta=-\omega (T-\tau)$; el significat del temps $\tau$  es pot veure en el gràfic que hi ha més avall.

    Quan es fa servir la funció cosinus, $\theta$ és positiu quan s'utilitza $\tau$, és a dir, el temps mesurat  des de l'origen ($t=0$) cap a l'esquerra, fins a trobar el primer valor màxim de la funció, i $\theta$ és negatiu quan s'utilitza $T-\tau$, és a dir, el temps mesurat des de l'origen cap a la dreta, fins a trobar també el primer valor màxim de la funció.

    Quan es fa servir la funció sinus, $\theta$ és positiu quan s'utilitza $\tau$, és a dir, el temps mesurat des de l'origen ($t=0$) cap a l'esquerra, fins a trobar el primer punt on la funció es fa zero (passant de valors negatius a positius), i $\theta$ és negatiu quan s'utilitza $T-\tau$, és a dir, el temps mesurat des de l'origen cap a la dreta, fins a trobar també el primer punt on la funció es fa zero (passant de valors negatius a positius).
    \item[] \input{Imatges/Not-Fasor.pdf_tex}
\end{list}
\index{fasor}

\pagebreak

Els símbols que representen els diferents conjunts de nombres i intervals, segons la norma internacional ISO 80000-2 \textit{Quantities and units --- Part 2: Mathematics}, són els següents:

\begin{list}{}
{\setlength{\labelwidth}{15mm} \setlength{\leftmargin}{20mm}
\setlength{\labelsep}{5mm}}
	 \item[$\mathsfb{N}$] Conjunt dels nombres naturals: $\{\,0,1,2,3,4,\ldots\,\}$. 
	 
	 L'exclusió del zero s'indica amb un asterisc: $\mathsfb{N}^* = \{ n \in \mathsfb{N} \mid n \ne 0 \}$. 
	 
	 És possible indicar altres restriccions, com per exemple:  $\mathsfb{N}_{> 5} = \{ n \in \mathsfb{N} \mid n > 5\}$. 
	 
	 També s'utilitzen els símbols $\vmathbb{N}$ i  $\vvmathbb{N}$.
	 
	 \item[$\mathsfb{Z}$] Conjunt dels nombres enters: $\{\,\ldots,-3,-2,-1,0,1,2,3,\ldots\,\}$.  
	 
	 L'exclusió del zero s'indica amb un asterisc: $\mathsfb{Z}^* = \{ n \in \mathsfb{Z} \mid n \ne 0 \}$. 
	 
	 És possible indicar altres restriccions, com per exemple:  $\mathsfb{Z}_{> -3} = \{ n \in \mathsfb{Z} \mid n > -3\}$.     
	 
	 També s'utilitzen els símbols $\vmathbb{Z}$ i $\vvmathbb{Z}$.
    
	 \item[$\mathsfb{Q}$] Conjunt dels nombres racionals. 
	 
	 L'exclusió del zero s'indica amb un asterisc: $\mathsfb{Q}^* = \{ r \in \mathsfb{Q} \mid r \ne 0 \}$. 
	 
	 És possible indicar altres restriccions, com per exemple:  $\mathsfb{Q}_{< 0} = \{ r \in \mathsfb{Q} \mid r < 0 \}$. 
	 
	 També s'utilitzen els símbols $\vmathbb{Q}$ i $\vvmathbb{Q}$.
	 
	 \item[$\mathsfb{R}$] Conjunt dels nombres reals. 
	 
	 L'exclusió del zero s'indica amb un asterisc: $\mathsfb{R}^* = \{ x \in \mathsfb{R} \mid x \ne 0 \}$. 
	 
	 És possible indicar altres restriccions, com per exemple:  $\mathsfb{R}_{> 0} = \{ x \in \mathsfb{R} \mid x > 0 \}$. 
	 
	 També s'utilitzen els símbols $\vmathbb{R}$ i $\vvmathbb{R}$.
	 
	\item[$\mathsfb{C}$] Conjunt dels nombres complexos.  
	
	L'exclusió del zero s'indica amb un asterisc: $\mathsfb{C}^* = \{ z \in \mathsfb{C} \mid z \ne 0 \}$.
	
	 També s'utilitzen els símbols $\vmathbb{C}$ i $\vvmathbb{C}$.
	 
	 \item[$\mathsfb{P}$] Conjunt dels nombres primers: $\{\,2,3,5,7,11,13,17,\ldots\,\}$. 
	 
	 També s'utilitzen els símbols $\vmathbb{P}$ i $\vvmathbb{P}$.
	 
	 \item[{$[a,b]$}] Interval tancat des de $a$ inclòs fins a $b$ inclòs: $[a,b] = \{x \in \mathsfb{R} \mid a \leq x \leq b\}$.
	 
	 \item[{$(a,b]$}] Interval semiobert esquerre des de $a$ exclòs fins a $b$ inclòs: $(a,b] = \{x \in \mathsfb{R} \mid a < x \leq b\}$. 
	 
	 També s'utilitza la notació $]a,b]$.
	 
	 \item[{$[a,b)$}] Interval semiobert dret des de $a$ inclòs fins a $b$ exclòs: $[a,b) = \{x \in \mathsfb{R} \mid a \leq x < b\}$. 
	 
	 També s'utilitza la notació $[a,b[$.
	 
	 \item[{$(a,b)$}] Interval obert des de $a$ exclòs fins a $b$ exclòs: $(a,b) = \{x \in \mathsfb{R} \mid a < x < b\}$. 
	 
	 També s'utilitza la notació $]a,b[$.
	 
	 \item[{$(-\infty,b]$}] Interval iŀlimitat tancat fins a $b$ inclòs: $(-\infty,b] = \{x \in \mathsfb{R} \mid x \leq b\}$.  
	 
	 També s'utilitza la notació $]-\infty,b]$.
	 
	 \item[{$(-\infty,b)$}] Interval iŀlimitat obert fins a $b$ exclòs: $(-\infty,b) = \{x \in \mathsfb{R} \mid x < b\}$. 
	 
	 També s'utilitza la notació $]-\infty,b[$.
	 
	 \item[{$[a,+\infty)$}] Interval iŀlimitat tancat des de $a$ inclòs: $[a, +\infty) = \{x \in \mathsfb{R} \mid  x \geq a\}$. 
	 
	 També s'utilitzen les notacions $[a, \infty)$, $[a, +\infty[$ i $[a, \infty[$.

	\item[{$(a,+\infty)$}] Interval iŀlimitat obert des de $a$ exclòs: $(a, +\infty) = \{x \in \mathsfb{R} \mid  x > a\}$. 
	
	També s'utilitzen les notacions $(a, \infty)$, $]a, +\infty[$ i $]a, \infty[$.
\end{list}
\index{n@$\mathsfb{N}$} \index{n@$\mathsfb{N}^*$}
\index{z@$\mathsfb{Z}$} \index{z@$\mathsfb{Z}^*$}
\index{q@$\mathsfb{Q}$} \index{q@$\mathsfb{Q}^*$}
\index{r@$\mathsfb{R}$} \index{r@$\mathsfb{R}^*$}
\index{c@$\mathsfb{C}$} \index{c@$\mathsfb{C}^*$}
\index{p@$\mathsfb{P}$}

