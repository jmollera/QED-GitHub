\chapter{Constants Físiques}\label{sec:const_fis} \index{constants físiques}

\section{Taula de valors}

En la taula \vref{taula:Const-Fis} es pot veure una recopilació de
constants físiques; les xifres entre parèntesis que hi apareixen representen l'error absolut del valor mesurat.

Els valors de les constants d'aquesta taula són els publicats
l'any 2018 pel \textit{Committee on Data for Science and Technology}
(CODATA), un comitè científic de l'\textit{International Council
for Science}.

A partir de l'any 2018 hi va haver un canvi important en el tractament d'algunes constants respecte de les publicades anteriorment pel CODATA l'any 2014. Això es va deure al fet que les definicions de les constants de l'any 2018 es van fer d'acord amb la renovació de les definicions de les unitats de l'SI que va tenir lloc l'any 2019 (vegeu l'apèndix \ref{sec:SI}). La qüestió més remarcable és que algunes constants que abans tenien un valor mesurat amb una certa precisió, es van redefinir amb un valor exacte; en aquesta categoria figuren les constants de Planck, Boltzmann i Avogadro, la càrrega fonamental, l'eficàcia lluminosa, la freqüència de transició hiperfina del cesi-133 i la massa atòmica relativa del carboni-12. Aquestes noves definicions van ocasionar l'efecte contrari en altres constants que abans tenien un valor exacte i que van passar a tenir un valor mesurat amb una certa precisió; en aquesta categoria figuren, per exemple, les permeabilitats elèctrica i magnètica en el buit, i la impedància característica del buit.

Podeu trobar  més informació a
les adreces: \href{http://www.codata.org/}{www.codata.org} i \href{http://physics.nist.gov/cuu/Constants/}{physics.nist.gov/cuu/Constants}.\index{CODATA}
\index{NIST}

\begin{ThreePartTable}
\begin{TableNotes}
    \item[a] {\footnotesize El valor numèric en si, és l'anomenat nombre d'Avogadro.}
    \item[b] {\footnotesize $K\ped{cd}$ és l'eficàcia lluminosa d'una radiació monocromàtica de freqüència \qty{540e12}{Hz}.}
    \item[c] {\footnotesize Donada una partícula X, $m(\mathrm{X})$ és la massa atòmica de la partícula X, $M(\mathrm{X})$ és la massa molar de la partícula X, i $A\ped{r}(\mathrm{X})$ és la massa atòmica relativa de la partícula X. Es compleixen les relacions següents: $M(\mathrm{X}) = m(\mathrm{X})\, N\ped{A}$ i $ A\ped{r}(\mathrm{X}) = \frac{m(\mathrm{X})}{m\ped{u}} = \frac{M(\mathrm{X})}{M\ped{u}}$.}
    \item[d] {\footnotesize Un «electró-volt» és l'energia cinètica que adquireix un electró després de creuar una diferència de potencial d'un volt en el buit.}
\end{TableNotes}
\begin{longtable}{lcll}
   \caption{\label{taula:Const-Fis} Constants físiques}\\
   \toprule[1pt]
   Magnitud & Símbol & Valor & Error relatiu\\
   \midrule
   \endfirsthead
   \caption[]{Constants físiques (\emph{ve de la pàgina anterior})} \\
   \toprule[1pt]
   Magnitud & Símbol & Valor & Error relatiu\\
   \midrule
   \endhead
   \midrule
   \multicolumn{4}{r}{\sffamily\bfseries\color{NavyBlue}(\emph{continua a la pàgina següent})}
   \endfoot
   \insertTableNotes
   \endlastfoot
   freqüència de la transició & $\Deltaup\nu\ped{Cs}$ & \qty{9 192 631 770}{Hz} & valor exacte \\
   hiperfina del cesi-133 & & & \\[0.8em]
   velocitat de la llum en el buit & $c$ & \qty{299792458}{m/s} & valor exacte\\[0.8em]
   constant de Planck & $h$ & \qty{6,62607015 e-34}{J/Hz} & valor exacte \\[0.8em]
   càrrega elemental & $e$ & \qty{1,602176634 e-19}{C} & valor exacte \\[0.8em]
   constant de Boltzmann & $k$ & \qty{1,380649e-23}{J/K} & valor exacte \\[0.8em]
   constant d'Avogadro & $N\ped{A}$\tnote{a} & \qty{6,02214076 e23}{mol^{-1}} & valor exacte \\[0.8em]
   eficàcia lluminosa & $K\ped{cd}$\tnote{b} & \qty{683}{lm/W} & valor exacte \\[0.8em]
   massa atòmica relativa & $A\ped{r}({}^{12}\mathrm{C})\,$\tnote{c} & 12 & valor exacte \\
   del carboni-12 & & & \\[0.8em]
   constant de Planck reduïda: $\frac{h}{2\piup}$ & $\hbar$ & \qty{1,054571817\dots e-34}{J.s} & valor exacte \\[0.8em]   
   constant d'Stefan-Boltzmann:  & $\sigma$ & \qty{5,670374419\dots e-8}{W/(m^2.K^4)} & valor exacte \\
   $\frac{\piup^2 k^4}{60\, \hbar^3 c^2}$ & & & \\[0.8em]
   constant molar dels gasos: $N\ped{A} k$ & $R$ & \qty{8,31446261815324}{\,J/(mol.K)} & valor exacte \\[0.8em]
   constant de Faraday: $N\ped{A} e$ & $F$ & \qty{96485,3321233100184}{C/mol} & valor exacte \\[0.8em]
   electró-volt & eV\tnote{d} & \qty{1,602176634e-19}{J} & valor exacte \\[0.8em]
   atmosfera estàndard  & -- & \qty{101325}{Pa} & valor exacte \\[0.8em]
   acceleració de la gravetat & $g\ped{n}$ & \qty{9,80665}{m/s^2} & valor exacte \\
   estàndard & & & \\[0.8em]
   constant de massa atòmica: & $m\ped{u}$\tnote{c} & \qty{1,6605390660(50)e-27}{kg} & \num{3e-10}  \\
   $\frac{1}{12}  m({}^{12}\mathrm{C})$ & & & \\[0.8em]
   constant de massa molar: & $M\ped{u}$\tnote{c} & \qty{0,99999999965(30)e-3}{kg/mol} & \num{3e-10}  \\
   $N\ped{A} m\ped{u}$ & & & \\[0.8em]
   massa molar del carboni-12: & $M({}^{12}\mathrm{C})\,$\tnote{c} & \qty{11,9999999958(36)e-3}{kg/mol} & \num{3e-10} \\
   $A\ped{r}({}^{12}\mathrm{C})\,M\ped{u} $  & & & \\[0.8em]
   constant gravitacional & $G$ &   \qty{6,67430(15) e-11}{m^3/(kg.s^2)} & \num{2,2e-5} \\
   de Newton & & & \\[0.8em]
   constant de l'estructura & $\alpha$ & \num{7,2973525693(11) e-3} & \num{1,5e-10} \\
   fina: $\frac{e^2}{4\piup\epsilon_0\hbar  c}$ & & & \\[0.8em]
   permeabilitat magnètica & $\mu_0$ & \qty{1,25663706212(19) e-6}{N/A^2} & \num{1,5e-10} \\
   en el buit: $\frac{4\piup\alpha\hbar}{e^2  c}$  & & & \\[0.8em]
   permeabilitat  elèctrica  & $\epsilon_0$ & \qty{8,8541878128(13) e-12}{F/m} & \num{1,5e-10} \\
   en el buit: $\frac{1}{\mu_0 c^2}$ & & & \\[0.8em]
   impedància característica  & $Z_0$ &  \qty{376,730313668(57)}{\ohm} & \num{1,5e-10}\\
   del buit: $\sqrt{\frac{\mu_0}{\epsilon_0}}=\mu_0 c$ & & &  \\[0.8em]
   massa de l'electró & $m\ped{e}$ & \qty{9,1093837015(28) e-31}{kg} & \num{3,0e-10} \\[0.8em]
   massa del  protó & $m\ped{p}$ & \qty{1,67262192369(51) e-27}{kg} & \num{3,1e-10} \\[0.8em]
   massa del neutró & $m\ped{n}$ & \qty{1,67492749804(95) e-27}{kg} & \num{5,7e-10} \\[0.8em]
   massa del deuteri & $m\ped{d}$ & \qty{3,3435837724(10) e-27}{kg} & \num{3,0e-10} \\[0.8em]
   massa del triti & $m\ped{t}$ & \qty{5,0073567446(15) e-27}{kg} & \num{3,0e-10} \\[0.8em]
   massa de la partícula $\alphaup$ & $m_\alphaup$ & \qty{6,6446573357(20) e-27}{kg} & \num{3,0e-10} \\[0.8em]
   radi de Bohr: $\frac{\hbar}{\alpha m\ped{e}c} = \frac{4\piup\epsilon_0\hbar^2}{m\ped{e}e^2}$ & $a_0$ & \qty{5,29177210903(80) e-11}{m} & \num{1,5e-10} \\[0.8em]
   longitud d'ona Compton   & $\lambdabar\ped{C}$ & \qty{3,861 592 6796(12)e-13}{m} & \num{3,0e-10} \\
   reduïda: $\frac{\hbar}{m\ped{e} c} = \alpha a_0$ & & & \\[0.8em]
\bottomrule[1pt]
\end{longtable}
\end{ThreePartTable}
\index{velocitat de la llum en el buit}  \index{constant!magnètica} \index{impedància característica del
buit} \index{atmosfera estàndard} \index{acceleració!de la gravetat
estàndard} \index{massa!molar del carboni-12}
\index{constant!gravitacional de Newton} \index{constant!de Planck}
\index{constant!de Planck redu\"{i}da} \index{constant!de Faraday}
\index{carrega elemental@càrrega elemental} \index{massa!de
l'electró} \index{massa!del protó} \index{massa!del neutró}
\index{massa!del deuteri} \index{massa!de la partícula $\alpha$}
\index{constant!d'Avogadro} \index{constant!molar
dels gasos} \index{constant!de Boltzmann}
\index{constant!d'Stefan-Boltzmann} \index{radi de Bohr}\index{longitud d'ona Compton reduïda}
\index{$\mu_0$} \index{$\epsilon_0$} \index{c@$c$} 
\index{gn@$g\ped{n}$} \index{Z0@$Z_0$} \index{F@$F$}
\index{m@$M({}^{12}\mathrm{C})$} \index{G@$G$} \index{h@$h$}
\index{h@$\hbar$} \index{e@$e$} \index{me@$m\ped{e}$}
\index{mp@$m\ped{p}$} \index{mn@$m\ped{n}$} \index{m@$m\ped{d}$}
\index{ma@$m_\alpha$} \index{NA@$N\ped{A}$} \index{R@$R$}
\index{k@$k$} \index{$\sigma$} \index{a0@$a_0$} \index{$\lambdabar\ped{C}$}
\index{eV}\index{electró-volt}\index{eficàcia lluminosa}\index{Kcd@$K\ped{cd}$}
\index{$\Deltaup\nu\ped{Cs}$}\index{massa!atòmica relativa del carboni-12}
\index{constant!de massa atòmica}\index{constant!de massa molar}
\index{Ar@$A\ped{r}({}^{12}\mathrm{C})$}\index{mu@$m\ped{u}$}\index{Mu@$M\ped{u}$}


\section{Error absolut i relatiu}\label{sec:err_abs_rel}

Tal com s'ha dit al principi, les xifres entre parèntesis indiquen l'error absolut del valor mesurat que les precedeix.\footnote{Més exactament, el valor entre parèntesis és la incertesa estàndard del valor mesurat, segons el document JCGM 100:2008 (GUM 1995 \textit{with minor
corrections}), \textit{Evaluation of measurement data --- Guide to the expression of uncertainty in
measurement}, el qual es pot obtenir a:  \href{https://www.bipm.org/documents/20126/2071204/JCGM\_100\_2008\_E.pdf/cb0ef43f-baa5-11cf-3f85-4dcd86f77bd6}{www.bipm.org/documents/20126/2071204/JCGM\_100\_2008\_E.pdf/cb0ef43f-baa5-11cf-3f85-4dcd86f77bd6}.} El nombre de xifres entre parèntesis determina la posició decimal d'aquest error; per exemple, en el cas de la  massa de l'electró tenim:
\index{error!absolut}\index{error!relatiu}

\[
    m\ped{e} = \qty{9,109 383 7015(28) e-31}{kg}
\]

Les dues xifres entre parèntesis, 28, determinen que la posició decimal de l'error absolut ha de correspondre amb la posició de les dues últimes xifres, 15, del valor; l'error absolut $\epsilon$  és doncs:

\[
    \epsilon = \qty{0,000 000 0028 e-31}{kg}
\]

Per tant, el valor de la massa de l'electró també es podria escriure així:

 \[
    m\ped{e} = \qty[separate-uncertainty,separate-uncertainty-units = repeat]{9,109 383 7015(28) e-31}{kg}
\]

O de forma més compacta:

\[
m\ped{e} = \qty[separate-uncertainty,separate-uncertainty-units = bracket]{9,109 383 7015(28) e-31}{kg}
\]

Finalment, l'error relatiu $\epsilon\ped{r}$ es calcula dividint l'error absolut pel valor sense l'error:

\[
    \epsilon\ped{r} = \frac{\qty{0,000 000 0028 e-31}{kg}}{\qty{9,109 383 7015 e-31}{kg}} =   \num{3,0e-10}
\]
