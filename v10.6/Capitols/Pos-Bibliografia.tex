\begin{thebibliography}{99}

    \addcontentsline{toc}{chapter}{Bibliografia}

    \bibitem{KOP} Helmut Kopka, Patrick W. Daly. \textbf{A Guide To \LaTeX}.  Addison-Wesley, 4th edition, 2004.
    \bibitem{GRZ} George Grätzer. \textbf{More Math Into \LaTeX}.  Springer, 4th edition, 2007.
    \bibitem{GOM} Michel Goossens, Frank Mittelbach. \textbf{The \LaTeX{} Companion}.  Addison-Wesley, 2nd edition, 2004.
    \bibitem{GOO} Michel Goossens, Frank Mittelbach, Sebstian Rahtz, Denis Roegel, Herbert Voß. \textbf{The \LaTeX{} Graphics Companion}.  Addison-Wesley, 2nd edition, 2008.
    \bibitem{VOS} Herbert Voß. \textbf{Typesetting tables with \LaTeX{}}.  UIT Cambrige Ltd., 2011.
    \bibitem{SPK} Scott Pakin. \textbf{The Comprehensive \LaTeX{} Symbol List}. CTAN.ORG.

    \bibitem{VALa} Gabriel Valiente Feruglio. \textbf{Composició de textos científics amb \LaTeX}.  Edicions UPC, 1998.
    \bibitem{VALb} Gabriel Valiente Feruglio. \textbf{Modern Catalan Typographical Conventions}.  TUGBoat, 16(3), 329-338, 1995.
    \bibitem{BEC} Claudio Beccari. \textbf{Typesetting mathematics for science and technology according to ISO 31/XL}.  TUGBoat, 18(1), 39-48, 1997.
    \bibitem{WIL} J. William Howard, Jr. \textbf{Graecum est: el uso del griego en textos electrónicos de carácter científico-técnico}.  Panace@, VI(19), 45-54, 2005.

    \bibitem{BUR} Richard Stevens Burington. \textbf{Handbook of Mathematical Tables and Formulas}.  McGraw-Hill, 4th edition, 1965.
    \bibitem{SCH} Joel L. Schiff. \textbf{The Laplace Transform: Theory and Applications}.  Springer, 1999.
    \bibitem{RJB} R. J. Beerends, H. G. ter Morsche, J. C. van den Berg, E. M. van de Vrie. \textbf{Fourier and Laplace Transforms}.  Cambridge University Press, 2003.
    \bibitem{JDH} Joe D. Hoffman. \textbf{Numerical Methods for Engineers and Scientists}.  Marcel Dekker, Inc., 2nd edition, 2001.
    \bibitem{EJB} E. Joseph Billo. \textbf{Excel${}^\circledR$ for Engineers and Scientists -- Numerical Methods}.  Wiley-INTERSCIENCE, 2007.
    \bibitem{AGVS} Amos Gilat, Vish Subramaniam. \textbf{Numerical Methods for Engineers and Scientists --
        An Introduction with Applications using MATLAB${}^\circledR$}.  Wiley, 3rd edition, 2013.
    \bibitem{WMF} Walter Mora Flores. \textbf{Introducción a los Métodos Numéricos}.  Instituto Tecnológico de Costa Rica, 2016.

    \bibitem{RASa} Enrique Ras. \textbf{Teoría de circuitos. Fundamentos}.  Marcombo Boixareu Editores, 3\textordfeminine\ edición, 1977.
    \bibitem{RASb} Enrique Ras. \textbf{Transformadores. De potencia, medida y protección}.  Marcombo Boixareu Editores, 6\textordfeminine\ edición, 1985.
    \bibitem{RASc} Enrique Ras. \textbf{Teoría de líneas eléctricas (Volumen I)}.  Marcombo Boixareu Editores, 2\textordfeminine\ edición, 1986.
    \bibitem{RASd} Enrique Ras. \textbf{Redes eléctricas y multipolos}.  Marcombo Boixareu Editores, 1980.
    \bibitem{RASe} Enrique Ras. \textbf{Análisis de Fourier y cálculo operacional aplicados a la electrotecnia}.  Marcombo Boixareu Editores, 1979.

    \bibitem{COR} Felipe Córcoles López, Joaquim Pedra Durán, Miquel Salichs Vivancos. \textbf{Transformadores}.  Edicions UPC, 2004.

    \bibitem{TRA} Jesús Trashorras Montecelos. \textbf{Subestaciones eléctricas}.  Paraninfo, 2015.


    \bibitem{CHA} Stephen J. Chapman. \textbf{Máquinas Eléctricas}.  McGraw-Hill, 4\textordfeminine\ edición, 2005.
    \bibitem{FIT} A. E. Fitzgerald, Charles Kingsley Jr., Stephen D. Umans. \textbf{Electric Machinery}.  McGraw-Hill, 6th edition, 2003.


    \bibitem{GRA} John J. Grainger, William D. Stevenson Jr. \textbf{Análisis de Sistemas de Potencia}.  McGraw-Hill, 1996.
    \bibitem{HAD} Hadi Saadat. \textbf{Power System Analysis}.  McGraw-Hill, 2nd edition, 2004.
    \bibitem{DUN} J. Duncan Glover, Mulukutla S. Sarma, Thomas J. Overbye. \textbf{Power System Analysis \& Design}.  CENGAGE Learning, 5th edition (SI), 2011.

    \bibitem{PMA} Paul M. Anderson. \textbf{Analysis of Faulted Power Systems}.  Wiley-INTERSCIENCE, 1995.
    \bibitem{BLAa} J. Lewis Blackburn. \textbf{Simmetrical Components for Power Systems Engineering}.  Marcel Dekker, Inc, 1993.
    \bibitem{BLAb} J. Lewis Blackburn, Thomas J. Domin. \textbf{Protective Relaying. Principles and Applications}.  CRC Press, 3rd edition, 2007.
    \bibitem{REI} Donald Reimert. \textbf{Protective Relaying for Power Generation Systems}.  CRC Press, 2006.


    \bibitem{TLE} Nasser D. Tleis. \textbf{Power Systems Modelling and Fault Analysis -- Theory and Practise}.  ELSEVIER, 2008.
    \bibitem{KAS} Ismail Kasikci. \textbf{Short Circuits in Power Systems. A practical Guide to IEC 60909}.  Wiley-VCH, 2002.
    \bibitem{JSch} Jürgen Schlabbach. \textbf{Short-Circuit Currents}.  The Institution of Engineering and Technology, 2005.
    \bibitem{RRop} Richard Roeper. \textbf{Corrientes de cortocircuito en redes trifásicas}.  Siemens Aktiengesellschaft \& Marcombo Boixareu Editores, 1985.

    \bibitem{JCD} J. C. Das. \textbf{Power System Analysis -- Short-Circuit, Load Flow and Harmonics}. Marcel Dekker, Inc., 2002.
    \bibitem{MAI} Mohamed A. Ibrahim. \textbf{Disturbance Analysis for Power Systems}. Wiley-IEEE Press, 2012.

    \bibitem{CAP} Robert Capella. \textbf{Protecciones eléctricas en MT}.  Publicación Técnica de Schneider 071, mayo 2003.
    \bibitem{LLO} Manuel Llorente Antón. \textbf{Líneas y cables}.  Publicación Técnica de Schneider 073, enero 2003.
    \bibitem{PAS} Jean Pasteau. \textbf{Envolventes y grados de protección}.  Cuaderno Técnico de Schneider 166, febrero 2001.
    \bibitem{FONa} Paola Fonti. \textbf{Transformadores de intensidad: cómo determinar sus especificaciones}.  Cuaderno Técnico de Schneider 194, agosto 2000.
    \bibitem{FONb} Paola Fonti. \textbf{Transformadores de intensidad: errores de especificación y soluciones}.  Cuaderno Técnico de Schneider 195, diciembre 2001.

    \bibitem{KNU} Knut Sjövall. \textbf{Instrument Transformers Application Guide, Edition 3}.  ABB, 2009.

\end{thebibliography}
