\chapter{Components Simètriques} \index{components simètriques} \label{sec:comp-sim}

\section{Introducció}
La teoria de les components simètriques és útil en l'estudi de
tensions i corrents trifàsics
 desequilibrats, com ara els que es produeixen en un curtcircuit on no intervenen les tres
 fases alhora (curtcircuit fase--terra, fase--fase, etc.).

\section{L'operador complex «a»}

Definim en primer lloc l'operador complex «a», el qual té un mòdul
igual a la unitat i un argument igual a \ang{120}: \index{a}
\begin{equation}
   \au \equiv \numpd{1}{120} = \eu^{\ju\frac{2\piup}{3}} =
   \cos \frac{2\piup}{3} + \ju \sin \frac{2\piup}{3} = - \frac{1}{2} + \ju \frac{\sqrt{3}}{2}
\end{equation}

A l'hora d'operar amb aquest valor, resulten útils les relacions
següents:
\begin{subequations}
\begin{align}
    \au^2 &= \numpd{1}{240}= - \frac{1}{2} - \ju \frac{\sqrt{3}}{2} \\
    \au^3 &= \numpd{1}{0} = 1 \\
    0 &= 1+ \au + \au^2
 \end{align}
\end{subequations}

\section{\texorpdfstring{Teorema de Fortescue--Stokvis}{Teorema de Fortescue-Stokvis}}\index{teorema!de Fortescue--Stokvis}

Tal com es veu en la Figura \vref{pic:Comp_sim}, aquest teorema
estableix que qualsevol sistema trifàsic asimètric (també anomenat
desequilibrat),  es pot descompondre  en la suma de tres sistemes
simètrics: un sistema directe o de seqüència positiva, un sistema
invers o de seqüència negativa, i un sistema homopolar o de
seqüència zero. Els fasors $\cmplx{\Upsilon}\ped{A}$,
$\cmplx{\Upsilon}\ped{B}$ i $\cmplx{\Upsilon}\ped{C}$ de la Figura \vref{pic:Comp_sim}, poden representar tant
tensions com corrents.

\begin{center}
    \input{Imatges/Cap-CompSim-CompSim.pdf_tex}
    \captionof{figure}{Components simètriques -- Teorema de Fortescue--Stokvis}
    \label{pic:Comp_sim}
\end{center}

\index{sistema!directe} \index{sistema!invers}
\index{sistema!homopolar} El sistema directe està format per tres
fasors que tenen la mateixa seqüència de fases que els fasors
originals, per exemple: A--B--C; els fasors
s'identifiquen mitjançant els subíndexs «1» o «d». El sistema
invers està format per tres fasors que tenen la seqüència de fases contrària
 que els fasors originals, per exemple: A--C--B; els fasors s'identifiquen mitjançant els
subíndexs «2» o «i». Finalment, el sistema homopolar està
format per tres fasors que estan en fase entre si; els fasors
s'identifiquen mitjançant el subíndex «0» o «h».

Les equacions següents expressen els fasors del sistema asimètric de la Figura \vref{pic:Comp_sim}, en funció
dels fasors dels tres sistemes simètrics de la mateixa figura:
\begin{subequations}
\begin{align}
   \cmplx{\Upsilon}\ped{A} &= \cmplx{\Upsilon}\ped{A,0}  +
   \cmplx{\Upsilon}\ped{A,1} + \cmplx{\Upsilon}\ped{A,2} \label{eq:c_sim_a}\\
   \cmplx{\Upsilon}\ped{B} &= \cmplx{\Upsilon}\ped{B,0} + \cmplx{\Upsilon}\ped{B,1} +
   \cmplx{\Upsilon}\ped{B,2}  =  \cmplx{\Upsilon}\ped{A,0} + \au^2
   \cmplx{\Upsilon}\ped{A,1} + \au \cmplx{\Upsilon}\ped{A,2} \label{eq:c_sim_b}\\
   \cmplx{\Upsilon}\ped{C} &= \cmplx{\Upsilon}\ped{C,0} + \cmplx{\Upsilon}\ped{C,1} +
   \cmplx{\Upsilon}\ped{C,2}  = \cmplx{\Upsilon}\ped{A,0} + \au
   \cmplx{\Upsilon}\ped{A,1} + \au^2 \cmplx{\Upsilon}\ped{A,2} \label{eq:c_sim_c}
\end{align}
\end{subequations}

O en forma matricial:
\begin{equation}\label{eq:c_sim_mat}
   \begin{pmatrix}
     \cmplx{\Upsilon}\ped{A} \\
     \cmplx{\Upsilon}\ped{B} \\
     \cmplx{\Upsilon}\ped{C}
   \end{pmatrix} =
   \begin{pmatrix}
     1 & 1 & 1 \\
     1 & \au^2 & \au\\
     1 & \au & \au^2
   \end{pmatrix} \cdot
   \begin{pmatrix}
     \cmplx{\Upsilon}\ped{A,0} \\
     \cmplx{\Upsilon}\ped{A,1} \\
     \cmplx{\Upsilon}\ped{A,2}
   \end{pmatrix}
\end{equation}

A partir del sistema d'equacions anterior, podem trobar els fasors
dels tres sistemes simètrics en funció dels fasors del sistema
asimètric:
\begin{subequations}
\begin{align}
   \cmplx{\Upsilon}\ped{A,0} &= \frac{1}{3} (\cmplx{\Upsilon}\ped{A} + \cmplx{\Upsilon}\ped{B} +
   \cmplx{\Upsilon}\ped{C}) & \cmplx{\Upsilon}\ped{B,0} &= \cmplx{\Upsilon}\ped{A,0} &
   \cmplx{\Upsilon}\ped{C,0} &= \cmplx{\Upsilon}\ped{A,0}
   \label{eq:c_sim_c2}\\
   \cmplx{\Upsilon}\ped{A,1} &= \frac{1}{3} (\cmplx{\Upsilon}\ped{A} + \au \cmplx{\Upsilon}\ped{B} +
   \au^2 \cmplx{\Upsilon}\ped{C}) & \cmplx{\Upsilon}\ped{B,1} &= \au^2 \cmplx{\Upsilon}\ped{A,1} &
   \cmplx{\Upsilon}\ped{C,1} &= \au \cmplx{\Upsilon}\ped{A,1} \label{eq:c_sim_a2} \\
   \cmplx{\Upsilon}\ped{A,2} &= \frac{1}{3} (\cmplx{\Upsilon}\ped{A} + \au^2 \cmplx{\Upsilon}\ped{B} +
   \au \cmplx{\Upsilon}\ped{C}) & \cmplx{\Upsilon}\ped{B,2} &= \au \cmplx{\Upsilon}\ped{A,2} &
   \cmplx{\Upsilon}\ped{C,2} &= \au^2 \cmplx{\Upsilon}\ped{A,2} \label{eq:c_sim_b2}
\end{align}
\end{subequations}

O en forma matricial:
\begin{equation}\label{eq:c_sim_mat1}
   \begin{pmatrix}
     \cmplx{\Upsilon}\ped{A,0} \\
     \cmplx{\Upsilon}\ped{A,1} \\
     \cmplx{\Upsilon}\ped{A,2}
   \end{pmatrix} =
   \begin{pmatrix}
     1 & 1 & 1 \\
     1 & \au^2 & \au\\
     1 & \au & \au^2
   \end{pmatrix}^{-1} \hspace{-2mm}\cdot
   \begin{pmatrix}
     \cmplx{\Upsilon}\ped{A} \\
     \cmplx{\Upsilon}\ped{B} \\
     \cmplx{\Upsilon}\ped{C}
   \end{pmatrix} =  \frac{1}{3} \times
   \begin{pmatrix}
     1 & 1 & 1 \\
     1 & \au & \au^2 \\
     1 & \au^2 & \au
   \end{pmatrix} \cdot
   \begin{pmatrix}
     \cmplx{\Upsilon}\ped{A} \\
     \cmplx{\Upsilon}\ped{B} \\
     \cmplx{\Upsilon}\ped{C}
   \end{pmatrix}
\end{equation}

\section{Corrent de neutre} \index{corrent de neutre}

Suposem un sistema trifàsic amb neutre de retorn, que pot ser el
terra, on qualsevol de les seves parts (generador, línia o consum)
poden ser desequilibrades; el corrent que circula pel neutre és
sempre la suma dels tres corrents de fase: $\cmplx{I}\ped{A}+
\cmplx{I}\ped{B}+\cmplx{I}\ped{C}$. A partir d'aquest fet, i
observant l'equació \eqref{eq:c_sim_c2}, es veu que el corrent de
retorn pel neutre és igual a tres vegades la component homopolar
del sistema de corrents de fase:
\begin{equation}
    \cmplx{I}\ped{A}+\cmplx{I}\ped{B}+\cmplx{I}\ped{C} =3 \cmplx{I}\ped{A,0}
\end{equation}

D'altra banda, quan un sistema trifàsic no té conductor neutre de retorn, tenim
$\cmplx{I}\ped{A}+ \cmplx{I}\ped{B}+\cmplx{I}\ped{C}=0$, i per tant,
observant la mateixa equació \eqref{eq:c_sim_c2}, es veu que el
sistema format pels corrents de fase no té sistema homopolar, és a dir: $\cmplx{I}\ped{A,0}=\cmplx{I}\ped{B,0}=\cmplx{I}\ped{C,0}=0$.

Finalment, també podem dir que el corrent total a terra en cas de
defecte a terra, és igual a tres vegades la component homopolar del
corrent de curtcircuit.

\section{Propietats de les tensions fase--fase i fase--neutre}\label{sec:comp-sim-neutre}
\index{tensió!fase--fase} \index{tensió!fase--neutre}

En la Figura \vref{pic:Comp_sim_tens} s'ha representat un sistema de
tensions fase--fase: $\cmplx{U}\ped{AB}$,
$\cmplx{U}\ped{BC}$ i $\cmplx{U}\ped{CA}$, i dos
sistemes de tensions fase--neutre, dels infinits que poden existir
depenent de la posició del punt neutre: $\cmplx{U}\ped{AG}$,
$\cmplx{U}\ped{BG}$ i $\cmplx{U}\ped{CG}$, i
$\cmplx{U}\ped{AN}$, $\cmplx{U}\ped{BN}$ i
$\cmplx{U}\ped{CN}$. El punt neutre G del primer sistema
coincideix amb el baricentre (intersecció de les mitjanes) del
triangle  format per les tres tensions fase--fase, mentre que el
punt neutre N del segon sistema està desplaçat respecte
d'aquest baricentre.\index{baricentre}

\begin{center}
    \input{Imatges/Cap-CompSim-Tensions.pdf_tex}
    \captionof{figure}{Components simètriques -- Tensions fase--fase i fase--neutre}
    \label{pic:Comp_sim_tens}
\end{center}

Atenent a l'equació \eqref{eq:c_sim_c2}, es veu que el sistema
format per les tensions fase--fase no té component homopolar, ja que
la seva suma  és sempre igual a zero: $\cmplx{U}\ped{AB} +
\cmplx{U}\ped{BC} + \cmplx{U}\ped{CA} = 0$. Si a més
d'aquesta consideració, tenim en compte el que s'ha dit en l'apartat
anterior, resulta que un sistema trifàsic desequilibrat sense conductor
neutre de retorn, es pot estudiar tenint en compte únicament un sistema directe
i un sistema invers, ja que en aquest cas tant les tensions fase--fase com els
corrents de fase, no tenen component homopolar.

Pel que fa a les components directa i inversa del sistema format per
les tensions fase--fase, es compleix el següent: les components
directa i inversa del sistema de tensions fase--fase, són
respectivament els fasors fase--fase de les components directa i
inversa del sistema de tensions fase--neutre.

Expressant-ho en forma matemàtica tenim:
\begin{subequations}
\begin{align}
   \cmplx{U}\ped{AB,0} &= 0 &
   \cmplx{U}\ped{BC,0} &= 0 &
   \cmplx{U}\ped{CA,0} &= 0 \\
   \cmplx{U}\ped{AB,1} &= (1-\au^2) \cmplx{U}\ped{AN,1} =
   \cmplx{U}\ped{AN,1} \sqrt{3}_{\angle\ang{30}} &
   \cmplx{U}\ped{BC,1} &= \au^2 \cmplx{U}\ped{AB,1} &
   \cmplx{U}\ped{CA,1} &= \au \cmplx{U}\ped{AB,1} \label{eq:c_sim_a3} \\
   \cmplx{U}\ped{AB,2} &= (1-\au) \cmplx{U}\ped{AN,2}  =
   \cmplx{U}\ped{AN,2} \sqrt{3}_{\angle\ang{-30}}&
   \cmplx{U}\ped{BC,2} &= \au \cmplx{U}\ped{AB,2} &
   \cmplx{U}\ped{CA,2} &= \au^2 \cmplx{U}\ped{AB,2} \label{eq:c_sim_b3}
\end{align}
\end{subequations}

En aquestes equacions s'han utilitzat les components directa i
inversa del sistema de tensions
$\cmplx{U}\ped{AN}$, $\cmplx{U}\ped{BN}$ i $\cmplx{U}\ped{CN}$,
però també es podrien haver utilitzat les components directa i
inversa del sistema de tensions
$\cmplx{U}\ped{AG}$, $\cmplx{U}\ped{BG}$ i $\cmplx{U}\ped{CG}$,
ja que es compleix el següent: tots el sistemes de tensió
fase--neutre que tinguin els mateixos extrems $A, B,
C$, tenen les mateixes components directa i inversa; en termes
més electrotècnics, es pot dir que qualsevol joc d'impedàncies en
estrella connectat a les mateixes fases $A, B, C$,
origina unes tensions fase--neutre, les components directa i inversa
de les quals, són independents de les característiques de les
impedàncies.

El sistema de tensions fase--neutre
$\cmplx{U}\ped{AG}$, $\cmplx{U}\ped{BG}$ i $\cmplx{U}\ped{CG}$,
el punt neutre $G$ del qual coincideix amb el baricentre del
triangle $A, B,
 C$, és l'únic que té un sistema homopolar nul; la resta de sistemes de tensions
 fase--neutre, com ara el format per les tensions $\cmplx{U}\ped{AN}$, $\cmplx{U}\ped{BN}$ i $\cmplx{U}\ped{CN}$,
 el punt neutre $N$ del qual està desplaçat respecte del punt $G$, tenen un sistema
 homopolar de valor:
\begin{equation}
    \cmplx{U}\ped{AN,0} = \cmplx{U}\ped{BN,0} =
    \cmplx{U}\ped{CN,0} = \cmplx{U}\ped{GN}\label{eq:tens-hom}
\end{equation}

Amb relació al paràgraf anterior, es pot afirmar que si es
connecten tres impedàncies idèntiques en estrella a un sistema
de tensions trifàsic, la tensió del punt neutre de l'estrella
coincidirà amb el baricentre $G$ del triangle format per les tensions
fase--fase d'aquest sistema de tensions, i per tant les tensions fase--neutre no tindran
component homopolar; de fet, $G$ és el punt neutre de les tensions
fase--fase del sistema de tensions trifàsic.

El sistema de tensions fase--neutre
$\cmplx{U}\ped{AG}$, $\cmplx{U}\ped{BG}$ i $\cmplx{U}\ped{CG}$, es pot determinar directament a partir del sistema de tensions fase--fase $\cmplx{U}\ped{AB}$, $\cmplx{U}\ped{BC}$ i $\cmplx{U}\ped{CA}$, utilitzant les equacions \eqref{eq:bari_x} i \eqref{eq:bari_y}, que ens donen les coordenades del punt $G$:
\begin{subequations}
\begin{align}
    \cmplx{U}\ped{AG} &= \frac{\cmplx{U}\ped{AB}-\cmplx{U}\ped{CA}}{3}\label{eq:uag}\\[1ex]
    \cmplx{U}\ped{BG} &= \frac{\cmplx{U}\ped{BC}-\cmplx{U}\ped{AB}}{3}\label{eq:ubg}\\[1ex]
    \cmplx{U}\ped{CG} &= \frac{\cmplx{U}\ped{CA}-\cmplx{U}\ped{BC}}{3}\label{eq:ucg}
\end{align}
\end{subequations}

\section{Potència} \index{potència complexa!trifàsica}

Tal com es veu en l'equació \eqref{eq:s_3f}, la qual fa referència a
la Figura \vref{pic:pot_comp_trif}, la potència complexa trifàsica
en un sistema desequilibrat $\cmplx{S}\ped{3F}$, es calcula a partir
de les tres tensions fase--neutre $\cmplx{U}\ped{AN}$,
$\cmplx{U}\ped{BN}$ i $\cmplx{U}\ped{CN}$, i dels tres
corrents de fase $\cmplx{I}\ped{A}$, $\cmplx{I}\ped{B}$ i
$\cmplx{I}\ped{C}$.


No obstant, si calculem els sistemes directe, invers i homopolar,
corresponents a les tensions i corrents anteriors,
$\cmplx{U}\ped{AN,1}$, $\cmplx{U}\ped{AN,2}$ i
$\cmplx{U}\ped{AN,0}$, i $\cmplx{I}\ped{A,1}$,
$\cmplx{I}\ped{A,2}$ i $\cmplx{I}\ped{A,0}$, podem
expressar la potència complexa trifàsica a partir d'aquests nous
valors, utilitzant les equacions \eqref{eq:c_sim_a},
\eqref{eq:c_sim_b} i \eqref{eq:c_sim_c}:
\begin{equation}
\begin{split}
   \cmplx{S}\ped{3F} &= \cmplx{U}\ped{AN} \,\cmplx{I}\ped{A}^* +
   \cmplx{U}\ped{BN} \,\cmplx{I}\ped{B}^* +  \cmplx{U}\ped{CN} \,\cmplx{I}\ped{C}^* = \\[1ex]
   &= \big(\cmplx{U}\ped{AN,0} + \cmplx{U}\ped{AN,1} +
   \cmplx{U}\ped{AN,2}\big) \big(\cmplx{I}\ped{A,0} + \cmplx{I}\ped{A,1} +
   \cmplx{I}\ped{A,2}\big)^* +  \\[1ex]
   &+ \big(\cmplx{U}\ped{AN,0} + \au^2 \cmplx{U}\ped{AN,1} +
   \au \cmplx{U}\ped{AN,2} \big) \big(\cmplx{I}\ped{A,0} + \au^2 \cmplx{I}\ped{A,1}
    + \au \cmplx{I}\ped{A,2} \big)^* + \\[1ex]
   &+ \big(\cmplx{U}\ped{AN,0} + \au \cmplx{U}\ped{AN,1} + \au^2
   \cmplx{U}\ped{AN,2} \big) \big(\cmplx{I}\ped{A,0} + \au
   \cmplx{I}\ped{A,1} + \au^2 \cmplx{I}\ped{A,2} \big)^* =  \\[1ex]
   &= 3\, \cmplx{U}\ped{AN,0} \, \cmplx{I}\ped{A,0}^* +
      3\, \cmplx{U}\ped{AN,1} \, \cmplx{I}\ped{A,1}^* +
      3\, \cmplx{U}\ped{AN,2}\,  \cmplx{I}\ped{A,2}^*
    \label{eq:c_sim_s}
\end{split}
\end{equation}

\begin{exemple}[Aplicació de les components simètriques - Impedàncies equilibrades]\label{ex:comp-sim}
    Es tracta de trobar la potència consumida per una càrrega trifàsica
    formada per tres resistències idèntiques de valor: $R\ped{A}=R\ped{B}=R\ped{C}=\SI{10}{\ohm}$,
    connectades en estrella a un sistema trifàsic sense neutre, i la
    tensió a què està sotmesa cada resistència; els valors de les
    tensions del sistema trifàsic són: $|\cmplx{U}\ped{AB}| =
    \SI{2760}{V}$, $|\cmplx{U}\ped{BC}| = \SI{1840}{V}$ i
    $|\cmplx{U}\ped{CA}| = \SI{2300}{V}$.

    Tal com s'ha explicat en la secció \vref{sec:comp-sim-neutre}, en ser idèntiques les tres impedàncies el punt neutre que es formarà coincidirà amb el baricentre del triangle de tensions A--B--C. Utilitzarem doncs la lletra «G» per designar aquest punt neutre, enlloc de la lletra «N».

    Assignem de forma arbitrària, tal com s'ha fet en la Figura
    \vref{pic:Comp_sim_tens}, un angle de fase igual a zero a la tensió
    $\cmplx{U}\ped{AB}$.

    \begin{center}
        \input{Imatges/Cap-CompSim-Tensions-Exemple1.pdf_tex}
    \end{center}

    A continuació trobem els angles $\varphi\ped{A}$ i $\varphi\ped{B}$,
    corresponents als vèrtexs  $A$ i $B$ del triangle de
    tensions, utilitzant la llei dels cosinus (vegeu la Secció
    \vref{sec:llei-s-c-t}): \index{llei dels cosinus}
    \begin{align*}
        \varphi\ped{A} &= \arccos \frac{|\cmplx{U}\ped{AB}|^2 + |\cmplx{U}\ped{CA}|^2 -
        |\cmplx{U}\ped{BC}|^2}{2 |\cmplx{U}\ped{AB}| \,|\cmplx{U}\ped{CA}|} =
        \arccos \frac{(\SI{2760}{V})^2 + (\SI{2300}{V})^2 - (\SI{1840}{V})^2}{2 \times \SI{2760}{V}
        \times \SI{2300}{V}} = \ang{41,41} \\[1ex]
        \varphi\ped{B} &= \arccos \frac{|\cmplx{U}\ped{BC}|^2 + |\cmplx{U}\ped{AB}|^2 -
        |\cmplx{U}\ped{CA}|^2}{2 |\cmplx{U}\ped{BC}| \,|\cmplx{U}\ped{AB}|} =
        \arccos \frac{(\SI{1840}{V})^2 + (\SI{2760}{V})^2 - (\SI{2300}{V})^2}{2 \times \SI{1840}{V}
        \times \SI{2760}{V}} = \ang{55,77}
    \end{align*}

    Els fasors corresponents a les tres tensions són doncs:
    \begin{align*}
    \cmplx{U}\ped{AB} &= \SIpd{2760}{0}{V} \\
    \cmplx{U}\ped{BC} &= \num{1840}_{\angle\ang{180} + \ang{55,77}}\si{\,V} =
    \SIpd{1840}{-124,23}{V} \\
    \cmplx{U}\ped{CA} &= \num{2300}_{\angle\ang{180} - \ang{41,41}}\si{\,V} = \SIpd{2300}{138,59}{V}
    \end{align*}

    Tal com s'ha dit anteriorment, el sistema de tensions fase--fase   té una component homopolar nuŀla.     Trobem a continuació les components directa, inversa i homopolar de les
    tensions $\cmplx{U}\ped{AB}$, $\cmplx{U}\ped{BC}$ i
    $\cmplx{U}\ped{CA}$, utilitzant les equacions
    \eqref{eq:c_sim_c2}, \eqref{eq:c_sim_a2} i \eqref{eq:c_sim_b2}:

    \begin{align*}
    \cmplx{U}\ped{AB,1} &= \frac{1}{3} \big(
    \SIpd{2760}{0}{V} + \numpd{1}{120} \times \SIpd{1840}{-124,23}{V} +
    \numpd{1}{240} \times \SIpd{2300}{138,59}{V}\big) = \SIpd{2267,09}{5,04}{V} \\[1ex]
    \cmplx{U}\ped{AB,2} &= \frac{1}{3} \big(
    \SIpd{2760}{0}{V} + \numpd{1}{240} \times \SIpd{1840}{-124,23}{V} +
    \numpd{1}{120} \times \SIpd{2300}{138,59}{V} \big) = \SIpd{539,77}{-21,66}{V} \\[1ex]
    \cmplx{U}\ped{AB,0} &= \frac{1}{3} \big(
    \SIpd{2760}{0}{V} + \SIpd{1840}{-124,23}{V} + \SIpd{2300}{138,59}{V}\big) = \SI{0}{V}
    \end{align*}

    Trobem ara les components directa i inversa
    de les tensions fase--neutre, utilitzant les equacions
    \eqref{eq:c_sim_a3} i \eqref{eq:c_sim_b3}; a més sabem que aquestes tensions fase--neutre no
    tenen component homopolar, ja que la càrrega trifàsica és equilibrada
    (tres resistències idèntiques):
    \begin{align*}
        \cmplx{U}\ped{AG,1} &=
        \frac{\cmplx{U}\ped{AB,1}}{\sqrt{3}_{\angle\ang{30}}} =
        \frac{\SIpd{2267,09}{5,04}{V}}{\sqrt{3}_{\angle\ang{30}}} =
        \SIpd{1308,91}{-24,96}{V} \\[1ex]
        \cmplx{U}\ped{AG,2} &=
        \frac{\cmplx{U}\ped{AB,2}}{\sqrt{3}_{\angle\ang{-30}}} =
        \frac{\SIpd{539,77}{137,42}{V}}{\sqrt{3}_{\angle\ang{-30}}} =
        \SIpd{311,64}{8,34}{V} \\[1ex]
        \cmplx{U}\ped{AG,0} &= \SI{0}{V}
    \end{align*}


    A partir d'aquests valors, podem calcular ara les components
    directa, inversa i homopolar del corrent que circula per les
    resistències, aplicant les lleis de Kirchhoff; donat que les tres resistències són idèntiques, les seves  components    directa, inversa i homopolar $R_1$,  $R_2$ i $R_0$ també ho són i tenen el mateix valor de $\SI{10}{\ohm}$.
    \begin{align*}
    \cmplx{I}\ped{A,1} &=
    \frac{\cmplx{U}\ped{AG,1}}{R_1} =
    \frac{\SIpd{1308,91}{-24,96}{V}}{\SI{10}{\ohm}} =
    \SIpd{130,89}{-24,96}{A} \\[1ex]
    \cmplx{I}\ped{A,2} &=
    \frac{\cmplx{U}\ped{AG,2}}{R_2} =
    \frac{\SIpd{311,64}{8,34}{V}}{\SI{10}{\ohm}} =
    \SIpd{31,16}{8,34}{A} \\[1ex]
    \cmplx{I}\ped{A,0} &=
    \frac{\cmplx{U}\ped{AG,0}}{R_0} =
    \frac{\SI{0}{V}}{\SI{10}{\ohm}} =
    \SI{0}{A}
    \end{align*}

    Podem ara  calcular ja  la potència consumida per la càrrega
    trifàsica, utilitzant l'equació \eqref{eq:c_sim_s}:
    \[
    \begin{split}
    \cmplx{S}\ped{3F} &=  3\,
    \cmplx{U}\ped{AG,0}\,  \cmplx{I}\ped{A,0}^* +
    3\, \cmplx{U}\ped{AG,1}\,
    \cmplx{I}\ped{A,1}^* + 3\, \cmplx{U}\ped{AG,2}\,
    \cmplx{I}\ped{A,2}^*  = \\
    &= \SI{0}{kW} + 3 \times \SIpd{1308,91}{-24,96}{V} \times
    \SIpd{130,89}{24,96}{A} + 3 \times
    \SIpd{311,64}{8,34}{V} \times \SIpd{31,16}{-8,34}{A} = \\
    &= \SI{543,11}{kW}
    \end{split}
    \]

    Utilitzarem ara les equacions \eqref{eq:c_sim_a},
    \eqref{eq:c_sim_b} i \eqref{eq:c_sim_c} per trobar les tensions $\cmplx{U}\ped{AG}$, $\cmplx{U}\ped{BG}$ i $\cmplx{U}\ped{CG}$, a què
    estan sotmeses les tres resistències, a partir de les  tensions $\cmplx{U}\ped{AG,0}$, $\cmplx{U}\ped{AG,1}$ i $\cmplx{U}\ped{AG,2}$:
    \begin{align*}
        \cmplx{U}\ped{AG} &= \SI{0}{V} + \SIpd{1308,91}{-24,96}{V} +
        \SIpd{311,64}{8,34}{V}  =
        \SIpd{1578,66}{-18,74}{V} \\[1ex]
        \cmplx{U}\ped{BG} &= \SI{0}{V} + \numpd{1}{240} \times
        \SIpd{1308,91}{-24,96}{V} +
        \numpd{1}{120} \times
        \SIpd{311,64}{8,34}{V}  =
        \SIpd{1362,86}{-158,16}{V}    \\[1ex]
        \cmplx{U}\ped{CG} &= \SI{0}{V} + \numpd{1}{120} \times
        \SIpd{1308,91}{-24,96}{V} +
        \numpd{1}{240} \times \SIpd{311,64}{8,34}{V}  =
        \SIpd{1039,96}{102,78}{V}
    \end{align*}

    Finalment, trobarem les mateixes tres tensions $\cmplx{U}\ped{AG}$, $\cmplx{U}\ped{BG}$ i $\cmplx{U}\ped{CG}$ a partir de les tensions $\cmplx{U}\ped{AB}$, $\cmplx{U}\ped{BC}$ i $\cmplx{U}\ped{CA}$, utilitzant les equacions \eqref{eq:uag}, \eqref{eq:ubg} i \eqref{eq:ucg}:
    \begin{align*}
        \cmplx{U}\ped{AG} &= \frac{\SIpd{2760}{0}{V}-\SIpd{2300}{138,59}{V}}{3}
        = \SIpd{1578,66}{-18,74}{V} \\[1.5ex]
        \cmplx{U}\ped{BG} &= \frac{\SIpd{1840}{-124,23}{V}-\SIpd{2760}{0}{V}}{3}  =
        \SIpd{1362,86}{-158,16}{V}  \\[1.5ex]
        \cmplx{U}\ped{CG} &= \frac{\SIpd{2300}{138,59}{V}-\SIpd{1840}{-124,23}{V}}{3}  = \SIpd{1039,96}{102,78}{V}
    \end{align*}
\end{exemple}


\begin{exemple}[Aplicació de les components simètriques - Impedàncies desequilibrades]\label{ex:comp-sim-deseq}
    Partim del mateix sistema de tensions trifàsic de l'exemple \vref{ex:comp-sim}, és a dir: $|\cmplx{U}\ped{AB}| =  \SI{2760}{V}$, $|\cmplx{U}\ped{BC}| = \SI{1840}{V}$ i
    $|\cmplx{U}\ped{CA}| = \SI{2300}{V}$, però considerem ara que les tres resistències connectades en estrella tenen valors diferents:  $R\ped{A}=\SI{5}{\ohm}$, $R\ped{B}=\SI{10}{\ohm}$ i $R\ped{C}=\SI{15}{\ohm}$.

    \begin{center}
        \input{Imatges/Cap-CompSim-Tensions-Exemple2.pdf_tex}
    \end{center}

    Es vol calcular en aquest cas  les components directa, inversa i homopolar de les tensions a què estan sotmeses les resistències.

    Els tres fasors $\cmplx{U}\ped{AB} $, $\cmplx{U}\ped{BC} $ i $\cmplx{U}\ped{CA}$ són les que s'han trobat en l'exemple \ref{ex:comp-sim}:
    \begin{align*}
        \cmplx{U}\ped{AB} &= \SIpd{2760}{0}{V} \\
        \cmplx{U}\ped{BC} &= \SIpd{1840}{-124,23}{V} \\
        \cmplx{U}\ped{CA} &= \SIpd{2300}{138,59}{V}
    \end{align*}

    En aquest cas, el punt neutre «N» que es formarà no coincidirà amb el baricentre del triangle de tensions A--B--C, donat que les tres resistències tenen valors diferents (càrrega  desequilibrada). Les tensions $\cmplx{U}\ped{AN} $, $\cmplx{U}\ped{BN} $ i $\cmplx{U}\ped{CN}$ poden calcular-se fàcilment utilitzant el teorema de Millman (vegeu la secció \vref{sec:millman}).

    Si tenim en compte que el punt «N» és el punt neutre de les tres resistències connectades en estrella, podem calcular $\cmplx{U}\ped{NA} $, $\cmplx{U}\ped{NB} $ i $\cmplx{U}\ped{NC}$ a partir del teorema de Millman, prenent com a punt de referència respectivament, els punts «A»,  «B»  i «C»:
    \begin{align*}
          \cmplx{U}\ped{NA} &= \frac{\dfrac{\cmplx{U}\ped{AA}}{R\ped{A}} + \dfrac{\cmplx{U}\ped{BA}}{R\ped{B}} + \dfrac{\cmplx{U}\ped{CA}}{R\ped{C}}}{\dfrac{1}{R\ped{A}} + \dfrac{1}{R\ped{B}} + \dfrac{1}{R\ped{C}}} = \frac{0 + \dfrac{-(\SIpd{2760}{0}{V})}{\SI{10}{\ohm}} + \dfrac{\SIpd{2300}{138,59}{V}}{\SI{15}{\ohm}}}{\dfrac{1}{\SI{5}{\ohm}} + \dfrac{1}{\SI{10}{\ohm}} + \dfrac{1}{\SI{15}{\ohm}}} = \SIpd{1101,65}{165,46}{V} \\[1.5ex]
          \cmplx{U}\ped{NB} &= \frac{\dfrac{\cmplx{U}\ped{AB}}{R\ped{A}} + \dfrac{\cmplx{U}\ped{BB}}{R\ped{B}} + \dfrac{\cmplx{U}\ped{CB}}{R\ped{C}}}{\dfrac{1}{R\ped{A}} + \dfrac{1}{R\ped{B}} + \dfrac{1}{R\ped{C}}} = \frac{\dfrac{\SIpd{2760}{0}{V}}{\SI{5}{\ohm}} + 0 + \dfrac{-(\SIpd{1840}{-124,23}{V})}{\SI{15}{\ohm}}}{\dfrac{1}{\SI{5}{\ohm}} + \dfrac{1}{\SI{10}{\ohm}} + \dfrac{1}{\SI{15}{\ohm}}} = \SIpd{1716,07}{9,28}{V} \\[1.5ex]
          \cmplx{U}\ped{NC} &= \frac{\dfrac{\cmplx{U}\ped{AC}}{R\ped{A}} + \dfrac{\cmplx{U}\ped{BC}}{R\ped{B}} + \dfrac{\cmplx{U}\ped{CC}}{R\ped{C}}}{\dfrac{1}{R\ped{A}} + \dfrac{1}{R\ped{B}} + \dfrac{1}{R\ped{C}}} = \frac{\dfrac{-(\SIpd{2300}{138,59}{V})}{\SI{5}{\ohm}} + \dfrac{\SIpd{1840}{-124,23}{V}}{\SI{10}{\ohm}} + 0}{\dfrac{1}{\SI{5}{\ohm}} + \dfrac{1}{\SI{10}{\ohm}} + \dfrac{1}{\SI{15}{\ohm}}} = \SIpd{1408,22}{-62.11}{V}
    \end{align*}

    Així doncs, tenim:
    \begin{align*}
        \cmplx{U}\ped{AN} &= -\cmplx{U}\ped{NA} =  \SIpd{1101,65}{-14,54}{V} \\
        \cmplx{U}\ped{BN} &= -\cmplx{U}\ped{NB} =  \SIpd{1716,07}{-170,72}{V} \\
        \cmplx{U}\ped{CN} &= -\cmplx{U}\ped{NC} =  \SIpd{1408,22}{117,89}{V}
    \end{align*}

    Les components directa, inversa i homopolar $\cmplx{U}\ped{AN,1}$, $\cmplx{U}\ped{AN,2}$ i
    $\cmplx{U}\ped{AN,0}$, les obtenim utilitzant les equacions
    \eqref{eq:c_sim_c2}, \eqref{eq:c_sim_a2} i \eqref{eq:c_sim_b2}:
    \begin{align*}
        \cmplx{U}\ped{AN,1} &= \frac{1}{3} \big(
        \SIpd{1101,65}{-14,54}{V} + \numpd{1}{120} \times \SIpd{1716,07}{-170,72}{V} +
        \numpd{1}{240} \times \SIpd{1408,22}{117,89}{V} \big) =\\[1ex]
        &= \SIpd{1308,91}{-24,96}{V} \\[1.5ex]
        \cmplx{U}\ped{AN,2} &= \frac{1}{3} \big(
        \SIpd{1101,65}{-14,54}{V} + \numpd{1}{240} \times \SIpd{1716,07}{-170,72}{V} +
        \numpd{1}{120} \times \SIpd{1408,22}{117,89}{V} \big) =\\[1ex]
        &= \SIpd{311,64}{8,34}{V} \\[1.5ex]
        \cmplx{U}\ped{AN,0} &= \frac{1}{3} \big(
        \SIpd{1101,65}{-14,54}{V} + \SIpd{1716,07}{-170,72}{V} + \SIpd{1408,22}{117,89}{V} \big) = \\[1ex]
        &= \SIpd{486,68}{151,73}{V}
    \end{align*}

    Com es pot veure, els valors de $\cmplx{U}\ped{AN,1}$ i $\cmplx{U}\ped{AN,2}$ són iguals respectivament als valors de $\cmplx{U}\ped{AG,1}$ i $\cmplx{U}\ped{AG,2}$, calculats en l'exemple \ref{ex:comp-sim}.
    Això és així, ja que tal com s'ha dit en la secció \vref{sec:comp-sim-neutre}, tots el sistemes de tensió fase–neutre que tenen els mateixos extrems A, B, C, tenen les mateixes components directa i inversa.

    Pel que fa a la tensió homopolar $\cmplx{U}\ped{AN,0}$, ha de complir-se l'equació \eqref{eq:tens-hom}, que ens diu que $\cmplx{U}\ped{AN,0}$ és igual a $\cmplx{U}\ped{GN}$. Utilitzant el valor de  $\cmplx{U}\ped{AG}$ calculat en l'exemple \ref{ex:comp-sim}, tenim:
    \[
        \cmplx{U}\ped{GN} = \cmplx{U}\ped{GA} + \cmplx{U}\ped{AN} = -(\SIpd{1578,66}{-18,74}{V}) +
        \SIpd{1101,65}{-14,54}{V} = \SIpd{486,68}{151,73}{V}
    \]
\end{exemple}

\section{Programes de càlcul de components simètriques}\label{sec:calcul-comp-sim}

En la secció \vref{sec:HP_ELC} es donen  una sèrie de programes escrits per a la calculadora \emph{HP Prime},
\index{HP Prime!exemples} alguns dels quals faciliten la resolució numèrica de les equacions que han aparegut en aquest capítol. En concret tenim:

\begin{itemize}
   \item \funsfbs{Triangle\_a\_Fasors}. A partir d'un triangle de tensions obté el fasors que el formen.
       \index{HP Prime!programes!TriangleaFasors@\funsfbs{Triangle\_a\_Fasors}}
   \item \funsfbs{FN\_a\_FF}. Obté les tres tensions fase--fase corresponents a tres tensions fase--neutre.
       \index{HP Prime!programes!FNaFF@\funsfbs{FN\_a\_FF}}
   \item \funsfbs{FF\_a\_FN}. Obté les tres tensions fase--neutre corresponents a tres tensions fase--fase, per a tres impedàncies qualssevol connectades en estrella.
       \index{HP Prime!programes!FFaFN@\funsfbs{FF\_a\_FN}}
   \item \funsfbs{FF\_a\_FG}. Obté les tres tensions fase--G corresponents a tres tensions fase--fase, on G és el baricentre del triangle que formen les tres tensions fase--fase.
       \index{HP Prime!programes!FFaFG@\funsfbs{FF\_a\_FG}}
   \item \funsfbs{ABC\_a\_A012}. Obté els fasors de seqüència homopolar, directa i inversa, corresponents a tres fasors.
       \index{HP Prime!programes!ABCaA012@\funsfbs{ABC\_a\_A012}}
   \item \funsfbs{A012\_a\_ABC}. Obté els  fasors corresponents a tres fasors de seqüència homopolar, directa i inversa.
       \index{HP Prime!programes!A012aABC@\funsfbs{A012\_a\_ABC}}
   \item \funsfbs{AN12\_a\_AB12}. Obté els fasors de seqüència directa i inversa de les tensions fase--fase, a partir dels fasors de seqüència directa i inversa de les tensions fase--neutre.
       \index{HP Prime!programes!AN12aAB12@\funsfbs{AN12\_a\_AB12}}
   \item \funsfbs{AB12\_a\_AN12}. Obté els fasors de seqüència directa i inversa de les tensions fase--neutre, a partir dels fasors de seqüència directa i inversa de les tensions fase--fase.
       \index{HP Prime!programes!AB12aAN12@\funsfbs{AB12\_a\_AN12}}
\end{itemize}

