\usepackage[T1]{fontenc}
\usepackage[widermath,partialup]{kpfonts}
\usepackage{pifont}   % Dingbats de llistes i enumeracions, i marques de correcte/incorrecte
\usepackage{marvosym} % S\'{\i}mbol d'una carta (e-mail)
\usepackage{alltt}    % Representaci\'{o} de codi de programa en lletra typewriter

\usepackage{amssymb} % S\'{\i}mbols i fonts blackboard
\usepackage{amscd}   % Definici\'{o} sentitis de corrent i tensi\'{o} en la secci\'{o} Notaci\'{o}
\usepackage{amsthm}  % Extensi\'{o} de l'entorn \newtheorem
\usepackage{beccari-v3}

\usepackage{siunitx}
\sisetup{
 %inter-unit-product =  \ensuremath{{}{\cdot}{}},
    inter-unit-product =  \ensuremath{{}\cdot{}},
    input-symbols = \piup\dots\pm,
    output-decimal-marker = {,},
    output-complex-root = \ensuremath{\mathrm{j}},
    complex-root-position = before-number,
    list-final-separator = \ensuremath{\text{~i~}},
    list-pair-separator = \ensuremath{\text{~i~}},
    range-phrase = \ensuremath{\text{~a~}},
    arc-separator = {\,}
}
\providecommand*{\SIpd}[3]{\ensuremath{\num{#1}_{\angle\ang{#2}}\,\si{#3}}}       % \SI with a complex in polar form (degrees)
\providecommand*{\SIpr}[3]{\ensuremath{\num{#1}_{\angle\SI{#2}{rad}}\,\si{#3}}}   % \SI with a complex in polar form (rad)
\providecommand*{\numpd}[2]{\ensuremath{\num{#1}_{\angle\ang{#2}}}}               % \num  with a complex in polar form (degrees)
