\chapter*{Historial}
\addcontentsline{toc}{chapter}{Historial}

Es presenta a continuaci\'{o} l'evoluci\'{o} que ha tingut aquest llibre, en
les successives versions que han aparegut.

\section*{Versi\'{o} 1.0 (8 de gener de 2005)}
\addcontentsline{toc}{section}{Versi\'{o} 1.0}

Despr\'{e}s de molts esfor\c{c}os, surt a la llum la primera versi\'{o} d'aquest
llibre, format pels cap\'{\i}tols 1, 2, 3, 4, 5, 6 i 7, i els ap\`{e}ndixs A,
B, C, D i E.

\section*{Versi\'{o} 1.1 (8 de febrer de 2005)}
\addcontentsline{toc}{section}{Versi\'{o} 1.1}

S'afegeix al llibre aquest apartat {"<}Historial{">}.

En l'apartat {"<}Notaci\'{o}{">}, s'especifica que el m\`{o}dul d'un nombre
complex \'{e}s igual a l'arrel quadrada \emph{positiva} de la suma dels
quadrats de les seves parts real i imagin\`{a}ria.

Es modifiquen les equacions \eqref{eq:p_3f_34} i \eqref{eq:q_3f_34}.

S'amplia la secci\'{o} corresponent a les difer\`{e}ncies entre les
normatives \textsf{CEI} i \textsf{ANSI}, que fan refer\`{e}ncia als
transformadors de mesura i protecci\'{o} (secci\'{o}
\ref{sec:comp_tt_ti_cei_ansi}).

Es revisa tot el text, fent-hi algunes petites modificacions i
correccions.

\section*{Versi\'{o} 1.2 (16 d'abril de 2005)}
\addcontentsline{toc}{section}{Versi\'{o} 1.2}

En l'apartat {"<}Notaci\'{o}{">}, s'afegeix l'explicaci\'{o} de la convenci\'{o}
seguida a l'hora de dibuixar les fletxes que representen les
tensions i els corrents.

S'afegeix l'ap\`{e}ndix F, on s'explica la designaci\'{o} de les classes de
refrigeraci\'{o} en els transformadors de pot\`{e}ncia.

\section*{Versi\'{o} 1.3 (24 d'octubre de 2005)}
\addcontentsline{toc}{section}{Versi\'{o} 1.3}

Els ap\`{e}ndixs A a F de la versi\'{o} 1.2, es desplacen tres lletres cap
avall, passant a ser els ap\`{e}ndixs D a I respectivament.

S'afegeix un nou ap\`{e}ndix A, dedicat a l'alfabet grec.

S'afegeix un nou ap\`{e}ndix B, dedicat al sistema internacional
d'unitats (SI).

S'afegeix un nou ap\`{e}ndix C, dedicat a les constants f\'{\i}siques.

En l'apartat {"<}Notaci\'{o}{">}, s'amplien les definicions corresponents al
conjugat i al m\`{o}dul d'un nombre complex, i s'inclouen les
definicions de $\mcmplx{V}^*$ i $\hermit{V}$.

S'ha ampliat la secci\'{o} \ref{sec:pot_complex}, corresponent a la
pot\`{e}ncia complexa.

 S'ha ampliat l'exemple de la secci\'{o}
\ref{sec:seccio_pu}.

En la secci\'{o} \ref{sec:EZS}, s'ha afegit el c\`{a}lcul de $R\ped{P}$ i
$\cmplx{Z}\ped{S}$.

 A l'hora de referir-se a la
relaci\'{o} de transformaci\'{o} d'un transformador, se substitueix el
s\'{\i}mbol {"<}$\ddot{u}${">} emprat en les versions anteriors, pel s\'{\i}mbol
{"<}$m${">}.

\section*{Versi\'{o} 1.4 (2 de desembre de 2005)}
\addcontentsline{toc}{section}{Versi\'{o} 1.4}

Es representa correctament la Figura \ref{pic:acobl}, ja que estava
tallada per la dreta.

Es corregeix l'equaci\'{o} \eqref{eq:cdt_trif_exact} i l'exemple que hi
ha a continuaci\'{o}, el qual en fa \'{u}s.

Es revisa tot el text, fent-hi algunes correccions.

\section*{Versi\'{o} 2.0 (3 d'agost de 2006)}
\addcontentsline{toc}{section}{Versi\'{o} 2.0}

S'ha modificat el criteri de colors utilitzat, a l'hora de ressaltar
els enlla\c{c}os interns del document (equacions, p\`{a}gines, etc.) i els
enlla\c{c}os externs; ara els enlla\c{c}os interns s\'{o}n de
\textcolor{red}{color vermell}, i els enlla\c{c}os externs s\'{o}n de
\textcolor{magenta}{color magenta}. A m\'{e}s totes els encap\c{c}alaments
de cap\'{\i}tols, seccions,
 subseccions, taules  i figures, s\'{o}n ara de
 \textcolor{NavyBlue}{color blau}.

S'han afegit nous cap\'{\i}tol i s'ha fet una reordenaci\'{o} que afecta a
diversos cap\'{\i}tols i ap\`{e}ndixs, segons es detalla a continuaci\'{o}:
\begin{dinglist}{'167}
   \item Els cap\'{\i}tols 1 i 2  de la versi\'{o} 1.4 mantenen la seva posici\'{o}.
   \item S'afegeix un nou cap\'{\i}tol 3, on es tracten les s\`{e}ries de Fourier.
   \item S'afegeix un nou cap\'{\i}tol 4, on es tracta la transformada de Laplace.
   \item El cap\'{\i}tol 3 de la versi\'{o} 1.4 es despla\c{c}a dos n\'{u}meros cap
    avall, passant a ser el cap\'{\i}tol 5.
   \item L'ap\`{e}ndix E de la versi\'{o} 1.4 es converteix en el cap\'{\i}tol 6.
   \item Els cap\'{\i}tols 4, 5, 6 i 7  de la versi\'{o} 1.4 es desplacen tres n\'{u}meros cap
    avall, passant a ser els cap\'{\i}tols 7, 8, 9 i 10 respectivament.
    \item L'ap\`{e}ndix G de la versi\'{o} 1.4 es converteix en el cap\'{\i}tol 11.
    \item Els ap\`{e}ndixs A, B, C i D de la versi\'{o} 1.4 mantenen la seva posici\'{o}.
    \item S'afegeix un nou ap\`{e}ndix E, on es tracten relacions trigonom\`{e}triques.
    \item L'ap\`{e}ndix F de la versi\'{o} 1.4 mant\'{e} la seva posici\'{o}.
    \item Els ap\`{e}ndixs H i I de la versi\'{o} 1.4 es desplacen una lletra cap
    amunt, passant a ser els ap\`{e}ndixs G i H respectivament.
\end{dinglist}


 A l'hora de referir-se a la font de corrent i a l'admit\`{a}ncia d'un circuit equivalent
 Norton, se substitueix el sub\'{\i}ndex {"<}Th{">} emprat en les versions
anteriors, pel sub\'{\i}ndex {"<}No{">}.

En l'apartat {"<}Notaci\'{o}{">} s'afegeixen els s\'{\i}mbols: $\mathbb{N}$,
$\mathbb{Z}$, $\mathbb{Z}^+$,  $\mathbb{Z}^*$, $\mathbb{Z}^-$,
$\mathbb{Q}$, $\mathbb{R}$, $\mathbb{R}^+$, $\mathbb{R}^-$ i
$\mathbb{C}$.

S'ha afegit el teorema de la superposici\'{o} en la secci\'{o}
\ref{sec:teoremes}.


S'ha afegit la bateria en la secci\'{o} \ref{sec:comp_elem}, com a un
dels components elementals d'un circuit el\`{e}ctric.

S'ha afegit la secci\'{o} \ref{sec:val_mitja_ef}, on es defineixen els
valors mitj\`{a} i efica\c{c}, i els factors d'amplitud, de forma i
d'arrissada.

S'ha afegit la secci\'{o} \ref{sec:div_tens_corr}, on es tracten els
circuits divisors de tensi\'{o} i divisors de corrent.

 S'ha modificat l'equaci\'{o} \eqref{eq:resistivitat}
i les taules \ref{taula:param-elc} i \ref{taula:AWG}.

S'ha afegit la secci\'{o} \ref{sec:conex_ti_tt}, on s'explica com
connectar correctament transformadors de corrent i de tensi\'{o}, a
aparells de mesura o de protecci\'{o}.

S'ha millorat l'explicaci\'{o} de la secci\'{o} \ref{sec:control-flux-pot}.

S'ha reestructurat la taula \ref{taula:SI-derivades}.

\section*{Versi\'{o} 2.1 (2 de gener de 2007)}
\addcontentsline{toc}{section}{Versi\'{o} 2.1}

S'adopta la compaginaci\'{o} moderna dels par\`{a}grafs en tot el llibre, consistent en separar-los per una l\'{\i}nia en blanc i en no entrar la primera l\'{\i}nia de text.

S'unifica la representaci\'{o} de les fonts de corrent: un cercle amb una fletxa a dins.

S'afegeix una nota a peu de p\`{a}gina en la secci\'{o} \ref{sec:T_N}, relacionant aquesta secci\'{o} amb la secci\'{o} \ref{sec:xarxes_Zth}.

Es millora l'explicaci\'{o} de la secci\'{o} \ref{sec:seccio_pu}, a l'hora que es trasllada de lloc (en les versions anteriors formava part del cap\'{\i}tol \ref{sec:calc_bas}).

Es millora l'explicaci\'{o} de la secci\'{o} \ref{sec:comp-sim-neutre}.

S'afegeix una nota a peu de p\`{a}gina en la secci\'{o} \ref{sec:EZS}, relacionant aquesta secci\'{o} amb el cap\'{\i}tol \ref{chap:flux_carregues}.

S'amplia la descripci\'{o} de l'equaci\'{o} \eqref{eq:awg_mm2}.

S'afegeix la secci\'{o} \ref{sec:sis_eq_no_lin}, on s'explica com resoldre sistemes d'equacions no lineals amb els programes \textit{Mathematica}${}^\circledR$ i \textit{MATLAB}${}^\circledR$.

Es millora l'explicaci\'{o} de la secci\'{o} \ref{sec:llei-s-c-t}, modificant la figura \ref{pic:llei-s-c-t} i numerant l'equaci\'{o} de la llei dels sinus.

\section*{Versi\'{o} 2.2 (10 de mar\c{c} de 2008)}
\addcontentsline{toc}{section}{Versi\'{o} 2.2}

Es canvia el color dels enlla\c{c}os interns, passant a ser de color negre com el text.

S'afegeixen les unitats que mancaven en alguns exemples.

En la secci\'{o} \ref{sec:MCM} s'introdueixen les unitats cmil i kcmil, equivalents a les unitats CM i MCM respectivament. Avui en dia \'{e}s m\'{e}s freq\"{u}ent veure escrit cmil i kcmil.

Es revisa l'ap\`{e}ndix B utilitzant les publicacions de l'any 2006 del {"<}Bureau
International des Poids et Mesures{">} (\textsf{BIPM}).\index{BIPM}

Es revisa l'ap\`{e}ndix C utilitzant les publicacions de l'any 2006 del {"<}Committee on Data for Science and Technology{">} (\textsf{CODATA}).\index{CODATA}
