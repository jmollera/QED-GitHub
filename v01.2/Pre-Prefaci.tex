\chapter*{Prefaci} \addcontentsline{toc}{chapter}{Prefaci}

   Voldria dir en primer lloc que no he intentat escriure un tractat complet
   d'electrot\`{e}cnia, d'electr\`{o}nica o de sistemes el\`{e}ctrics de pot\`{e}ncia, sin\'{o} que m\'{e}s aviat
   he volgut
   fer una recopilaci\'{o} de q\"{u}estions te\`{o}riques i pr\`{a}ctiques relacionades amb els camps mencionats
   anteriorment.

   Les fonts d'informaci\'{o} utilitzades en la realitzaci\'{o} d'aquest llibre s\'{o}n molt diverses,
   i inclouen llibres sobre les diferents mat\`{e}ries, articles de revistes o d'Internet,
   apunts de classe d'assignatures impartides a l'\textsf{ETSEIB},\index{ETSEIB} i d'altres.

   Pel qu\`{e} fa al llibre en si mateix, s'ha escrit utilitzant el sistema de composici\'{o} de
   textos \LaTeX,\index{LaTex@\LaTeX} el qual
   permet integrar molt f\`{a}cilment text, f\'{o}rmules i gr\`{a}fics, obtenint uns resultat de
   gran qualitat. S'han utilitzat diversos paquets d'ampliaci\'{o}, com ara
   l'\AmS-\LaTeX,\index{AmSLaTex@\AmS-\LaTeX}
   per tal de millorar la presentaci\'{o} de certes parts del
   llibre, com per exemple les taules, les cap\c{c}aleres, les f\'{o}rmules matem\`{a}tiques, etc.

   Aquest llibre, est\`{a} pensat per ser utilitzat tant de forma directa en la pantalla d'un
   ordinador, com en paper despr\'{e}s d'haver-lo impr\`{e}s; per tal d'estalviar paper, el text
   t\'{e} un format pensat per ser impr\`{e}s en paper DIN-A4,\index{DIN-A4} a dues cares.

    El contingut del llibre \'{e}s molt divers, i va des de temes for\c{c}a te\`{o}rics, fins a
    d'altres bastant m\'{e}s pr\`{a}ctics. He procurat donar exemples de tots els conceptes
    que s'hi expliquen, excepci\'{o} feta dels molt elementals, perqu\`{e} crec que \'{e}s important
     no quedar-se nom\'{e}s amb la teoria de  la resoluci\'{o} d'un problema determinat, sin\'{o} que
     \'{e}s molt \'{u}til veure exemples resolts pas a pas.

    Encara que he fet tots els esfor\c{c}os possibles per e{\l.l}iminar qualsevol
    mena  d'error en aquest text, \'{e}s gaireb\'{e} inevitable que n'hagi quedat algun,
    per tant, si alg\'{u} troba algun error, far\`{a} b\'{e} d'avisar-me!


   Nom\'{e}s em resta dir, que espero que els que llegeixin aquest llibre el trobin
   \'{u}til i interessant.

\vspace*{1cm}
\hfill
\begin{minipage}[b]{25mm}
\PSforPDF{
    %Created by jPicEdt 1.x
    %PsTricks format (pstricks.sty needed)
    %Sat Sep 11 18:21:26 CEST 2004
    %Retocat per JMB
    \psset{xunit=1mm,yunit=1mm,runit=1mm}
    \begin{pspicture}(0,0)(25.00,13.00)
    \newrgbcolor{userLineColour}{0.00 0.00 0.00}
    \psbezier[linewidth=1.00,linecolor=userLineColour]{c-c}(8.19,0.90)(2.61,16.06)(18.56,16.06)(13.38,1.50)
    \psbezier[linewidth=1.00,linecolor=userLineColour]{c-c}(12.31,1.50)(6.99,16.06)(22.95,16.06)(17.76,1.50)
    \psbezier[linewidth=1.00,linecolor=userLineColour]{c-c}(12.31,1.50)(12.67,0.62)(13.02,0.62)(13.38,1.50)
    \psbezier[linewidth=1.00,linecolor=userLineColour]{c-c}(16.70,1.50)(17.05,0.62)(17.41,0.62)(17.76,1.50)
    \psbezier[linewidth=1.00,linecolor=userLineColour]{c-c}(16.70,1.50)(11.38,16.06)(27.34,16.06)(21.70,1.06)
    \psline[linewidth=1.00,linecolor=userLineColour,linearc=0.1]{c-c}(5.50,12.25)(5.50,0.80)(1.45,5.93)(24.50,5.93)
    \end{pspicture}
}
\end{minipage}

{ \fontfamily{pzc}\selectfont\LARGE

\hfill Josep Mollera Barriga

\hfill Badalona, 16 d'abril de 2005 }

\hfill \ifpdf
    ({\Large\textcolor[rgb]{0.00,0.00,1.00}{\Letter}}\;\href{mailto:jmollerab@ati.es}{jmollerab@ati.es})
\else
    (jmollera@anacnv.com)
\fi
