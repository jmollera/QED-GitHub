\chapter*{Historial} \addcontentsline{toc}{chapter}{Historial}

Es presenta a continuaci\'{o} l'evoluci\'{o} que ha tingut aquest llibre, en
les successives versions que han aparegut.

\section*{Versi\'{o} 1.0 (8 de gener de 2005)}

Despr\'{e}s de molts esfor\c{c}os, surt a la llum la primera versi\'{o} d'aquest
llibre, format pels cap\'{\i}tols 1, 2, 3, 4, 5, 6 i 7, i els ap\`{e}ndixs A,
B, C, D i E.

\section*{Versi\'{o} 1.1 (8 de febrer de 2005)}

S'afegeix al llibre aquest apartat {"<}Historial{">}.

En l'apartat {"<}Notaci\'{o}{">}, s'especifica que el m\`{o}dul d'un n\'{u}mero
complex \'{e}s igual a l'arrel quadrada \emph{positiva} de la suma dels
quadrats de les seves parts real i imagin\`{a}ria.

Es modifiquen les equacions \eqref{eq:p_3f_34} i \eqref{eq:q_3f_34}.

S'amplia la secci\'{o} corresponent a les difer\`{e}ncies entre les
normatives \textsf{CEI} i \textsf{ANSI}, que fan refer\`{e}ncia als
transformadors de mesura i protecci\'{o} (Secci\'{o}
\ref{sec:comp_tt_ti_cei_ansi}).

Es revisa tot el text, fent algunes petites modificacions i
correccions.

\section*{Versi\'{o} 1.2 (16 d'abril de 2005)}

En l'apartat {"<}Notaci\'{o}{">}, s'afegeix l'explicaci\'{o} de la convenci\'{o}
seguida a l'hora de dibuixar les fletxes que representen les
tensions i els corrents.

S'afegeix l'ap\`{e}ndix F, on s'explica la designaci\'{o} de les classes de
refrigeraci\'{o} en els transformadors de pot\`{e}ncia.
