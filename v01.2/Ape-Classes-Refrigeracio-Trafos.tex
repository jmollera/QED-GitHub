\chapter{Designaci\'{o} de les Classes de Refrigeraci\'{o} en els Transformadors de Pot\`{e}ncia}
\index{CEI!60076-2} \index{ANSI!C57.12} \index{refrigeraci\'{o}!en
transformadors de pot\`{e}ncia} \index{convecci\'{o} natural}

Les classes de refrigeraci\'{o} utilitzades en els transformadors de
pot\`{e}ncia, es designen mitjan\c{c}ant quatre lletres.

Actualment, la definici\'{o} i l'\'{u}s d'aquestes lletres, \'{e}s coincident
entre la norma europea (\textsf{CEI 60076-2}) i la norma americana
(\textsf{ANSI C57.12}).

Es defineix a continuaci\'{o} el significat d'aquestes lletres:

\textbf{1a lletra}. Indica l'element refrigerant intern, que est\`{a} en
contacte amb els debanats del transformador. Els valors possibles
s\'{o}n els seg\"{u}ents:
\begin{list}{}
   {\setlength{\labelwidth}{4.5mm} \setlength{\leftmargin}{4.5mm} \setlength{\labelsep}{2mm}}
   \item[\textbf{O}] L'element refrigerant \'{e}s un oli mineral o un l\'{\i}quid sint\`{e}tic a\"{\i}llant, amb una temperatura d'ignici\'{o}
   inferior o igual a 300\unit{\celsius}.
   \item[\textbf{K}] L'element refrigerant \'{e}s un l\'{\i}quid sint\`{e}tic a\"{\i}llant, amb una temperatura d'ignici\'{o}
   superior a 300\unit{\celsius}.
   \item[\textbf{L}] L'element refrigerant \'{e}s un l\'{\i}quid sint\`{e}tic a\"{\i}llant, amb una temperatura d'ignici\'{o}
   no mesurable.
\end{list}

\textbf{2a lletra}. Indica el mecanisme de circulaci\'{o} de l'element
refrigerant intern. Els valors possibles s\'{o}n els seg\"{u}ents:
\begin{list}{}
   {\setlength{\labelwidth}{4.5mm} \setlength{\leftmargin}{4.5mm} \setlength{\labelsep}{2mm}}
   \item[\textbf{N}] Circulaci\'{o} mitjan\c{c}ant convecci\'{o} natural,
    a trav\'{e}s de l'equip refrigerant i pels debanats del transformador.
   \item[\textbf{F}] Circulaci\'{o} for\c{c}ada a trav\'{e}s de l'equip refrigerant (mitjan\c{c}ant bombes),
    i circulaci\'{o} mitjan\c{c}ant convecci\'{o} natural pels debanats del
    transformador. Aquest tipus de circulaci\'{o} tamb\'{e} s'anomena {"<}de flux no
    dirigit{">}.
   \item[\textbf{D}] Circulaci\'{o} for\c{c}ada a trav\'{e}s de l'equip refrigerant (mitjan\c{c}ant bombes),
    i dirigida per aquest equip refrigerant cap als debanats del
    transformador i, de manera opcional, tamb\'{e} cap a altres parts del transformador. Aquest
    tipus de circulaci\'{o} tamb\'{e} s'anomena {"<}de flux dirigit{">}.
\end{list}
\pagebreak

\textbf{3a lletra}. Indica l'element refrigerant extern. Els valors
possibles s\'{o}n els seg\"{u}ents:
\begin{list}{}
   {\setlength{\labelwidth}{4.5mm} \setlength{\leftmargin}{4.5mm} \setlength{\labelsep}{2mm}}
   \item[\textbf{A}] L'element refrigerant \'{e}s l'aire.
   \item[\textbf{W}] L'element refrigerant \'{e}s l'aigua.
\end{list}

\textbf{4a lletra}. Indica el mecanisme de circulaci\'{o} de l'element
refrigerant extern. Els valors possibles s\'{o}n els seg\"{u}ents:
\begin{list}{}
   {\setlength{\labelwidth}{4.5mm} \setlength{\leftmargin}{4.5mm} \setlength{\labelsep}{2mm}}
   \item[\textbf{N}] Circulaci\'{o} mitjan\c{c}ant convecci\'{o} natural.
   \item[\textbf{F}] Circulaci\'{o} for\c{c}ada, mitjan\c{c}ant ventiladors (en el cas de
   l'aire) o bombes (en el cas de l'aigua).
\end{list}

En la Taula \vref{taula:classes-refrigeracio-trafos} es presenta una
comparativa, entres diverses designacions antigues de classes de
refrigeraci\'{o} (segons les normes americanes) i les designacions
equivalents actuals:
\begin{table}[htb]
   \caption{\label{taula:classes-refrigeracio-trafos}
   Classes de refrigeraci\'{o} en els transformadors de pot\`{e}ncia}
   \begin{center}\begin{tabular}{cc}
   \toprule[1pt]
   Designaci\'{o} antiga & Designaci\'{o} actual \\
   (normes \textsf{\textsf{ANSI}})     & (normes \textsf{\textsf{CEI}} i
   \textsf{\textsf{ANSI}}) \\
   \midrule
   OA & ONAN   \\
   FA & ONAF   \\
   FOA & OFAF  \\
   FOW & OFWF  \\
   FOA & ODAF  \\
   FOW & ODWF \\
   \bottomrule[1pt]
   \end{tabular} \end{center}
\end{table}

En el cas d'un transformador on puguem seleccionar que la circulaci\'{o}
sigui natural o for\c{c}ada (amb la pot\`{e}ncia corresponent en cada cas),
les designacions s\'{o}n del tipus: ONAN/ONAF, ONAN/OFAF, etc.

En el cas dels transformadors secs, l'element refrigerant sempre \'{e}s
l'aire, ja sigui en circulaci\'{o} natural o for\c{c}ada, i per tant les
designacions s\'{o}n simplement AN o AF.
