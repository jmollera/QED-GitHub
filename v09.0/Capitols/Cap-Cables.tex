\chapter{Cables}\index{cables}

\section{Introducció}
Es tracten en aquest capítol qüestions relatives als cables elèctrics.

\section{Resistència}\index{cables!resistència}

\subsection{Resistència d'un conductor}

La resistència $R$ d'un conductor depèn de la resistivitat $\rho$
del material, de la llargada $l$ del conductor i de la seva secció
$S$.
\begin{equation}
   R= \rho \frac{l}{S}
\end{equation}
\index{$\rho$}

\index{resistivitat!variació amb la temperatura}La resistivitat no
és un valor constant sinó que depèn de la temperatura, a major
temperatura major resistivitat. Coneixent la resistivitat $\rho_1$ a una
temperatura $T_1$ es pot calcular la resistivitat $\rho_2$ a una altra
temperatura $T_2$, a partir del coeficient de variació de la
resistivitat amb la temperatura $\alpha_1$ donat a la temperatura $T_1$.
\begin{equation}
   \rho_2 = \rho_1 [1 + \alpha_1 (T_2 - T_1)]\label{eq:resistivitat}
\end{equation}
\index{$\alpha$}

\index{resistivitat!valors}En la Taula
\vref{taula:param-elc} es donen valors de resistivitat i de
coeficients de variació de la resistivitat amb la temperatura a
\SI{20}{\celsius} i a \SI{0}{\celsius}, de diversos materials.

\begin{center}
   \captionof{table}{Paràmetres elèctrics d'alguns materials}
   \label{taula:param-elc}
   \begin{tabular}{lccc}
   \toprule[1pt]
   Material & $\rho_{\SI{20}{\celsius}} / (\si{\ohm.mm^2/m})$ & $\alpha_{\SI{20}{\celsius}} / \si{\celsius^{-1}}$ &
   $\alpha_{\SI{0}{\celsius}} / \si{\celsius^{-1}}$
   \\
   \midrule
      Alumini & \num{0,02825} & \num{0,00391} & \num{0,00424} \\
      Coure   & \num{0,01723} & \num{0,00393} & \num{0,00427} \\
      Plata   & \num{0,01645} & \num{0,00380} & \num{0,00412} \\
   \bottomrule[1pt]
   \end{tabular}
\end{center}

La resistència així calculada és vàlida quan el corrent que circula
pel cable és corrent continu.

Quan el corrent que circula pel cable és
corrent altern cal tenir en compte l'efecte peŀlicular, el qual
li provoca un augment de la resistència causat perquè el corrent
tendeix a circular més per la zona perifèrica del conductor que no pas per
la zona central; l'efecte és important per a valors elevats de la
secció del conductor o de la freqüència del corrent.\index{efecte peŀlicular}

\index{resistència!efectiva}La resistència efectiva es troba a
partir de la resistència calculada anteriorment per a corrent
continu, i d'un factor k que té en compte l'efecte peŀlicular.
\begin{equation}
   R\ped{efectiva} = \text{k} R
\end{equation}

En la Taula \vref{taula:const_r_ef} es donen valors\footnote{Valors obtinguts del llibre «Teoría de Circuitos. Fundamentos, 3ª edición», Enrique Ras, Marcombo Boixareu Editores (pàg. 114).} de k per a conductors de coure i d'alumini, per a diversos valors del producte de la secció del conductor per la freqüència del corrent.

\begin{center}
   \captionof{table}{Valors de k pel càlcul de la resistència efectiva}
   \label{taula:const_r_ef}
   \begin{tabular}{S[table-format=5.0]S[table-format=1.3]S[table-format=1.3]}
   \toprule[1pt]
   {Secció$\,\cdot\,$Freqüència} & \multicolumn{2}{c}{k} \\
   \cmidrule(rl){1-1} \cmidrule(rl){2-3}
    \si{mm^2.Hz} & \multicolumn{1}{c}{Cu} & \multicolumn{1}{c}{Al} \\
   \midrule
  5000 &  1,000 & 1,000 \\
  10000 & 1,008 & 1,000 \\
  15000 & 1,025 & 1,006 \\
  20000 & 1,045 & 1,015 \\
  25000 & 1,070 & 1,026 \\
  30000 & 1,096 & 1,040 \\
  35000 & 1,126 & 1,053 \\
  40000 & 1,158 & 1,069 \\
  45000 & 1,195 & 1,085 \\
  50000 & 1,230 & 1,104 \\
  75000 & 1,433 & 1,206 \\
  100000 & 1,622 & 1,330 \\
   \bottomrule[1pt]
  \end{tabular}
\end{center}

\subsection{Resistència d'un cable}

La resistència d'un cable $R\ped{Cable}$ depèn del nombre de conductors per fase $n$ (o
per pol, en corrent continu), de la resistència de cada conductor $R\ped{Conductor}$ i del
tipus de tensió elèctrica a la qual estigui sotmès el cable (monofàsica, trifàsica,
contínua, etc.).

\subsubsection*{Corrent continu o altern monofàsic}
\begin{equation}\label{eq:r_cc_mono}
    R\ped{Cable} = 2\, \frac{R\ped{Conductor}}{n}
\end{equation}

El valor multiplicatiu 2, prové del fet que cal tenir en compte tant el conductor d'anada
com el de tornada.

\subsubsection*{Corrent altern trifàsic equilibrat}
\vspace{-5mm}
\begin{equation}\label{eq:r_trifas}
R\ped{Cable} = \frac{R\ped{Conductor\;fase}}{n}
\end{equation}

Atès que no circula corrent pel neutre, la seva resistència no  té cap influència.

\subsubsection*{Corrent altern trifàsic desequilibrat}
\begin{equation}
    R\ped{Cable\;fase} = \frac{R\ped{Conductor\;fase}}{n\ped{fase}} \qquad\qquad
    R\ped{Cable\; neutre} = \frac{R\ped{Conductor\;neutre}}{n\ped{neutre}}
\end{equation}

En aquest cas cal tenir en compte que els corrents que circulen per les fase i pel neutre
són diferents.


\section{Caiguda de tensió}\index{cables!caiguda de tensió}

La caiguda de tensió $\Delta U$ en un cable es defineix com la diferència entre els mòduls de les tensions a l'origen $|{\cmplx{U}\ped{O}}|$ i al final $|\cmplx{U}\ped{F}|$ del cable.
\begin{equation}
   \Delta U \equiv |\cmplx{U}\ped{O}| - |\cmplx{U}\ped{F}|
\end{equation}

\subsection{Caiguda de tensió en corrent continu}\index{cables!caiguda de tensió!en corrent continu}

En corrent continu la caiguda de tensió depèn del corrent $I$ que circula pel cable i de la  resistència del propi cable, calculada segons l'equació \eqref{eq:r_cc_mono}.
\begin{equation}
   \Delta U = I R\ped{Cable}
\end{equation}

\subsection{Caiguda de tensió en corrent altern}\index{cables!caiguda de tensió!en corrent altern}

\index{factor!de potència}En corrent altern la caiguda de tensió
depèn del  corrent $\cmplx{I}$ que circula pel cable, de la
resistència i la reactància del propi cable i del factor de
potència $\cos \varphi$. El diagrama fasorial d'aquestes magnituds
es pot veure en la Figura \vref{pic:cdt_ca}.

\begin{center}
   \input{Imatges/Cap-Cables-Caiguda-Tensio.pdf_tex}
   \captionof{figure}{Caiguda de tensió en corrent altern}
   \label{pic:cdt_ca}
\end{center}

Quan es tracta de corrent monofàsic, la resistència del cable es calcula segons l'equació
\eqref{eq:r_cc_mono}; la reactància del cable $X\ped{Cable}$ es calcula de forma anàloga
amb la mateixa equació, a partir de la reactància dels conductors $X\ped{Conductor}$.

Pel que fa al corrent trifàsic, se suposa equilibrat, i per tant s'utilitza l'equació
\eqref{eq:r_trifas} per calcular la resistència del cable $R\ped{Cable}$ (i de forma
anàloga la reactància $X\ped{Cable}$). Addicionalment, els corrents fan referència als
corrents de fase i les tensions a les tensions fase--neutre; l'angle $\varphi$ és per
tant l'angle entre la tensió final fase--neutre i el corrent de fase.

Disposem en aquest cas de dues equacions, una d'exacta i una altre d'aproximada (per a valors elevats de $\cos \varphi$).

\begin{subequations}
\begin{align}
   \Delta U &= |\cmplx{I}| \,( R\ped{Cable} \cos \varphi + X\ped{Cable} \sin \varphi) + |\cmplx{U}\ped{O}| - \sqrt{|\cmplx{U}\ped{O}|^2 - |\cmplx{I}|^2 ( X\ped{Cable} \cos \varphi - R\ped{Cable} \sin \varphi )^2} \label{eq:cdt_trif_exact} \\
   \Delta U &\approx |\cmplx{I}| \,( R\ped{Cable} \cos \varphi + X\ped{Cable} \sin \varphi) \qquad \text{si} \; \cos \varphi \gtrapprox \num{0,8} \label{eq:cdt_trif_aprox}
\end{align}
\end{subequations}

\begin{exemple}[Càlcul de la caiguda de tensió en un sistema trifàsic]
       Es tracta de calcular la caiguda de tensió en un sistema trifàsic on $|\cmplx{U}\ped{O}| = \SI{380}{V}$ (fase--fase), $|\cmplx{I}|=\SI{630}{A}$ i $\cos \varphi = \num{0,87}$(i). La unió entre els extrems origen  i final està formada per tres cables unipolars en paraŀlel de $\SI{240}{mm^2}$ de secció cadascun i $\SI{400}{m}$ de llargada; els valors per fase de resistència i inductància són $\SI{0,095}{\ohm/km}$ i $\SI{0,102}{\ohm/km}$ respectivament.

    A partir de l'equació \eqref{eq:r_trifas} calculem els valors de $R\ped{Cable}$ i de $X\ped{Cable}$:

    \[
       R\ped{Cable} = \frac{\SI{0,095}{\ohm/km}\times \SI{0,4}{km}}{3} = \SI{0,0127}{\ohm}
    \]
    \[
       X\ped{Cable} = \frac{\SI{0,102}{\ohm/km}\times \SI{0,4}{km}}{3} = \SI{0,0136}{\ohm}
    \]

    Obtenim a continuació el valor de $\sin \varphi$:

    \[
       \sin \varphi = \sqrt{1-\num{0,87}^2} = \num{0,49}
    \]

    Calculem en primer lloc la caiguda de tensió de forma aproximada utilitzant l'equació \eqref{eq:cdt_trif_aprox}:

    \[
       \Delta U \approx \SI{630}{A} \times ( \SI{0,0127}{\ohm} \times \num{0,87} + \SI{0,0136}{\ohm} \times \num{0,49} ) = \SI{11,16}{V}
    \]

    que en tant per cent respecte la tensió a l'origen representa:

    \[
        \Delta u = \frac{\SI{11,16}{V}}{380/\sqrt{3}\unit{V}} \times 100 = \SI{5,09}{\%}
    \]

    Finalment, calculem la caiguda de tensió exacta utilitzant l'equació \eqref{eq:cdt_trif_exact}:

    \[ \begin{split}
       \Delta U &=  \SI{630}{A} \times( \SI{0,0127}{\ohm} \times \num{0,87} + \SI{0,0136}{\ohm} \times \num{0,49}) + \frac{380}{\sqrt{3}}\unit{V} \,- \\
        & \quad - \sqrt{\left(380/\sqrt{3}\unit{V}\right)^2 - \left(\SI{630}{A}\right)^2 \times  \left( \SI{0,0136}{\ohm} \times \num{0,87} - \SI{0,0127}{\ohm} \times \num{0,49} \right)^2 } \,= \SI{11,19}{V}
    \end{split} \]

    que en tant per cent respecte la tensió a l'origen representa:

    \[
        \Delta u = \frac{\SI{11,19}{V}}{380/\sqrt{3}\unit{V}} \times 100 = \SI{5,10}{\%}
    \]
\end{exemple}

\section{Capacitat tèrmica en curtcircuit}\index{cables!capacitat tèrmica en curtcircuit}\label{ces:cables_Icc_termica}

Quan hi ha un curtcircuit en un cable, tot el calor generat no es transmet a l'exterior en els instants inicials sinó que s'acumula en la massa del conductor, incrementant la seva temperatura (procés adiabàtic). En aquestes condicions, la norma CEI 60724 dóna la següent equació per a cables de tensió assignada d'\SI{1}{kV} i \SI{3}{kV}:\index{CEI!60724}

\begin{equation}
   I\ped{cc}^2 \,t\ped{cc} = K^2 S^2 \ln \frac{\beta + \theta\ped{f}}{\beta + \theta\ped{i}}
\end{equation}
Amb:

\begin{list}{}
   {\setlength{\labelwidth}{10mm} \setlength{\leftmargin}{12mm} \setlength{\labelsep}{2mm}}
   \item[\hspace{5mm}$\boldsymbol{I\ped{cc}}$\hfill] Corrent de curtcircuit
   \item[\hspace{5mm}$\boldsymbol{t\ped{cc}}$\hfill] Temps màxim que pot durar el curtcircuit sense que es malmeti el cable
   \item[\hspace{5mm}$\boldsymbol{K}$\hfill] Paràmetre que depèn del material del conductor
   \item[\hspace{5mm}$\boldsymbol{S}$\hfill] Secció del cable
   \item[\hspace{5mm}$\boldsymbol{\beta}$\hfill] Invers del coeficient de variació de la resistivitat amb la temperatura ($\betaup = 1/ \alphaup$)
   \item[\hspace{5mm}$\boldsymbol{\theta\ped{i}}$\hfill] Temperatura inicial del curtcircuit (depèn del material de l'aïllament del conductor)
   \item[\hspace{5mm}$\boldsymbol{\theta\ped{f}}$\hfill] Temperatura final del curtcircuit (depèn del material de l'aïllament del conductor)
\end{list}

La norma CEI 60724 dóna valors per a $K$, $\beta$, $\theta\ped{i}$ i $\theta\ped{f}$, per a diferents materials de conductor i d'aïllament, arribant finalment a la fórmula:


\begin{equation}\label{eq:Icc_termica}
   I\ped{cc} = S\, \frac{\text{C}}{\sqrt{t\ped{cc}}} \quad
   \begin{cases}
   I\ped{cc} &:\; \text{expressat en\unit{A}} \\
   S         &:\; \text{expressat en\unit{mm^2}} \\
   t\ped{cc} &:\; \text{expressat en\unit{s}} \\
   \text{C}  &:\; \text{paràmetre que depèn del tipus de cable}
   \end{cases}
\end{equation}

En la Taula \vref{taula:const_termica} es donen valors de C per a diferents materials del conductor i de l'aïllament, segons la norma CEI 60724.

\begin{center}
   \captionof{table}{Valors de C pel càlcul de curtcircuits en cables} \label{taula:const_termica}
   \begin{tabular}{c>{\hspace{2.5em}}cc}
   \toprule[1pt]
   \renewcommand*{\multirowsetup}{\centering}
   \multirow{2}{25mm}{\rule{0mm}{4mm}Material del\\conductor} & \multicolumn{2}{c}{C, segons el material de l'aïllament} \\ \cmidrule(rl){2-3}
    & PVC & EPR i XLPE \\
   \midrule
   Cu & 115 & 143 \\
   Al & 76 & 94 \\
   \bottomrule[1pt]
   \end{tabular}
\end{center}


\begin{exemple}[Càlcul de la capacitat tèrmica d'un cable]
       Es tracta de calcular el temps màxim durant el qual un cable de coure de $\SI{50}{mm^2}$ amb aïllament d'EPR, pot suportar un corrent de curtcircuit de $\SI{15}{kA}$.

    A partir de l'equació \eqref{eq:Icc_termica} calculem el temps màxim demanat:
    \[
       t\ped{cc} = \left( \frac{S\,C}{I\ped{cc}} \right) ^2 = \left( \frac{\SI{50}{mm^2}\times 143}{\SI{15000}{A}} \right) ^2 = \SI{0,23}{s}
    \]
\end{exemple}

\section{Conversió entre unitats americanes i unitats SI}

\subsection{«Mils» (mil), «circular mils» (cmil o CM) i «thousand circular mils» (kcmil o MCM)}\label{sec:MCM}
\index{mil}\index{cmil}\index{CM}\index{kcmil}\index{MCM}
\index{mils@\guillemotleft mils\guillemotright} \index{circular mils@\guillemotleft circular mils\guillemotright} \index{thousand circular mils@\guillemotleft thousand circular mils\guillemotright}

\index{mils@\guillemotleft mils\guillemotright!difinició} \index{circular mils@\guillemotleft circular mils\guillemotright!difinició} \index{thousand circular mils@\guillemotleft thousand circular mils\guillemotright!difinició}Les definicions d'aquestes tres unitats utilitzades en la mesura de diàmetres i seccions de conductors són:
\begin{align}
  \SI{1}{mil} &\equiv \text{Una miŀlèsima de polsada} \\
  \SI{1}{cmil} = \SI{1}{CM} &\equiv  \text{Àrea d'un cercle de diàmetre igual a \SI{1}{mil}} \\
  \SI{1}{kcmil} = \SI{1}{MCM} &\equiv \SI{1000}{cmil} = \SI{1000}{CM}
\end{align}

\index{mils@\guillemotleft mils\guillemotright!equivalències} \index{circular mils@\guillemotleft circular mils\guillemotright!equivalències} \index{thousand circular mils@\guillemotleft thousand circular mils\guillemotright!equivalències}Es donen a continuació algunes conversions d'aquestes unitats:
\begin{align}
   \SI{1}{mil} &= \SI{e-3}{in}  \\
  \SI{1}{mil} &= \SI{e-3}{in} \times \frac{\SI{25,4}{mm}}{\SI{1}{in}} = \SI{25,4e-3}{mm}  \\
  \SI{1}{cmil} &= \frac{\piup}{4}\unit{mil^2} = \SI{0,785398}{mil^2}   \\
   \SI{1}{cmil} &= \frac{\piup}{4}\times \SI{e-6}{in^2} = \SI{0,785398e-6}{in^2} \\
   \SI{1}{cmil} &= \frac{\piup}{4} \times \SI{e-6}{in^2} \times \frac{\SI{645,16}{mm^2}}{\SI{1}{in^2}} = \SI{506,7075e-6}{mm^2}
   \\[1ex]
   \SI{1}{kcmil} &= \SI{785,398}{mil^2}  = \SI{0,785398e-3}{in^2} = \SI{0,5067075}{mm^2}
\end{align}

Una relació útil entre diàmetres  i seccions és la següent: la secció $S$ d'un cercle expressada en «circular mils» és igual al quadrat del diàmetre $d$ del cercle expressat en «mils».
\begin{equation}
   S = d^2 \quad
   \begin{cases}
   S &:\; \text{expressat en\unit{cmil}} \\
   d &:\; \text{expressat en\unit{mil}}
   \end{cases}
\end{equation}


En la Taula \vref{taula:MCM} es relacionen els diàmetres i seccions en diverses unitats, dels conductors usualment disponibles compresos entre $\SI{2000}{kcmil}$ i $\SI{250}{kcmil}$.

\begin{center}
    \captionof{table}{Dimensions de cables definits en kcmil}
    \label{taula:MCM}
    \begin{tabular}{S[table-format=4.0]S[table-format=1.6]S[table-format=4.4]
                    S[table-format=4.5]S[table-format=1.7]S[table-format=2.5]}
    \toprule[1pt]
    \multicolumn{3}{c}{Secció} &   \multicolumn{3}{c}{Diàmetre}         \\
    \cmidrule(rl){1-3} \cmidrule(rl){4-6}
    \multicolumn{1}{c}{kcmil}  &    \multicolumn{1}{c}{\si{in^2}}  & \multicolumn{1}{c}{\si{mm^2}}  & \multicolumn{1}{c}{mil}
           &    \multicolumn{1}{c}{in} &   \multicolumn{1}{c}{mm}   \\
    \midrule
    2000 &   1,570796 &   1013,4150 & 1414,21356 &  1,4142136 &   35,92102 \\
    1750 &   1,374447 &   886,7381  & 1322,87566 &  1,3228757 &   33,60104 \\
    1600 &   1,256637 &   810,7320  & 1264,91106 &  1,2649111 &   32,12874 \\
    1500 &   1,178097 &   760,0612  & 1224,74487 &  1,2247449 &   31,10852 \\
    1250 &   0,981748 &   633,3843  & 1118,03399 &  1,1180340 &   28,39806 \\
    1000 &   0,785398 &   506,7075  & 1000,00000 &  1,0000000 &   25,40000 \\
     800 &   0,628319 &   405,3660  &  894,42719 &  0,8944272 &   22,71845 \\
     750 &   0,589049 &   380,0306  &  866,02540 &  0,8660254 &   21,99705 \\
     700 &   0,549779 &   354,6952  &  836,66003 &  0,8366600 &   21,25116 \\
     600 &   0,471239 &   304,0245  &  774,59667 &  0,7745967 &   19,67476 \\
     500 &   0,392699 &   253,3537  &  707,10678 &  0,7071068 &   17,96051 \\
     450 &   0,353429 &   228,0184  &  670,82039 &  0,6708204 &   17,03884 \\
     400 &   0,314159 &   202,6830  &  632,45553 &  0,6324555 &   16,06437 \\
     350 &   0,274889 &   177,3476  &  591,60798 &  0,5916080 &   15,02684 \\
     300 &   0,235619 &   152,0122  &  547,72256 &  0,5477226 &   13,91215 \\
     250 &   0,196350 &   126,6769  &  500,00000 &  0,5000000 &   12,70000 \\
    \bottomrule[1pt]
    \end{tabular}
\end{center}

D'aquesta taula es pot extreure la següent relació aproximada:
\begin{equation}
  \text{Secció en}\unit{mm^2} \approx \dfrac{\text{Secció en}\unit{kcmil}}{2}
\end{equation}



\subsection{«American Wire Gauge» (AWG)}\label{sec:awg}
\index{AWG (\guillemotleft American Wire Gauge\guillemotright)}

\index{AWG (\guillemotleft American Wire Gauge\guillemotright)!definició}L'«American Wire Gauge», anomenat també «Brown \& Sharp Gauge», és un sistema de numeració de conductors circulrs segons el seu diàmetre. A cada número AWG li correspon un valor de diàmetre; els successius diàmetres formen una progressió geomètrica descendent (en augmentar el número AWG, disminueix el diàmetre).

La raó d'aquesta progressió geomètrica s'obté de la següent consideració: hi ha dos valors de referència: 36 AWG, el qual té assignat un diàmetre de \SI{5}{mil}, i 0000 AWG, el qual té assignat un diàmetre de \SI{460}{mil}. Entre aquest dos valors de referència hi ha una diferencia de 39 unitats (vegeu la Taula \vref{taula:AWG}), i per tant, sent $r\ped{d}$ la raó de diàmetres buscada, tenim:

\begin{equation}
   \SI{460}{mil} \times r\ped{d}^{39} = \SI{5}{mil} \quad \rightarrow \quad r\ped{d} = \left( \frac{\SI{5}{mil}}{\SI{460}{mil}} \right)^{1/39} = \left( \frac{1}{92} \right)^{1/39} = 92^{-1/39}
\end{equation}

En ser la secció d'un conductor proporcional al quadrat del seu diàmetre, les seccions dels successius números AWG formen una progressió geomètrica  descendent de raó $r\ped{S}$ igual a: \begin{equation}
   r\ped{S} = r\ped{d}^2 = 92^{-2/39}
\end{equation}

Finalment, en ser la resistència d'un conductor inversament proporcional a la seva secció, les resistències dels successius números AWG formen una progressió geomètrica ascendent de raó $r\ped{R}$ igual a:
\begin{equation}
   r\ped{R} = \frac{1}{r\ped{S}} = 92^{2/39}
\end{equation}

A partir d'aquestes raons, i coneixent el diàmetre $d$, la secció $S$ i la resistència $R$ d'un número AWG $n$, podem calcular aquests paràmetres per a un altre número AWG, $k$ unitats posterior o $k$ unitats anterior:

\begin{equation}
   \begin{array}{rllllll}
     \text{AWG:}         & & n & & n+k                & & n-k \\
     \text{Diàmetre:}    & & d & & d\times 92^{-k/39}  & & d\times 92^{k/39} \\
     \text{Secció:}      & & S & & S\times 92^{-2k/39} & & S\times 92^{2k/39} \\
     \text{Resistència:} & & R & & R\times 92^{2k/39}  & & R\times 92^{-2k/39}
   \end{array}
\end{equation}

Per a alguns valors particulars de $k$ es compleixen de forma aproximada les següents relacions:

\begin{list}{}
   {\setlength{\labelwidth}{15mm} \setlength{\leftmargin}{17mm} \setlength{\labelsep}{2mm}}

   \item[$\boldsymbol{k=6}$\hfill] En augmentar en 6  unitats el número AWG, el diàmetre es divideix per 2
                 $(92^{-6/39}\approx \num{0,5})$.

   \item[$\boldsymbol{k=-6}$\hfill] En disminuir en 6 unitats el número AWG, el diàmetre es multiplica per 2
                 $(92^{6/39}\approx 2)$.

   \item[$\boldsymbol{k=20}$\hfill] En augmentar en 20  unitats el número AWG, el diàmetre es divideix per 10
                 $(92^{-20/39}\approx \num{0,1})$.

   \item[$\boldsymbol{k=-20}$\hfill] En disminuir en 20 unitats el número AWG, el diàmetre es multiplica per 10
                 $(92^{20/39}\approx 10)$.

   \item[$\boldsymbol{k=3}$\hfill] En augmentar en 3 unitats el número AWG, la secció es divideix per 2
                 $(92^{-2\times 3/39}\approx \num{0,5})$ i la resistència es multiplica per 2
                 $(92^{2\times 3/39}\approx 2)$.

   \item[$\boldsymbol{k=-3}$\hfill] En disminuir en 3 unitats el número AWG, la secció es multiplica per 2
                  $(92^{2\times 3/39}\approx 2)$ i la resistència es divideix per 2
                  $(92^{-2\times 3/39}\approx \num{0,5})$.

   \item[$\boldsymbol{k=10}$\hfill] En augmentar en 10 unitats el número AWG, la secció es divideix per 10
                 $(92^{-2\times 10/39}\approx \num{0,1})$ i la resistència es multiplica per 10
                 $(92^{2\times 10/39}\approx 10)$.

   \item[$\boldsymbol{k=-10}$\hfill] En disminuir en 10 unitats el número AWG, la secció es multiplica per 10
                  $(92^{2\times 10/39}\approx 10)$ i la resistència es divideix per 10
                  $(92^{-2\times 10/39}\approx \num{0,1})$.
\end{list}

\index{AWG (\guillemotleft American Wire Gauge\guillemotright)!equivalències}En la Taula \vref{taula:AWG} es relacionen els diàmetres i seccions en diverses unitats dels conductors compresos entre 0000 AWG i 40 AWG.

\begin{longtable}{S[table-format=4.0]S[table-format=3.3]
                  S[table-format=2.6]S[table-format=2.4]
                  S[table-format=6.3]S[table-format=1.3e1]
                  S[table-format=3.6]}
\caption{\label{taula:AWG}Dimensions de cables AWG} \\
\toprule[1pt]
    \multicolumn{1}{c}{Cable}  &    \multicolumn{3}{c}{Diàmetre} &   \multicolumn{3}{c}{Secció}         \\
    \cmidrule(rl){1-1} \cmidrule(rl){2-4} \cmidrule(rl){5-7}
      \multicolumn{1}{c}{AWG} &   \multicolumn{1}{c}{mil}  & \multicolumn{1}{c}{in}  & \multicolumn{1}{c}{mm}  &   \multicolumn{1}{c}{cmil} & \multicolumn{1}{c}{\si{in^2}}  & \multicolumn{1}{c}{\si{mm^2}} \\
\midrule \endfirsthead
\caption[]{Dimensions de cables AWG (\emph{ve de la pàgina anterior})} \\
\toprule[1pt]
    \multicolumn{1}{c}{Cable} & \multicolumn{3}{c}{Diàmetre} & \multicolumn{3}{c}{Secció} \\
    \cmidrule(rl){1-1} \cmidrule(rl){2-4} \cmidrule(rl){5-7}
    \multicolumn{1}{c}{AWG} &   \multicolumn{1}{c}{mil}  & \multicolumn{1}{c}{in}  & \multicolumn{1}{c}{mm}  &   \multicolumn{1}{c}{cmil} & \multicolumn{1}{c}{\si{in^2}}  & \multicolumn{1}{c}{\si{mm^2}} \\
\midrule \endhead
\midrule
\multicolumn{7}{r}{\sffamily\bfseries\color{NavyBlue}(\emph{continua a la pàgina següent})}
\endfoot
\endlastfoot
{0000} & 460,000 &   0,460000 &    11,6840 & 211600,000 &  1,662e-1 & 107,219303 \\
{\phantom{0}000} & 409,642 &   0,409642 &    10,4049 & 167806,429 &  1,318e-1 &  85,028773 \\
{\phantom{00}00}  &  364,797 &   0,364797 &     9,2658 & 133076,548 &  1,045e-1 &  67,430882 \\
  0  &  324,861 &   0,324861 &     8,2515 & 105534,501 &  8,289e-2 &  53,475121 \\
 1 &    289,297 &   0,289297 &     7,3481 &  83692,664 &  6,573e-2 &  42,407699 \\
 2 &    257,626 &   0,257626 &     6,5437 &  66371,300 &  5,213e-2 &  33,630834 \\
 3 &    229,423 &   0,229423 &     5,8273 &  52634,834 &  4,134e-2 &  26,670464 \\
 4 &    204,307 &   0,204307 &     5,1894 &  41741,321 &  3,278e-2 &  21,150639 \\
 5 &    181,941 &   0,181941 &     4,6213 &  33102,372 &  2,600e-2 &  16,773220 \\
 6 &    162,023 &   0,162023 &     4,1154 &  26251,375 &  2,062e-2 &  13,301768 \\
 7 &    144,285 &   0,144285 &     3,6649 &  20818,287 &  1,635e-2 &  10,548782 \\
 8 &    128,490 &   0,128490 &     3,2636 &  16509,652 &  1,297e-2 &   8,365564 \\
 9 &    114,424 &   0,114424 &     2,9064 &  13092,749 &  1,028e-2 &   6,634194 \\
10 &    101,897 &   0,101897 &     2,5882 &  10383,022 &  8,155e-3 &   5,261155 \\
11 &     90,742 &   0,090742 &     2,3048 &   8234,111 &  6,467e-3 &   4,172286 \\
12 &     80,808 &   0,080808 &     2,0525 &   6529,947 &  5,129e-3 &   3,308773 \\
13 &     71,962 &   0,071962 &     1,8278 &   5178,483 &  4,067e-3 &   2,623976 \\
14 &     64,084 &   0,064084 &     1,6277 &   4106,724 &  3,225e-3 &   2,080908 \\
15 &     57,068 &   0,057068 &     1,4495 &   3256,780 &  2,558e-3 &   1,650235 \\
16 &     50,821 &   0,050821 &     1,2908 &   2582,744 &  2,028e-3 &   1,308696 \\
17 &     45,257 &   0,045257 &     1,1495 &   2048,209 &  1,609e-3 &   1,037843 \\
18 &     40,303 &   0,040303 &     1,0237 &   1624,304 &  1,276e-3 &   0,823047 \\
19 &     35,891 &   0,035891 &     0,9116 &   1288,131 &  1,012e-3 &   0,652706 \\
20 &     31,961 &   0,031961 &     0,8118 &   1021,535 &  8,023e-4 &   0,517619 \\
21 &     28,462 &   0,028462 &     0,7229 &    810,114 &  6,363e-4 &   0,410491 \\
22 &     25,347 &   0,025347 &     0,6438 &    642,449 &  5,046e-4 &   0,325534 \\
23 &     22,572 &   0,022572 &     0,5733 &    509,486 &  4,001e-4 &   0,258160 \\
24 &     20,101 &   0,020101 &     0,5106 &    404,040 &  3,173e-4 &   0,204730 \\
25 &     17,900 &   0,017900 &     0,4547 &    320,419 &  2,517e-4 &   0,162359 \\
26 &     15,941 &   0,015941 &     0,4049 &    254,104 &  1,996e-4 &   0,128756 \\
27 &     14,196 &   0,014196 &     0,3606 &    201,513 &  1,583e-4 &   0,102108 \\
28 &     12,641 &   0,012641 &     0,3211 &    159,807 &  1,255e-4 &   0,080976 \\
29 &     11,258 &   0,011258 &     0,2859 &    126,733 &  9,954e-5 &   0,064217 \\
30 &     10,025 &   0,010025 &     0,2546 &    100,504 &  7,894e-5 &   0,050926 \\
31 &      8,928 &   0,008928 &     0,2268 &     79,703 &  6,260e-5 &   0,040386 \\
32 &      7,950 &   0,007950 &     0,2019 &     63,207 &  4,964e-5 &   0,032028 \\
33 &      7,080 &   0,007080 &     0,1798 &     50,126 &  3,937e-5 &   0,025399 \\
34 &      6,305 &   0,006305 &     0,1601 &     39,752 &  3,122e-5 &   0,020142 \\
35 &      5,615 &   0,005615 &     0,1426 &     31,524 &  2,476e-5 &   0,015974 \\
36 &      5,000 &   0,005000 &     0,1270 &     25,000 &  1,963e-5 &   0,012668 \\
37 &      4,453 &   0,004453 &     0,1131 &     19,826 &  1,557e-5 &   0,010046 \\
38 &      3,965 &   0,003965 &     0,1007 &     15,723 &  1,235e-5 &   0,007967 \\
39 &      3,531 &   0,003531 &     0,0897 &     12,469 &  9,793e-6 &   0,006318 \\
40 &      3,145 &   0,003145 &     0,0799 &      9,888 &  7,766e-6 &   0,005010 \\
\bottomrule[1pt]
\end{longtable}

\index{AWG (\guillemotleft American Wire Gauge\guillemotright)!conversió a\unit{mm^2}} Es donen a continuació les fórmules per passar directament d'un número AWG a la seva secció $S$ equivalent expressada en\unit{mm^2}, i al seu diàmetre $d$ equivalent expressat en\unit{mm}.
\begin{align}
   S/\unit{mm^2} &= \frac{ \piup \times \num{25,4}^2 \times 460^2}{4 \times 10^6 \times 92^{\frac{2(\text{AWG}+3)}{39}}} =
   \frac{\num{53,4751207}}{92^{\text{AWG}/\num{19,5}}}\label{eq:awg_mm2}\\[2ex]
   d/\unit{mm} &= \sqrt{\frac{\num{25,4}^2 \times 460^2}{10^6 \times 92^{\frac{2(\text{AWG}+3)}{39}}}} =
   \frac{\num{8,2514628}}{92^{\text{AWG}/\num{39}}}
\end{align}

En aquestes dues fórmules cal utilitzar els valors -1, -2, i -3  pels números 00 AWG,
000 AWG i 0000 AWG respectivament.


\begin{exemple}[Secció d'un conductor AWG]
    Es tracte de calcular la secció del conductor 14 AWG, utilitzant l'equació \eqref{eq:awg_mm2}.

    Utilitzant aquesta equació tenim:
    $S = \dfrac{\num{53,4751207}}{92^{\text{14}/\num{19,5}}} = \SI{2,1}{mm^2}$
\end{exemple}


A més de les sigles «AWG», hi ha altres formes alternatives d'escriptura. En el cas dels conductors compresos entre 1 AWG i 40 AWG, també es pot veure escrit (prenent com a exemple el conductor 4 AWG):
\begin{itemize}
   \item \#4 (on el símbol «\#» s'utilitza com a abreviació de «number»)
   \item No. 4 (on «No.» és l'abreviació de «number»)
   \item No. 4 AWG
   \item 4 ga. (on «ga.» és l'abreviació de «gauge»)
\end{itemize}

En el cas dels conductors 0 AWG, 00 AWG, 000 AWG i 0000 AWG també es pot veure escrit (prenent com a exemple el conductor 000 AWG):
\begin{itemize}
   \item 3/0
   \item 3/0 AWG
   \item \#000
   \item \#3/0
\end{itemize}
\break

En el cas de cables formats per més d'un conductor, el cable es denomina utilitzant la secció dels conductors, seguida del nombre de conductors que formen el cable. Per exemple: \#14/2 o 14-2 identifica un cable format per dos conductors de 14 AWG.

En el cas d'un conductor format per múltiples fils, el cable es denomina utilitzant l'AWG total (suma de les seccions de cada fil), seguit del nombre de fils i de l'AWG de cada fil. Per exemple: 22 AWG 7/30 identifica un conductor de 22 AWG, format per 7 fils de  30 AWG cadascun.

