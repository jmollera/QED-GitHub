\chapter{Constants Físiques}\label{sec:const_fis} \index{constants físiques}

\section{Taula de valors}

En la Taula \vref{taula:Const-Fis} es pot veure una recopilació de
constants físiques; les xifres entre parèntesis que hi apareixen representen l'error absolut del valor.

Els valors de les constants d'aquesta taula són els recomanats
l'any 2018 pel «Committee on Data for Science and Technology»
(CODATA), un comitè científic de l'«International Council
for Science».

Podeu trobar  més informació a
les adreces: \href{http://www.codata.org/}{www.codata.org} i \href{http://physics.nist.gov/cuu/Constants/}{physics.nist.gov/cuu/Constants}.\index{CODATA}
\index{NIST}

\begin{ThreePartTable}
\begin{TableNotes}
    \item[a] {\footnotesize El valor numèric en si, es l'anomenat número d'Avogadro.}
    \item[b] {\footnotesize $K\ped{cd}$ és l'eficàcia lluminosa d'una radiació monocromàtica de freqüència \SI{540e12}{Hz}.}
    \item[c] {\footnotesize Un «electró-volt» és l'energia cinètica que adquireix un electró després de creuar una diferència de potencial d'un volt en el buit.}
    \item[d] {\footnotesize Donada una partícula X, $m(\mathrm{X})$ és la massa atòmica de la partícula X, $M(\mathrm{X})$ és la massa molar de la partícula X, i $A\ped{r}(\mathrm{X})$ és la massa atòmica relativa de la partícula X. Es compleixen les relacions següents: $M(\mathrm{X}) = m(\mathrm{X})\, N\ped{A}$ i $ A\ped{r}(\mathrm{X}) = \frac{m(\mathrm{X})}{m\ped{u}} = \frac{M(\mathrm{X})}{M\ped{u}}$.}
\end{TableNotes}
\begin{longtable}{lcll}
   \caption{\label{taula:Const-Fis} Constants físiques}\\
   \toprule[1pt]
   Magnitud & Símbol & Valor & Error relatiu\\
   \midrule
   \endfirsthead
   \caption[]{Constants físiques (\emph{ve de la pàgina anterior})} \\
   \toprule[1pt]
   Magnitud & Símbol & Valor & Error relatiu\\
   \midrule
   \endhead
   \midrule
   \multicolumn{4}{r}{\sffamily\bfseries\color{NavyBlue}(\emph{continua a la pàgina següent})}
   \endfoot
   \insertTableNotes
   \endlastfoot
   freqüència de la transició & $\Deltaup\nu\ped{Cs}$ & \SI{9 192 631 770}{Hz} & exacte \\
   hiperfina del cesi-133 & & & \\[0.5em]
   velocitat de la llum en el buit & $c$ & \SI{299792458}{m/s} & exacte\\[0.5em]
   constant de Planck & $h$ & \SI{6,62607015 e-34}{J.s} & exacte \\[0.5em]
   càrrega elemental & $e$ & \SI{1,602176634 e-19}{C} & exacte \\[0.5em]
   constant de Boltzmann & $k$ & \SI{1,380649e-23}{J/K} & exacte \\[0.5em]
   constant d'Avogadro & $N\ped{A}$\tnote{a} & \SI{6,02214076 e23}{mol^{-1}} & exacte \\[0.5em]
   eficàcia lluminosa & $K\ped{cd}$\tnote{b} & \SI{683}{lm/W} & exacte \\[0.5em]
   constant de Planck reduïda: & $\hbar$ & \SI{1,054571817\dots e-34}{J.s} & exacte\\
   $h/(2\piup)$ & & & \\[0.5em]
   constant d'Stefan-Boltzmann:  & $\sigma$ & \SI{5,670374419\dots e-8}{W/(m^2.K^4)} & exacte \\
   $\piup^2 k^4 / (60\, \hbar^3 c^2)$ & & & \\[0.5em]
   constant molar dels gasos: $N\ped{A} k$ & $R$ & \SI{8,31446261815324}{\,J/(mol.K)} & exacte \\[0.5em]
   constant de Faraday: $N\ped{A} e$ & $F$ & \SI{96485,3321233100184}{C/mol} & exacte \\[0.5em]
   electró-volt & eV\tnote{c} & \SI{1,602176634e-19}{J} & exacte \\[0.5em]
   atmosfera estàndard  & -- & \SI{101325}{Pa} & exacte \\[0.5em]
   acceleració de la gravetat & $g\ped{n}$ & \SI{9,80665}{m/s^2} & exacte \\
   estàndard & & & \\[0.5em]
   massa atòmica relativa & $A\ped{r}({}^{12}\mathrm{C})\,$\tnote{d} & 12 & exacte \\
   del carboni-12 & & & \\[0.5em]
   constant de massa atòmica: & $m\ped{u}$\tnote{d} & \SI{1,6605390660(50)e-27}{kg} & \num{3e-10}  \\
   $\frac{1}{12}  m({}^{12}\mathrm{C})$ & & & \\[0.5em]
   constant de massa molar: & $M\ped{u}$\tnote{d} & \SI{0,99999999965(30)e-3}{kg/mol} & \num{3e-10}  \\
   $N\ped{A} m\ped{u}$ & & & \\[0.5em]
   massa molar del carboni-12: & $M({}^{12}\mathrm{C})\,$\tnote{d} & \SI{11,9999999958(36)e-3}{kg/mol} & \num{3e-10} \\
   $A\ped{r}({}^{12}\mathrm{C})\,M\ped{u} $  & & & \\[0.5em]
   constant gravitacional & $G$ &   \SI{6,67430(15) e-11}{m^3/(kg.s^2)} & \num{2,2e-5} \\
   de Newton & & & \\[0.5em]
   constant de l'estructura & $\alpha$ & \num{7,2973525693(11) e-3} & \num{1,5e-10} \\
   fina: $e^2/(4\piup\epsilon_0\hbar  c)$ & & & \\[0.5em]
   permeabilitat magnètica & $\mu_0$ & \SI{1,25663706212(19) e-6}{N/A^2} & \num{1,5e-10} \\
   en el buit: $4\piup\alpha\hbar/(e^2  c)$  & & & \\[0.5em]
   permeabilitat  elèctrica  & $\epsilon_0$ & \SI{8,8541878128(13) e-12}{F/m} & \num{1,5e-10} \\
   en el buit: $1/(\mu_0 c^2)$ & & & \\[0.5em]
   impedància característica  & $Z_0$ &  \SI{376,730313668(57)}{\ohm} & \num{1,5e-10}\\
   del buit: $\sqrt{\mu_0/\epsilon_0}=\mu_0 c$ & & &  \\[0.5em]
   massa de l'electró & $m\ped{e}$ & \SI{9,1093837015(28) e-31}{kg} & \num{3,0e-10} \\[0.5em]
   massa del  protó & $m\ped{p}$ & \SI{1,67262192369(51) e-27}{kg} & \num{3,1e-10} \\[0.5em]
   massa del neutró & $m\ped{n}$ & \SI{1,67492749804(95) e-27}{kg} & \num{5,7e-10} \\[0.5em]
   massa del deuteri & $m\ped{d}$ & \SI{3,3435837724(10) e-27}{kg} & \num{3,0e-10} \\[0.5em]
   massa del triti & $m\ped{t}$ & \SI{5,0073567446(15) e-27}{kg} & \num{3,0e-10} \\[0.5em]
   massa de la partícula $\alphaup$ & $m_\alphaup$ & \SI{6,6446573357(20) e-27}{kg} & \num{3,0e-10} \\[0.5em]
   radi de Bohr:  & $a_0$ & \SI{5,29177210903(80) e-11}{m} & \num{1,5e-10} \\
   $\hbar/(\alpha m\ped{e}c) = 4\piup\epsilon_0\hbar^2/(m\ped{e}e^2)$ & & & \\[0.5em]
   longitud d'ona Compton:  & $\lambda\ped{C}$ & \SI{2,42631023867(73) e-12}{m} & \num{3,0e-10} \\
   $h/(m\ped{e} c)$ & & & \\[0.5em]
\bottomrule[1pt]
\end{longtable}
\end{ThreePartTable}
\index{velocitat de la llum en el buit}  \index{constant!magnètica} \index{impedància característica del
buit} \index{atmosfera estàndard} \index{acceleració!de la gravetat
estàndard} \index{massa!molar del carboni-12}
\index{constant!gravitacional de Newton} \index{constant!de Planck}
\index{constant!de Planck redu\"{i}da} \index{constant!de Faraday}
\index{carrega elemental@càrrega elemental} \index{massa!de
l'electró} \index{massa!del protó} \index{massa!del neutró}
\index{massa!del deuteri} \index{massa!de la partícula $\alpha$}
\index{constant!d'Avogadro} \index{constant!molar
dels gasos} \index{constant!de Boltzmann}
\index{constant!d'Stefan-Boltzmann} \index{radi de Bohr}\index{longitud d'ona Compton}
\index{$\mu_0$} \index{$\epsilon_0$} \index{c@$c$} \index{atm}
\index{gn@$g\ped{n}$} \index{Z0@$Z_0$} \index{F@$F$}
\index{m@$M({}^{12}\mathrm{C})$} \index{G@$G$} \index{h@$h$}
\index{h@$\hbar$} \index{e@$e$} \index{me@$m\ped{e}$}
\index{mp@$m\ped{p}$} \index{mn@$m\ped{n}$} \index{m@$m\ped{d}$}
\index{ma@$m_\alpha$} \index{NA@$N\ped{A}$} \index{R@$R$}
\index{k@$k$} \index{$\sigma$} \index{a0@$a_0$} \index{$\lambda\ped{C}$}
\index{eV}\index{electró-volt}\index{eficàcia lluminosa}\index{Kcd@$K\ped{cd}$}
\index{$\Deltaup\nu\ped{Cs}$}\index{massa!atòmica relativa del carboni-12}
\index{constant!de massa atòmica}\index{constant!de massa molar}
\index{Ar@$A\ped{r}({}^{12}\mathrm{C})$}\index{mu@$m\ped{u}$}\index{Mu@$M\ped{u}$}


\section{Error absolut i relatiu}\label{err_abs_rel}

Tal com s'ha dit al principi, les xifres entre parèntesis indiquen l'error absolut del valor que les precedeix. El nombre de xifres entre parèntesis determina la posició decimal d'aquest error; per exemple, en el cas de la  massa de l'electró tenim:
\index{error!absolut}\index{error!relatiu}
\[
    m\ped{e} = \SI{9,109 383 7015(28) e-31}{kg}
\]

Les dues xifres entre parèntesis, 28, determinen que la posició decimal de l'error absolut ha de correspondre a la posició de les dues últimes xifres, 15, del valor; l'error absolut $\epsilon$  és doncs:
\[
    \epsilon = \SI{0,000 000 0028 e-31}{kg}
\]

Per tant, el valor de la massa de l'electró també es pot escriure's així:
 \[
    m\ped{e} = \SI[separate-uncertainty]{9,109 383 7015(28) e-31}{kg}
\]

Finalment, l'error relatiu $\epsilon\ped{r}$ es calcula dividint l'error absolut pel valor sense l'error:
\[
    \epsilon\ped{r} = \frac{\SI{0,000 000 0028 e-31}{kg}}{\SI{9,109 383 7015 e-31}{kg}} =   \num{3,0e-10}
\]
