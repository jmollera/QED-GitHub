\addtocontents{xms}{\protect\addvspace{10pt}}
\chapter{Cables}\index{cables}

\section{Introducció}
Es tracten en aquest capítol qüestions relatives als cables elèctrics.

\section{Resistència}\index{cables!resistència}

\subsection{Resistència d'un conductor}

La resistència $R$ d'un conductor depèn de la resistivitat $\rho$
del material, i de la llargada $l$ i  la secció
$S$  del conductor.
\begin{equation}
   R= \rho \frac{l}{S}
\end{equation}
\index{$\rho$}

\index{resistivitat!variació amb la temperatura}La resistivitat no
és un valor constant sinó que depèn de la temperatura, a major
temperatura major resistivitat. Coneixent la resistivitat $\rho_1$ a una
temperatura $T_1$ es pot calcular la resistivitat $\rho_2$ a una altra
temperatura $T_2$, a partir del coeficient de variació de la
resistivitat amb la temperatura $\alpha_1$ donat a la temperatura $T_1$.
\begin{equation}
   \rho_2 = \rho_1 [1 + \alpha_1 (T_2 - T_1)]\label{eq:resistivitat}
\end{equation}
\index{$\alpha$}

\index{resistivitat!valors}En la taula
\vref{taula:param-elc} es donen valors de resistivitat i de
coeficients de variació de la resistivitat amb la temperatura a
\qty{20}{\degreeCelsius} i a \qty{0}{\degreeCelsius}, de diversos materials.

\begin{center}
   \captionof{table}{Paràmetres elèctrics d'alguns materials}
   \label{taula:param-elc}
   \begin{tabular}{lccc}
   \toprule[1pt]
   Material & $\rho_{\qty{20}{\degreeCelsius}} / \unit[per-mode = fraction]{\ohm\milli\metre\squared\per\metre}$ & $\alpha_{\qty{20}{\degreeCelsius}} / \unit{\degreeCelsius^{-1}}$ &
   $\alpha_{\qty{0}{\degreeCelsius}} / \unit{\degreeCelsius^{-1}}$
   \\
   \midrule
      Alumini & \num{0,02825} & \num{0,00391} & \num{0,00424} \\
      Coure   & \num{0,01723} & \num{0,00393} & \num{0,00427} \\
      Plata   & \num{0,01645} & \num{0,00380} & \num{0,00412} \\
   \bottomrule[1pt]
   \end{tabular}
\end{center}

La resistència així calculada és vàlida quan el corrent que circula
pel cable és corrent continu.

Quan el corrent que circula pel cable és
corrent altern cal tenir en compte l'efecte peŀlicular, el qual
 provoca un augment de la resistència causat perquè el corrent
tendeix a circular més per la zona perifèrica del conductor que no pas per
la zona central; l'efecte és important per a valors elevats de la
secció del conductor o de la freqüència del corrent.\index{efecte peŀlicular}

\index{resistència!efectiva}La resistència efectiva es troba a
partir de la resistència calculada anteriorment per a corrent
continu, i d'un factor $k$ que té en compte l'efecte peŀlicular.
\begin{equation}
   R\ped{efectiva} = k R
\end{equation}

En la taula \vref{taula:const_r_ef} es donen valors de $k$ per a conductors de coure i d'alumini, i per a diversos valors del producte de la secció del conductor per la freqüència del corrent. Aquests valors s'han obtingut de la  referència \cite{RASa}, pàgina 114.

\begin{center}
   \captionof{table}{Valors de $k$ per al càlcul de la resistència efectiva}
   \label{taula:const_r_ef}
   \begin{tabular}{S[table-format=6.0]S[table-format=1.3]S[table-format=1.3]}
   \toprule[1pt]
   {Secció$\,\times\,$Freqüència} & \multicolumn{2}{c}{$k$} \\
   \cmidrule(rl){1-1} \cmidrule(rl){2-3}
    \unit{mm^2.Hz} & \multicolumn{1}{c}{Cu} & \multicolumn{1}{c}{Al} \\
   \midrule
  5000 &  1,000 & 1,000 \\
  10000 & 1,008 & 1,000 \\
  15000 & 1,025 & 1,006 \\
  20000 & 1,045 & 1,015 \\
  25000 & 1,070 & 1,026 \\
  30000 & 1,096 & 1,040 \\
  35000 & 1,126 & 1,053 \\
  40000 & 1,158 & 1,069 \\
  45000 & 1,195 & 1,085 \\
  50000 & 1,230 & 1,104 \\
  75000 & 1,433 & 1,206 \\
  100000 & 1,622 & 1,330 \\
   \bottomrule[1pt]
  \end{tabular}
\end{center}

\subsection{Resistència d'un cable}

La resistència d'un cable $R\ped{Cable}$ depèn del nombre de conductors per fase $n$ (o
per pol, en corrent continu), de la resistència de cada conductor $R\ped{Conductor}$ i del
tipus de tensió elèctrica a la qual estigui sotmès el cable (monofàsica, trifàsica,
contínua, etc.).

\subsubsection*{Corrent continu o altern monofàsic}
\begin{equation}\label{eq:r_cc_mono}
    R\ped{Cable} = 2\, \frac{R\ped{Conductor}}{n}
\end{equation}

El valor multiplicatiu 2 prové del fet que cal tenir en compte tant el conductor d'anada
com el de tornada.

\subsubsection*{Corrent altern trifàsic equilibrat}
\vspace{-5mm}
\begin{equation}\label{eq:r_trifas}
R\ped{Cable} = \frac{R\ped{Conductor\;fase}}{n}
\end{equation}

Atès que no circula corrent pel neutre, la seva resistència no  té cap influència.

\subsubsection*{Corrent altern trifàsic desequilibrat}
\begin{equation}
    R\ped{Cable\;fase} = \frac{R\ped{Conductor\;fase}}{n\ped{fase}} \qquad\qquad
    R\ped{Cable\; neutre} = \frac{R\ped{Conductor\;neutre}}{n\ped{neutre}}
\end{equation}

En aquest cas cal tenir en compte que els corrents que circulen per les fases i pel neutre
són diferents.


\section{Caiguda de tensió}\index{cables!caiguda de tensió}

La caiguda de tensió $\Delta U$ en un cable es defineix com la diferència entre els mòduls de les tensions a l'origen $|{\cmplx{U}\ped{O}}|$ i al final $|\cmplx{U}\ped{F}|$ del cable.
\begin{equation}
   \Delta U \equiv |\cmplx{U}\ped{O}| - |\cmplx{U}\ped{F}|
\end{equation}

\subsection{Caiguda de tensió en corrent continu}\index{cables!caiguda de tensió!en corrent continu}

En corrent continu la caiguda de tensió depèn del corrent $I$ que circula pel cable i de la  resistència del  cable mateix, calculada segons l'equació \eqref{eq:r_cc_mono}.
\begin{equation}
   \Delta U = I R\ped{Cable}
\end{equation}

\subsection{Caiguda de tensió en corrent altern}\index{cables!caiguda de tensió!en corrent altern}

\index{factor!de potència}En corrent altern la caiguda de tensió
depèn del  corrent $\cmplx{I}$ que circula pel cable, de la
resistència i la reactància del  cable mateix, i del factor de
potència $\cos \varphi$. El diagrama fasorial d'aquestes magnituds
es pot veure en la figura \vref{pic:cdt_ca}.

\begin{center}
   \input{Imatges/Cap-Cables-Caiguda-Tensio.pdf_tex}
   \captionof{figure}{Caiguda de tensió en corrent altern}
   \label{pic:cdt_ca}
\end{center}

Quan es tracta de corrent monofàsic, la resistència del cable es calcula segons l'equació
\eqref{eq:r_cc_mono}; la reactància del cable $X\ped{Cable}$ es calcula de forma anàloga
amb la mateixa equació, a partir de la reactància dels conductors $X\ped{Conductor}$.

Pel que fa al corrent trifàsic, se suposa equilibrat i per tant s'utilitza l'equació
\eqref{eq:r_trifas} per calcular la resistència del cable $R\ped{Cable}$ (i de forma
anàloga la reactància $X\ped{Cable}$). Addicionalment, els corrents fan referència als
corrents de fase i les tensions a les tensions fase-neutre; l'angle $\varphi$ és per
tant l'angle entre la tensió final fase--neutre i el corrent de fase.

Disposem en aquest cas de dues equacions, una d'exacta i una altra d'aproximada (per a valors elevats de $\cos \varphi$).

\begin{subequations}
\begin{align}
   \Delta U &= |\cmplx{I}| \,( R\ped{Cable} \cos \varphi + X\ped{Cable} \sin \varphi) + |\cmplx{U}\ped{O}| - \sqrt{|\cmplx{U}\ped{O}|^2 - |\cmplx{I}|^2 ( X\ped{Cable} \cos \varphi - R\ped{Cable} \sin \varphi )^2} \label{eq:cdt_trif_exact} \\[2ex]
   \Delta U &\approx |\cmplx{I}| \,( R\ped{Cable} \cos \varphi + X\ped{Cable} \sin \varphi), \qquad \cos \varphi \gtrapprox \num{0,8} \label{eq:cdt_trif_aprox}
\end{align}
\end{subequations}


\begin{exemple}[Càlcul de la caiguda de tensió en un sistema trifàsic]
	\addcontentsxms{Càlcul de la caiguda de tensió en un sistema trifàsic}
       Es tracta de calcular la caiguda de tensió en un sistema trifàsic on $|\cmplx{U}\ped{O}| = \qty{380}{V}$ (fase--fase), $|\cmplx{I}|=\qty{630}{A}$ i $\cos \varphi = \num{0,87}$(i). La unió entre els extrems origen  i final es fa amb tres cables unipolars en paraŀlel de $\qty{240}{mm^2}$ de secció cadascun i $\qty{400}{m}$ de llargada; els valors per fase de resistència i inductància són $\qty{0,095}{\ohm/km}$ i $\qty{0,102}{\ohm/km}$ respectivament.

    A partir de l'equació \eqref{eq:r_trifas} calculem els valors de $R\ped{Cable}$ i de $X\ped{Cable}$:

    \[
       R\ped{Cable} = \frac{\qty{0,095}{\ohm/km}\times \qty{0,4}{km}}{3} = \qty{0,0127}{\ohm}
    \]
    \[
       X\ped{Cable} = \frac{\qty{0,102}{\ohm/km}\times \qty{0,4}{km}}{3} = \qty{0,0136}{\ohm}
    \]

    Obtenim a continuació el valor de $\sin \varphi$:

    \[
       \sin \varphi = \sqrt{1-\num{0,87}^2} = \num{0,49}
    \]

    Calculem en primer lloc la caiguda de tensió de forma aproximada utilitzant l'equació \eqref{eq:cdt_trif_aprox}:

    \[
       \Delta U \approx \qty{630}{A} \times ( \qty{0,0127}{\ohm} \times \num{0,87} + \qty{0,0136}{\ohm} \times \num{0,49} ) = \qty{11,16}{V}
    \]

    Valor que en tant per cent respecte de la tensió a l'origen representa:

    \[
        \Delta u = \frac{\qty{11,16}{V}}{\frac{380}{\sqrt{3}}\unit{\,V}} \times 100 = \qty{5,09}{\percent}
    \]

    Finalment, calculem la caiguda de tensió exacta utilitzant l'equació \eqref{eq:cdt_trif_exact}:

    \[ \begin{split}
       \Delta U &=  \qty{630}{A} \times( \qty{0,0127}{\ohm} \times \num{0,87} + \qty{0,0136}{\ohm} \times \num{0,49}) + \qty[parse-numbers = false]{\frac{380}{\sqrt{3}}}{V}  \\
        & \quad - \sqrt{\left(\qty[parse-numbers = false]{\frac{380}{\sqrt{3}}}{V}\right)^2 - \left(\qty{630}{A}\right)^2 \times  \left( \qty{0,0136}{\ohm} \times \num{0,87} - \qty{0,0127}{\ohm} \times \num{0,49} \right)^2 } \,= \qty{11,19}{V}
    \end{split} \]

    Valor que en tant per cent respecte de la tensió a l'origen representa:

    \[
        \Delta u = \frac{\qty{11,19}{V}}{\qty[parse-numbers = false]{\frac{380}{\sqrt{3}}}{V}} \times 100 = \qty{5,10}{\percent}
    \]
\end{exemple}

\section{Capacitat tèrmica en curtcircuit}\index{cables!capacitat tèrmica en curtcircuit}\label{ces:cables_Icc_termica}

Quan hi ha un curtcircuit en un cable, tota la calor generada no es transmet a l'exterior en els instants inicials sinó que s'acumula en la massa del conductor, incrementant-ne la temperatura (procés adiabàtic). En aquestes condicions, la norma CEI 60724 \textit{Short-circuit temperature limits of electric cables with rated voltages of 1 kV (Um = 1,2 kV) and 3 kV (Um = 3,6 kV)}, dona la següent equació per a cables de tensió assignada d'\qty{1}{kV} i \qty{3}{kV}:\index{CEI!60724-00@60724}

\begin{equation}
   I\ped{cc}^2 \,t\ped{cc} = K^2 S^2 \ln \frac{\beta + \theta\ped{f}}{\beta + \theta\ped{i}}
\end{equation}
Amb:

\begin{list}{}
   {\setlength{\labelwidth}{10mm} \setlength{\leftmargin}{12mm} \setlength{\labelsep}{2mm}}
   \item[\hspace{5mm}$\boldsymbol{I\ped{cc}}$\hfill] Corrent de curtcircuit
   \item[\hspace{5mm}$\boldsymbol{t\ped{cc}}$\hfill] Temps màxim que pot durar el curtcircuit sense que es malmeti el cable
   \item[\hspace{5mm}$\boldsymbol{K}$\hfill] Paràmetre que depèn del material del conductor
   \item[\hspace{5mm}$\boldsymbol{S}$\hfill] Secció del cable
   \item[\hspace{5mm}$\boldsymbol{\beta}$\hfill] Invers del coeficient de variació de la resistivitat amb la temperatura ($\beta = 1/ \alpha$)
   \item[\hspace{5mm}$\boldsymbol{\theta\ped{i}}$\hfill] Temperatura inicial del curtcircuit (depèn del material de l'aïllament del conductor)
   \item[\hspace{5mm}$\boldsymbol{\theta\ped{f}}$\hfill] Temperatura final del curtcircuit (depèn del material de l'aïllament del conductor)
\end{list}

La mateixa norma CEI 60724 dona valors per a $K$, $\beta$, $\theta\ped{i}$ i $\theta\ped{f}$ per a diferents materials del conductor i de l'aïllament, arribant finalment a la fórmula següent, la qual utilitza el paràmetre $C$:
\begin{equation}\label{eq:Icc_termica}
   I\ped{cc}/{\scriptstyle\unit{A}} = C\, \frac{S/{\scriptstyle\unit{mm^2}}}{\sqrt{t\ped{cc}/{\scriptstyle\unit{s}}}}
\end{equation}


En la taula \vref{taula:const_termica} es donen valors de $C$ per a diferents materials del conductor i de l'aïllament, segons la norma CEI 60724.

\begin{center}
   \captionof{table}{Valors de $C$ per al càlcul de curtcircuits en cables} \label{taula:const_termica}
   \begin{tabular}{c>{\hspace{2.5em}}cc}
   \toprule[1pt]
   \renewcommand*{\multirowsetup}{\centering}
   \multirow{2}{25mm}{\rule{0mm}{4mm}Material del\\conductor} & \multicolumn{2}{c}{$C$, segons el material de l'aïllament} \\ \cmidrule(rl){2-3}
    & PVC & EPR i XLPE \\
   \midrule
   Cu & 115 & 143 \\
   Al & 76 & 94 \\
   \bottomrule[1pt]
   \end{tabular}
\end{center}


\begin{exemple}[Càlcul de la capacitat tèrmica d'un cable]\label{ex:cap-termica-cable}
	\addcontentsxms{Càlcul de la capacitat tèrmica d'un cable}
       Es tracta de calcular el temps màxim que un cable de coure de $\qty{50}{mm^2}$ amb aïllament d'EPR, pot suportar un corrent de curtcircuit de $\qty{15}{kA}$.

    A partir de la taula \vref{taula:const_termica} tenim: $C=143$. Utilitzant l'equació \eqref{eq:Icc_termica} calculem el temps màxim demanat:
    \[
       \num{15000} = 143 \times \frac{50}{\sqrt{t\ped{cc}/{\scriptstyle\unit{s}}}} \quad \Rightarrow \quad
       t\ped{cc} =  \left(143\times \frac{50}{\num{15000}} \right) ^2 \,\unit{s} = \qty{0,23}{s}
    \]
\end{exemple}

\section{Conversió entre unitats americanes i unitats SI}

\subsection{\textit{Mils} (mil), \textit{circular mils} (cmil o CM) i \textit{thousand circular mils} (kcmil o MCM)}\label{sec:MCM}
\index{mil}\index{cmil}\index{CM}\index{kcmil}\index{MCM}
\index{mils@\textit{mils}} \index{circular mils@\textit{circular mils}} \index{thousand circular mils@\textit{thousand circular mils}}

\index{mils@\textit{mils}!difinició} \index{circular mils@\textit{circular mils}!difinició} \index{thousand circular mils@\textit{thousand circular mils}!difinició}Les definicions d'aquestes tres unitats utilitzades en la mesura de diàmetres i seccions de conductors són:\footnote{Actualment són molt més usats els símbols cmil i kcmil, que no pas els seus equivalents respectius CM i MCM.}
\begin{align}
  \qty{1}{mil} &\equiv \text{Una miŀlèsima de polsada} \\
  \qty{1}{cmil} = \qty{1}{CM} &\equiv  \text{Àrea d'un cercle de diàmetre igual a \qty{1}{mil}} \\
  \qty{1}{kcmil} = \qty{1}{MCM} &\equiv \qty{1000}{cmil} = \qty{1000}{CM}
\end{align}

\index{mils@\textit{mils}!conversions} \index{circular mils@\textit{circular mils}!conversions} \index{thousand circular mils@\textit{thousand circular mils}!conversions}Es donen a continuació algunes conversions d'aquestes unitats:
\begin{align}
   \qty{1}{mil} &= \qty{e-3}{in}  \\
  \qty{1}{mil} &= \qty{e-3}{in} \times \frac{\qty{25,4}{mm}}{\qty{1}{in}} = \qty{25,4e-3}{mm}  \\
  \qty{1}{cmil} &= \qty[parse-numbers=false]{\frac{\piup}{4}}{mil^2} \approx \qty{0,785398}{mil^2}   \\
   \qty{1}{cmil} &= \frac{\piup}{4}\times \qty{e-6}{in^2} \approx \qty{0,785398e-6}{in^2} \\
   \qty{1}{cmil} &= \frac{\piup}{4} \times \qty{e-6}{in^2} \times \frac{\qty{645,16}{mm^2}}{\qty{1}{in^2}} \approx \qty{506,7075e-6}{mm^2}
   \\[1ex]
   \qty{1}{kcmil} &= \qty{785,398}{mil^2}  = \qty{0,785398e-3}{in^2} \approx \qty{0,5067075}{mm^2}\label{eq:kcmil_mm2}
\end{align}

Una relació útil entre diàmetres  i seccions és la següent: la secció $S$ d'un cercle expressada en cmil és igual al quadrat del diàmetre $d$ del cercle expressat en mil.
\begin{equation}
   S/{\scriptstyle\unit{cmil}}  = (d/{\scriptstyle\unit{mil}})^2
\end{equation}

En la taula \vref{taula:MCM} es relacionen els diàmetres i seccions en diverses unitats, dels conductors usualment disponibles compresos entre $\qty{2000}{kcmil}$ i $\qty{250}{kcmil}$.

\begin{center}
    \captionof{table}{Dimensions de cables definits en kcmil}
    \label{taula:MCM}
    \begin{tabular}{S[table-format=4.0]S[table-format=1.6]S[table-format=4.4]
                    S[table-format=4.5]S[table-format=1.7]S[table-format=2.5]}
    \toprule[1pt]
    \multicolumn{3}{c}{Secció} &   \multicolumn{3}{c}{Diàmetre}         \\
    \cmidrule(rl){1-3} \cmidrule(rl){4-6}
    \multicolumn{1}{c}{kcmil}  &    \multicolumn{1}{c}{\unit{in^2}}  & \multicolumn{1}{c}{\unit{mm^2}}  & \multicolumn{1}{c}{mil}
           &    \multicolumn{1}{c}{in} &   \multicolumn{1}{c}{mm}   \\
    \midrule
    2000 &   1,570796 &   1013,4150 & 1414,21356 &  1,4142136 &   35,92102 \\
    1750 &   1,374447 &   886,7381  & 1322,87566 &  1,3228757 &   33,60104 \\
    1600 &   1,256637 &   810,7320  & 1264,91106 &  1,2649111 &   32,12874 \\
    1500 &   1,178097 &   760,0612  & 1224,74487 &  1,2247449 &   31,10852 \\
    1250 &   0,981748 &   633,3843  & 1118,03399 &  1,1180340 &   28,39806 \\
    1000 &   0,785398 &   506,7075  & 1000,00000 &  1,0000000 &   25,40000 \\
     800 &   0,628319 &   405,3660  &  894,42719 &  0,8944272 &   22,71845 \\
     750 &   0,589049 &   380,0306  &  866,02540 &  0,8660254 &   21,99705 \\
     700 &   0,549779 &   354,6952  &  836,66003 &  0,8366600 &   21,25116 \\
     600 &   0,471239 &   304,0245  &  774,59667 &  0,7745967 &   19,67476 \\
     500 &   0,392699 &   253,3537  &  707,10678 &  0,7071068 &   17,96051 \\
     450 &   0,353429 &   228,0184  &  670,82039 &  0,6708204 &   17,03884 \\
     400 &   0,314159 &   202,6830  &  632,45553 &  0,6324555 &   16,06437 \\
     350 &   0,274889 &   177,3476  &  591,60798 &  0,5916080 &   15,02684 \\
     300 &   0,235619 &   152,0122  &  547,72256 &  0,5477226 &   13,91215 \\
     250 &   0,196350 &   126,6769  &  500,00000 &  0,5000000 &   12,70000 \\
    \bottomrule[1pt]
    \end{tabular}
\end{center}

D'aquesta taula es pot extreure la següent relació aproximada entre una secció $S$ expressada en $\unit{mm^2}$ i la mateixa secció $S$ expressada en $\unit{kcmil}$:
\begin{equation}
  S/{\scriptstyle\unit{mm^2}} \approx \frac{ S/{\scriptstyle\unit{kcmil}}}{2}
\end{equation}



\subsection{\textit{American Wire Gauge} (AWG)}\label{sec:awg}
\index{AWG (\textit{American Wire Gauge})}

\index{AWG (\textit{American Wire Gauge})!definició}
L'\textit{American Wire Gauge}, anomenat també \textit{Brown \& Sharp Gauge}, és un sistema de numeració de conductors circulars segons el seu diàmetre. A cada número AWG li correspon un valor de diàmetre; els successius diàmetres formen una progressió geomètrica descendent (en augmentar el número AWG disminueix el diàmetre).

La raó d'aquesta progressió geomètrica s'obté de la següent consideració: hi ha dos valors de referència: 36 AWG, el qual té assignat un diàmetre de \qty{5}{mil}, i 0000 AWG, el qual té assignat un diàmetre de \qty{460}{mil}. Entre aquests dos valors de referència hi ha una diferència de 39 unitats (vegeu la taula \vref{taula:AWG}), i per tant, sent $r\ped{d}$ la raó de diàmetres buscada, tenim:

\begin{equation}
   \qty{460}{mil} \times r\ped{d}^{39} = \qty{5}{mil} \quad \Rightarrow \quad r\ped{d} = \left( \frac{\qty{5}{mil}}{\qty{460}{mil}} \right)^{1/39} = \left( \frac{1}{92} \right)^{1/39} = 92^{-1/39}
\end{equation}

En ser la secció d'un conductor proporcional al quadrat del seu diàmetre, les seccions dels successius números AWG formen una progressió geomètrica  descendent de raó $r\ped{S}$ igual a: \begin{equation}
   r\ped{S} = r\ped{d}^2 = 92^{-2/39}
\end{equation}

Finalment, en ser la resistència d'un conductor inversament proporcional a la seva secció, les resistències dels successius números AWG formen una progressió geomètrica ascendent de raó $r\ped{R}$ igual a:
\begin{equation}
   r\ped{R} = \frac{1}{r\ped{S}} = 92^{2/39}
\end{equation}

A partir d'aquestes raons, i coneixent el diàmetre $d$, la secció $S$ i la resistència $R$ d'un número AWG $n$, podem calcular aquests tres paràmetres per a un altre número AWG, $k$ unitats posterior o $k$ unitats anterior:

\begin{equation}
   \begin{array}{rllllll}
     \text{AWG:}         & & n & & n+k                & & n-k \\
     \text{Diàmetre:}    & & d & & d\, 92^{-k/39}  & & d\, 92^{k/39} \\
     \text{Secció:}      & & S & & S\, 92^{-2k/39} & & S\, 92^{2k/39} \\
     \text{Resistència:} & & R & & R\, 92^{2k/39}  & & R\, 92^{-2k/39}
   \end{array}
\end{equation}

Per a alguns valors particulars de $k$ es compleixen de forma aproximada les següents relacions:

\begin{list}{}
   {\setlength{\labelwidth}{15mm} \setlength{\leftmargin}{17mm} \setlength{\labelsep}{2mm}}

   \item[$\boldsymbol{k=6}$\hfill] En augmentar en 6  unitats un número AWG, el diàmetre es divideix per 2
                 $(92^{-6/39}\approx \num{0,5})$.

   \item[$\boldsymbol{k=-6}$\hfill] En disminuir en 6 unitats un número AWG, el diàmetre es multiplica per 2
                 $(92^{6/39}\approx 2)$.

   \item[$\boldsymbol{k=20}$\hfill] En augmentar en 20  unitats un número AWG, el diàmetre es divideix per 10
                 $(92^{-20/39}\approx \num{0,1})$.

   \item[$\boldsymbol{k=-20}$\hfill] En disminuir en 20 unitats un número AWG, el diàmetre es multiplica per 10
                 $(92^{20/39}\approx 10)$.

   \item[$\boldsymbol{k=3}$\hfill] En augmentar en 3 unitats un número AWG, la secció es divideix per 2
                 $(92^{-2\times 3/39}\approx \num{0,5})$ i la resistència es multiplica per 2
                 $(92^{2\times 3/39}\approx 2)$.

   \item[$\boldsymbol{k=-3}$\hfill] En disminuir en 3 unitats un número AWG, la secció es multiplica per 2
                  $(92^{2\times 3/39}\approx 2)$ i la resistència es divideix per 2
                  $(92^{-2\times 3/39}\approx \num{0,5})$.

   \item[$\boldsymbol{k=10}$\hfill] En augmentar en 10 unitats un número AWG, la secció es divideix per 10
                 $(92^{-2\times 10/39}\approx \num{0,1})$ i la resistència es multiplica per 10
                 $(92^{2\times 10/39}\approx 10)$.

   \item[$\boldsymbol{k=-10}$\hfill] En disminuir en 10 unitats un número AWG, la secció es multiplica per 10
                  $(92^{2\times 10/39}\approx 10)$ i la resistència es divideix per 10
                  $(92^{-2\times 10/39}\approx \num{0,1})$.
\end{list}

\index{AWG (\textit{American Wire Gauge})!equivalències}En la taula \vref{taula:AWG} es relacionen els diàmetres i seccions en diverses unitats dels conductors compresos entre 0000 AWG i 40 AWG.

\begin{longtable}{S[table-format=4.0]S[table-format=3.3]
                  S[table-format=2.6]S[table-format=2.4]
                  S[table-format=6.3]S[table-format=1.3e1]
                  S[table-format=3.6]}
\caption{\label{taula:AWG}Dimensions de cables AWG} \\
\toprule[1pt]
    \multicolumn{1}{c}{Cable}  &    \multicolumn{3}{c}{Diàmetre} &   \multicolumn{3}{c}{Secció}         \\
    \cmidrule(rl){1-1} \cmidrule(rl){2-4} \cmidrule(rl){5-7}
      \multicolumn{1}{c}{AWG} &   \multicolumn{1}{c}{mil}  & \multicolumn{1}{c}{in}  & \multicolumn{1}{c}{mm}  &   \multicolumn{1}{c}{cmil} & \multicolumn{1}{c}{\unit{in^2}}  & \multicolumn{1}{c}{\unit{mm^2}} \\
\midrule \endfirsthead
\caption[]{Dimensions de cables AWG (\emph{ve de la pàgina anterior})} \\
\toprule[1pt]
    \multicolumn{1}{c}{Cable} & \multicolumn{3}{c}{Diàmetre} & \multicolumn{3}{c}{Secció} \\
    \cmidrule(rl){1-1} \cmidrule(rl){2-4} \cmidrule(rl){5-7}
    \multicolumn{1}{c}{AWG} &   \multicolumn{1}{c}{mil}  & \multicolumn{1}{c}{in}  & \multicolumn{1}{c}{mm}  &   \multicolumn{1}{c}{cmil} & \multicolumn{1}{c}{\unit{in^2}}  & \multicolumn{1}{c}{\unit{mm^2}} \\
\midrule \endhead
\midrule
\multicolumn{7}{r}{\sffamily\bfseries\color{NavyBlue}(\emph{continua a la pàgina següent})}
\endfoot
\endlastfoot
{0000} & 460,000 &   0,460000 &    11,6840 & 211600,000 &  1,662e-1 & 107,219303 \\
{\phantom{0}000} & 409,642 &   0,409642 &    10,4049 & 167806,429 &  1,318e-1 &  85,028773 \\
{\phantom{00}00}  &  364,797 &   0,364797 &     9,2658 & 133076,548 &  1,045e-1 &  67,430882 \\
  0  &  324,861 &   0,324861 &     8,2515 & 105534,501 &  8,289e-2 &  53,475121 \\
 1 &    289,297 &   0,289297 &     7,3481 &  83692,664 &  6,573e-2 &  42,407699 \\
 2 &    257,626 &   0,257626 &     6,5437 &  66371,300 &  5,213e-2 &  33,630834 \\
 3 &    229,423 &   0,229423 &     5,8273 &  52634,834 &  4,134e-2 &  26,670464 \\
 4 &    204,307 &   0,204307 &     5,1894 &  41741,321 &  3,278e-2 &  21,150639 \\
 5 &    181,941 &   0,181941 &     4,6213 &  33102,372 &  2,600e-2 &  16,773220 \\
 6 &    162,023 &   0,162023 &     4,1154 &  26251,375 &  2,062e-2 &  13,301768 \\
 7 &    144,285 &   0,144285 &     3,6649 &  20818,287 &  1,635e-2 &  10,548782 \\
 8 &    128,490 &   0,128490 &     3,2636 &  16509,652 &  1,297e-2 &   8,365564 \\
 9 &    114,424 &   0,114424 &     2,9064 &  13092,749 &  1,028e-2 &   6,634194 \\
10 &    101,897 &   0,101897 &     2,5882 &  10383,022 &  8,155e-3 &   5,261155 \\
11 &     90,742 &   0,090742 &     2,3048 &   8234,111 &  6,467e-3 &   4,172286 \\
12 &     80,808 &   0,080808 &     2,0525 &   6529,947 &  5,129e-3 &   3,308773 \\
13 &     71,962 &   0,071962 &     1,8278 &   5178,483 &  4,067e-3 &   2,623976 \\
14 &     64,084 &   0,064084 &     1,6277 &   4106,724 &  3,225e-3 &   2,080908 \\
15 &     57,068 &   0,057068 &     1,4495 &   3256,780 &  2,558e-3 &   1,650235 \\
16 &     50,821 &   0,050821 &     1,2908 &   2582,744 &  2,028e-3 &   1,308696 \\
17 &     45,257 &   0,045257 &     1,1495 &   2048,209 &  1,609e-3 &   1,037843 \\
18 &     40,303 &   0,040303 &     1,0237 &   1624,304 &  1,276e-3 &   0,823047 \\
19 &     35,891 &   0,035891 &     0,9116 &   1288,131 &  1,012e-3 &   0,652706 \\
20 &     31,961 &   0,031961 &     0,8118 &   1021,535 &  8,023e-4 &   0,517619 \\
21 &     28,462 &   0,028462 &     0,7229 &    810,114 &  6,363e-4 &   0,410491 \\
22 &     25,347 &   0,025347 &     0,6438 &    642,449 &  5,046e-4 &   0,325534 \\
23 &     22,572 &   0,022572 &     0,5733 &    509,486 &  4,001e-4 &   0,258160 \\
24 &     20,101 &   0,020101 &     0,5106 &    404,040 &  3,173e-4 &   0,204730 \\
25 &     17,900 &   0,017900 &     0,4547 &    320,419 &  2,517e-4 &   0,162359 \\
26 &     15,941 &   0,015941 &     0,4049 &    254,104 &  1,996e-4 &   0,128756 \\
27 &     14,196 &   0,014196 &     0,3606 &    201,513 &  1,583e-4 &   0,102108 \\
28 &     12,641 &   0,012641 &     0,3211 &    159,807 &  1,255e-4 &   0,080976 \\
29 &     11,258 &   0,011258 &     0,2859 &    126,733 &  9,954e-5 &   0,064217 \\
30 &     10,025 &   0,010025 &     0,2546 &    100,504 &  7,894e-5 &   0,050926 \\
31 &      8,928 &   0,008928 &     0,2268 &     79,703 &  6,260e-5 &   0,040386 \\
32 &      7,950 &   0,007950 &     0,2019 &     63,207 &  4,964e-5 &   0,032028 \\
33 &      7,080 &   0,007080 &     0,1798 &     50,126 &  3,937e-5 &   0,025399 \\
34 &      6,305 &   0,006305 &     0,1601 &     39,752 &  3,122e-5 &   0,020142 \\
35 &      5,615 &   0,005615 &     0,1426 &     31,524 &  2,476e-5 &   0,015974 \\
36 &      5,000 &   0,005000 &     0,1270 &     25,000 &  1,963e-5 &   0,012668 \\
37 &      4,453 &   0,004453 &     0,1131 &     19,826 &  1,557e-5 &   0,010046 \\
38 &      3,965 &   0,003965 &     0,1007 &     15,723 &  1,235e-5 &   0,007967 \\
39 &      3,531 &   0,003531 &     0,0897 &     12,469 &  9,793e-6 &   0,006318 \\
40 &      3,145 &   0,003145 &     0,0799 &      9,888 &  7,766e-6 &   0,005010 \\
\bottomrule[1pt]
\end{longtable}

\index{AWG (\textit{American Wire Gauge})!conversió a \unit{mm^2}} Es donen a continuació les equacions per passar directament d'un número AWG a la seva secció equivalent $S$  expressada en \unit{mm^2}, i al seu diàmetre equivalent $d$  expressat en \unit{mm}.
\begin{align}
   S/{\scriptstyle\unit{mm^2}}  &= \frac{ \piup \times \num{25,4}^2 \times 460^2}{4 \times 10^6 \times 92^{\frac{2(\text{AWG}+3)}{39}}} =
   \frac{ \piup \times \num{25,4}^2 \times 460^2}{4 \times 10^6 \times 92^{6/39} \times 92^{\text{AWG}/\num{19,5}}} \approx
   \frac{\num{53,4751207321}}{92^{\text{AWG}/\num{19,5}}}\label{eq:awg_mm2}\\[2ex]
   d/{\scriptstyle\unit{mm}}  &= \sqrt{\frac{\num{25,4}^2 \times 460^2}{10^6 \times 92^{\frac{2(\text{AWG}+3)}{39}}}} =
   \frac{\num{25,4} \times 460}{10^3 \times 92^{3/39} \times 92^{\text{AWG}/39}} \approx
   \frac{\num{8,25146280217}}{92^{\text{AWG}/39}}\label{eq:awg_mm}
\end{align}

En aquestes dues equacions cal utilitzar els valors $-1, -2$ i $-3$  pels números 00 AWG,
000 AWG i 0000 AWG respectivament.


\begin{exemple}[Secció en \unit{mm^2} d'un conductor AWG]
	\addcontentsxms{Secció en \unit{mm^2} d'un conductor AWG}
    Es tracta de calcular la secció $S$ en \unit{mm^2} d'un conductor 14 AWG.

    Utilitzant l'equació \eqref{eq:awg_mm2} tenim:
    \[
        S = \dfrac{\num{53,4751207321}}{92^{14/\num{19,5}}} 
        \,\unit{mm^2} = \qty{2,1}{mm^2}
    \]
\end{exemple}


Es donen a continuació dues equacions que poden considerar-se les inverses de les equacions \eqref{eq:awg_mm2} i \eqref{eq:awg_mm}, ja que ens permeten trobar el número AWG aproximat, corresponent a una secció $S$ donada en \unit{mm^2} o a un diàmetre $d$ donat en \unit{mm}:
\begin{align}
   \text{AWG} &= \frac{\num{19,5}}{\ln 92} \times \ln \frac{\piup \times
   \num{25,4}^2 \times 460^2}{4 \times 10^6 \times 92^{6/39} \times S/{\scriptstyle\unit{mm^2}}} \approx
   \num{4,31245284200} \times \ln \frac{\num{53,4751207321}}{S/{\scriptstyle\unit{mm^2}}}\label{eq:mm2_awg} \\[2ex]
   \text{AWG} &= \frac{39}{\ln 92} \times \ln \frac{\num{25,4} \times 460}{10^3 \times 92^{3/39} \times d/{\scriptstyle\unit{mm}}} \approx
   \num{8,62490568399} \times \ln \frac{\num{8,25146280217}}{d/{\scriptstyle\unit{mm}}}\label{eq:mm_awg}
\end{align}

Aquestes dues equacions ens donaran en general un valor decimal que caldrà arrodonir al valor enter  més proper.

Si obtenim com a resultat els números $-1, -2$ o $-3$, cal recordar que aquests valors equivalen als números 00 AWG,
000 AWG i 0000 AWG respectivament. Si s'obtenen números més negatius ($-4, -5,$ $-6, ...$) això ens indica que no hi ha cap número AWG corresponent a la nostra secció o diàmetre, ja que el màxim valor possible és 0000 AWG.


\begin{exemple}[Número AWG corresponent a una secció en \unit{mm^2}]\label{ex:mm2-a-AWG}
	\addcontentsxms{Número AWG corresponent a una secció en \unit{mm^2}}
    Es tracta de calcular el número AWG aproximat, corresponent a un conductor de $S=\qty{4}{mm^2}$.

    Utilitzant l'equació \eqref{eq:mm2_awg} tenim:
    \[
        \text{AWG} = \num{4,31245284200} \times \ln \dfrac{\num{53,4751207321}}{4} =
        \num{11,18} \, \approx \, 11
    \]
\end{exemple}

A més de les sigles «AWG», hi ha altres formes alternatives d'escriptura. En el cas dels conductors compresos entre 1 AWG i 40 AWG, també es pot veure escrit (prenent com a exemple el conductor 4 AWG):
\begin{itemize}
   \item \#4 (on el símbol «\#» s'utilitza com a substitut de \textit{number})
   \item No. 4 (on «No.» és l'abreviació de \textit{number})
   \item No. 4 AWG
   \item 4 ga. (on «ga.» és l'abreviació de \textit{gauge})
\end{itemize}

En el cas dels conductors 0 AWG, 00 AWG, 000 AWG i 0000 AWG també es pot veure escrit (prenent com a exemple el conductor 000 AWG):
\begin{itemize}
   \item 3/0
   \item 3/0 AWG
   \item \#000
   \item \#3/0
\end{itemize}


En el cas de cables formats per més d'un conductor, el cable es denomina utilitzant la secció dels conductors, seguida del nombre de conductors que formen el cable. Per exemple: \#14/2 o 14-2 identifica un cable format per dos conductors de 14 AWG.

En el cas d'un conductor format per múltiples fils, el cable es denomina utilitzant l'AWG total (suma de les seccions de cada fil), seguit del nombre de fils i de l'AWG de cada fil. Per exemple: 22 AWG 7/30 identifica un conductor de 22 AWG, format per 7 fils de  30 AWG.

