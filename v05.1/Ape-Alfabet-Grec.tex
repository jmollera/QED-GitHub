\chapter{Alfabet Grec} \index{alfabet grec}

 En la Taula \vref{taula:alfabet-grec} es pot veure l'alfabet grec,
 amb els noms de les lletres en diversos idiomes.

\begin{table}[h]
   \caption{\label{taula:alfabet-grec} Alfabet grec}
   \begin{center}\begin{tabular}{cccllll}
   \toprule[1pt]
   \renewcommand*{\multirowsetup}{\centering}
   \multirow{2}{15mm}{\rule{0mm}{4.5mm}N\'{u}mero\\d'ordre} & \multicolumn{2}{c}{Lletra} &
   \multicolumn{4}{c}{Nom} \\
   \cmidrule(rl){2-3} \cmidrule(rl){4-7}
    & min\'{u}scula & maj\'{u}scula & catal\`{a} & castell\`{a} &  angl\`{e}s & franc\`{e}s\\
   \midrule
   1  & $\alphaup$ & A & alfa & alfa &  alpha & alpha\\
   2  & $\betaup$ & B & beta & beta &  beta & b\^{e}ta\\
   3  & $\gammaup$ & $\Gammaup$ & gamma & gamma &  gamma & gamma\\
   4  & $\deltaup$ & $\Deltaup$ & delta & delta &  delta & delta\\
   5  & $\epsilonup$, $\varepsilonup$ & E & \`{e}psilon & \'{e}psilon &  epsilon & epsilon\\
   6  & $\zetaup$ & Z & zeta & dseda &  zeta & z\^{e}ta\\
   7  & $\etaup$ & H & eta & eta &  eta & \^{e}ta\\
   8  & $\thetaup$, $\varthetaup$ & $\Thetaup$ & theta & zeta &  theta & th\^{e}ta\\
   9  & $\iotaup$ & I & iota & iota &  iota & iota\\
   10 & $\kappaup$, $\varkappaup$ & K & kappa & kappa &  kappa & kappa\\
   11 & $\lambdaup$ & $\Lambdaup$ & lambda & lambda &  lambda &lambda\\
   12 & $\muup$ & M & mi & mi &  mu & mu\\
   13 & $\nuup$ & N & ni & ni &  nu & nu\\
   14 & $\xiup$ & $\Xiup$ & ksi & xi &  xi & ksi, xi\\
   15 & o & O & \`{o}micron & \'{o}micron &  omicron & omicron\\
   16 & $\piup$, $\varpiup$ & $\Piup$ & pi & pi &  pi & pi\\
   17 & $\rhoup$, $\varrhoup$ & P & rho, ro & ro &  rho & rh\^{o}\\
   18 & $\sigmaup$, $\varsigmaup$ & $\Sigmaup$ & sigma & sigma &  sigma &sigma\\
   19 & $\tauup$ & T & tau & tau & tau &tau\\
   20 & $\upsilonup$ & $\Upsilonup$ & \'{\i}psilon & \'{\i}psilon &  upsilon &upsilon\\
   21 & $\phiup$, $\varphiup$ & $\Phiup$ & fi & fi &  phi & phi\\
   22 & $\chiup$ & X & khi & ji &  chi & khi\\
   23 & $\psiup$ & $\Psiup$ & psi & psi &  psi & psi\\
   24 & $\omegaup$ & $\Omegaup$ & omega & omega &  omega & om\'{e}ga\\
   \bottomrule[1pt]
   \end{tabular} \end{center}
\end{table}

Les dues grafies de la lletra min\'{u}scula \`{e}psilon  ($\epsilonup,
\varepsilonup$) s\'{o}n totalment equivalents entre s\'{\i}; el mateix passa
amb les dues grafies de les lletres min\'{u}scules theta ($\thetaup,
\varthetaup$), kappa $(\kappaup, \varkappaup)$, rho ($\rhoup,\varrhoup$) i fi ($\phiup, \varphiup$).

La lletra sigma min\'{u}scula t\'{e} dues variants: $\varsigmaup$, escrita en
grec al final d'una paraula, i $\sigmaup$, escrita en grec a l'inici o
en mig d'una paraula. En els textos t\`{e}cnics i cient\'{\i}fics s'utilitza
majorit\`{a}riament la variant $\sigmaup$.

La variant $\varpiup$ de la lletra pi, es denomina {"<}pi d\`{o}rica{">} en
catal\`{a}, {"<}pi d\'{o}rica{">} en castell\`{a}, {"<}dorian pi{">} en angl\`{e}s i {"<}pi dorique{">} en franc\`{e}s.

Pel que fa als noms de les lletres, alguns poden sorprendre; aix\`{o} no
\'{e}s estrany ja que algunes lletres han rebut hist\`{o}ricament noms
diversos, i fins i tot contradictoris respecte dels actuals.

Els noms anglesos de les lletres s\'{o}n els m\'{e}s uniformes, ja que no
s'ha observat cap variaci\'{o} en les diverses fonts consultades.

Els noms catalans de les lletres s\'{o}n els que apareixen en el DIEC2 {"<}Diccionari de la llengua catalana, 2a edici\'{o} (2007){">}. Altres noms utilitzats en
les diverses fonts consultades s\'{o}n:
\begin{multicols}{3}
\begin{list}{}
   {\setlength{\labelwidth}{16mm} \setlength{\leftmargin}{16mm} \setlength{\labelsep}{2mm}}
   \item[B, $\betaup :$] vita.
   \item[Z, $\zetaup :$] zita.
   \item[H, $\etaup :$] ita.
   \item[$\Thetaup$, $\thetaup :$] thita.
   \item[T, $\tauup :$] taf.
   \item[$\xiup$, $\Xiup$:] csi.\footnote{Aquest nom apareix juntament amb {"<}ksi{">} en el {"<}Gran Diccionari de la Llengua Catalana{">} (1999).}
\end{list}
\end{multicols}

Els noms castellans de les lletres s\'{o}n els que apareixen en el D.R.A.E.
{"<}Diccionario de la Lengua Espa\~{n}ola, 22\textordfeminine\
edici\'{o}n (2001){">}. Altres noms utilitzats en les diverses fonts
consultades s\'{o}n:
\begin{multicols}{3}
\begin{list}{}
   {\setlength{\labelwidth}{16mm} \setlength{\leftmargin}{16mm} \setlength{\labelsep}{2mm}}
   \item[Z, $\zetaup :$] zeta\footnote{Aquests noms eren els que apareixien en les edicions
   del D.R.A.E anteriors a la 21a (1992).}, dseta, dzeta.
   \item[$\Thetaup$, $\thetaup :$] theta\footnotemark[2], thita.
   \item[K, $\kappaup :$] cappa.
   \item[M, $\muup :$] my\footnotemark[2], mu.
   \item[N, $\nuup :$] ny\footnotemark[2], nu.
   \item[O, o :] omicr\'{o}n.
   \item[P, $\rhoup :$] rho.
   \item[$\Upsilonup$, $\upsilonup :$] \'{u}psilon.
   \item[$\Phiup$, $\phiup :$] phi.
\end{list}
\end{multicols}

Els noms francesos de les lletres s\'{o}n els que apareixen en el {"<}Dictionnaire de l'Acad\'{e}mie fran\c{c}aise, neuvi\`{e}me \'{e}dition{">}. 